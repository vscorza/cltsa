% Cite GS, realizability 
% Cite FDS 

In this section we introduce the models and definitions for Fair Discrete Systems (FDS). Fair Discrete Systems ~\cite{kesten2000verification},~\cite{bloem2012synthesis} are used as a symbolic representation of a finite state transition system over Boolean variables. 
%Game structures were introduced  to model open systems where properties are split between assumptions and guarantees.

In the following definitions we will denote $2^{\gsV}$ as $\Sigma_{\gsV}$, and $s|_{\gsY}$ as the projection of $s$ to $\gsY$, i.e., $s_{\gsY} = \{y \in \gsY | y \in s \}$.

\begin{definition}
	\label{def:FDS} \emph{(Fair Discrete System)} 
	A \emph{Fair Discrete System} (FDS) \fdsDef, where:
	\begin{itemize}
		\item $\gsV = \{v_1,\ldots,v_n \}$: A finite set of Boolean variables. We define a \emph{state} $s$ to be an interpretation of $\gsV$, i.e., $s \in \Sigma_{\gsV}$.
		\item $\theta$: The \emph{initial condition}. This is an assertion over $\gsV$ characterizing all the initial states of the FDS. A state is called \emph{initial} if it satisfies $\theta$.
		\item $\rho$: A \emph{transition relation}. This is an assertion $\rho(\gsV \cup \gsV')$, relating a state $s \in \Sigma_{\gsV}$ to its $\fdsD$-successors $s' \in \Sigma_{\gsV}$, i.e. $(s,s') \models \rho$.
		\item $\mathcal{J} = \{J_1, \ldots, J_m \}$: A set of \emph{justice requirements} (weak fairness). Each requirement $J \in \mathcal{J}$ is an assertion over $\gsV$ that is intended to hold infinitely many times in every computation.
		\item $\mathcal{C} = \{(P_1,Q_1), \ldots, (P_n,Q_n) \}$: A set of \emph{compassion requirements} (strong fairness). Each requirement $(P,Q) \in \mathcal{C}$ consists of a pair of assertions, such that if a computation contains infinitely many $P$-states, it should also hold infinitely many $Q$-states.
	\end{itemize}
\end{definition}

\begin{definition}
	\label{def:fds_run} \emph{(Run of a FDS)} 
	A \emph{run} of the FDS $\fdsD$ is a maximal sequence of states $\sigma = s_0,s_1,\ldots$
	satisfying:
	\begin{itemize}
		\item \emph{initiality}, i.e., $s_0\models \theta$ and
		\item \emph{consecution}, i.e., for every $j\geq 0, (s_j, s_{j+1})\models \rho$.
	\end{itemize}
	A sequence $\sigma$ is maximal if either $\sigma$ is infinite or $\sigma=s_0,\ldots,s_k$ and $s_k$ has no $\fdsD$-successor, i.e., for all $s_{k+1} \in \Sigma_{\gsV}, (s_k,s_{k+1}) \not\models \rho$
\end{definition}


\begin{definition}
	\label{def:complete_fds} (\emph{Complete w.r.t.} $\gsX$)
	We say that an FDS $\fdsD$ \emph{is complete with respect to} $\gsX \subseteq \gsV$, if:
	\begin{itemize}
		\item for every assignment $s_{\gsX} \in \Sigma_{\gsX}$, there exists a state $s \in \Sigma_{\gsV}$ such that $s|_{\gsX} = s_{\gsX}$ and $s \models \theta$
		\item for all states $s \in \Sigma_{\gsV}$ and assignments $s'_{\gsX} \in \Sigma_{\gsX}$, there exists a state $s' \in \Sigma_{\gsV'}$ such that $s'|_{\gsX}=s'_{\gsX}$ and $(s,s') \models \rho$
	\end{itemize}
\end{definition}

\begin{definition}
	\label{def:fds_realizability} (\emph{Realizability of $\varphi$ by an FDS})
	Given an LTL formula $\varphi$ over sets of input and output variables $\gsX$ and $\gsY$ respectively, we say that an FDS \fdsDef \emph{realizes} $\varphi$ if:
	\begin{itemize}
		\item $\gsV$ contains $\gsX$ and $\gsY$ 
		\item $\fdsD$ is complete with respect to $\gsX$
		\item $\fdsD \models \varphi$
	\end{itemize}
	Such an FDS is called a \emph{controller} for $\varphi$. We say that the specification is realizable ~\cite{pnueli1989synthesis}, if there exists a fairness-free FDS $\fdsD$, i.e., an FDS where $\mathcal{J} =\mathcal{C} = \emptyset$, that realizes it, otherwise we say that the specification is \emph{unrealizable}.
\end{definition}

\begin{definition}
	\label{def:fds_control_problem} \emph{(FDS Control Problem)} 
	Let $\varphi$ be a LTL formula over sets of input and output variables $\gsX$ and $\gsY$, then \fdsControlProblemDef constitutes an FDS control problem. If a solution to $\fdsControlProblem$ exists, such a solution will be the FDS \fdsDef that realizes $\varphi$.
\end{definition}
In the context of an FDS control problem, $x \in \gsX$ will be called an \emph{environmental variable} and $y \in \gsY$ will be called a \emph{system variable}.