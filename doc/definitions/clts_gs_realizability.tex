We introduce an embedding from CLTS control problem into LTL formulas over boolean variables.
%The first thing that will be addressed is the relation between fluent valuations and states in any given CLTS model. Assume $E = \langle S_E, \Sigma_E, \Delta_E, s^E_0 \rangle$ is the automaton that represents the plant, and $\mathcal{F}=\{fl_1,\ldots,fl_k\}$ with $fl_i = \langle I_i, T_i, init_i \rangle$ the set of fluents associated with a control problem. We will construct a new plant that is compatible with the valuation of the fluents at each step of the execution. To do this we build an equivalent automaton for each fluent, compose them with each other and then with the original plant $E$. This way we have a consistent mechanism to define the set of states satisfying any given fluent.  The automaton for the candidate fluent $fl_i$  automaton is constructed as follows:
%\[ FL_i = \langle S_{fl_i}, \Sigma_{fl_i}, \Delta_{fl_i}, s_{0_i}\rangle \]
%Where:
%\[S_{fl_i}= \{s^{\bot}_i, s^{\top}_i\} \]
%\[\Delta_{fl_i} \text{is the minimal relation s.t.}\]
%\[ \forall \alpha \in I_i: \{(s^{\bot}_i,\alpha,s^{\top}_i), (s^{\top}_i,\alpha,s^{\top}_i)\} \in \Delta_{fl_i} \]
%\[ \forall \beta \in T_i: \{(s^{\top}_i,\beta,s^{\bot}_i), (s^{\bot}_i,\beta,s^{\bot}_i)\} \in \Delta_{fl_i} \]
%\[
%s_{0_i} = \begin{cases}
%s^{\top}_i & init_i = \top \\
%s^{\bot}_i & otherwise
%\end{cases}
%\]
%
%The fluent compatible plant $E_G$ is the result of the synchronous parallel composition of the fluent automata with the original plant:
%\[E_G = (E \parallel_s FL_1 ||_s \ldots ||_s FL_k) \]
%And the function that determines which states satisfy which fluent is defined as:
%\[ fl(i) = \{s \in S_{E_G} : s_{(i + 1)} = s^{\top}_i \} \]
Atomic satisfaction of LTL formula $fl_i$ at step $j$ of trace $\pi$ can be redefined as:
\[ \pi_j \models fl_i \]
Where $fl_i$ is the system variable introduced for fluent $FL_i$ in the CLTS control problem.
Now we introduce the CLTS to LTL translation.
\begin{definition}
	\label{def:val_ltl} \emph{(LTL over $\mathcal{F}$ to LTL over $\mathcal{V}$ translation)} 
	Let $\varphi$ be a LTL formula over fluents $\mathcal{F}$, and if $\mathcal{V}$ is the minimal set of boolean variables containing $v_{fl_i}$ for each $f \in \mathcal{F}$, $val(\varphi)$ is an equivalent LTL formula over boolean variables according to the following inductive definition:\\
	
	\begin{tabular}{ l c l }
		$val(fl_i)$ & $\triangleq$ & $v_{fl_i}$\\	
		$val(\neg \varphi)$ & $\triangleq$ & $\neg val(\varphi)$\\
		$val(\varphi \vee \psi$ & $\triangleq$ & $val(\varphi) \vee val(\psi)$\\
		$val(\bigcirc \varphi)$ & $\triangleq$ & $\bigcirc val(\varphi)$\\
		$val(\varphi \U \psi)$ & $\triangleq$ & $val(\varphi) \U val(\psi)$\\
	\end{tabular}	
\end{definition}

\begin{definition}
	\label{def:clts_to_ltl_translation} \emph{(CLTS control problem to LTL embedding)} 
	Let $\langle A, \mathcal{C}, \mathcal{F}(\varphi), \varphi \rangle$ be a CLTS control problem, $ltl(\mathcal{I})$ is the embedding into a tuple consiting of a LTL formula over boolean variables and two disjoint sets of boolean environmental and system variables.
\end{definition}


The embedding from CLTS to LTL is defined as the strict realizability formula of the induced game structure. The alphabet is split in two sets:
\[\gsX = \{x_i : a_i \in \Sigma_A \setminus \mathcal{C} \}\]
%\[\gsY = \{y_i : a_i \in \mathcal{C} \} \cup \{s_i : i \in 1..\lfloor log_2(|S_A|)\rfloor \} \cup \{v_{fl_i} : fl_i \in \mathcal{F}\}\]
\[\gsY = \{y_i : a_i \in \mathcal{C} \} \cup \{s_i : i \in 1..|S_A| \} \cup \{v_{fl_i} : fl_i \in \mathcal{F}\}\]
We will write $s(j)$ to represent the valuation of the system variables representing a given state $j$.
%\[s(j) = \bigwedge\limits^{\lfloor log_2(|S_A|)\rfloor - 1}_{i=0}  s_{bit}(i,j)\]
%\[
%s_{bit}(i,j) = \begin{cases}
%s_i & j / 2^i \\
%\neg s_i & otherwise
%\end{cases}
%\]
To represent the value of fluent $f$ at state $i$ we will use $f_{val}(f,i)$:
\[
f_{val}(f,i) = \begin{cases}
v_{fl_i} & s_i \in fl(f) \\
\neg v_{fl_i} & otherwise
\end{cases}
\]
The initial condition is as follows:
\[\gsTheE = \bigwedge_{i = 1}^{|\Sigma_A \setminus \mathcal{C}|}\neg x_i\]
\[\gsTheS = (\bigwedge_{i = 1}^{|\mathcal{C}|}\neg y_i) \wedge s(0) \wedge (\bigwedge_{f = 1}^{|\mathcal{F}|}f_{val}(f,0))\]
Transition relations are constructed as a conjunction of safety formulas that are added for each state. These can be split into two types, enabling and state updating formulae. The first type restrict the set of variables that can be set to $true$ in a mutually exclusive fashion at each state. The second type defines the state reached for each combination of current state and enabled variables. 
We will use the following two functions to build the conjunction of system or environment variables related to the label in any given automaton transition:
\[\delta_e(l) = \{\bigwedge_{\sigma_i \in \Sigma \setminus \mathcal{C}}var(l, \sigma_i)\}\]
\[\delta_s(l) = \{\bigwedge_{\sigma_i \in \Sigma}var(l, \sigma_i)\}\]
\[
var(l, \sigma_i) = \begin{cases}
\sigma_i & \sigma_i \in l \\
\neg \sigma_i & otherwise
\end{cases}
\]
And the mutually exclusive condition needed for any given state $s_i$ will be described as:
\[mux_e(s_i,L) = s(i) \wedge \bigvee_{l \in L}(\delta_e(l) \wedge \bigwedge_{l' \in L \setminus \{l\}}(\neg \delta_e(l')) ) \]
\[mux_s(s_i,L) = s(i) \wedge \bigvee_{l \in L}(\delta_s(l) \wedge \bigwedge_{l' \in L \setminus \{l\}}(\neg \delta_s(l')) ) \]
Then we can build the transition relations $\gsRhoE$ and $\gsRhoS$ from $\Delta_A$ as follows:

\[ \gsRhoE = \bigwedge_{s_i \in S_A} \bigcirc(mux_e(s_i,\cup_{(s_i,l,s_j)}l))\]
\[ \gsRhoS = (\bigwedge_{s_i \in S_A} \bigcirc(mux_s(s_i,\cup_{(s_i,l,s_j)}l)) \wedge (\bigwedge_{(s_k,l',s_l) \in \Delta_A} (s(k) \wedge \delta_s(l') \implies \bigcirc(s(l))\]

Embedding $ltl$ is defined as:

\[ \varphi_{ltl}(\theta_e \implies \theta_s) \wedge (\theta_e \implies \square((\boxdot \rho_e) \implies \rho_s)) \wedge (\theta_e \wedge \rho_e \implies val(\varphi)) \]

\[ltl(\mathcal{I}) = \langle \varphi_{ltl}, \gsX, \gsY\rangle \]

%And we can prove:
%
%\[E\parallel C \models  \varphi \iff gs(C) \models (\theta_e \implies \theta_s) \wedge (\theta_e \implies \square((\boxdot \rho_e) \implies \rho_s)) \wedge (\theta_e \wedge \rho_e \implies val(\varphi))
%\]
%\[clts(G) \parallel clts(D) \models  fl(\varphi) \iff D \models \varphi \]
In order to prove the correctness of our embedding we need to present and prove the following realizability-preserving theorem.

\begin{theorem}(\emph{$ltl$ preserves realizability})\label{theorem:gs_preserves_realizability}\\
	Let $I = \langle E, \mathcal{C}, \mathcal{F}, \varphi \rangle$ be a CLTS control problem with $E = \langle S_e,\Sigma,$ $\Delta_e,$ $s_{e_0}\rangle$, $\Sigma = \mathcal{C}\uplus \mathcal{U}$ then if $ltl(\mathcal{I}) = \langle \varphi_{ltl}, \gsX, \gsY\rangle$:
	\small
	\[(\exists D= \langle \mathcal{V}_d, \theta_d, \rho_d, \mathcal{J}_d, \mathcal{C}_d\rangle  \text{ where }  \gsX, \gsY \in \mathcal{V}_d, \mathcal{D} \text{ is fairness-free, complete w.r.t. } \gsX \text{: } D \models \varphi_{ltl})\] \[ \iff (\exists C \text{ legal w.r.t. } E,\mathcal{C} \text{: } C \parallel E \text{ is deadlock-free and } C \parallel E \models \varphi)  \]
	\normalsize
\end{theorem}