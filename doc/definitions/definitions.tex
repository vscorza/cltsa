\documentclass{article}
%Packages being imported that add functionality to LaTeX.
\usepackage[utf8]{inputenc}
\usepackage{geometry}
\usepackage{amssymb,amsmath,amsthm,mathtools,bussproofs,turnstile}
\usepackage{enumerate}
\usepackage{listings}

%Theorem Environments
\newtheorem{theorem}{Theorem}[section]
\newtheorem{lemma}[theorem]{Lemma}
\theoremstyle{definition}
\newtheorem{definition}{Definition}[section]
\newtheorem{corollary}[theorem]{Corollary}

\newcommand{\automaton}[1]{$#1 = (S_{#1}, \Sigma_{#1}, \Delta_{#1}, s_{0}^{#1})$}
\newcommand{\ltsComposition}[3]{$#1 \parallel_{#3} #2 = (S_{#1}\times S_{#2}, \Sigma_{#1} \cup \Sigma_{#2}, \Delta, (s_{0}^{#1},s_{0}^{#2}))$}

\title{Concurrent CLTS parallel composition} %Change this to the appropriate number.
\author{-} %Change this to your name.
\date{}

\begin{document}

\maketitle

%Counters for setting the appropriate numbering.
\setcounter{section}{1} %This gives the chapter number.
\setcounter{theorem}{1} %This gives the item number.
%Note that the item number should be one less than you desire..

\begin{definition} 
	Let \automaton{M},\automaton{N}, with $\Delta_M : S_M \times \mathcal{P}(\Sigma_M) \times S_M$, be two CLTS instances, then CLTS concurrent parallel composition is defined as \ltsComposition{M}{N}{c} where $\Delta$ is the smallest relation s.t:
	\begin{center}
		\begin{equation}
		\AxiomC{$(s, l_M, s') \in \Sigma_M,(t, l_N, t') \in \Sigma_N  $}\RightLabel{$l_M \cap \Sigma_N = l_N \cap \Sigma_M$}
		\UnaryInfC{$((s,t),l_M \cup l_N,(s',t')) \in \Delta$}
		\DisplayProof
		\end{equation}	
		\begin{equation}
		\AxiomC{$(s, l_M, s') \in \Sigma_M $}\RightLabel{$l_M \cap \Sigma_N = \emptyset$}
		\UnaryInfC{$((s,t),l_M,(s',t))$}
		\DisplayProof
		\quad\quad
		\AxiomC{$(t, l_N, t') \in \Sigma_N $}\RightLabel{$l_N \cap \Sigma_M = \emptyset$}
		\UnaryInfC{$((s,t),l_N,(s,t'))$}
		\DisplayProof
		\end{equation}
	\end{center}
\end{definition}

\begin{definition} 
	Let \automaton{M},\automaton{N}, with $\Delta_M : S_M \times \mathcal{P}(\Sigma_M) \times S_M$, be two CLTS instances, then CLTS asynchronous parallel composition is defined as \ltsComposition{M}{N}{a} where $\Delta$ is the smallest relation s.t:
	\begin{center}
		\begin{equation}
		\AxiomC{$(s, l_M, s') \in \Sigma_M,(t, l_N, t') \in \Sigma_N  $}\RightLabel{$l_M \cap \Sigma_N = l_N \cap \Sigma_M \neq \emptyset$}
		\UnaryInfC{$((s,t),l_M \cup l_N,(s',t')) \in \Delta$}
		\DisplayProof
		\end{equation}	
		\begin{equation}
		\AxiomC{$(s, l_M, s') \in \Sigma_M $}\RightLabel{$l_M \cap \Sigma_N = \emptyset$}
		\UnaryInfC{$((s,t),l_M,(s',t))$}
		\DisplayProof
		\quad\quad
		\AxiomC{$(t, l_N, t') \in \Sigma_N $}\RightLabel{$l_N \cap \Sigma_M = \emptyset$}
		\UnaryInfC{$((s,t),l_N,(s,t'))$}
		\DisplayProof
		\end{equation}
	\end{center}
\end{definition}

\begin{definition} 
	Let \automaton{M},\automaton{N}, with $\Delta_M : S_M \times \mathcal{P}(\Sigma_M) \times S_M$, be two CLTS instances, then CLTS synchronous parallel composition is defined as \ltsComposition{M}{N}{s} where $\Delta$ is the smallest relation s.t:
	\begin{center}
		\begin{equation}
		\AxiomC{$(s, l_M, s') \in \Sigma_M,(t, l_N, t') \in \Sigma_N  $}\RightLabel{$l_M \cap \Sigma_N = l_N \cap \Sigma_M$}
		\UnaryInfC{$((s,t),l_M \cup l_N,(s',t')) \in \Delta$}
		\DisplayProof
		\end{equation}	
	\end{center}
\end{definition}

%\begin{definition} 
%	Let $M_1 \ldots M_N$ be CLTS instances s.t. $M_i= (S_{i}, \Sigma_{i}, \Delta_{i}, s_{0}^{i})$ with $\Delta_i : S_i \times l \subseteq \Sigma_i \times S_i$, the concurrent parallel composition \ltsComposition{}{M_i}{} where $\Delta_i$ is the smallest relation s.t:
%	\begin{center}
%		\begin{equation}
%		\AxiomC{$\forall i,j \in 1 \ldots N, i \neq j: ((s_i, l_i, t_i) \in \Delta_i \wedge \forall a \in l_i : a \in (\Sigma_i \cap \Sigma_j) \rightarrow (a \in l_i \wedge a \in l_j)) \vee (l_i = \emptyset \wedge t_i = s_i)$}
%		\UnaryInfC{$((s_1 \ldots s_N),\cup_{i=1}^{N}l_i,(t_1 \ldots t_N)) \in \Delta$}
%		\DisplayProof
%		\end{equation}		
%	\end{center}
%\end{definition}

%\begin{figure}[ht]
%	\begin{center}
%		\renewcommand{\ttdefault}{pcr}
\begin{lstlisting}[escapeinside={[*}{*]},basicstyle=\scriptsize\ttfamily,columns=flexible,frame=lines,mathescape=true,xleftmargin=3.0ex,keywordstyle=\textbf,morekeywords={if,while,do,else,fork,int,null, algorithm, is, input, output, return},numbers=left,numberstyle=\scriptsize]
algorithm compose_automata is
	input: $A_1 \ldots A_N$  the set of CLTS automata to be composed
	input: $type$ an element of $\{synch,asynch,concurrent\}$ the type of composition to be applied
	output: $A$ the CLTS automaton that results from applying $\parallel_{type}$ to $A$
	$s_0$ = $(s^1_{0},\ldots,s^N_{0})$
	$frontier$ = [$s_0$]
	$A$ = create_automaton($s_0$)
	$visited$ = $[]$
	while(|$frontier$| > 0)
		$\Delta_{union}$ = $\emptyset$
		$\Sigma_{union}$ = $\emptyset$		
		
		$s$ = $frontier$.pop()
		$visited$.push($s$)
		for $s_i \in s$ 
			if $\Delta_{union} = \emptyset$
				$\Delta_{union}$ = $\Delta_i(s)$
			else if $\Delta_i(s) \cap \Sigma_{union} = \emptyset$		
				if $type$ = $asynch$
					$\Delta_{union}$ = $\Delta_{union} \cup \Delta_i(s)$
				else
					$\Delta_{union}$ = $\Delta$_cross_product($\Delta_{union}$, $\Delta_i(s)$, $type$)
			else
				$\Delta_{union}$ = $\Delta$_partial_composition($\Delta_{union}$, $\Delta_i(s)$, $\Sigma_{union}$)
		$\Sigma_{union}$ = $\Sigma_{union} \cup \Sigma_i$
		$\Delta$_expand_frontier($frontier$,$\Delta_{union}$, $visited$)				
		add_transitions($A$, $\Delta_{partial}$)
	return $A$

algorithm $\Delta$_cross_product is
	input: $\Delta$ the composite set of transitions already computed for a given state
	input: $\Delta_{i}(s)$ the local set of transitions for a given state $s$ at automaton $A_i$
	input: $type$ an element of $\{synch,asynch,concurrent\}$ the type of composition to	
	output: $\Delta'$ the cross product of the computed set $\Delta$ with $\Delta_{i}(s)$
	$\Delta'$ = $\emptyset$
	for $d \in \Delta$
		for $d' \in \Delta_{i}(s)$
			$\Delta'$ = $\Delta' \cup$ {$s(d)$,($label(d) \cup label(d')$,($t(d)_1, \ldots, t(d')_i, \ldots, t(d)_N$))}
	if $type \neq synch$
		return $\Delta \cup \Delta' \cup \Delta_{i}(s)$
	else
		return $\Delta'$
	
algorithm $\Delta$_partial_composition is
	input: $\Delta$ the composite set of transitions already computed for a given state
	input: $\Delta_{i}(s)$ the local set of transitions for a given state $s$ at automaton $A_i$
	input: $\Sigma$ the composite set of labels already computed for a given state		
	output: $\Delta'$ the partial composition of the computed set $\Delta$ with $\Delta_{i}$ according to $type$
	$\Delta'$ = $\emptyset$
	for $d \in \Delta$
		for $d' \in \Delta_{i}(s)$
			if $\Sigma \cap label(d') = \Sigma_i \cap label(d)$
				$\Delta'$ = $\Delta' \cup$ {$s(d)$,($label(d) \cup label(d')$,($t(d)_1, \ldots, t(d')_i, \ldots, t(d)_N$))}
	return $\Delta'$
				

\end{lstlisting} 
%		\caption{Concurrent Composition Algorithm}
%		\label{fig:dfs-code}
%	\end{center}
%\end{figure}
\begin{figure}[ht]
	\begin{center}
		\renewcommand{\ttdefault}{pcr}
\begin{lstlisting}[escapeinside={[*}{*]},basicstyle=\scriptsize\ttfamily,columns=flexible,frame=lines,mathescape=true,xleftmargin=3.0ex,keywordstyle=\textbf,morekeywords={if,while,do,else,fork,int,null, algorithm, is, input, output, return, for},numbers=left,numberstyle=\scriptsize]
algorithm compose_automata is
	input: $A_1 \ldots A_N$  the set of CLTS automata to be composed
	input: $type$ an element of $\{synch,asynch,concurrent\}$ the type of composition to be applied
	output: $A$ the CLTS automaton that results from applying $\parallel_{type}$ to $A_1 \ldots A_N$
	$s_0$ = $(s^1_{0},\ldots,s^N_{0})$
	$frontier$ = [$s_0$]
	$A$ = create_automaton($s_0$)
	$visited$ = $[]$
	while(|$frontier$| > 0)
		$\Delta_{I}$ = $\emptyset$
		$\Delta_{a}$ = $\emptyset$
		$\Sigma_{a}$ = $\emptyset$		
		$s$ = $frontier$.pop()
		if $s \in visited$
			continue
		$visited$.push($s$)
		$\Delta_{I}$ = $\Delta_0(s)$
		for $s_{i > 0} \in s$ 
			for $d:(s,l,t) \in \Delta_{I}$
				for $d':(s',l',t') \in \Delta_{i}(s)$
					if $\Sigma_{a} \cap l' = \Sigma_i \cap l \wedge \Sigma_i \cap l \neq \emptyset $
						$\Delta_{a}$ = $\Delta_{a} \cup$ {($s$,$l \cup l'$,($t_1, \ldots, t\prime_i, \ldots, t_N$))}
					if $\Sigma_{a} \cap l' = \Sigma_i \cap l \wedge \Sigma_i \cap l = \emptyset$
						if $type \neq asynch$
							$\Delta_{a}$ = $\Delta_{a} \cup$ {($s$,$l \cup l'$,($t_1, \ldots, t\prime_i, \ldots, t_N$))}
						if $type \neq synch$							
							$\Delta_{a}$ = $\Delta_{a} \cup$ {$(s$,$l$,($t_1, \ldots, t_N$))} 
								$\cup$ {$(s$,$l'$,($s_1, \ldots, t'_i, \ldots, s_N$))}
					if $l \cap l' = \emptyset \wedge type \neq synch$
						if $\Sigma_a \cap l' = \emptyset $
							$\Delta_{a}$ = $\Delta_{a} \cup$ {$(s$,$l'$,($s_1, \ldots, t'_i, \ldots, s_N$))} 
						if $\wedge \Sigma_i \cap l = \emptyset$
							$\Delta_{a}$ = $\Delta_{a} \cup$ {$(s$,$l$,($t_1, \ldots,  t_N$))} 							
			$\Delta_{I} = \Delta_{a}$						
			$\Delta_{a} = \emptyset$						
			$\Sigma_{a}$ = $\Sigma_{a} \cup \Sigma_i$
		for $d:(s,l,t) \in \Delta_{I}$
			if $t \notin visited \wedge t \notin frontier$ 
				 $frontier$.push($t$)
			add_transition($A$, $d$)			
	return $A$

\end{lstlisting} 
		\caption{Concurrent Composition Algorithm (closer to code version)}
		\label{fig:dfs-code2}
	\end{center}
\end{figure}
\end{document}