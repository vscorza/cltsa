\documentclass{article}
%Packages being imported that add functionality to LaTeX.
\usepackage[utf8]{inputenc}
\usepackage{geometry}
\usepackage{amssymb,amsmath,amsthm,mathtools,bussproofs,turnstile}
\usepackage{enumerate}
\usepackage{listings}
\usepackage{../common/vaucanson-g/vaucanson-g}

%\newcommand{\implies}{\Rightarrow}
\newcommand{\traces}{\emph{Tr} }
\newcommand{\true}{\emph{true} }
\newcommand{\false}{\emph{false} }
\newcommand{\assume}{\phi}
\newcommand{\guarantee}{\gamma}
\newcommand{\invariant}{\rho}
\newcommand{\fluentp}[1]{\dot{#1}}
\newcommand{\fluent}{\emph{fl}\xspace}
\newcommand{\set}[1]{\{#1\}}
\newcommand{\gr}{{GR(1)}\xspace}
\newcommand{\sgr}{{SGR(1)}\xspace}
\newcommand{\sublts}{\sqsubseteq}
\newcommand{\subgame}{\subseteq\xspace}
\newcommand{\altsubgame}{\sqsubseteq\xspace}

\newcommand{\act}{a}

%MTS
\newcommand{\transition}[1]{\stackrel{#1}{\longrightarrow}}
\newcommand{\nottransition}[1]{\stackrel{#1}{\not\longrightarrow}}

%CASE STUDY
\newcommand{\drill}{Drill}
\newcommand{\arm}{Arm}

%UTIL
\newcommand{\fix}[1]{\fxfatal{#1}}
\newcommand{\comments}[1]{}
%\newcommand{\todo}[1]{ {\bf TODO: #1 }}

%\newcommand{\G}{{\bf G}}
\newcommand{\F}{{\bf F}}
\newcommand{\X}{{\bf X}}
\newcommand{\Uu}{{\bf U}}
\newcommand{\Gg}{{\bf G}}

\newcommand{\exception}{\epsilon}
\newcommand{\extension}{\uparrow}
\newcommand{\simulates}[2]{#1 \geq #2}
\newcommand{\notsimulates}[2]{#1 \not\geq #2}
\newcommand{\deterministic}[2]{\emph{det}( #1 , #2 )}
\newcommand{\simState}[3]{\emph{simState}( #1 , #2 , #3 )}
\newcommand{\initialState}[1]{\emph{init}( #1 )}
\newcommand{\states}[1]{\emph{states}( #1 )}
\newcommand{\executes}[1]{\triangleright #1 }

\newcommand*\circleRef[1]{\tikz[baseline=(char.base)]{\node[shape=circle,draw,inner sep=2pt] (char) {#1};}}
\newcommand{\listRef}[1]{\circleRef{\scriptsize{#1}}}

%\newtheorem{theorem}{Theorem}[section]
\newtheorem{property}{Property}[section]
%\newtheorem{corollary}{Corollary}[theorem]
%\newtheorem{lemma}[theorem]{Lemma}
\theoremstyle{definition}
%\newtheorem{definition}{Definition}[section]
\newenvironment{proofsketch}{%
  \renewcommand{\proofname}{Proof sketch}\proof}{\endproof}

%game defs
\newcommand{\counterS}{{$f_1^{\neg\varphi}$}\xspace}
\newcommand{\winS}{{$f_0^{\varphi}$}\xspace}
\newcommand{\ltsind}{{$\overline{S_{\neg\varphi}}$}\xspace}
\newcommand{\strategy}{\ensuremath{\sigma}\xspace}
\newcommand{\tran}{t}
\newcommand{\upd}{u}

\newcommand{\automaton}[1]{$#1 = (S_{#1}, \Sigma_{#1}, \Delta_{#1}, s_{0}^{#1})$}
\newcommand{\ltsComposition}[3]{$#1 \parallel_{#3} #2 = (S_{#1}\times S_{#2}, \Sigma_{#1} \cup \Sigma_{#2}, \Delta, (s_{0}^{#1},s_{0}^{#2}))$}
\newcommand{\gsV}{{\mathcal{V}}}
\newcommand{\gsX}{{\mathcal{X}}}
\newcommand{\gsY}{{\mathcal{Y}}}
\newcommand{\gsTheE}{{\theta_e}}
\newcommand{\gsTheS}{{\theta_s}}
\newcommand{\gsRhoE}{{\rho_e}}
\newcommand{\gsRhoS}{{\rho_s}}


\title{Concurrent CLTS} %Change this to the appropriate number.
\author{Fundamentals} %Change this to your name.
\date{}

\begin{document}

\maketitle

%Counters for setting the appropriate numbering.
\setcounter{section}{1} %This gives the chapter number.
\setcounter{theorem}{1} %This gives the item number.
%Note that the item number should be one less than you desire..

\begin{definition}
	\label{def:CLTS} \emph{(Concurrent Labelled Transition Systems)} 
	A \emph{Concurrent Labelled Transition System} (CLTS) is $E =  (S, \Sigma, \Delta, s_0)$, where $S$ is a finite set of states, $\Sigma \subseteq Act$ is its {\em communicating alphabet}, $\Delta \subseteq (S \times \mathcal{P}(\Sigma) \times S)$ is a transition relation, and $s_0 \in S$ is the initial state.  We denote $\Delta(s)=\{\act~|~(s,\act,s') \in \Delta\}$. 
	An CLTS is deterministic if $(s,\act,s')$ and $(s,\act,s'')$ are in $\Delta$ implies $s'=s''$.
	An execution of $E$ is a word $\varepsilon=s_0, \act_0, s_1, \ldots$ where $(s_i, \act_i, s_{i+1}) \in \Delta$. 
	A word $\pi$ is a trace (induced by $\varepsilon$) of $E$ if its the result of removing every $s_i$ from an execution $\varepsilon$ of $E$. 
	We denote the set of infinite traces of $E$ by $\traces(E)$. 
\end{definition}

We use linear temporal logics of fluents (FLTL) over CLTS models. %~\cite{DBLP:conf/sigsoft/GiannakopoulouM03}. 
A \emph{fluent} \emph{fl} is defined by a pair of sets and a Boolean value: $\emph{\fluent} = \langle I_{\emph{\fluent}}, T_{\emph{\fluent}}, \emph{Init}_{\emph{\fluent}} \rangle$, where $I_{\emph{\fluent}}\subseteq \mathcal{P}(Act)$ is the set of initiating actions, $T_{\emph{\fluent}} \mathcal{P}(Act)$ is the set of terminating actions and $I_{\emph{\fluent}}\cap T_{\emph{\fluent}}=\emptyset$. 
A fluent may be initially \true or \false as indicated by \emph{Init}$_{\emph{\fluent}}$. 
Every actions $\ell\in Act$ induces a fluent, namely $\fluentp{\ell}=\langle \{\ell\}, \{Act\setminus \set{\ell}\}, \false\rangle$. 
Finally, the alphabet of a fluent is the union of its terminating and initiating actions.

Let $\mathcal{F}$ be the set of all possible fluents over $\mathcal{P}(Act)$. 
An FLTL formula is defined inductively using the standard Boolean connectives and temporal operators $X$~(next), $U$ (strong until) as follows: 
$\varphi ::= \fluent \mid \neg \varphi \mid \varphi \vee \psi \mid \X \varphi \mid \varphi U \psi,$
where $\fluent\in\mathcal{F}$. 
As usual we introduce $\wedge$, $\F$ (eventually), and $\G$ (always) as syntactic sugar. 
Let $\Pi$ be the set of infinite traces over \emph{Act}.
The trace $\pi=\ell_0,\ell_1,\ldots$ satisfies a fluent $\emph{Fl}$ at position $i$, denoted $\pi,i \models \emph{Fl}$, if and only if one of the following conditions holds:
\begin{list}{-}%{\leftmargin=3em}
	\item $\emph{Init}_{\emph{Fl}} \wedge (\forall j \in \mathbb{N} \cdot 0 \leq j \leq i \rightarrow T_{\fluent} \not\subseteq \ell_j)$
	\item $\exists j \in \mathbb{N} \cdot (j \leq i \wedge I_{\fluent} \subseteq \ell_j) \wedge (\forall k \in \mathbb{N} \cdot j < k \leq i \ \rightarrow T_{\fluent} \not\subseteq \ell_k)$
\end{list}

Given an infinite trace $\pi$, the satisfaction of a formula $\varphi$ at position $i$, denoted $\pi,i\models\varphi$, is defined as follows:

\begin{tabular}{ l c l }
$\pi, i \models_d \neg \varphi$ & $\triangleq$ & $\pi, i \not\models_d \varphi$\\
$\pi, i \models_d \varphi \vee \psi$ & $\triangleq$ & $(\pi, i \models_d \varphi) \vee (\pi, i \models_d \psi)$\\
$\pi, i \models_d \X \varphi$ & $\triangleq$ & $\pi, i +1 \models_d \varphi$\\
$\pi, i \models_d \varphi \U \psi$ & $\triangleq$ & $\exists j \geq i . \pi,j \models_d \psi \wedge \forall i \leq k \le k. \pi, k \models_d \varphi$\\
\end{tabular}
  
We say that $\varphi$ holds in $\pi$, denoted $\pi\models\varphi$, if $\pi,0\models\varphi$. 
A formula $\varphi \in \mbox{FLTL}$ holds in an CLTS $E$ (denoted $E \models \varphi$) if it holds on every infinite trace produced by $E$.
A \emph{fluent} \fluent \space is defined by a set of initiating actions $I_{\fluent}$, a set of terminating actions $T_{\fluent}$, and an initial value \emph{Initially}$_{\fluent}$.

That is,
%\begin{equation*}
$ \fluent = \langle I_{\fluent}, T_{\fluent} \rangle_{\emph{initially}_{\fluent}} $, 
where 
$I_{\fluent},T_{\fluent} \subseteq \mathcal{P}(\emph{Act})$ 
and $I_{\fluent} \cap T_{\fluent} = \emptyset.$\\
%\end{equation*}
%A fluent may be initially \true or \false as indicated by the \emph{Initially}$_{\fluent}$ attribute,
%When we omit \emph{Initially}$_{\fluent}$, we assume the fluent is
%initially \emph{false}. We use $\fluentp{\ell}$ as short for the
%fluent defined as $\fluent = \langle \ell,
%\emph{Act}\setminus\set{\ell} \rangle$.
%Given the set of fluents $\Phi$, an FLTL formula is defined
%inductively using the standard boolean connectives and temporal
%operators $\mathbf{X}$ (next), $\mathbf{U}$ (strong until) as
%follows:\\
%%\begin{equation*}
%$\varphi ::= \fluent \mid \neg \varphi \mid \varphi \vee \psi \mid
%\mathbf{X} \varphi \mid \varphi \mathbf{U} \psi$,
%%\end{equation*}
%where $\fluent \in \Phi$.

When having to compose concurrent systems we define three different semantics that can be applied pairwise. Synchronous semantic $A ||_s B$ (Figure~\ref{fig:synchronous_composition})captures the composition of two components that share a single synchronizing event (implicit), where all participants should be able to make progress concurrently. The main motivation being digital components sharing a single clock. 
Asynchronous semantic $A ||_a B$ (Figure~\ref{fig:asynchronous_composition}) captures the interleaving interpretation of concurrency as found in LTS systems. Concurrent semantic $A ||_c B$ (Figure~\ref{fig:concurrent_composition}) captures the behavioral superset of the latter two. It can be used when composing two processes that may not share an implicit synchronizing event, as in the synchronous semantic, but can be observed by a third component that over samples the other two, allowing for the possibility of both concurrent an independent occurrence when observed as a composed system. A synthetic example is shown in Figure~\ref{fig:concurrent_systems}, motivation for this semantic can be found in clock domain crossing examples and micro architecture buffers.

\begin{figure}[bt]
	\centering
	%\SmallPicture
	%\ShowFrame
		\begin{VCPicture}{(-3,-3)(3,2.5)}
			\SetStateLabelScale{1}
			\SetEdgeLabelScale{1}
			\State[1_a]{(-3,1.5)}{A}
			\State[1_b]{(-3,-1.5)}{B}			
			\State[1_c]{(2.5,0)}{C}						
			\Initial[n]{A}
			\Initial[n]{B}
			\Initial[w]{C}						
			%\ChgEdgeLineStyle{dashed} %\EdgeLineDouble
			%\ChgEdgeLineWidth{1.5}
			%\EdgeL{1}{2}{req}
			%\ArcR[.3]{6}{1}{reset}        
			\CLoopSW[.5]{A}{data_a}        
			%\CLoopSE[.5]{A}{idle_a}
			\CLoopSW[.5]{B}{data_b}        
			%\CLoopSE[.5]{B}{idle_b}				
			%\CLoopSW[.5]{C}{idle_c}        
			\CLoopSE[.5]{C}{data_b}					
			\CLoopNW[.5]{C}{data_a}					
			\CLoopNE[.5]{C}{<data_a,data_b,goal>}					
			%\RstEdgeLineWidth{1}
			%\RstEdgeLineStyle %\EdgeLineSimple
			%\EdgeL{2}{3}{grant}
			%\EdgeL[.75]{2}{5}{\overline{grant}}
			%\VArcR{arcangle=-20}{3}{6}{timeout}
			%\ArcR[.6]{3}{6}{timeout}
			%\ArcL{5}{1}{hready}
			%\VArcR{arcangle=-30}{3}{4}{hready}
		\end{VCPicture}

	\caption{A, B and C systems}
	\label{fig:concurrent_systems}
	%%\vspace*{-4mm}
	\MediumPicture
\end{figure}
\begin{figure}[bt]
	\centering
\minipage{0.32\textwidth}	
\centering
	%\SmallPicture
	%\ShowFrame
	\begin{VCPicture}{(-1.5,-1.5)(1.5,1.5)}
		\SetStateLabelScale{.8}
		\SetEdgeLabelScale{1}
		\State[1_{a\parallel_a b}]{(0,0)}{C}
		\Initial[n]{C}
		%\ChgEdgeLineStyle{dashed} %\EdgeLineDouble
		%\ChgEdgeLineWidth{1.5}
		%\EdgeL{1}{2}{req}
		%\ArcR[.3]{6}{1}{reset}        
		\CLoopSW[.5]{C}{data_a}        
		\CLoopSE[.5]{C}{data_b}					
		%\RstEdgeLineWidth{1}
		%\RstEdgeLineStyle %\EdgeLineSimple
		%\EdgeL{2}{3}{grant}
		%\EdgeL[.75]{2}{5}{\overline{grant}}
		%\VArcR{arcangle=-20}{3}{6}{timeout}
		%\ArcR[.6]{3}{6}{timeout}
		%\ArcL{5}{1}{hready}
		%\VArcR{arcangle=-30}{3}{4}{hready}
	\end{VCPicture}
	\caption{A $\parallel_a$ B}
	\label{fig:asynchronous_composition}
\endminipage\hfill
\minipage{0.32\textwidth}%
\centering
	%%\vspace*{-4mm}
	%\SmallPicture
	%\ShowFrame
	\begin{VCPicture}{(-1.5,-1.5)(1.5,1.5)}
		\SetStateLabelScale{.8}
		\SetEdgeLabelScale{1}
		\State[1_{a \parallel_s b}]{(0,0)}{C}
		\Initial[n]{C}
		%\ChgEdgeLineStyle{dashed} %\EdgeLineDouble
		%\ChgEdgeLineWidth{1.5}
		%\EdgeL{1}{2}{req}
		%\ArcR[.3]{6}{1}{reset}        
		\CLoopS[.5]{C}{<data_a, data_b>}        
		%\RstEdgeLineWidth{1}
		%\RstEdgeLineStyle %\EdgeLineSimple
		%\EdgeL{2}{3}{grant}
		%\EdgeL[.75]{2}{5}{\overline{grant}}
		%\VArcR{arcangle=-20}{3}{6}{timeout}
		%\ArcR[.6]{3}{6}{timeout}
		%\ArcL{5}{1}{hready}
		%\VArcR{arcangle=-30}{3}{4}{hready}
	\end{VCPicture}
	\caption{A $\parallel_s$ B}
	\label{fig:synchronous_composition}
\endminipage\hfill
\minipage{0.32\textwidth}%	
\centering
	%%\vspace*{-4mm}
	%\SmallPicture
	%\ShowFrame
	\begin{VCPicture}{(-1.5,-1.5)(1.5,1.5)}
		\SetStateLabelScale{.8}
		\SetEdgeLabelScale{1}
		\State[1_{a\parallel_c b}]{(0,0)}{C}
		\Initial[n]{C}
		%\ChgEdgeLineStyle{dashed} %\EdgeLineDouble
		%\ChgEdgeLineWidth{1.5}
		%\EdgeL{1}{2}{req}
		%\ArcR[.3]{6}{1}{reset}        
		\CLoopSW[.5]{C}{data_a}        
		\CLoopSE[.5]{C}{data_b}					
		\CLoopNE[.5]{C}{<data_a, data_b>}					
		%\RstEdgeLineWidth{1}
		%\RstEdgeLineStyle %\EdgeLineSimple
		%\EdgeL{2}{3}{grant}
		%\EdgeL[.75]{2}{5}{\overline{grant}}
		%\VArcR{arcangle=-20}{3}{6}{timeout}
		%\ArcR[.6]{3}{6}{timeout}
		%\ArcL{5}{1}{hready}
		%\VArcR{arcangle=-30}{3}{4}{hready}
	\end{VCPicture}
	\caption{A $\parallel_c$ B}
	\label{fig:concurrent_composition}
\endminipage\hfill	
	%%\vspace*{-4mm}
\end{figure}
%\begin{figure}[bt]
%\centering
%\SmallPicture
%%\ShowFrame
%\VCDraw{
%    \begin{VCPicture}{(-4,-1.5)(4,2.3)}
%        \SetEdgeLabelScale{1.4}
%        \State[1]{(-3,0)}{1}
%        \State[2]{(0,0)}{2}
%        \State[3]{(3,1)}{3}
%        \State[4]{(-0.5,3)}{4}
%        \State[6]{(0,1.5)}{6}        
%        \State[5]{(3,-1)}{5}
%		\Initial[w]{1}
%        \ChgEdgeLineStyle{dashed} %\EdgeLineDouble
%        %\ChgEdgeLineWidth{1.5}
%        \EdgeL{1}{2}{req}
%        \ArcR[.3]{6}{1}{reset}        
%        \VArcR{arcangle=-30}{4}{1}{process}        
%        %\RstEdgeLineWidth{1}
%        \RstEdgeLineStyle %\EdgeLineSimple
%        \EdgeL{2}{3}{grant}
%        \EdgeL[.75]{2}{5}{\overline{grant}}
%        %\VArcR{arcangle=-20}{3}{6}{timeout}
%        \ArcR[.6]{3}{6}{timeout}
%        \ArcL{5}{1}{hready}
%        \VArcR{arcangle=-30}{3}{4}{hready}
%    \end{VCPicture}
%}
%\caption{Bus Access example ($E$)}
%\label{fig:req_grant}
%%%\vspace*{-4mm}
%\MediumPicture
%\end{figure}
%\begin{figure}[bt]
%\centering
%\SmallPicture
%%\ShowFrame
%\VCDraw{
%    \begin{VCPicture}{(-4,-1.5)(4,2)}
%        \SetEdgeLabelScale{1.4}
%        \State[1]{(-3,0)}{1}
%        \State[2]{(0,-1)}{2}
%        \State[3]{(3,0)}{3}
%        \State[6]{(0,1)}{6}        
%		\Initial[w]{1}
%        \ChgEdgeLineStyle{dashed} %\EdgeLineDouble
%        %\ChgEdgeLineWidth{1.5}
%        \ArcR{1}{2}{req}
%        \ArcR{6}{1}{reset}        
%        %\RstEdgeLineWidth{1}
%        \RstEdgeLineStyle %\EdgeLineSimple
%        \ArcR{2}{3}{grant}
%        %\VArcR{arcangle=-20}{3}{6}{timeout}
%        \ArcR{3}{6}{timeout}
%    \end{VCPicture}
%}
%\caption{Minimized Bus Access ($E_1$)}
%\label{fig:req_grant_sub_1}
%%\vspace*{-4mm}
%\MediumPicture
%\end{figure}
%\begin{figure}[bt]
%\centering
%\SmallPicture
%%\ShowFrame
%\VCDraw{
%    \begin{VCPicture}{(-4,-1.5)(4,2)}
%        \SetEdgeLabelScale{1.4}
%        \State[1]{(-3,0)}{1}
%        \State[2]{(0,0)}{2}
%        \State[3]{(3,1)}{3}
%        \State[6]{(0,1.5)}{6}        
%        \State[5]{(3,-1)}{5}
%		\Initial[w]{1}
%        \ChgEdgeLineStyle{dashed} %\EdgeLineDouble
%        %\ChgEdgeLineWidth{1.5}
%        \EdgeL{1}{2}{req}
%        \ArcR[.3]{6}{1}{reset}        
%        %\RstEdgeLineWidth{1}
%        \RstEdgeLineStyle %\EdgeLineSimple
%        \EdgeL{2}{3}{grant}
%        \EdgeL[.75]{2}{5}{\overline{grant}}
%        %\VArcR{arcangle=-20}{3}{6}{timeout}
%        \ArcR[.7]{3}{6}{timeout}
%        \ArcL{5}{1}{hready}
%    \end{VCPicture}
%}
%\caption{Minimized Bus Access ($E_2$)}
%\label{fig:req_grant_sub_2}
%%\vspace*{-4mm}
%\MediumPicture
%\end{figure}

\begin{definition} 
	Let \automaton{M},\automaton{N}, with $\Delta_M : S_M \times \mathcal{P}(\Sigma_M) \times S_M$, be two CLTS instances, then CLTS concurrent parallel composition is defined as \ltsComposition{M}{N}{c} where $\Delta$ is the smallest relation s.t:
	\begin{center}
		\begin{equation}
		\AxiomC{$(s, l_M, s') \in \Sigma_M,(t, l_N, t') \in \Sigma_N  $}\RightLabel{$l_M \cap \Sigma_N = l_N \cap \Sigma_M$}
		\UnaryInfC{$((s,t),l_M \cup l_N,(s',t')) \in \Delta$}
		\DisplayProof
		\end{equation}	
		\begin{equation}
		\AxiomC{$(s, l_M, s') \in \Sigma_M $}\RightLabel{$l_M \cap \Sigma_N = \emptyset$}
		\UnaryInfC{$((s,t),l_M,(s',t))$}
		\DisplayProof
		\quad\quad
		\AxiomC{$(t, l_N, t') \in \Sigma_N $}\RightLabel{$l_N \cap \Sigma_M = \emptyset$}
		\UnaryInfC{$((s,t),l_N,(s,t'))$}
		\DisplayProof
		\end{equation}
	\end{center}
\end{definition}

\begin{definition} 
	Let \automaton{M},\automaton{N}, with $\Delta_M : S_M \times \mathcal{P}(\Sigma_M) \times S_M$, be two CLTS instances, then CLTS asynchronous parallel composition is defined as \ltsComposition{M}{N}{a} where $\Delta$ is the smallest relation s.t:
	\begin{center}
		\begin{equation}
		\AxiomC{$(s, l_M, s') \in \Sigma_M,(t, l_N, t') \in \Sigma_N  $}\RightLabel{$l_M \cap \Sigma_N = l_N \cap \Sigma_M \neq \emptyset$}
		\UnaryInfC{$((s,t),l_M \cup l_N,(s',t')) \in \Delta$}
		\DisplayProof
		\end{equation}	
		\begin{equation}
		\AxiomC{$(s, l_M, s') \in \Sigma_M $}\RightLabel{$l_M \cap \Sigma_N = \emptyset$}
		\UnaryInfC{$((s,t),l_M,(s',t))$}
		\DisplayProof
		\quad\quad
		\AxiomC{$(t, l_N, t') \in \Sigma_N $}\RightLabel{$l_N \cap \Sigma_M = \emptyset$}
		\UnaryInfC{$((s,t),l_N,(s,t'))$}
		\DisplayProof
		\end{equation}
	\end{center}
\end{definition}

\begin{definition} 
	Let \automaton{M},\automaton{N}, with $\Delta_M : S_M \times \mathcal{P}(\Sigma_M) \times S_M$, be two CLTS instances, then CLTS synchronous parallel composition is defined as \ltsComposition{M}{N}{s} where $\Delta$ is the smallest relation s.t:
	\begin{center}
		\begin{equation}
		\AxiomC{$(s, l_M, s') \in \Sigma_M,(t, l_N, t') \in \Sigma_N  $}\RightLabel{$l_M \cap \Sigma_N = l_N \cap \Sigma_M$}
		\UnaryInfC{$((s,t),l_M \cup l_N,(s',t')) \in \Delta$}
		\DisplayProof
		\end{equation}	
	\end{center}
\end{definition}

%\begin{definition} 
%	Let $M_1 \ldots M_N$ be CLTS instances s.t. $M_i= (S_{i}, \Sigma_{i}, \Delta_{i}, s_{0}^{i})$ with $\Delta_i : S_i \times l \subseteq \Sigma_i \times S_i$, the concurrent parallel composition \ltsComposition{}{M_i}{} where $\Delta_i$ is the smallest relation s.t:
%	\begin{center}
%		\begin{equation}
%		\AxiomC{$\forall i,j \in 1 \ldots N, i \neq j: ((s_i, l_i, t_i) \in \Delta_i \wedge \forall a \in l_i : a \in (\Sigma_i \cap \Sigma_j) \rightarrow (a \in l_i \wedge a \in l_j)) \vee (l_i = \emptyset \wedge t_i = s_i)$}
%		\UnaryInfC{$((s_1 \ldots s_N),\cup_{i=1}^{N}l_i,(t_1 \ldots t_N)) \in \Delta$}
%		\DisplayProof
%		\end{equation}		
%	\end{center}
%\end{definition}

%\begin{figure}[ht]
%	\begin{center}
%		\renewcommand{\ttdefault}{pcr}
\begin{lstlisting}[escapeinside={[*}{*]},basicstyle=\scriptsize\ttfamily,columns=flexible,frame=lines,mathescape=true,xleftmargin=3.0ex,keywordstyle=\textbf,morekeywords={if,while,do,else,fork,int,null, algorithm, is, input, output, return},numbers=left,numberstyle=\scriptsize]
algorithm dfs_min is
	input: $E$ the LTS representing the original plant
	output: $E'$ the LTS that satisfies the problem statement
	$E$ = $E$.prune_mixed_states()
	$E_{new}$ = $E'$ = $E$ 
	$\mathcal{T}_u$ = $E$.non_controllable_transitions() 
	to_remove = []
	while(|$\mathcal{T}_u$| > 0)
		$t$ = $\mathcal{T}_u$.pop()
		while($\neg E'$.contains($t$) or $d_{out}$($E'$,t.from) == 1)
			if (|$\mathcal{T}_u$| == 0)
				return $E$
			$t$ = $\mathcal{T}_u$.pop() 
		to_remove.add($t$)
		$E_{new}$ = $E$.remove(to_remove)
		
		is_realizable = $E_{new}$.realizable() 
		
		if ($\neg$ is_realizable):
			$E'$ = $E_{new}$
		else
			to_remove.remove($t$) 
	return $E'$  
\end{lstlisting} 
%		\caption{Concurrent Composition Algorithm}
%		\label{fig:dfs-code}
%	\end{center}
%\end{figure}
%\begin{figure}[ht]
%	\begin{center}
%		\renewcommand{\ttdefault}{pcr}
\begin{lstlisting}[escapeinside={[*}{*]},basicstyle=\scriptsize\ttfamily,columns=flexible,frame=lines,mathescape=true,xleftmargin=3.0ex,keywordstyle=\textbf,morekeywords={if,while,do,else,fork,int,null, algorithm, is, input, output, return, for},numbers=left,numberstyle=\scriptsize]
algorithm compose_automata is
	input: $A_1 \ldots A_n$  the set of CLTS automata to be composed
	input: $type_2 \ldots type_N$ a set of elements of $\{synch,asynch,concurrent\}$ indicating the type of composition 
		to be applied from left to right, pair-wise
	output: $A$ the CLTS automaton that results from applying $A_1 \parallel_{type_2} A_2 \parallel_{type_3} \ldots \parallel_{type_n} A_n$
	$s_0$ = $(s^1_{0},\ldots,s^n_{0})$
	$frontier$ = [$s_0$]
	$A$ = create_automaton($s_0$)
	$visited$ = $[]$
	$accum$ = $\emptyset$
	while(|$frontier$| > 0)
		$s$ = $frontier$.pop()
		if $s \in visited$
			continue
		$visited$.push($s$)
		$idxs$ = ($0_1,\ldots,0_n$)
		while($idxs \leq (|\Delta_1(s_1)|+1,\ldots,|\Delta_n(s_n)|+1)$)
			for $i \in 1\ldots n$
			if $i = 0$
				if $idxs[i] = |\Delta_i(s_i)|+1$
					$accum$ = $\emptyset$
				else
					$accum$ = $l(\Delta_i(s_i)[idxs[i]])$
			else
				if overlaps($accum,\Sigma_i) \neq$ overlap($l(\Delta_i(s_i)[idxs[i]])$)
					increment($idxs$)
					break
				if $type_i = asynch$ $\wedge $ overlaps($accum,\Sigma_i) \neq$ overlap($l(\Delta_i(s_i)[idxs[i]])$) = $\emptyset$
					increment($idxs$)
					break					
				if $type_i = synch$ $\wedge $ overlaps($accum,\Sigma_i) = \emptyset$ $\wedge$ $idxs[i] = |\Delta_i(s_i)|+1$
					increment($idxs$)
					break	
			if $idxs[i] < |\Delta_i(s_i)|+1$
				$accum$ = $accum \cup l(\Delta_i(s_i)[idxs[i]])$
			$\sigma$ = $(s, accum, t:(t_1,\ldots,t_n))$
			if $i = n \wedge t \notin visited \wedge t \notin frontier$ 
				 $frontier$.push($t$)
			add_transition($A$, $\sigma$)			
	return $A$

\end{lstlisting} 
%		\caption{Concurrent Composition Algorithm (closer to code version)}
%		\label{fig:dfs-code2}
%	\end{center}
%\end{figure}
\end{document}