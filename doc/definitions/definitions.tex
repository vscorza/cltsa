\documentclass{article}
%Packages being imported that add functionality to LaTeX.
\usepackage[utf8]{inputenc}
\usepackage{geometry}
\usepackage{amssymb,amsmath,amsthm,mathtools,bussproofs,turnstile}

\usepackage{enumerate}

%Theorem Environments
\newtheorem{theorem}{Theorem}[section]
\newtheorem{lemma}[theorem]{Lemma}
\theoremstyle{definition}
\newtheorem{definition}{Definition}[section]
\newtheorem{corollary}[theorem]{Corollary}

\newcommand{\automaton}[1]{$#1 = (S_{#1}, \Sigma_{#1}, \Delta_{#1}, s_{0}^{#1})$}
\newcommand{\ltsComposition}[3]{$#1 \parallel_{#3} #2 = (S_{#1}\times S_{#2}, \Sigma_{#1} \cup \Sigma_{#2}, \Delta, (s_{0}^{#1},s_{0}^{#2}))$}

\title{Concurrent CLTS parallel composition} %Change this to the appropriate number.
\author{-} %Change this to your name.
\date{}

\begin{document}

\maketitle

%Counters for setting the appropriate numbering.
\setcounter{section}{1} %This gives the chapter number.
\setcounter{theorem}{1} %This gives the item number.
%Note that the item number should be one less than you desire..

\begin{definition} 
Let \automaton{M},\automaton{N}, with $\Delta_M : S_M \times l \subseteq \Sigma_M \times S_M$, be two CLTS instances, then CLTS sequential parallel composition is defined as \ltsComposition{M}{N}{s} where $\Delta$ is the smallest relation s.t:
\begin{center}
	\begin{equation}
		\AxiomC{$(s, l_m, s') \in \Sigma_m $}\RightLabel{$l_m \cap \Sigma_n = \emptyset$}
		\UnaryInfC{$((s,t),l_m,(s',t))$}
		\DisplayProof
		\quad\quad
		\AxiomC{$(t, l_n, t') \in \Sigma_n $}\RightLabel{$l_n \cap \Sigma_m = \emptyset$}
		\UnaryInfC{$((s,t),l_n,(s,t'))$}
		\DisplayProof
	\end{equation}
	\begin{equation}
		\AxiomC{$(s, l_m, s') \in \Sigma_m,(t, l_n, t') \in \Sigma_n  $}\RightLabel{$l_m \cap \Sigma_n = l_n \cap \Sigma_m \neq \emptyset$}
		\UnaryInfC{$((s,t),l_m \cup l_n,(s',t'))$}
		\DisplayProof
	\end{equation}	
\end{center}
\end{definition}
\begin{definition} 
	Let \automaton{M},\automaton{N}, with $\Delta_M : S_M \times l \subseteq \Sigma_M \times S_M$, be two CLTS instances, then CLTS concurrent parallel composition is defined as \ltsComposition{M}{N}{c} where $\Delta$ is the smallest relation s.t:
	\begin{center}
		\begin{equation}
		\AxiomC{$(s, l_m, s') \in \Sigma_m,(t, l_n, t') \in \Sigma_n  $}\RightLabel{$l_m \cap \Sigma_n = l_n \cap \Delta_m$}
		\UnaryInfC{$((s,t),l_m \cup l_n,(s',t'))$}
		\DisplayProof
		\end{equation}	
	\end{center}
\end{definition}
\begin{definition} 
	Let \automaton{M},\automaton{N}, with $\Delta_M : S_M \times l \subseteq \Sigma_M \times S_M$, be two CLTS instances, then CLTS parallel composition is defined as \ltsComposition{M}{N}{} where $\Delta$ is the smallest relation s.t:
	\begin{center}
		\begin{equation}
			\AxiomC{$(s, l_m, s') \in \Sigma_m $}\RightLabel{$l_m \cap \Sigma_n = \emptyset$}
			\UnaryInfC{$((s,t),l_m,(s',t))$}
			\DisplayProof
			\quad\quad
			\AxiomC{$(t, l_n, t') \in \Sigma_n $}\RightLabel{$l_n \cap \Sigma_m = \emptyset$}
			\UnaryInfC{$((s,t),l_n,(s,t'))$}
			\DisplayProof
		\end{equation}		
		\begin{equation}
		\AxiomC{$(s, l_m, s') \in \Sigma_m,(t, l_n, t') \in \Sigma_n  $}\RightLabel{$l_m \cap \Sigma_n = l_n \cap \Sigma_m$}
		\UnaryInfC{$((s,t),l_m \cup l_n,(s',t'))$}
		\DisplayProof
		\end{equation}	
	\end{center}
\end{definition}
\end{document}