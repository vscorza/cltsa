
We can now introduce algorithm ~\ref{fig:gs_to_clts_algorithm}, that takes a game structure, represented as a set 
$\gsTheE$, $\gsTheS$, $\gsRhoE$, $\gsRhoS$, $\varphi$ of OBDDs, one for each formula, and produces the CLTS automaton $A$ that preserves its behavior. Since the state space is finite ($2^{|\gsV|}$ at most) and the number of states reached by applying $\gsRhoS(\gsRhoE(v))$ increases monotonically (when starting from $\gsTheS(\gsTheE)$), a least fixed point exists, the algorithm computes such fixed point while progressively creating the transition relation $\Delta$. \texttt{valuations}($\psi$) gives the set of valuations over $\gsV$ for a formula $\psi$ represented with an OBDD. Each valuation induces a unique state in $A$. \texttt{restrict}($\psi$, $s$) fixes a subset of variables in $\psi$ according to $s$ a valuation. If $\varphi$ is the set of propositional formulas involved in the winning condition, \texttt{evaluate\_condition}($A$, $\varphi$, $s$) evaluates the satisfaction of each such formulas according to valuation $s$ and maps the result to $s$ as a state in $A$.
\newpage
\renewcommand{\ttdefault}{pcr}
\begin{figure}[bt]
\begin{lstlisting}[escapeinside={[*}{*]},basicstyle=\scriptsize\ttfamily,columns=flexible,frame=lines,mathescape=true,xleftmargin=3.0ex,keywordstyle=\textbf,morekeywords={if,while,do,else,fork,int,null, algorithm, is, input, output, return, for},numbers=left,numberstyle=\scriptsize]
algorithm gs_to_clts is
	input: $GS$ a set of OBDD structure representing LTL formulas
	output: $A$ a CLTS automaton equivalent to the game structure induced by $GS$
	
	$A$ $\leftarrow$ create_automaton($s_0$)
	$visited \leftarrow \{s_0\}$ 
	$frontier \leftarrow$ valuations($\gsTheE$)
	for $s \in frontier$:
		add_transition($A$, $s_0$, $s$)
		evaluate_condition($A$, $\varphi$, $s$)		
	$new\_frontier \leftarrow $ [] 
	for $s \in frontier$:
		add($visited$,$s$)
		$current \leftarrow$ valuations(restrict($\gsTheS$, $s$))
		for $s' \in current$:
			add_transition($A$, $s$, $s'$)
			evaluate_condition($A$, $\varphi$, $s'$)
			if $s' \not\in new\_frontier$:			
				add($new\_frontier$, $s'$)
	$frontier \leftarrow new\_frontier$
	while $frontier \neq$ []:
		while $frontier \neq$ []:		
			$s_{sys} \leftarrow$ pop($frontier$)
			if $s_{sys} \in frontier$:
				continue
			$current \leftarrow$ valuations(restrict($\gsRhoE$, $s_{sys}$))
			for $s' \in current$:
				add_transition($A$, $s_{sys}$, $s'$)
				evaluate_condition($A$, $\varphi$, $s'$)			
				if $s' \not\in new\_frontier$:			
					add($new\_frontier$, $s'$)				
		$frontier \leftarrow new\_frontier$		
		while $frontier \neq$ []:		
			$s_{env} \leftarrow$ pop($frontier$)
			if $s_{env} \in frontier$:
			continue
			$current \leftarrow$ valuations(restrict($\gsRhoS$, $s_{env}$))
			for $s' \in current$:
				add_transition($A$, $s_{env}$, $s'$)
				evaluate_condition($A$, $\varphi$, $s'$)			
				if $s' \not\in new\_frontier$:			
					add($new\_frontier$, $s'$)				
		$frontier \leftarrow new\_frontier$				
	return $A$
\end{lstlisting}
\caption{Game Structure to CLTS translation algorithm}
\label{fig:gs_to_clts_algorithm}
%%\vspace*{-4mm}
\MediumPicture
\end{figure}

