\[\xymatrix@C+1pc{
	\langle E = \langle S, \Sigma, \Delta, s_0 \rangle, \mathcal{C} \rangle \ar@{<->}[d]^{t^{-1}}_{t}
	& \varphi_{CLTS}\ar@{<->}[d]^{f^{-1}}_{f}
	& C\ar@{<->}[d]^{t^{-1}}_{t}
	& \exists C: E\parallel C \models  \varphi_{CLTS}\ar@{<=>}[d]^{t^{-1}}_{t}\\
	G = \langle \mathcal{X},\mathcal{Y},\theta_e,\theta_s,\rho_e,\rho_s, \emptyset \rangle
	& \varphi_{FDS}
	& D
	&\exists D: D \models \varphi_G\\
}\]
Donde $C$ es legal respecto a $\langle E,\mathcal{C} \rangle$ y $D$ predica sobre $\mathcal{V}$, valiendo $\mathcal{X} \cup \mathcal{Y} \subseteq \mathcal{V}$,$D$ es completo respecto de $X$ y
\[\varphi_G = (\theta_e \implies \theta_s) \wedge (\theta_e \implies \square((\boxdot \rho_e) \implies \rho_s)) \wedge (\theta_e \wedge \rho_e \implies \varphi_{FDS}) \]
De aquí que habría que encontrar $t$, $t^{-1}$,$f$,$f^{-1}$ y con eso mostrar la doble implicación de los problemas de síntesis.
\newpage

La idea es que no podemos traducir CLTS a LTS sin meter deadlocks.
Supongamos la existencia de una traducción $lts:CLTS \rightarrow LTS$ que debería cumplir con lo siguiente, si $A=\langle S, \Sigma, \Delta, s_0 \rangle$ es un CLTS y $A'=lts(A)=\langle S', \Sigma', \Delta', s'_0 \rangle$ es su traducción, querríamos que se cumpla que se pueden proyectar los estados del CLTS, que al menos contiene el alfbeto original y que hay caminos consistentes entre los estados proyectados:

\[S \subseteq S', \Sigma \subseteq \Sigma', s_0 = s'_0\]
Para los estados proyectados pedimos que la relación de transición permita reconstruir un camino donde aparezcan los elementos de la etiqueta orginal preservando cardinalidad:
\[ \forall (s,l,s') \in \Delta, \exists s \xRightarrow[]{l} s' \in \Delta': |l'\downarrow_{\Sigma}| = |l| \]
Queremos ver si se cumple la siguiente propiedad:
\[\exists lts: CLTS \rightarrow LTS | \forall A,B:CLTS, lts(A) \parallel lts(B) = lts(A \parallel_s B)\]
Para esto vamos a usar tres CLTSs de referencia:
\[M_x = \langle \lbrace 1,2 \rbrace, \lbrace a,b,c,x \rbrace, \lbrace\langle 1, \lbrace b,x \rbrace, 2 \rangle\rbrace, 1 \rangle\]
\[M_2 = \langle \lbrace 1,2 \rbrace, \lbrace b,c \rbrace, \lbrace \langle 1, \lbrace b,c \rbrace, 2 \rangle\rbrace, 1 \rangle\]
\[M_3 = \langle \lbrace 1,2 \rbrace, \lbrace a,b \rbrace, \lbrace \langle 1, \lbrace a,b \rbrace, 2 \rangle\rbrace, 1 \rangle\]
\begin{figure}[bt]
	\centering
\minipage{0.32\textwidth}%
\centering
\begin{VCPicture}{(-1.5,-1)(1.5,1)}
	\SetStateLabelScale{.8}
	\SetEdgeLabelScale{1}
	\State[1]{(-1,-0.3)}{1}
	\State[2]{(1,-0.3)}{2}
	\Initial[w]{1}
	\ArcL{1}{2}{<b,x>}        
\end{VCPicture}
\caption{$M_x$}
\label{fig:m_x}
\endminipage\hfill
\minipage{0.32\textwidth}%
\centering
\begin{VCPicture}{(-1.5,-1)(1.5,1)}
	\SetStateLabelScale{.8}
	\SetEdgeLabelScale{1}
	\State[1]{(-1,-0.3)}{1}
	\State[2]{(1,-0.3)}{2}
	\Initial[w]{1}
	\ArcL{1}{2}{<b,c>}        
\end{VCPicture}
\caption{$M_2$}
\label{fig:m_x}
\endminipage\hfill
\minipage{0.32\textwidth}%
\centering
\begin{VCPicture}{(-1.5,-1)(1.5,1)}
	\SetStateLabelScale{.8}
	\SetEdgeLabelScale{1}
	\State[1]{(-1,-0.3)}{1}
	\State[2]{(1,-0.3)}{2}
	\Initial[w]{1}
	\ArcL{1}{2}{<a,b>}        
\end{VCPicture}
\caption{$M_3$}
\label{fig:m_x}
\endminipage\hfill
\end{figure}

Aplicando la traducción sobre $M_1$ podemos decir que:
\[M'_x = lts(M_x) \implies ( 1 \xRightarrow[]{l'} 2 \in \Delta_{M'_x} \wedge |l'\downarrow_{\lbrace a,b,x\rbrace}|= |\lbrace b,x \rbrace | )\]
Si este es el caso vale alguna de las dos opciones siguientes:
\[1 \xRightarrow[]{l'} 2: 1 \xRightarrow[]{p}\xrightarrow{b}\xRightarrow[]{p'}\xrightarrow{x}\xRightarrow[]{p''}2\]
\[1 \xRightarrow[]{l'} 2: 1 \xRightarrow[]{r}\xrightarrow{x}\xRightarrow[]{r'}\xrightarrow{b}\xRightarrow[]{r''}2\]
Y para $M_2$:
\[M'_2 = lts(M_2) \implies ( 1 \xRightarrow[]{l''} 2 \in \Delta_{M'_2} \wedge |l''\downarrow_{\lbrace b,c\rbrace}|= |\lbrace b,c \rbrace | )\]
Si este es el caso vale alguna de las dos opciones siguientes:
\[1 \xRightarrow[]{l''} 2: 1 \xRightarrow[]{u}\xrightarrow{b}\xRightarrow[]{u'}\xrightarrow{c}\xRightarrow[]{u''}2\]
\[1 \xRightarrow[]{l''} 2: 1 \xRightarrow[]{v}\xrightarrow{c}\xRightarrow[]{v'}\xrightarrow{b}\xRightarrow[]{v''}2\]
Pero recordando que $M_{\parallel}=M_x \parallel_s M_2=\langle \lbrace 1 \rbrace, \lbrace a,b,x \rbrace, \emptyset, 1 \rangle$ bloquea, los pares de prefijos entre $p,r$ y $u,v$ debería bloquear, con lo cual, por ej. debería valer $p_1 \in \Sigma_{M'_2}$, $u_1 \in \Sigma_{M'_x}$ y $p_1 \neq u_1$. Ahora, para un:
\[M_y = \langle \lbrace 1,2 \rbrace, \lbrace a,b,c,y \rbrace, \lbrace\langle 1, \lbrace b,y \rbrace, 2 \rangle\rbrace, 1 \rangle\]
En el que reemplazamos $x$ por $y$ valiendo $x \neq y$, podemos reproducir el razonamiento sobre:
\[1 \xRightarrow[]{l'''} 2: 1 \xRightarrow[]{w}\xrightarrow{b}\xRightarrow[]{w'}\xrightarrow{y}\xRightarrow[]{w''}2\]
Y notar que no puede valer $p_1 = w_1$, porque si así fuera la composición $lts(M_x) || lts(M_y)$ introduciría deadlock al permitir $1 \xrightarrow{p_1} 2$. De aquí se desprende que no hay forma de producir un alfabeto finito que permita evitar deadlocks frente a la composición con un autómata de alfabeto arbitrario.

Y si queremos preservar alfabeto finito tendríamos también la siguiente situación:

\[lts(M_2 \parallel_s M_3) \neq lts(M_2)\parallel lts(M_3)\]

$lts(M_2)\parallel lts(M_3)$ no puede producir prefijo que bloquee el elemento $a,b,c$ por haber extendido su alfabeto a posteriori ($\Sigma = \lbrace a,b,c \rbrace$)
y la composición no bloquearía con:

\[M_c = \langle \lbrace 1,2 \rbrace, \lbrace a,b,c \rbrace, \lbrace\langle 1, \lbrace c \rbrace, 2 \rangle\rbrace, 1 \rangle\]

Un ejemplo que muestra que no la composición asíncrona no preserva las propiedades de la conjunción se puede construir con dos autómatas que representen las fórmulas $a \iff b$ y $a \neq x$ el primero sobre el alfabeto $\{a,b\}$ y el segundo sobre $\{a,x\}$. Esto indicaría que la composición asíncrona no es congruente con la conjunción si se preserva el alcance finito de los alfabetos.