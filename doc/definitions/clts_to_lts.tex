\[\xymatrix@C+1pc{
	\langle E = \langle S, \Sigma, \Delta, s_0 \rangle, \mathcal{C} \rangle \ar@{<->}[d]^{t^{-1}}_{t}
	& \varphi_{CLTS}\ar@{<->}[d]^{f^{-1}}_{f}
	& C\ar@{<->}[d]^{t^{-1}}_{t}
	& \exists C: E\parallel C \models  \varphi_{CLTS}\ar@{<=>}[d]^{t^{-1}}_{t}\\
	G = \langle \mathcal{X},\mathcal{Y},\theta_e,\theta_s,\rho_e,\rho_s, \emptyset \rangle
	& \varphi_{FDS}
	& D
	&\exists D: D \models \varphi_G\\
}\]
Donde $C$ es legal respecto a $\langle E,\mathcal{C} \rangle$ y $D$ predica sobre $\mathcal{V}$, valiendo $\mathcal{X} \cup \mathcal{Y} \subseteq \mathcal{V}$,$D$ es completo respecto de $X$ y
\[\varphi_G = (\theta_e \implies \theta_s) \wedge (\theta_e \implies \square((\boxdot \rho_e) \implies \rho_s)) \wedge (\theta_e \wedge \rho_e \implies \varphi_{FDS}) \]
De aquí que habría que encontrar $t$, $t^{-1}$,$f$,$f^{-1}$ y con eso mostrar la doble implicación de los problemas de síntesis.
\newpage

La idea es que no podemos traducir CLTS a LTS sin meter deadlocks.
Supongamos la existencia de una traducción $lts:CLTS \rightarrow LTS$ que debería cumplir con lo siguiente, si $A=\langle S, \Sigma, \Delta, s_0 \rangle$ es un CLTS y $A'=lts(A)=\langle S', \Sigma', \Delta', s'_0 \rangle$ es su traducción, querríamos que se cumpla que se pueden proyectar los estados del CLTS, que al menos contiene el alfbeto original y que hay caminos consistentes entre los estados proyectados:

\[S \subseteq S', \Sigma \subseteq \Sigma', s_0 = s'_0\]
Para los estados proyectados pedimos que la relación de transición permita reconstruir un camino donde aparezcan los elementos de la etiqueta orginal preservando cardinalidad:
\[ \forall (s,l,s') \in \Delta, \exists s \xRightarrow[]{l} s' \in \Delta': |l'\downarrow_{\Sigma}| = |l| \]
Queremos ver si se cumple la siguiente propiedad:
\[\exists lts: CLTS \rightarrow LTS | \forall A,B:CLTS, lts(A) \parallel lts(B) = lts(A \parallel_s B)\]
Para esto vamos a usar tres CLTSs de referencia:
\[M_1 = \langle \lbrace 1,2 \rbrace, \lbrace a,b,c \rbrace, \lbrace\langle 1, \lbrace a,b \rbrace, 2 \rangle\rbrace, 1 \rangle\]
\[M_2 = \langle \lbrace 1,2 \rbrace, \lbrace b,c \rbrace, \lbrace \langle 1, \lbrace b,c \rbrace, 2 \rangle\rbrace, 1 \rangle\]
\[M_3 = \langle \lbrace 1,2 \rbrace, \lbrace a,b \rbrace, \lbrace \langle 1, \lbrace a,b \rbrace, 2 \rangle\rbrace, 1 \rangle\]

Aplicando la traducción sobre $M_1$ podemos decir que:
\[M'_1 = lts(M_1) \implies ( 1 \xRightarrow[]{l'} 2 \in \Delta_{M'_1} \wedge |l'\downarrow_{\lbrace a,b,c\rbrace}|= |\lbrace a, b \rbrace | )\]
Si este es el caso vale alguna de las dos opciones siguientes:
\[1 \xRightarrow[]{l'} 2: 1 \xRightarrow[]{p}\xrightarrow{a}\xRightarrow[]{p'}\xrightarrow{b}\xRightarrow[]{p''}2\]
\[1 \xRightarrow[]{l'} 2: 1 \xRightarrow[]{r}\xrightarrow{b}\xRightarrow[]{r'}\xrightarrow{a}\xRightarrow[]{r''}2\]
Y para $M_2$:
\[M'_2 = lts(M_2) \implies ( 1 \xRightarrow[]{l''} 2 \in \Delta_{M'_2} \wedge |l''\downarrow_{\lbrace b,c\rbrace}|= |\lbrace b,c \rbrace | )\]
Si este es el caso vale alguna de las dos opciones siguientes:
\[1 \xRightarrow[]{l''} 2: 1 \xRightarrow[]{u}\xrightarrow{b}\xRightarrow[]{u'}\xrightarrow{c}\xRightarrow[]{u''}2\]
\[1 \xRightarrow[]{l''} 2: 1 \xRightarrow[]{v}\xrightarrow{c}\xRightarrow[]{v'}\xrightarrow{b}\xRightarrow[]{v''}2\]
Pero recordando que $M_{\parallel}=M_1 \parallel_s M_2=\langle \lbrace 1 \rbrace, \lbrace a,b,c \rbrace, \emptyset, 1 \rangle$ bloquea, los pares de prefijos entre $p,r$ y $u,v$ debería bloquear, con lo cual, por ej. debería valer $p_1 \in \Sigma_{M'_2}$, $u_1 \in \Sigma_{M'_1}$ y $p_1 \neq u_1$.

\textbf{Suponiendo que la extensión del alfabeto se consigue por combinación de elementos de $\Sigma$} (por ej. concatenación, que si así no fuese no hay cota de finitud para la extensión, porque cualquier autómata que tenga el mismo alfabeto más un elemento $x$ y está compuesto por una única transición $1 \xrightarrow{x} 2$ tendría que bloquear siempre, entonces un elemento particular debería ser agregado para $x$), tendríamos la siguiente situación:

\[lts(M_2 \parallel_s M_3) \neq lts(M_2)\parallel lts(M_3)\]

Porque $lts(M_2)\parallel lts(M_3)$ no puede producir prefijo $abc$ por haber extendido su alfabeto a poteriori ($\Sigma = \lbrace a,b,c \rbrace$)
pero la composición no bloquearía con:

\[M_1 = \langle \lbrace 1,2 \rbrace, \lbrace a,b,c \rbrace, \lbrace\langle 1, \lbrace c \rbrace, 2 \rangle\rbrace, 1 \rangle\]