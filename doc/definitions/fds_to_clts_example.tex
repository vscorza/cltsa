Now we show an example of an embedding from an FDS control problem into a CLTS control problem, for which a strategy is built, converted into a CLTS controller and then translated into an FDS controller. 
The FDS control problem handles a request $r$ to process a data packet by granting access to one of two clients, represented by signals $g_1$ and $g_2$.
The property to be satisfied is captured by a conjunction of an initial statement, a safety restriction and two liveness properties:
\[\varphi: (\overline{r}\wedge \overline{g}_1\wedge \overline{g}_2)\wedge (\square(g_1 \iff \overline{g}_2))\wedge ((\square\Diamond(x \implies g_1))\wedge(\square\Diamond(x \implies g_1)))\]
The FDS control problem is noted as $\mathcal{K}=\langle \{r\}, \{g_1,g_2\}, \varphi \rangle$. The automaton depicted in figure ~\ref{fig:fds_to_clts_K} represents the behavior of the initial and safety terms ($(\overline{r}\wedge \overline{g}_1\wedge \overline{g}_2)\wedge (\square(g_1 \iff \overline{g}_2))$).

\begin{figure}[bt]
	\centering
	%\minipage{0.32\textwidth}%
	%\centering
	\VCDraw{
	\begin{VCPicture}{(-2,-2)(4,3.5)}
		\SetStateLabelScale{1}
		\SetEdgeLabelScale{1.2}
		\StateVar[0:\overline{r g_1 g_2}]{(-2,1.3)}{0}
		\StateVar[1:\overline{r g_1}g_2]{(1,2.9)}{1}
		\StateVar[2:r g_1 \overline{g_2}]{(4,1.3)}{2}
		\StateVar[3:r\overline{g_1} g_2]{(1,-0.3)}{3}
		\Initial[w]{0}
		\LoopN{0}{}		
		\LoopN{1}{}				
		\LoopN{2}{}						
		\LoopS{3}{}						
		\EdgeL{0}{1}{}        
		\EdgeL{0}{2}{}        		
		\EdgeL{0}{3}{}        		
		\EdgeL{1}{0}{}        		
		\EdgeL{1}{2}{}        				
		\EdgeL{1}{3}{}        						
		\EdgeL{2}{0}{}        						
		\EdgeL{2}{1}{}        				
		\EdgeL{2}{3}{}        		
		\EdgeL{3}{0}{}				
		\EdgeL{3}{1}{}				
		\EdgeL{3}{2}{}		
	\end{VCPicture}
	}
	\caption{Automaton representation of $(\overline{r}\wedge \overline{g}_1\wedge \overline{g}_2)\wedge (\square(g_1 \iff \overline{g}_2))$}
	\label{fig:fds_to_clts_K}
	%\endminipage\hfill
\end{figure}

The CLTS automaton $E$ in \cltsCPEmbedding that captures all possible interaction between the FDS control problem variables is depicted in figure ~\ref{fig:fds_to_clts_E}, note that label information is omitted to allow for a more readable representation. Note that the LTL formula of the CLTS embedding is:

\[\varphi: \bigcirc(\overline{\hat{r}}\wedge \overline{\hat{g}}_1\wedge \overline{\hat{g}}_2)\wedge (\square(\hat{g}_1 \iff \overline{\hat{g}}_2))\wedge ((\square\Diamond(\hat{x} \implies \hat{g}_1))\wedge(\square\Diamond(\hat{x} \implies \hat{g}_1)))\]

\begin{figure}[bt]
	\centering
	%\minipage{0.32\textwidth}%
	%\centering
	\VCDraw{
		\begin{VCPicture}{(-6.5,-8)(6.5,4)}
			\SetStateLabelScale{1}
			\SetEdgeLabelScale{1.2}
			\StateVar[0:\overline{r g_1 g_2}]{(-4,1)}{0}
			\StateVar[1:\overline{r g_1}g_2]{(-2,3)}{1}			
			\StateVar[2:\overline{r} g_1 \overline{g_2}]{(2,3)}{2}			
			\StateVar[3:\overline{r} g_1 g_2]{(4,1)}{3}						
			\StateVar[4:r\overline{g_1 g_2}]{(4,-1)}{4}									
			\StateVar[5:r\overline{g_1} g_2]{(2,-3)}{5}												
			\StateVar[6:r g_1\overline{g_2}]{(-2,-3)}{6}
			\StateVar[7:r g_1 g_2]{(-4,-1)}{7}			
			\StateVar[s_0:\overline{r g_1 g_2}]{(0,-6)}{s0}						
			\Initial[s]{s0}
			\CLoopN{0}{}						
			\CLoopN{1}{}			
			\CLoopN{2}{}						
			\CLoopN{3}{}									
			\CLoopS{4}{}												
			\CLoopS{5}{}												
			\CLoopS{6}{}												
			\CLoopS{7}{}																				
			\EdgeL{0}{1}{}        
			\EdgeL{0}{2}{}        
			\EdgeL{0}{3}{}        
			\EdgeL{0}{4}{}        
			\EdgeL{0}{5}{}        
			\EdgeL{0}{6}{}        
			\EdgeL{0}{7}{}        
			\EdgeL{1}{0}{}        																	
			\EdgeL{1}{2}{}        																	
			\EdgeL{1}{3}{}
			\EdgeL{1}{4}{}
			\EdgeL{1}{5}{}
			\EdgeL{1}{6}{}
			\EdgeL{1}{7}{}
			\EdgeL{2}{0}{}
			\EdgeL{2}{1}{}
			\EdgeL{2}{3}{}						
			\EdgeL{2}{4}{}
			\EdgeL{2}{5}{}
			\EdgeL{2}{6}{}
			\EdgeL{2}{7}{}
			\EdgeL{3}{0}{}
			\EdgeL{3}{1}{}
			\EdgeL{3}{2}{}
			\EdgeL{3}{4}{}
			\EdgeL{3}{5}{}
			\EdgeL{3}{6}{}
			\EdgeL{3}{7}{}
			\EdgeL{4}{0}{}
			\EdgeL{4}{1}{}
			\EdgeL{4}{2}{}
			\EdgeL{4}{3}{}
			\EdgeL{4}{5}{}
			\EdgeL{4}{6}{}
			\EdgeL{4}{7}{}
			\EdgeL{5}{0}{}			
			\EdgeL{5}{1}{}
			\EdgeL{5}{2}{}
			\EdgeL{5}{3}{}
			\EdgeL{5}{4}{}
			\EdgeL{5}{6}{}
			\EdgeL{5}{7}{}
			\EdgeL{6}{0}{}
			\EdgeL{6}{1}{}
			\EdgeL{6}{2}{}
			\EdgeL{6}{3}{}
			\EdgeL{6}{4}{}
			\EdgeL{6}{5}{}
			\EdgeL{6}{7}{}			
			\EdgeL{7}{0}{}			
			\EdgeL{7}{1}{}			
			\EdgeL{7}{2}{}			
			\EdgeL{7}{3}{}			
			\EdgeL{7}{4}{}			
			\EdgeL{7}{5}{}			
			\EdgeL{7}{6}{}		
			\ArcR{s0}{6}{}	
			\ArcL{s0}{5}{}	
			\VArcL[.5]{arcangle=-70,ncurv=.8}{s0}{4}{}	
			\VArcL[.5]{arcangle=70,ncurv=.8}{s0}{7}{}								
			\VArcL[.5]{arcangle=-90,ncurv=1.2}{s0}{3}{}										
			\VArcL[.5]{arcangle=90,ncurv=1.2}{s0}{0}{}							
			\VArcL[.5]{arcangle=110,ncurv=2}{s0}{1}{}										
			\VArcL[.5]{arcangle=-110,ncurv=2}{s0}{2}{}													
			%	\VArcR[.5]{arcangle=45,ncurv=.6}{6}{0}{}        						
			%	\VArcL[.5]{arcangle=-45,ncurv=.6}{6}{1}{}        									
		\end{VCPicture}
	}	
	\caption{Automaton $E$ in \cltsCPEmbedding}
	\label{fig:fds_to_clts_E}
	%\endminipage\hfill
\end{figure}




\footnotesize
\vspace{1em}
\begin{tabular}{ r c l }
	$\theta_{e}$ &$=$& $(\bar{\hat{p}} \wedge \bar{\hat{r}})$\\
	$\theta_{s}$ &$=$& $\bar{s_0} \wedge \bar{s_1} \wedge ((\bar{\hat{P1}} \wedge \bar{\hat{P2}} \wedge \bar{\hat{R}})) \wedge \bar{\hat{e}} \wedge \bar{\hat{a}} \wedge \bar{\hat{b}}$\\	
	$\rho_{env.enabling}$ &$=$& $((\bar{s_0} \wedge \bar{s_1} ) \implies ((\hat{p}'\wedge \bar{\hat{r}}') \vee (\bar{\hat{p}}' \wedge \hat{r}'))) \wedge$\\
	&& $((s_0 \wedge \bar{s_1}) \implies (\bar{\hat{p}}' \wedge \bar{\hat{r}}')) \wedge$\\	
	&& $((\bar{s_0} \wedge s_1) \implies (\bar{\hat{p}}' \wedge \bar{\hat{r}}')) \wedge$\\	
	&& $((s_0 \wedge s_1) \implies (\bar{\hat{p}}' \wedge \bar{\hat{r}}'))$\\			
	$\rho_{sys.enabling}$ &$=$& $((\bar{s_0} \wedge \bar{s_1} ) \implies (\bar{\hat{P1}}' \wedge\bar{\hat{P2}}' \wedge \bar{\hat{R}}')) \wedge$\\
	&& $((s_0 \wedge \bar{s_1} ) \implies ((\hat{P1}'\wedge \bar{\hat{P2}}' \wedge \bar{\hat{R}}') \vee (\bar{\hat{P1}}'\wedge \hat{P2}' \wedge \bar{\hat{R}}'))) \wedge$\\		
	&& $((\bar{s_0} \wedge s_1 ) \implies ((\bar{\hat{P1}}'\wedge \bar{\hat{P2}}' \wedge \hat{R}')))\wedge$\\	
	&& $((s_0 \wedge \bar{s_1} ) \implies ((\bar{\hat{P1}}'\wedge \bar{\hat{P2}}' \wedge \hat{R}')))$\\
	$\rho_{update.states}$ &$=$& $((\bar{s_0} \wedge \bar{s_1} \wedge \hat{p}'\wedge \bar{\hat{r}}' \wedge \bar{\hat{P1}}' \wedge \bar{\hat{P2}}' \wedge \bar{\hat{R}}') \implies (\bar{s_0}' \wedge \bar{s_1}')) \wedge$\\
	&&$((\bar{s_0} \wedge \bar{s_1} \wedge \bar{\hat{p}}' \wedge \hat{r}'\wedge \bar{\hat{P1}}' \wedge \bar{\hat{P2}}' \wedge \bar{\hat{R}}') \implies (s_0' \wedge \bar{s_1}')) \wedge$\\	
	&&$((s_0 \wedge \bar{s_1} \wedge \bar{\hat{p}}' \wedge \bar{\hat{r}}' \wedge \bar{\hat{P1}}' \wedge \bar{\hat{P2}}' \wedge \hat{R}')\implies (\bar{s_0}' \wedge \bar{s_1}')) \wedge$\\	
	&&$((s_0 \wedge \bar{s_1} \wedge \bar{\hat{p}}' \wedge \bar{\hat{r}}' \wedge \hat{P1}'\wedge \bar{\hat{P2}}' \wedge \bar{\hat{R}}') \implies (\bar{s_0}' \wedge s_1')) \wedge$\\	
	&&$((s_0 \wedge \bar{s_1} \wedge \bar{\hat{p}}' \wedge \bar{\hat{r}}' \wedge \bar{\hat{P1}}'\wedge \hat{P2}' \wedge \bar{\hat{R}}') \implies (s_0' \wedge s_1')) \wedge$\\		
	&&$((\bar{s_0} \wedge s_1 \wedge \bar{\hat{p}}' \wedge \bar{\hat{r}}' \wedge \bar{\hat{P1}}' \wedge \bar{\hat{P2}}' \wedge \hat{R}')\implies (\bar{s_0}' \wedge \bar{s_1}')) \wedge$\\		
	&&$((s_0 \wedge s_1 \wedge \bar{\hat{p}}' \wedge \bar{\hat{r}}' \wedge \bar{\hat{P1}}' \wedge \bar{\hat{P2}}' \wedge \hat{R}')\implies (\bar{s_0}' \wedge \bar{s_1}'))$\\			
	$\rho_{update.propositions}$ &$=$& $((\bar{s_0} \wedge \bar{s_1}) \implies (\bar{\hat{e}}' \wedge \bar{\hat{a}}' \wedge \bar{\hat{b}}')) \wedge$\\
	&& $((s_0 \wedge \bar{s_1}) \implies (\hat{e}' \wedge \bar{\hat{a}}' \wedge \bar{\hat{b}}')) \wedge$\\
	&& $((\bar{s_0} \wedge s_1) \implies (\bar{\hat{e}}' \wedge \hat{a}' \wedge \bar{\hat{b}}')) \wedge$\\
	&& $((s_0 \wedge s_1) \implies (\bar{\hat{e}}' \wedge \bar{\hat{a}}' \wedge \hat{b}'))$\\	
	$\gsRhoE$&$=$&$\rho_{env.enabling}$\\		
	$\gsRhoS$&$=$&$\rho_{sys.enabling} \wedge  \rho_{update.states} \wedge  \rho_{update.propositions}$\\	
\end{tabular}
\vspace{1em}
\normalsize

If we keep only the behavior that conforms to $\theta_e, \theta_s, \rho_e$ and $\rho_s$ this translation induces the Kripke automaton depicted in figure ~\ref{fig:clts_to_fds_K}.

\begin{figure}[bt]
	\centering
	%\minipage{0.32\textwidth}%
	%\centering
	\VCDraw{
		\begin{VCPicture}{(-6.5,-2.5)(6.5,3.5)}
			\SetStateLabelScale{1}
			\SetEdgeLabelScale{1.2}
			\StateVar[0:\overline{s_0 s_1 e a b p r P1 P2 R}]{(-6,1.3)}{0}
			\StateVar[1:\overline{s_0 s_1 e a b}p \overline{r P1 P2 R}]{(-2,1.3)}{1}
			\StateVar[2:\overline{s_0 s_1 e a b p r P1 P2} R]{(2,-0.3)}{2}
			\StateVar[3:s_0\overline{s_1} e  \overline{a b p} r \overline{P1 P2 R}]{(2,2.9)}{3}
			\StateVar[4:\overline{s_0}s_1 \overline{e} a  \overline{b p r } P1 \overline{ P2 R}]{(6,2.9)}{4}
			\StateVar[5:s_0 s_1 \overline{e a} b  \overline{ p r P1} P2 \overline{R}]{(6,-0.3)}{5}			
			\Initial[w]{0}
			\CLoopSW{1}{}			
			\EdgeL{0}{1}{}        
			\LArcL{0}{3}{}  			
			\EdgeL{2}{1}{}        		
			\EdgeL{1}{3}{}        		
			\ArcR{3}{2}{}        		
			\ArcR{2}{3}{}        					
			\ArcL{3}{4}{}        		
			\EdgeR{3}{5}{}        					
			\EdgeL{4}{2}{}        		
			\ArcL{5}{2}{}        		
		%	\VArcR[.5]{arcangle=45,ncurv=.6}{6}{0}{}        						
		%	\VArcL[.5]{arcangle=-45,ncurv=.6}{6}{1}{}        									
		\end{VCPicture}
	}	
	\caption{Kripke structure induced from $\theta$ and $\rho$}
	\label{fig:clts_to_fds_K}
	%\endminipage\hfill
\end{figure}

We then use $\theta_e$, $\theta_s$, $\rho_e$ and $\rho_s$ to write \varphiLTL and solve \fdsEmbeddingDef. Following the construction of the FDS controller from a given strategy as in ~\cite{bloem2012synthesis}, we describe the transformation of $f$ into $\hat{f}$ as:
\[\hat{f} = \bigwedge_{m \in M}\bigwedge_{s \in W_s}\bigwedge_{s'_{\gsX} \in \Sigma_{\gsX}} ((m \wedge s \wedge s'_{\gsX})\implies f(m,s,s'_{\gsX})')\]
Where $M$ is the memory domain and $W_s$ is the set of winning states. We can see that this transformation maps the function:
\[f: M \times \Sigma \times \Sigma_{\gsX} \mapsto M \times \Sigma_{\gsY}\ \]
To:
\[\hat{f}: M \times \Sigma \times \Sigma_{\gsX'} \mapsto M' \times \Sigma_{\gsY'}\ \]
While restricting only those states within the winning region.
The FDS controller is built as $\mathcal{D} = \langle \hat{\gsV}, \hat{\theta}, \hat{\rho} \rangle$ where:

\vspace{1em}
\begin{tabular}{ l c l }
	$\hat{\gsV}$ & $=$ & $\gsX \cup \gsY \cup \mathcal{M}$\\	
	$\hat{\theta}$ & $=$ & $(\theta_e \implies (\theta_s \wedge m_0 \wedge W_s))$\\
	$\hat{\rho}$ & $=$ & $((W_s \wedge \rho_e) \implies \hat{f})$\\	
\end{tabular}
\vspace{1em}

We describe $\hat{f}$ extensively only for the part of its domain that satisfies $\theta_e$, $\rho_e$ and $W_s$ since any other values are assigned arbitrarily. The rightmost term indicates the transition from the automaton in figure ~\ref{fig:clts_to_fds_K} being kept for each row. 

\vspace{1em}
\begin{tabular}{ r r r l l r r l l l}
	$\hat{f}($ & $\overline{m},$ & $\overline{s_0 s_1eabprP1P2R},$ &$p'\overline{r'}$ & $) =$ & $($ & $\overline{s_0' s_1'e'a'b'P1'P2'R'},$ & $\overline{m'}$ & $)$ & $((0,0) \rightarrow (0,0))$\\
	$\hat{f}($ & $\overline{m},$ & $\overline{s_0 s_1eab}p\overline{rP1P2R},$ &$p'\overline{r'}$ & $) =$ & $($ & $\overline{s_0' s_1'e'a'b'P1'P2'R'},$ & $\overline{m'}$ & $)$ & $((0,0) \rightarrow (0,0))$\\	
	$\hat{f}($ & $\overline{m},$ & $\overline{s_0 s_1 eabp rP1P2R},$ &$\overline{p'}r'$ & $) =$ & $($ & $s_0'\overline{s_1'}e'\overline{a'b'P1'P2'R'},$ & $\overline{m'}$ & $)$ & $((0,0) \rightarrow (1,0))$\\
	$\hat{f}($ & $\overline{m},$ & $\overline{s_0 s_1eab} p \overline{rP1P2R},$ &$\overline{p'}r'$ & $) =$ & $($ & $s_0'\overline{s_1'}e'\overline{a'b'P1'P2'R'},$ & $\overline{m'}$ & $)$ & $((0,0) \rightarrow (1,0))$\\
	$\hat{f}($ & $\overline{m},$ & $s_0\overline{s_1}e\overline{abp} r \overline{P1P2R},$ &$\overline{p'r'}$ & $) =$ & $($ & $\overline{s_0'}s_1'\overline{e'}a'\overline{b'}P1'\overline{P2'R'},$ & $m'$ & $)$ & $((1,0) \rightarrow (2,1))$\\	
	$\hat{f}($ & $m,$ & $\overline{s_0}s_1\overline{e}a\overline{b p r}P1\overline{P2R},$ &$\overline{p'r'}$ & $) =$ & $($ & $\overline{s_0's_1'e'a'b'P1'P2'}R',$ & $m'$ & $)$ & $((2,1) \rightarrow (0,1))$\\	
	$\hat{f}($ & $m,$ & $\overline{s_0s_1eab p rP1P2}R,$ &$p'\overline{r'}$ & $) =$ & $($ & $\overline{s_0's_1'e'a'b'P1'P2'R'},$ & $m'$ & $)$ & $((0,1) \rightarrow (0,1))$\\			
	$\hat{f}($ & $m,$ & $\overline{s_0s_1eab p rP1P2}R,$ &$\overline{p'}r'$ & $) =$ & $($ & $s_0'\overline{s_1'}e'\overline{a'b'P1'P2'R'},$ & $m'$ & $)$ & $((0,1) \rightarrow (1,1))$\\				
	$\hat{f}($ & $m,$ & $s_0\overline{s_1}e\overline{ab p} r\overline{P1P2R},$ &$\overline{p'r'}$ & $) =$ & $($ & $s_0's_1'\overline{e'a'}b'\overline{P1'}P2'\overline{R'},$ & $\overline{m'}$ & $)$ & $((1,1) \rightarrow (3,0))$\\					
	$\hat{f}($ & $\overline{m},$ & $s_0s_1\overline{ea}b\overline{prP1}P2\overline{R},$ &$\overline{p'r'}$ & $) =$ & $($ & $\overline{s_0's_1'e'a'b'P1'P2'}R',$ & $\overline{m'}$ & $)$ & $((3,0) \rightarrow (0,0))$\\						
	$\hat{f}($ & $\overline{m},$ & $\overline{s_0s_1eabprP1P2}R,$ &$p'\overline{r'}$ & $) =$ & $($ & $\overline{s_0's_1'e'a'b'P1'P2'R'},$ & $\overline{m'}$ & $)$ & $((0,0) \rightarrow (0,0))$\\	
	$\hat{f}($ & $\overline{m},$ & $\overline{s_0s_1eabprP1P2}R,$ &$\overline{p'}r'$ & $) =$ & $($ & $s_0'\overline{s_1'}e'\overline{a'b'P1'P2'R'},$ & $\overline{m'}$ & $)$ & $((0,0) \rightarrow (1,0))$\\		
\end{tabular}
\vspace{1em}



If we restrict the behavior of the model that was shown in figure ~\ref{fig:clts_to_fds_K} after applying $\hat{f}$ we get the Kripke automaton from figure  ~\ref{fig:clts_to_fds_F}.

\begin{figure}[bt]
	\centering
	%\minipage{0.32\textwidth}%
	%\centering
	\VCDraw{
		\begin{VCPicture}{(-8.5,-2.5)(8.5,3.5)}
			\SetStateLabelScale{1}
			\SetEdgeLabelScale{1.2}
			\StateVar[0:\overline{s_0 s_1 e a b p r P1 P2 R m}]{(-10,-0.3)}{0}
			\StateVar[1:\overline{s_0 s_1 e a b}p \overline{r P1 P2 R m}]{(-5,-0.3)}{1}
			\StateVar[2:\overline{s_0 s_1 e a b p r P1 P2} R \overline{m}]{(-5,2.9)}{2}
			\StateVar[3:s_0\overline{s_1} e  \overline{a b p} r \overline{P1 P2 R m}]{(0,-0.3)}{3}
			\StateVar[4:\overline{s_0}s_1 \overline{e} a  \overline{b p r } P1 \overline{ P2 R}m]{(5,-0.3)}{4}
			\StateVar[5:\overline{s_0 s_1 e a b}p \overline{r P1 P2 R}m]{(10,2.9)}{1b}
			\StateVar[6:\overline{s_0 s_1 e a b p r P1 P2} R m]{(10,-0.3)}{2b}
			\StateVar[7:s_0\overline{s_1} e  \overline{a b p} r \overline{P1 P2 R}m]{(5,2.9)}{3b}
			\StateVar[8:s_0 s_1 \overline{e a} b  \overline{ p r P1} P2		\overline{R m}]{(0,2.9)}{5b}					
			\Initial[w]{0}
			\CLoopS{1}{}			
			\CLoopN{1b}{}					
			\EdgeL{0}{1}{}        
			\LArcR{0}{3}{}  							
			\EdgeL{1}{3}{}        			
			\ArcR{2}{1}{}        			
			\EdgeL{2}{3}{}        			
			\EdgeL{3}{4}{}        			
			\EdgeL{4}{2b}{}        			
			\EdgeL{2b}{3b}{}        			
			\ArcR{2b}{1b}{}        						
			\EdgeL{1b}{3b}{}        									
			\EdgeL{3b}{5b}{}        						
			\EdgeL{5b}{2}{}        						
			%	\VArcR[.5]{arcangle=45,ncurv=.6}{6}{0}{}        						
			%	\VArcL[.5]{arcangle=-45,ncurv=.6}{6}{1}{}        									
		\end{VCPicture}
	}	
	\caption{Kripke structure induced from $\theta$ and $\rho$ after strengthening through $\hat{f}$}
	\label{fig:clts_to_fds_F}
	%\endminipage\hfill
\end{figure}

We can get our CLTS controller from \fdsD by applying $clts(\fdsD)$ to get the CLTS depicted in figure ~\ref{fig:clts_to_fds_C} which is a solution to \controlProblem.

\begin{figure}[H]
	\centering
	%\minipage{0.32\textwidth}%
	%\centering
	\VCDraw{
		\begin{VCPicture}{(-3.5,-2.5)(3.5,3.5)}
			\SetStateLabelScale{1}
			\SetEdgeLabelScale{1.2}
			\State[0,0]{(-3,2.3)}{0}
			\State[1,0]{(0,2.3)}{1}
			\State[2,0]{(3,2.3)}{2}
			\State[0,1]{(3,-0.3)}{0b}
			\State[1,1]{(0,-0.3)}{1b}			
			\State[3,0]{(-3,-0.3)}{3}			
			\Initial[w]{0}
			\EdgeL{0}{1}{<r>}        
			\LoopN{0}{<p>}		
			\EdgeL{1}{2}{<P1>}        
			\EdgeL{2}{0b}{<R>}        
			\EdgeL{0b}{1b}{<r>}    
			\LoopS{0b}{<p>}					    						
			\EdgeL{1b}{3}{<P2>}    			
			\EdgeL{3}{0}{<R>}    						
		\end{VCPicture}
	}
	\caption{Controller $C$ translated from $\fdsD$}
	\label{fig:clts_to_fds_C}
	%\endminipage\hfill
\end{figure}