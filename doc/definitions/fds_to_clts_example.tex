Now we show an example of an embedding from an FDS control problem into a CLTS control problem, for which a strategy is built, converted into a CLTS controller and then translated into an FDS controller. 
The FDS control problem handles a request $r$ to process a data packet by granting access to one of two clients, represented by signals $g_1$ and $g_2$.
The property to be satisfied is captured by a conjunction of an initial statement, a safety restriction and two liveness properties:
\[\varphi: (\overline{r}\wedge \overline{g}_1\wedge \overline{g}_2)\wedge (\square(g_1 \iff \overline{g}_2))\wedge ((\square\Diamond(x \implies g_1))\wedge(\square\Diamond(x \implies g_1)))\]
The FDS control problem is noted as $\mathcal{K}=\langle \{r\}, \{g_1,g_2\}, \varphi \rangle$. The automaton depicted in figure ~\ref{fig:fds_to_clts_K} represents the behavior of the initial and safety terms ($(\overline{r}\wedge \overline{g}_1\wedge \overline{g}_2)\wedge (\square(g_1 \iff \overline{g}_2))$).

\begin{figure}[bt]
	\centering
	%\minipage{0.32\textwidth}%
	%\centering
	\VCDraw{
	\begin{VCPicture}{(-2,-2)(4,3.5)}
		\SetStateLabelScale{1}
		\SetEdgeLabelScale{1.2}
		\StateVar[0:\overline{r g_1 g_2}]{(-2,1.3)}{0}
		\StateVar[1:\overline{r g_1}g_2]{(1,2.9)}{1}
		\StateVar[2:r g_1 \overline{g_2}]{(4,1.3)}{2}
		\StateVar[3:r\overline{g_1} g_2]{(1,-0.3)}{3}
		\Initial[w]{0}
		\LoopN{0}{}		
		\LoopN{1}{}				
		\LoopN{2}{}						
		\LoopS{3}{}						
		\EdgeL{0}{1}{}        
		\EdgeL{0}{2}{}        		
		\EdgeL{0}{3}{}        		
		\EdgeL{1}{0}{}        		
		\EdgeL{1}{2}{}        				
		\EdgeL{1}{3}{}        						
		\EdgeL{2}{0}{}        						
		\EdgeL{2}{1}{}        				
		\EdgeL{2}{3}{}        		
		\EdgeL{3}{0}{}				
		\EdgeL{3}{1}{}				
		\EdgeL{3}{2}{}		
	\end{VCPicture}
	}
	\caption{Automaton representation of $(\overline{r}\wedge \overline{g}_1\wedge \overline{g}_2)\wedge (\square(g_1 \iff \overline{g}_2))$}
	\label{fig:fds_to_clts_K}
	%\endminipage\hfill
\end{figure}

The CLTS automaton $E$ in \cltsCPEmbedding that captures all possible interaction between the FDS control problem variables is depicted in figure ~\ref{fig:fds_to_clts_E}, note that label information is omitted to allow for a more readable representation. Note that the LTL formula of the CLTS embedding is:

\[\varphi: \bigcirc(\overline{\hat{r}}\wedge \overline{\hat{g}}_1\wedge \overline{\hat{g}}_2)\wedge (\square(\hat{g}_1 \iff \overline{\hat{g}}_2))\wedge ((\square\Diamond(\hat{x} \implies \hat{g}_1))\wedge(\square\Diamond(\hat{x} \implies \hat{g}_1)))\]

\begin{figure}[bt]
	\centering
	%\minipage{0.32\textwidth}%
	%\centering
	\VCDraw{
		\begin{VCPicture}{(-6.5,-8)(6.5,4)}
			\SetStateLabelScale{1}
			\SetEdgeLabelScale{1.2}
			\StateVar[0:\overline{r g_1 g_2}]{(-4,1)}{0}
			\StateVar[1:\overline{r g_1}g_2]{(-2,3)}{1}			
			\StateVar[2:\overline{r} g_1 \overline{g_2}]{(2,3)}{2}			
			\StateVar[3:\overline{r} g_1 g_2]{(4,1)}{3}						
			\StateVar[4:r\overline{g_1 g_2}]{(4,-1)}{4}									
			\StateVar[5:r\overline{g_1} g_2]{(2,-3)}{5}												
			\StateVar[6:r g_1\overline{g_2}]{(-2,-3)}{6}
			\StateVar[7:r g_1 g_2]{(-4,-1)}{7}			
			\StateVar[s_0:\overline{r g_1 g_2}]{(0,-6)}{s0}						
			\Initial[s]{s0}
			\CLoopN{0}{}						
			\CLoopN{1}{}			
			\CLoopN{2}{}						
			\CLoopN{3}{}									
			\CLoopS{4}{}												
			\CLoopS{5}{}												
			\CLoopS{6}{}												
			\CLoopS{7}{}																				
			\EdgeL{0}{1}{}        
			\EdgeL{0}{2}{}        
			\EdgeL{0}{3}{}        
			\EdgeL{0}{4}{}        
			\EdgeL{0}{5}{}        
			\EdgeL{0}{6}{}        
			\EdgeL{0}{7}{}        
			\EdgeL{1}{0}{}        																	
			\EdgeL{1}{2}{}        																	
			\EdgeL{1}{3}{}
			\EdgeL{1}{4}{}
			\EdgeL{1}{5}{}
			\EdgeL{1}{6}{}
			\EdgeL{1}{7}{}
			\EdgeL{2}{0}{}
			\EdgeL{2}{1}{}
			\EdgeL{2}{3}{}						
			\EdgeL{2}{4}{}
			\EdgeL{2}{5}{}
			\EdgeL{2}{6}{}
			\EdgeL{2}{7}{}
			\EdgeL{3}{0}{}
			\EdgeL{3}{1}{}
			\EdgeL{3}{2}{}
			\EdgeL{3}{4}{}
			\EdgeL{3}{5}{}
			\EdgeL{3}{6}{}
			\EdgeL{3}{7}{}
			\EdgeL{4}{0}{}
			\EdgeL{4}{1}{}
			\EdgeL{4}{2}{}
			\EdgeL{4}{3}{}
			\EdgeL{4}{5}{}
			\EdgeL{4}{6}{}
			\EdgeL{4}{7}{}
			\EdgeL{5}{0}{}			
			\EdgeL{5}{1}{}
			\EdgeL{5}{2}{}
			\EdgeL{5}{3}{}
			\EdgeL{5}{4}{}
			\EdgeL{5}{6}{}
			\EdgeL{5}{7}{}
			\EdgeL{6}{0}{}
			\EdgeL{6}{1}{}
			\EdgeL{6}{2}{}
			\EdgeL{6}{3}{}
			\EdgeL{6}{4}{}
			\EdgeL{6}{5}{}
			\EdgeL{6}{7}{}			
			\EdgeL{7}{0}{}			
			\EdgeL{7}{1}{}			
			\EdgeL{7}{2}{}			
			\EdgeL{7}{3}{}			
			\EdgeL{7}{4}{}			
			\EdgeL{7}{5}{}			
			\EdgeL{7}{6}{}		
			\ArcR{s0}{6}{}	
			\ArcL{s0}{5}{}	
			\VArcL[.5]{arcangle=-70,ncurv=.8}{s0}{4}{}	
			\VArcL[.5]{arcangle=70,ncurv=.8}{s0}{7}{}								
			\VArcL[.5]{arcangle=-90,ncurv=1.2}{s0}{3}{}										
			\VArcL[.5]{arcangle=90,ncurv=1.2}{s0}{0}{}							
			\VArcL[.5]{arcangle=110,ncurv=2}{s0}{1}{}										
			\VArcL[.5]{arcangle=-110,ncurv=2}{s0}{2}{}													
			%	\VArcR[.5]{arcangle=45,ncurv=.6}{6}{0}{}        						
			%	\VArcL[.5]{arcangle=-45,ncurv=.6}{6}{1}{}        									
		\end{VCPicture}
	}	
	\caption{Automaton $E$ in \cltsCPEmbedding}
	\label{fig:fds_to_clts_E}
	%\endminipage\hfill
\end{figure}


A candidate CLTS controller $C$ that is a solution to \cltsCPEmbedding is depicted in figure ~\ref{fig:fds_to_clts_C} (label information omitted to improve readability).


Applying the translation $fds'$ over $C$ we get:

\footnotesize
\vspace{1em}
\begin{tabular}{ r c l }
	$\gsV_{d}$ &$=$& $\gsX \cup \gsY \cup \{s_{(2,0)},s_{(6,1)},s_{(5,0)},s_{(2,1)},s_{(1,0)}\}$\\	
	$\theta_{d}$ &$=$& $\bar{r} \wedge \bar{g}_1 \wedge g_2 \wedge s_{(2,0)}  \wedge \bar{s}_{(6,1)}  \wedge \bar{s}_{(5,0)}  \wedge \bar{s}_{(2,1)}  \wedge \bar{s}_{(1,0)}$\\						
	$\rho_{d}$ &$=$& $\square((s_{(2,0)}\implies (\bar{r}' \wedge \bar{g}'_1 \wedge g_2 \wedge s'_{(2,0)}  \wedge \bar{s}'_{(6,1)}  \wedge \bar{s}'_{(5,0)}  \wedge \bar{s}'_{(2,1)}  \wedge \bar{s}'_{(1,0)}) \vee$\\
	&&$\qquad(r' \wedge g'_1 \wedge \bar{g}_2 \wedge \bar{s}'_{(2,0)}  \wedge s'_{(6,1)}  \wedge \bar{s}'_{(5,0)}  \wedge \bar{s}'_{(2,1)}  \wedge \bar{s}'_{(1,0)})) \wedge$\\
	&& $(s_{(6,1)}\implies (r' \wedge \bar{g}'_1 \wedge g_2 \wedge \bar{s}'_{(2,0)}  \wedge \bar{s}'_{(6,1)}  \wedge s'_{(5,0)}  \wedge \bar{s}'_{(2,1)}  \wedge \bar{s}'_{(1,0)}) \vee$\\
	&&$\qquad(\bar{r}' \wedge \bar{g}'_1 \wedge g_2 \wedge \bar{s}'_{(2,0)}  \wedge \bar{s}'_{(6,1)}  \wedge \bar{s}'_{(5,0)}  \wedge s'_{(2,1)}  \wedge \bar{s}'_{(1,0)})) \wedge$\\		
	&& $(s_{(5,0)}\implies (\bar{r}' \wedge \bar{g}'_1 \wedge g_2 \wedge \bar{s}'_{(2,0)}  \wedge \bar{s}'_{(6,1)}  \wedge \bar{s}'_{(5,0)}  \wedge \bar{s}'_{(2,1)}  \wedge s'_{(1,0)}) \vee$\\
	&&$\qquad(r' \wedge g'_1 \wedge \bar{g}_2 \wedge \bar{s}'_{(2,0)}  \wedge s'_{(6,1)}  \wedge \bar{s}'_{(5,0)}  \wedge \bar{s}'_{(2,1)}  \wedge \bar{s}'_{(1,0)})) \wedge$\\			
	&& $(s_{(1,0)}\implies (r' \wedge g'_1 \wedge \bar{g}_2 \wedge \bar{s}'_{(2,0)}  \wedge s'_{(6,1)}  \wedge \bar{s}'_{(5,0)}  \wedge \bar{s}'_{(2,1)}  \wedge \bar{s}'_{(1,0)})) \wedge$\\				
	&& $(s_{(2,0)}\implies (r' \wedge \bar{g}'_1 \wedge \bar{g}_2 \wedge \bar{s}'_{(2,0)}  \wedge \bar{s}'_{(6,1)}  \wedge s'_{(5,0)}  \wedge \bar{s}'_{(2,1)}  \wedge \bar{s}'_{(1,0)})))$\\	
	$\mathcal{J}_d$ & $=$ & $\emptyset$\\
$\mathcal{C}_d$ & $=$ & $\emptyset$\\					
\end{tabular}
\vspace{1em}
\normalsize

Then, \fdsD can be represented as the automaton in figure ~\ref{fig:fds_to_clts_D}, where we use $mux'(s_i)$ to abbreviate the conjunction of state variables used as memory.

\begin{figure}[bt]
	\centering
	%\minipage{0.32\textwidth}%
	%\centering
	\VCDraw{
		\begin{VCPicture}{(-2.5,-1.5)(6.5,3.5)}
			\SetStateLabelScale{1}
			\SetEdgeLabelScale{1.2}
			\StateVar[0:\overline{r g_1} g_2 mux(s_{2,0})]{(-2,2.9)}{1}
			\StateVar[1:r g_1\overline{g_2} mux(s_{6,1})]{(2,2.9)}{2}
			\StateVar[2:r \overline{g_1}g_2 mux(s_{5,0})]{(6,2.9)}{3}
			\StateVar[3:\overline{r g_1}g_2 mux(s_{1,0})]{(6,-0.3)}{4}
			\StateVar[4:\overline{r g_1} g_2 mux(s_{2,1})]{(2,-0.3)}{5}				
			\Initial[w]{1}
			\LoopS{1}{}
			\EdgeL{1}{2}{}        						
			\EdgeL{2}{3}{}        			
			\EdgeL{3}{2}{}      			
			\EdgeL{2}{5}{}        			
			\EdgeL{3}{4}{}        						
			\EdgeL{4}{2}{}        			
			\EdgeL{5}{3}{}        						
			%	\VArcR[.5]{arcangle=45,ncurv=.6}{6}{0}{}        						
			%	\VArcL[.5]{arcangle=-45,ncurv=.6}{6}{1}{}        									
		\end{VCPicture}
	}
	\caption{Automaton representation of \fdsD}
	\label{fig:fds_to_clts_D}
	%\endminipage\hfill
\end{figure}