When writing a specification or an abstraction of a digital system it is convenient to describe its behavior with game structures. This formalism captures the semantic of a two player game where the first participant, the environment, observes the state of both input and output variables at a given point of time to define a new valuation for the input, only then the second player, the system, observes both the previous and the current partial states to define the new valuation for the output. 

The set of possible actions for each player is described as relations between current and next (primed) state's variables. Satisfaction of $\gsRhoE: \gsX \times \gsY \times \gsX'$ describes the values that $\gsX'$ can take when $\gsX$ and $\gsY$ are fixed. Both the initial conditions and the transition relations can be described as boolean formulas which in turn can be represented with OBDDs. OBDDs can be composed, restricted over a subset of known values, checked for satisfiability and equality among other operations. Algorithms that work with OBDDs can and do process sets of states at each step, but finite state machines, such as CLTS, can only be processed one state at a time.

\begin{definition}
	\label{def:GS} \emph{(Game Structures)} 
	A \emph{Game Structure} (GS) $G =  \langle \gsV, \gsX, \gsY, \gsTheE, \gsTheS, \gsRhoE, \gsRhoS, \varphi \rangle$ consists of the following:
	\begin{itemize}
		\item $\gsV = \{v_1,\ldots,v_n\}$ : A finite set of typed state variables over finite domains, without loss of generality we assume that they are all Boolean. In the context of a game structure a state is defined as an assignment of $\gsV$, $s \in \Sigma_{\gsV}$.
		\item $\gsX \subseteq \gsV$ is a set of input variables.
		\item $\gsY = \gsV \setminus \gsX$ is a set of output variables.
		\item $\gsTheE$ is an assertion over $\gsX$ characterizing the initial states of the environment.
		\item $\gsTheS$ is an assertion over $\gsV$ characterizing the initial states of the system.		
		\item $\gsRhoE(\gsV,\gsX')$ is the transition relation of the environment, it identifies a valuation $s_{\gsX} \in \Sigma_{\gsX}$ as a possible input in state $s$ if $(s,s_{\gsX}) \models \gsRhoE$.
		\item $\gsRhoS(\gsV,\gsX',\gsY')$ is the transition relation of the system, it identifies a valuation $s_{\gsY} \in \Sigma_{\gsY}$ as a possible output in state $s$ reading input $s_{\gsX} $if $(s,s_{\gsX}, s_{\gsY}) \models \gsRhoS$.		
		\item $\varphi$ is the winning condition, given by an LTL formula.
	\end{itemize} 
\end{definition}

\begin{definition}
	\label{def:GS_play} \emph{(Play over Game Structure)} 
	Let $G =  \langle \gsV, \gsX, \gsY, \gsTheE, \gsTheS, \gsRhoE, \gsRhoS, \varphi \rangle$ be a game structure, $\sigma = \sigma_1,\sigma_2,\ldots \in (2^{\mathcal{V}})^\omega$ is a play over $G$ if the following holds:
	\begin{itemize} 
		\item The initial condition is satisfied, i.e. $\theta_e(\sigma_1|_{\mathcal{X}}) \wedge \theta_s(\sigma_1)$
		\item The valuations are updated according to $\rho$,
		 $\forall i \in 2\ldots: \rho_e(\sigma_{i-1}, \sigma_i|_{\mathcal{X}}) \wedge \rho_s(\sigma_{i-1}, \sigma_i|_{\mathcal{X}},\sigma_i|_{\mathcal{Y}})$		
	\end{itemize}
\end{definition}

A play $\sigma$ satisfies LTL formula $\varphi$  if $\sigma_1 \models \varphi$ according to the following inductive definition:

\begin{tabular}{ l c l }
	$\sigma_i \models_d v$ & $\triangleq$ & $v \in \sigma_i$ for any $v \in \mathcal{V}$\\	
	$\sigma_i \models_d \neg \varphi$ & $\triangleq$ & $\sigma_i \not\models_d \varphi$\\
	$\sigma_i \models_d \varphi \vee \psi$ & $\triangleq$ & $(\sigma_i \models_d \varphi) \vee (\sigma_i \models_d \psi)$\\
	$\sigma_i \models_d \bigcirc \varphi$ & $\triangleq$ & $\sigma_{i+1} \models_d \varphi$\\
	$\sigma_i \models_d \varphi \U \psi$ & $\triangleq$ & $\exists j \geq i . \sigma_j \models_d \psi \wedge \forall i \leq k \le k. \sigma_k \models_d \varphi$\\
\end{tabular}

A game structure $G$ satisfies LTL formula $\varphi$ if it is satisfied by all of its plays.

Many non trivial digital designs require the definition of state machines to some extent or another, while at the same time high level specifications can involve interactions with digital or truly concurrent components. Under these circumstances a mixed specification over both a finite state process algebra and a LTL game structure definition is desirable.  In order to reason about their composed behavior a proper common representation is needed. Our approach is to transform the game structure into its equivalent finite state automaton and then compose both using one of the parallel composition semantics.

\textcolor{blue}{We now present} the background required to relate the model checking approach in game structures to an equivalent control problem in the CLTS domain.  The main idea behind this is that any game structure specified as a set of LTL formulas induces an equivalent automaton. A pair of translations from a set of LTL formulas to a Kripke structure and from a Kripke structure to a CLTS are introduced first and then a direct transformation between a set of LTL formulas and a CLTS structure.

\begin{definition}
	\label{def:Kripke} \emph{(Kripke Structures)} 
	Let $AP$ be a set of atomic propositions, a \emph{Kripke Structure} (KS)  $K$ over $AP$ is a tuple $K =  \langle S, I, R, L \rangle$ consisting of the following:
	\begin{itemize}
		\item $S$ a finite set of states
		\item $I \subseteq$ a set of initial states
		\item $R \subseteq S \times S$ a transition relation
		\item $L: S \rightarrow 2^{AP}$ a labeling function
	\end{itemize} 
\end{definition}

The translation between game structures and Kripke interprets valuations as states, and relations between those as transitions. Let $G$ be the game structure to be translated and $K=\langle S, I, R, L \rangle$ the target Kripke structure over $AP$. 
Assume that $|AP|=|\mathcal{V}|$ and that for each $v_i \in \mathcal{V}$ exists a proposition $\cdot(v_i) \in AP$. Valuations over $AP$ in $G$ are mapped to states in $S$ as follows, given a valuation $v \in \mathcal{V}$, $\cdot(v)$ represents a unique state in $S$ such that  $|L(\cdot(v))| = |v|$ and for each $v_i \in v$ it holds $\cdot(v_i) \in L(\cdot(v))$. The translation starts with $K$ having an initial state $s_0$ with no valuation associated to it. For every $x \in 2^{|X|}$ where $\theta_e(x)$ holds, a new state $\cdot(x)$ and transition $(s_0, \cdot(x))$ are added to $K$, then for each $x \in 2^{|X|}$, $y \in 2^{|Y|}$ such that $\theta_e(x)$ and $\theta_s(x \cup y)$ hold, state $\cdot(x \cup y)$ and transition $(\cdot(x), \cdot(x \cup y)$ are added to $K$.  For every $v,x' \in 2^{|V|\times|X'|}$ satisfying $\rho_e(v,x')$, a new state $\cdot(v \cup x')$ and transition $(\cdot(v), \cdot(v\cup x'))$ are added to $K$, and for each valuation $v,x',y' \in 2^{|V|\times|X'|\times|Y'|}, $ satisfying $\rho_e(v, x')$ and $\rho_s(v,x',y')$, state $\cdot(x \cup y)$ and transition $(\cdot(v \cup x'), \cdot(x \cup y))$ are added to $K$. 
If the Kripke structure is constructed according to the previous translation, the following properties hold, describing the relation between plays over $G$ and paths over $K$:
	\[\begin{aligned}[t]
	\forall x, y: \theta_e(x) \wedge \theta_s(x,y)& \implies\\
	&(s_0, \cdot (x)), (\cdot(x), \cdot (x \cup y)) \in R \\
	&\wedge \forall x_i \in x: \cdot(x_i) \in L(\cdot(x)) \wedge \forall v_i \in x \cup y : \cdot(v_i) \in L(\cdot(x \cup y))\\
	\forall x, y : \exists v \rho_e(v,x) \wedge \rho_s(v,x,y)& \implies\\
	&(\cdot(v), \cdot(v \cup x)), (\cdot(v \cup x),\cdot(x \cup y)) \in R\\
	&\wedge \forall v_i \in v: \cdot(v_i) \in L(\cdot(v)) \\
	&\wedge \forall w_i \in v \cup x: \cdot(w_i) \in L(\cdot(v \cup x))\\
	&\wedge \forall z_i \in x \cup y: \cdot(z_i) \in L(\cdot(x \cup y))	
	\end{aligned}
	\] 

A translation between LTL formulas over atomic propositions in $K$ and boolean variables in $G$ is straight forward, since the only consideration to be taken is regarding the atomic case of formula $\varphi$ (consisting of a single variable $v_i$). Suppose that $\cdot(\varphi)$ is the translation from boolean variables into atomic propositions, we say that $\varphi$ is satisfied in a play $\sigma=s_0s_1\ldots s_i$ over $G$ if $v_i \in s_i$. In our case this also implies that $\cdot(\varphi)$ is satisfied in the path $\rho=\cdot(s_0)\cdot(s_1)\ldots \cdot(s_i)$ over $K$, since $\cdot(v_i)$ appears in $L(\cdot(s_i))$. It follows from this observation, that if $K$ is the Kripke structure constructed from $G$ following the previous translation, then LTL satisfaction is preserved between structures.

The second step of our composed translation is to transform $K$ into a CLTS instance $M$.
The set of states is kept as is and the labels are moved from $L$ in $K$ to $\Delta$ in $M$. If $M=\langle S, \Sigma, \Delta, s_0 \rangle$ is constructed fom $K =\langle S,I,R,L \rangle$ over $AP$, then both state sets are equal, $s_0$ is the only element in $I$, (since it was, in turn, constructed from $G$) and both $\Sigma$ and $\Delta$ are defined as follows:

\[\forall a \in AP: a\uparrow \in \Sigma \wedge a\downarrow \in \Sigma \]
\[\forall (s, s') \in R: (s, \delta(s,s'), s') \in \Delta \]

Where we add labels to the transitions in $\Delta$, indicating the change of individual variables between valuations. If a variable keeps its value then it will not take part in the transition, otherwise if, for instance, variable $v$ is present in $s$ but not in $s'$, the change is made explicit by adding $v\downarrow$ to the concurrent label.

\[\delta(s,s'): \lbrace label(a,s,s') | a \in L(s) \neq a \in L(s') \rbrace\]
\[
label (a,s,s') = \begin{cases}
a\uparrow & \text{if } a \in L(s') \\
a\downarrow & \text{if } a \in L(s)
\end{cases}
\]

Now if $\cdot(\varphi)$ is the translation from atomic propositions into fluents, and a set of fluents $\mathcal{F}$ exists containing, for each proposition $a$ a fluent $\langle a\uparrow, a\downarrow \rangle$, we say that, again observing only the atomic case where the formula consists of a single variable, $\varphi$ is satisfied in a path $\rho=s_0 s_1\ldots s_i$ over $K$ if $v_i$ appears in $L(s_i)$. In our case this also implies that $\cdot(\varphi)$ is satisfied in the path $\pi=\cdot(s_0)\cdot(s_1)\ldots \cdot(s_i)$ over $M$ since, from $v_i \in L(s_i)$ follows that at some point the proposition started appearing in states leading to $s_i$, suppose that $s_j$ was the first state were $v_i$ appeared and was kept until reaching $s_i$ ($\forall j \leq k \leq i: v_i \in L(s_k)$), then transition $(s_{j-1},s_j) \in R$ was translated as $(s_{j-1}, \delta(s_{j-1},s_j),s_j)$ and $v_i\uparrow \in \delta(s_{j-1},s_j)$ since its value changed from one state to the other. It follows from this observation, that if $M$ is the CLTS structure constructed from $K$ following the previous translation, then LTL satisfaction is preserved between structures.


\newpage
\subsubsection{CLTS preserves realizability w.r.t. game structures}
Given a reactive system problem expressed as a game structure:
\[G = \langle \gsV, \gsX, \gsY, \gsTheE, \gsTheS, \gsRhoE, \gsRhoS, \varphi \rangle\]
We want to show that an equi-realizable CLTS control problem $\mathcal{I}$  exists, starting with the strict realizability formula:
\[\varphi_G = (\theta_e \implies \theta_s) \wedge (\theta_e \implies \square((\boxdot \rho_e) \implies \rho_s)) \wedge (\theta_e \wedge \rho_e \implies \varphi_{FDS}) \]
and since the system wins $G$ iff there is a fairness-free FDS $\mathcal{D}$ (complete w.r.t. $\mathcal{X}$) s.t. $\mathcal{D} \models \varphi_G$ we can introduce the following translations:
\[E = clts(\varphi_G)\]
\[\mathcal{F} = fluents(\varphi_G)\]
\[\mathcal{C} = \sigma(\mathcal{Y})\]
\[C= ctrl(\mathcal{D},\varphi_G)\] 
\[\varphi_{CLTS} = fl(\varphi)\]
in order to prove that $\mathcal{I} = \langle E, \mathcal{C}, \mathcal{F}, \varphi_{CLTS} \rangle$ is an equi-realizable control problem w.r.t. $G$, i.e.: 
\[\mathcal{D}\models \varphi_G \iff E \parallel C \models \varphi_{CLTS}\]
Where $C$ is a legal CLTS w.r.t. $E$ and $\mathcal{C}$.
\newpage
\subsubsection{GS $\rightarrow$ CLTS}
\input{gs_to_clts}

%\newpage
%\subsubsection{Mixed Environments}
%
\begin{definition}
	\label{def:mixed_env} \emph{(Mixed Environment Definition)} 
	A mixed environment $D = \langle E, G \rangle$ is defined over both CTLS instances and game structures as follows,
	let $E_1,\ldots,E_n$ be a set of CLTS automata and $G_1,\ldots,G_m$ a set of game structures,
	then each component is defined as $G=\langle \mathcal{V}, \mathcal{X}, \cap_{i \in 1..m}\theta_{e_i}, \cap_{i \in 1..m}\theta_{s_i},$ $\cap_{i \in 1..m}\rho_{e_i},$ $\cap_{i \in 1..m}\rho_{s_i}\rangle$ and $E = \parallel_{*,i \in 1..n} E_i$, where $\parallel_*$ stands for either the asynchronous or synchronous composition semantic.
	The composed behavior of $D$ is defined by the automaton $E \parallel_* clts(G)$, thus $\langle E, G \rangle \models \varphi$ means $(E \parallel_* clts(G)) \models \varphi$.
\end{definition}


\begin{definition}
	\label{def:mixed_control_problem} \emph{(Mixed Control Problem)} 
	Let $D= \langle E, G \rangle$ be a mixed environment, if $\mathcal{F}$ is a set of fluents over $\Sigma$ and $\varphi$ an LTL formula over $\mathcal{F}$, then $I = \langle E, G, \mathcal{F}, \varphi \rangle$ conforms a mixed control problem. If $I$ is consistent and a solution to $I$ exists, such a solution will be a CLTS $M$, legal w.r.t. $D$, such that, $D \parallel M \models \varphi$.
\end{definition}

\begin{definition}
	\label{def:consistent_mixed_control_problem} \emph{(Consistent Mixed Control Problem)} 
	Let $I = \langle E, G, \mathcal{F}, \varphi \rangle$ be a mixed control problem, $I$ is consistent if $\forall v \in \mathcal{V}: v\uparrow, v\downarrow \in \Sigma$, let
	$\mathcal{Y} = \mathcal{V}\setminus \mathcal{X}$, $v \in \mathcal{Y}$ implies $v\uparrow, v\downarrow \in \mathcal{C}$, $\dot{v} = \lbrace v\uparrow, v\downarrow \rbrace \in \mathcal{F}$ and both $E$ and $clts(G)$ are consistent w.r.t. $\mathcal{F}$.
\end{definition}

\newpage
\subsubsection{CLTS $\rightarrow$ GS}

We introduce now a translation from CLTS to LTL based reactive systems specifications. We will note $\Sigma \setminus \controlSet$ as $\nonControlSet$. In order to show the relation between a LTL based reactive system specification and a CLTS control problem we first introduce a set of embeddings, then present and prove a realizability preserving theorem between the original control problem and its embedding into a reactive system specification.


In order to embed the CLTS control problem $\controlProblem$ into a LTL reactive system specification, the LTL formula over atomic propositions is translated simply by replacing each occurrence of an atomic proposition $\proposition \in \propositions$ with the boolean variable variable $v_{\proposition}$.
%\begin{definition}
%	\label{def:val_ltl} \emph{(LTL over $\propositions$ to LTL over $\mathcal{V}$ translation)} 
%	Let $\varphi$ be a LTL formula over set of atomic propositions $\propositions$, and $\mathcal{V}$ the minimal set of boolean variables containing a variable $v_{\proposition}$ for each $\proposition \in \propositions$, then $val(\varphi)$ is an equivalent LTL formula over boolean variables and is defined as follows:\\
%	
%	\begin{tabular}{ l c l }
%		$val(\proposition)$ & $\triangleq$ & $v_{\proposition}$\\	
%		$val(\neg \varphi)$ & $\triangleq$ & $\neg val(\varphi)$\\
%		$val(\varphi \vee \psi$ & $\triangleq$ & $val(\varphi) \vee val(\psi)$\\
%		$val(\bigcirc \varphi)$ & $\triangleq$ & $\bigcirc val(\varphi)$\\
%		$val(\varphi \U \psi)$ & $\triangleq$ & $val(\varphi) \U val(\psi)$\\
%	\end{tabular}	
%\end{definition}

%\begin{definition}
%	\label{def:clts_to_gs_translation} \emph{(CLTS to Game Structure translation)} 
%	Let $A = \langle S_A, \Sigma_A, \Delta_A, s_{0_A} \rangle$ be a clts, and $\mathcal{C}\subseteq \Sigma_A$ its controllable alphabet, $gs(A,\mathcal{C})=B$ is a game structure such that $A \models \varphi$ $\iff$ $B \models val(\varphi)$.
%\end{definition}
%
%Let $A$ be the CLTS to be translated and $B =  \langle \gsX, \gsY, \gsTheE, \gsTheS, \gsRhoE, \gsRhoS, \varphi \rangle$ the game structure resulting after applying $gs(A)$. 
Next we present the decomposition of the plant $E$ as a set of initial and transition formulae.
The alphabet of the translation will be comprised of two sets:
\[\gsX = \{x_i : c_i \in \nonControlSet \}\]
\[\gsY = \{y_i : u_i \in \controlSet \} \cup \{\varState{j} : s_j \in S_A \} \cup \{v_{\proposition} : \proposition \in \propositions\}\]
In the previous sets, $x_i$ is a boolean variable representing the occurrence of an environmental action, $y_i$ is a boolean variable representing the occurrence of a system action, $\varState{j}$ represents the fact that the plant is currently on state $j$ and $v_{\proposition}$ indicates that state $j$ satisfies proposition $\proposition$, the latest will be called natural variables in the context of this embedding.
%From now on we will write $s(j)$ to represent the valuation of the system variables representing a given state $j$, i.e.: 
%\[s(j) = \bigwedge\limits^{\lfloor log_2(|S_A|)\rfloor - 1}_{i=0}  s_{bit}(i,j)\]
%\[
%s_{bit}(i,j) = \begin{cases}
%s_i & j / 2^i \\
%\neg s_i & otherwise
%\end{cases}
%\]
To embed satisfaction of $\proposition$ at state $i$ we use $v(\proposition,i)$:
\[
v(\proposition,i) = \begin{cases}
v_{\proposition} & \proposition \in \valuations(s_i) \\
\neg v_{\proposition} & otherwise
\end{cases}
\]
We will use the following two functions to build the conjunction of system or environment variables related to the label in any given automaton transition (note that here $l$ is a label composed of several actions or events):
\[
var(l, l_i) = \begin{cases}
\varLabel{i} & l_i \in l \\
\neg \varLabel{i} & otherwise
\end{cases}
\]
\[label_e(l) = \{\bigwedge_{l_i \in \nonControlSet}var(l, l_i)\]
\[label_s(l) = \{\bigwedge_{l_i \in \Sigma}var(l, l_i)\}\]
And the enabling formula, which is a mutually exclusive condition over label variables for any given state $s_i$ is described as (note again that $L$ is a set of labels enabled for a particular state, each one consisting of several actions or events):
\[labels_e(L) = \bigvee_{l \in L}(label_e(l) \wedge \bigwedge_{l' \in L \setminus \{l\}}(\neg label_e(l')) ) \]
\[labels_s(L) = \bigvee_{l \in L}(label_s(l) \wedge \bigwedge_{l' \in L \setminus \{l\}}(\neg label_s(l')) ) \]
The initial condition formula will be defined as:
\[\gsTheE = \bigvee_{(s_0,l,s_j) \in \Delta_A}labels_e(l)\]
\[\gsTheS = (\varState{0} \wedge\bigwedge_{s_j \neq s_0, s_q \in S}\neg \varState{j})  \wedge \bigvee_{(s_0,l,s_j) \in \Delta_A}labels_s(l) \wedge \bigwedge_{\proposition \in \propositions}v(\proposition,0)\]
\[\theta = \gsTheE \wedge \gsTheS\]
Transition relations are constructed as a conjunction of safety formulae that are defined for each state. These can be split into two types, enabling and updating formulae. The first type restrict the set of variables that can be set to $true$ in a mutually exclusive fashion at each state, representing the label of the transition picked from any given state. The second type defines the state reached for each combination of current state and enabled natural variables. 

Then we can build the transition relations $\gsRhoE$ and $\gsRhoS$ from $\Delta_A$ as a conjunction of the enabling and transition formulae:

\vspace{1em}
\begin{tabular}{ l c l }
	$\rho_{env.enabling}$ &$=$& $\bigwedge_{s_i \in S_A} \varState{i} \implies \bigvee_{(s_i,l,s_j) \in \Delta_A}labels_e(l)$\\
	$\rho_{sys.enabling}$ &$=$& $\bigwedge_{s_i \in S_A} \varState{i} \implies \bigvee_{(s_i,l,s_j) \in \Delta_A}labels_s(l)$\\
	&&\\
	$\rho_{update.states}$&$=$&$\bigwedge_{(s_i,l,s_j) \in \Delta_A} (\varState{i} \wedge label_s(l) \implies \bigcirc(\varState{j} \wedge\bigwedge_{s_q \neq s_j, s_q \in S}\neg \varState{q}) ) $\\
	$\rho_{update.propositions}$&$=$&$\bigwedge_{(s_i,l,s_j) \in \Delta_A} (\varState{i} \wedge label_s(l) \implies \bigcirc(\bigwedge_{\proposition \in \propositions}v(\proposition,j) ) )$\\
	&&\\
	$\gsRhoS$&$=$&$\rho_{sys.enabling} \wedge  \rho_{update.states} \wedge  \rho_{update.propositions}$\\	
\end{tabular}
\vspace{1em}

The first formula, $\rho_{env.enabling}$ restricts the set of environment label variables that can be enabled for any given state, $\rho_{sys.enabling}$ does the same for system label variables, $\rho_{update.states}$ defines the next state according to the current one and the selected label, and $\rho_{update.propositions}$ updates the natural variables according to the next state, $\gsRhoS$ is the system transition formula, defined as a conjunction of the previous ones.
%
%Now we have the following mapping:
%
%\[\xymatrix@C+1pc{
%	\langle E = \langle S, \Sigma, \Delta, s_0 \rangle, \mathcal{C} \rangle \ar@{<->}[d]^{clts}_{gs}
%	& \varphi_{CLTS}\ar@{<->}[d]^{fl}_{val}
%	& C\ar@{<->}[d]^{clts}_{gs}
%	& \exists C: E\parallel C \models  \varphi_{CLTS}\ar@{<=>}[d]\\
%	G = \langle \mathcal{X},\mathcal{Y},\theta_e,\theta_s,\rho_e,\rho_s, \emptyset \rangle
%	& \varphi_G
%	& D
%	&\exists D: D \models \varphi_G\\
%}\]

%The following will be used to later define an FDS:
%\[\theta = \theta_e \wedge \theta_s, \rho = \rho_e \wedge \rho_s \]

The LTL equivalent formula to control problem \controlProblemDef can be defined as:

\[ \varphiLtl \]

Now we can show that realizability is preserved between CLTS control problem $\controlProblem$ and embedding $ltl(\controlProblem)$.

\begin{theorem}(\emph{$ltl$ preserves realizability})\label{theorem:gs_preserves_realizability}\\
	Let \controlProblemDef be a CLTS control problem with \cltsDef{E} and $\Sigma = \mathcal{C}\uplus \mathcal{U}$ then if $ltl(\controlProblem) = \langle \varphi_{ltl}, \gsX, \gsY\rangle$:
	\small
	\[(\exists D:FDS|D= \langle \mathcal{V}_d, \theta_d, \rho_d, \mathcal{J}_d, \mathcal{C}_d\rangle, \text{s.t. }    \gsX, \gsY \in \mathcal{V}_d, D \text{ is fairness-free, complete w.r.t. } \gsX \text{: } D \models \varphi_{ltl}\]
	% \wedge \exists \sigma \in D: \sigma \models \theta_e \wedge \square \rho_e) \]
	 \[ \iff (\exists C:CLTS| C \text{ legal w.r.t. } E,\mathcal{C} \text{: } C \parallel E \text{ is deadlock-free and } C \parallel E \models \varphi)  \]
	\normalsize
\end{theorem}

%, with an additional condition for LTL based reactive systems ($\sigma \models \theta_d \wedge \square \rho_d$) ruling out the cases where $val(\varphi)$ is never satisfied, since $\varphi_{LTL}$ could be realized by achieving $\neg\theta \vee \diamond \neg \rho$. In such a case no controller can be built for $\mathcal{I}$ s.t. $\varphi$ is satisfied since it will violate the safety restrictions related to the behavior of the plant $E_G$.

The proof is split in two parts, first we prove that if a controller exists for $\controlProblem$, a DFS exists for $\varphi_{LTL}$. Then we prove the opposite direction. For the ($\controlProblem \rightarrow \varphi_{LTL}$) step we construct a FDS from controller $C$ through a specific embedding $fds(C)$ and then prove that each property holds. For the ($\varphi_{LTL} \rightarrow \mathcal{I}$) we construct a controller from FDS $D$ through a specific embedding $ctrl(D)$ and then prove each property for this controller. The outline of the proof is as follows:

\begin{description}
	\item[($\controlProblem \rightarrow \varphi_{LTL}$)] Given \controlProblemDef and $ltl(\controlProblem) = \langle \varphi_{ltl}, \gsX, \gsY\rangle$
		\begin{itemize}
			\item $fds(C)$ is a fairness-free FDS			
			\item $fds(C)$ is complete w.r.t. $\mathcal{X}$
			\item $fds(C) \models \varphi_{LTL}$
		%	\item $\exists \sigma \in D: \sigma \models \theta_e \wedge \square \rho_e$			
		\end{itemize}
	\item[($\varphi_{LTL} \rightarrow \mathcal{I}$)] Given $\varphi_{ltl}, \gsX, \gsY$ and FDS $D$
		\begin{itemize}
			\item $ctrl(D)$ is a deadlock-free CLTS			
			\item $ctrl(D)$ is legal w.r.t. $E$ and $\mathcal{C}$
			\item $ctrl(D) \models \varphi$
		\end{itemize}	
\end{description}

First,  for the $\mathcal{I} \rightarrow \varphi_{LTL}$ step, $fds$ constructs FDS $D=\langle \mathcal V_d, \theta_d, \rho_d, \mathcal{J}_d, \mathcal{C}_d \rangle$ from a control problem $\controlProblem$ and controller $C$ using the formulas defined for $\varphi_{LTL}$, i.e. $\gsTheE, \gsTheS, \gsRhoE, \gsRhoS$, with the consideration that instead of using $E$ when building them we will use $C$ and the elements of the FDS are going to be defined as follows:

	\begin{tabular}{ l c l }
	$V_d$ & $=$ & $\mathcal{X} \cup \mathcal{Y}$\\	
	$\theta_d$ & $=$ & $(\theta_e \implies \theta_s)$\\
	$\rho_d$ & $=$ & $(\rho_e \implies \rho_s)$\\	
	$\mathcal{J}_d$ & $=$ & $\emptyset$\\
	$\mathcal{C}_d$ & $=$ & $\emptyset$\\
\end{tabular}

We can easily see that $fds(C)$ is fairness-free and complete w.r.t. $\mathcal{X}$ by construction, the first property is obvious, the second holds since no restriction is placed upon environment variables. Now, to prove that $fds(C)$ satisfies $\varphi_{LTL}$ we observe all paths in $D$ do, i.e.:

\[\forall \sigma \in D: \sigma \models \varphi_{LTL} \]

We split the proof in two again, first assuming $\sigma \not\models \theta_e \wedge \square \rho_e$, i.e.: $\sigma \models \neg\theta_e \vee \Diamond \neg\rho_e$, in this case $\varphi_{LTL}$ is trivially satisfied since all the precedents in the general conjunction terms are falsified. Now if $\sigma \models \theta_e \wedge \square \rho_e$ holds it should also satisfy $\theta_s$, $\square\rho_s$ and $val(\varphi)$. Since $\sigma$ is a play in $fds(C)$ then $\theta_d=(\theta_e \implies \theta_s)$ and $\rho_d=(\rho_e \implies \rho_s)$ hold, and since it also satisfies $\theta_e \wedge \square \rho_e$ we observe the following:
\[((\theta_e \implies \theta_s) \wedge \theta_e))\implies \theta_s\]
\[((\rho_e \implies \rho_s) \wedge \rho_e))\implies \rho_s\]
To show that $\sigma$ also satisfies $val(\varphi)$ we prove that for every play $\sigma$ in $fds(C)$ there is an execution $\execution$ in $C$ that conform to it.
For a play $\sigma \in 2^{|\mathcal{V}|^{\omega}}$ we define $exec(\sigma) \in S^{\omega}$ as follows, suppose that $\sigma = \sigma_1 \sigma_2 \ldots \sigma_i \ldots$ and
$exec(\sigma)= \execution_1 \execution_2 \ldots \execution_i \ldots$, then $\execution_i = s_j \iff \sigma_i \models \varState{j}$, to prove that such execution is possible in $C$ we observe that for each pair ($\sigma_i$,$\sigma_{i+1}$), since $\sigma \models \rho_e \wedge \rho_s$, and $\rho_s$ forces
plays to assign states according to transitions in $\Delta_C$, if $\sigma_i \models s_j$ then there exists a transition $(j, l, k)$ s.t.$\sigma_{i+1} \models s_k$ and $\sigma_{i+i} \models \delta_s(l) \wedge \bigwedge_{l' \in L \setminus \{l\}}(\neg \delta_s(l'))$. Then, for the base case we know that $\sigma_0 \models s_0$ by construction, therefore we proved that $exec(\sigma) \in C$.
Since for each play $\sigma$ that satisfies $\theta_e \wedge \square \rho_e$ we have a trace $\pi$ in controller $C$, and since $C \models \varphi$ we can show that $\sigma \models val(\varphi)$.


This is proved over the algebraic construction of $\varphi$, and we will only show the base case of atomic satisfaction, where
$exec(\sigma) \models \proposition$ implies $\sigma \models v_{\proposition}$. Again, for any given state $s_i \in exec(\sigma)$, satisfaction of the natural variable $v_{\proposition}$ will be preserved by  $\rho_{update.propositions}$ through $\bigwedge_{\proposition \in \propositions}v(\proposition,j)$.\\
\\

Now,  for the $\varphi_{LTL} \rightarrow \mathcal{I}$ direction, we propose an embedding $clts$ from an FDS into a CLTS that will preserve only the behavior that satisfies $\theta_e \wedge \square \rho_e$, since otherwise we will be keeping traces in $clts(D)$ that are not feasible in $clts(D)\parallel E$. We can show that even after removing the set of plays that satisfy $\neg \theta_e \vee \Diamond \rho_e$, a non empty controller exists, since this would contradict the existance of a complete w.r.t. $\gsX$ FDS $D$ that satifsfies $\varphi_{LTL}$, because the environment would be able to win by satisfying $\theta_e \wedge \square \rho_e$ while the system can not achieve $val(\varphi)$. After observing this and considering that  $D= \langle \mathcal{V}_d, \theta_d, \rho_d, \mathcal{J}_d, \mathcal{C}_d\rangle$ is a solution to $\varphi_{LTL}$, which in turn was built from \controlProblemDef,  we can present the embedding $clts$ from FDS into CLTS as \cltsEmbeddingDef{D} as follows :

\vspace{1em}
\begin{tabular}{ l c l }
	$S$ &$=$& $S_E$\\
	$\Sigma$ &$=$&$\Sigma_E$\\	
	$s_0$&$=$&$s_0^{E}$\\
	$\valuations$&$=$&$\valuations_E$\\
	&&\\
	$\Delta$&$=$&$\{(s_i,l,s_j)| \rho_d \models \varState{i} \wedge \bigcirc \varState{j} \wedge labels_e(l) \wedge labels_s(l) \}$\\
\end{tabular}
\vspace{1em}

%\newpage
%\subsubsection{GS $\rightarrow$ CLTS algorithm}
%
We can now introduce algorithm ~\ref{fig:gs_to_clts_algorithm}, that takes a game structure, represented as a set 
$\gsTheE$, $\gsTheS$, $\gsRhoE$, $\gsRhoS$, $\varphi$ of OBDDs, one for each formula, and produces the CLTS automaton $A$ that preserves its behavior. Since the state space is finite ($2^{|\gsV|}$ at most) and the number of states reached by applying $\gsRhoS(\gsRhoE(v))$ increases monotonically (when starting from $\gsTheS(\gsTheE)$), a least fixed point exists, the algorithm computes such fixed point while progressively creating the transition relation $\Delta$. \texttt{valuations}($\psi$) gives the set of valuations over $\gsV$ for a formula $\psi$ represented with an OBDD. Each valuation induces a unique state in $A$. \texttt{restrict}($\psi$, $s$) fixes a subset of variables in $\psi$ according to $s$ a valuation. If $\varphi$ is the set of propositional formulas involved in the winning condition, \texttt{evaluate\_condition}($A$, $\varphi$, $s$) evaluates the satisfaction of each such formulas according to valuation $s$ and maps the result to $s$ as a state in $A$.
\newpage
\renewcommand{\ttdefault}{pcr}
\begin{figure}[H]
\begin{lstlisting}[escapeinside={[*}{*]},basicstyle=\scriptsize\ttfamily,columns=flexible,frame=lines,mathescape=true,xleftmargin=3.0ex,keywordstyle=\textbf,morekeywords={if,while,do,else,fork,int,null, algorithm, is, input, output, return, for},numbers=left,numberstyle=\scriptsize]
algorithm gs_to_clts is
	input: $GS$ a set of OBDD structure representing LTL formulas
	output: $A$ a CLTS automaton equivalent to the game structure induced by $GS$
	
	$A$ $\leftarrow$ create_automaton($s_0$)
	$visited \leftarrow \{s_0\}$ 
	$frontier \leftarrow$ valuations($\gsTheE$)
	for $s \in frontier$:
		add_transition($A$, $s_0$, $s$)
		evaluate_condition($A$, $\varphi$, $s$)		
	$new\_frontier \leftarrow $ [] 
	for $s \in frontier$:
		add($visited$,$s$)
		$current \leftarrow$ valuations(restrict($\gsTheS$, $s$))
		for $s' \in current$:
			add_transition($A$, $s$, $s'$)
			evaluate_condition($A$, $\varphi$, $s'$)
			if $s' \not\in new\_frontier$:			
				add($new\_frontier$, $s'$)
	$frontier \leftarrow new\_frontier$
	while $frontier \neq$ []:
		while $frontier \neq$ []:		
			$s_{sys} \leftarrow$ pop($frontier$)
			if $s_{sys} \in frontier$:
				continue
			$current \leftarrow$ valuations(restrict($\gsRhoE$, $s_{sys}$))
			for $s' \in current$:
				add_transition($A$, $s_{sys}$, $s'$)
				evaluate_condition($A$, $\varphi$, $s'$)			
				if $s' \not\in new\_frontier$:			
					add($new\_frontier$, $s'$)				
		$frontier \leftarrow new\_frontier$		
		while $frontier \neq$ []:		
			$s_{env} \leftarrow$ pop($frontier$)
			if $s_{env} \in frontier$:
			continue
			$current \leftarrow$ valuations(restrict($\gsRhoS$, $s_{env}$))
			for $s' \in current$:
				add_transition($A$, $s_{env}$, $s'$)
				evaluate_condition($A$, $\varphi$, $s'$)			
				if $s' \not\in new\_frontier$:			
					add($new\_frontier$, $s'$)				
		$frontier \leftarrow new\_frontier$				
	return $A$
\end{lstlisting}
\caption{Game Structure to CLTS translation algorithm}
\label{fig:gs_to_clts_algorithm}
%%\vspace*{-4mm}
\MediumPicture
\end{figure}


\newpage
%%%%% CLTS 2 GS

