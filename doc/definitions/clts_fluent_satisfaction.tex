
We use linear temporal logics of fluents (FLTL) over CLTS models. %~\cite{DBLP:conf/sigsoft/GiannakopoulouM03}. 
A \emph{fluent} \emph{fl} is defined by a pair of sets and a Boolean value: $\emph{\fluent} = \langle I_{\emph{\fluent}}, T_{\emph{\fluent}}, \emph{Init}_{\emph{\fluent}} \rangle$, where $I_{\emph{\fluent}}\subseteq \mathcal{P}(Act)$ is the set of initiating actions, $T_{\emph{\fluent}} \mathcal{P}(Act)$ is the set of terminating actions and $I_{\emph{\fluent}}\cap T_{\emph{\fluent}}=\emptyset$. 
A fluent may be initially \true or \false as indicated by \emph{Init}$_{\emph{\fluent}}$. 
Every actions $\ell\in Act$ induces a fluent, namely $\fluentp{\ell}=\langle \{\ell\}, \{Act\setminus \set{\ell}\}, \false\rangle$. 
Finally, the alphabet of a fluent is the union of its terminating and initiating actions.

Let $\mathcal{F}$ be the set of all possible fluents over $\mathcal{P}(Act)$. 
An FLTL formula is defined inductively using the standard Boolean connectives and temporal operators $X$~(next), $U$ (strong until) as follows: 
$\varphi ::= \fluent \mid \neg \varphi \mid \varphi \vee \psi \mid \X \varphi \mid \varphi U \psi,$
where $\fluent\in\mathcal{F}$. 
As usual we introduce $\wedge$, $\F$ (eventually), and $\G$ (always) as syntactic sugar. 
Let $\Pi$ be the set of infinite traces over \emph{Act}.
The trace $\pi=\ell_0,\ell_1,\ldots$ satisfies a fluent $\emph{Fl}$ at position $i$, denoted $\pi,i \models \emph{Fl}$, if and only if one of the following conditions holds:
\begin{itemize}%{\leftmargin=3em}
	\item $\emph{Init}_{\emph{Fl}} \wedge (\forall j \in \mathbb{N} \cdot 0 \leq j \leq i \rightarrow \nexists t \in T_{\fluent}: t \subseteq \ell_j)$
	\item $\exists j \in \mathbb{N} \cdot (j \leq i \wedge \exists i \in I_{\fluent}: i \subseteq \ell_j) \wedge (\forall k \in \mathbb{N} \cdot j < k \leq i \ \rightarrow \nexists t \in T_{\fluent}: t \subseteq \ell_k)$
\end{itemize}

%Suppose that $\fluent \in \mathcal{F}$, in order to avoid having a label $l_i$ in our trace $\pi$ where both the initiation and termination of $\fluent$ are triggered, we will require that our model $E$ satisfies a notion of consistency with respect to a set of fluents, we say that a model $E$ is consistent with respect to a set of fluents $\mathcal{F}$, noted $\cong(E,\mathcal{F})$ if the following holds: let $\pi$ be a trace of $E$,
%$\forall \pi_i \in \pi, fl \in F, \neg(\exists i \in I_{fl}: (i \subseteq \pi_i) \wedge \exists t \in T_{fl}: (t \subseteq \pi_i))$.
\textcolor{blue}{Suppose that} $\fluent \in \mathcal{F}$, in order to avoid having a label $l_i$ in our trace $\pi$ where both the initiation and termination of $\fluent$ are triggered, we will require that our model $E$ satisfies a notion of consistency with respect to a set of fluents, we say that a model $E$ is consistent with respect to a set of fluents $\mathcal{F}$, noted $\cong(E,\mathcal{F})$ if the following safety condition holds: 
\[\forall \fluent \in \mathcal{F}, (s,l,s') \in \Delta_E: \square(\neg(\exists i \in I_{\fluent}: i \subseteq l) \vee \neg(\exists t \in T_{\fluent}: t \subseteq l))\]



Given an infinite trace $\pi$, the satisfaction of a formula $\varphi$ at position $i$, denoted $\pi,i\models\varphi$, is defined as follows:

\begin{tabular}{ l c l }
$\pi, i \models_d \neg \varphi$ & $\triangleq$ & $\pi, i \not\models_d \varphi$\\
$\pi, i \models_d \varphi \vee \psi$ & $\triangleq$ & $(\pi, i \models_d \varphi) \vee (\pi, i \models_d \psi)$\\
$\pi, i \models_d \X \varphi$ & $\triangleq$ & $\pi, i +1 \models_d \varphi$\\
$\pi, i \models_d \varphi \U \psi$ & $\triangleq$ & $\exists j \geq i . \pi,j \models_d \psi \wedge \forall i \leq k \le k. \pi, k \models_d \varphi$\\
\end{tabular}
  
We say that $\varphi$ holds in $\pi$, denoted $\pi\models\varphi$, if $\pi,0\models\varphi$. 
A formula $\varphi \in \mbox{FLTL}$ holds in an CLTS $E$ (denoted $E \models \varphi$) if it holds on every infinite trace produced by $E$.
A \emph{fluent} \fluent \space is defined by a set of initiating actions $I_{\fluent}$, a set of terminating actions $T_{\fluent}$, and an initial value \emph{Initially}$_{\fluent}$.

That is,
%\begin{equation*}
$ \fluent = \langle I_{\fluent}, T_{\fluent} \rangle_{\emph{initially}_{\fluent}} $, 
where 
$I_{\fluent},T_{\fluent} \subseteq \mathcal{P}(\emph{Act})$ 
and $I_{\fluent} \cap T_{\fluent} = \emptyset.$\\
%\end{equation*}
%A fluent may be initially \true or \false as indicated by the \emph{Initially}$_{\fluent}$ attribute,
%When we omit \emph{Initially}$_{\fluent}$, we assume the fluent is
%initially \emph{false}. We use $\fluentp{\ell}$ as short for the
%fluent defined as $\fluent = \langle \ell,
%\emph{Act}\setminus\set{\ell} \rangle$.
%Given the set of fluents $\Phi$, an FLTL formula is defined
%inductively using the standard boolean connectives and temporal
%operators $\mathbf{X}$ (next), $\mathbf{U}$ (strong until) as
%follows:\\
%%\begin{equation*}
%$\varphi ::= \fluent \mid \neg \varphi \mid \varphi \vee \psi \mid
%\mathbf{X} \varphi \mid \varphi \mathbf{U} \psi$,
%%\end{equation*}
%where $\fluent \in \Phi$.
