
To show that the CLTS and FDS control problems are equivalent we  still need to show that an FDS control problem can be translated into an equivalent CLTS control problem. We will again introduce a set of embeddings that constitute the translation and then state and prove a realizability preserving theorem.

In order to embed the FDS control problem \fdsControlProblemDef into the CLTS control problem \cltsCPEmbeddingDef, the first thing to note is that the LTL formula over Boolean variables $\varphi$ is translated as $\varphiCLTS$ by replacing each occurrence of an Boolean variables $v \in \gsV$ with the proposition $\dot{v} \in \propositions$, then a next operator is added enclosing the translated formula in order to account for the initial transition that goes from the distinguished state $s_0$ to all the initial states that conform to the original specification ($\varphiCLTS = \bigcirc \dot{\varphi}$).

%\begin{definition}
%	\label{def:val_ltl} \emph{(LTL over $\propositions$ to LTL over $\mathcal{V}$ translation)} 
%	Let $\varphi$ be a LTL formula over set of atomic propositions $\propositions$, and $\mathcal{V}$ the minimal set of boolean variables containing a variable $v_{\proposition}$ for each $\proposition \in \propositions$, then $val(\varphi)$ is an equivalent LTL formula over boolean variables and is defined as follows:\\
%	
%	\begin{tabular}{ l c l }
%		$val(\proposition)$ & $\triangleq$ & $v_{\proposition}$\\	
%		$val(\neg \varphi)$ & $\triangleq$ & $\neg val(\varphi)$\\
%		$val(\varphi \vee \psi$ & $\triangleq$ & $val(\varphi) \vee val(\psi)$\\
%		$val(\bigcirc \varphi)$ & $\triangleq$ & $\bigcirc val(\varphi)$\\
%		$val(\varphi \U \psi)$ & $\triangleq$ & $val(\varphi) \U val(\psi)$\\
%	\end{tabular}	
%\end{definition}

%\begin{definition}
%	\label{def:clts_to_gs_translation} \emph{(CLTS to Game Structure translation)} 
%	Let $A = \langle S_A, \Sigma_A, \Delta_A, s_{0_A} \rangle$ be a clts, and $\mathcal{C}\subseteq \Sigma_A$ its controllable alphabet, $gs(A,\mathcal{C})=B$ is a game structure such that $A \models \varphi$ $\iff$ $B \models val(\varphi)$.
%\end{definition}
%
%Let $A$ be the CLTS to be translated and $B =  \langle \gsX, \gsY, \gsTheE, \gsTheS, \gsRhoE, \gsRhoS, \varphi \rangle$ the game structure resulting after applying $gs(A)$. 
Let \fdsControlProblemDef, the alphabet of the embedding will be comprised of two sets, \controlSet and \nonControlSet, declared as follows:
\[\controlSet = \{y\uparrow | y \in \gsY \} \cup \{y\downarrow | y \in \gsY \}\]
\[\nonControlSet = \{x\uparrow | x \in \gsX \} \cup \{x\downarrow | x \in \gsX \}\]
\[\Sigma = \controlSet \cup \nonControlSet \]
For each variable $v$ in \fdsControlProblem we are adding two actions, $v\uparrow$ that indicates that the variable was set to $true$ when going to current state to the next and $v\downarrow$ added to the transitions from a state where $v$ held into one where it does not.
We will use \enumStateSetDef to relate a state in the FDS domain to one in the CLTS control problem, it assigns a unique integer to each possible valuation over \gsV.
Let \cltsDef be the CLTS describing the behavior of our problem, we can build it as follows:

\vspace{1em}
\begin{tabular}{ l c l }
	$S$ &$=$& $\{s_{i+1} : i \in [0\ldots2^{|\gsV|-1}] \}\cup \{s_0\}$\\
	$\Sigma$ &$=$&$\controlSet \cup \nonControlSet$\\	
	$\Delta$&$=$&$\{(s_i,\actionLabel,s_j) | s_i = \enumStateSet{\hat{s}_i} \wedge  s_j = \enumStateSet{\hat{s}_j} \wedge (\forall v \in \gsV: (v\uparrow \in \actionLabel \iff v \in \hat{s}_j \wedge v \not\in \hat{s}_i)$\\
	&&$\wedge (\forall v \in \gsV: (v\downarrow \in \actionLabel \iff v \not\in \hat{s}_j \wedge v \in \hat{s}_i))\}$\\
	&&$\cup \{ (s_0,\actionLabel,s_i)| s_i = \enumStateSet{\hat{s}_i} \wedge (\forall v \in \gsV: (v\uparrow \in \actionLabel \iff v \in \hat{s}_i)\}$\\
	$\propositions$&$=$&$\{\dot{v} | v \in \gsV\}$\\
	$\valuations(s)$&$=$&$\{\dot{v}| v \in \hat{s} \wedge s = \enumStateSet{\hat{s}}\}$\\
\end{tabular}
\vspace{1em}
\\
With $E$ we are allowing the whole domain of $\gsV \times \gsV'$ to be executed, expecting the CLTS controller that is a solution to \cltsCPEmbedding to to leave only the traces that are conforming to runs on \fdsD the FDS controller that is a solution to \fdsControlProblem. $S$ is the set of states to be related through \enumStateSetDef to valuations over \gsV while adding a distinguished initial state $s_0$ that allows $E$ to emulate the existence of multiple initial states by adding transitions from $s_0$ to those states that relate to valuations satisfying $\theta$. A proposition $\dot{v}$ is added for each variable $v$ so that the valuations over states in $S$ conform to those in $\Sigma_{\gsV}$. The transition relation imposes no restriction containing all possible pairs of states other than the ones going back to $s_0$.

Now we can prove that realizability is preserved between FDS control problem $\fdsControlProblem$ and the embedding \cltsCPEmbedding.

\begin{theorem}(\emph{$clts$ preserves realizability})\label{theorem:clts_preserves_realizability}\\
	Let \fdsControlProblemDef be an FDS control problem  then if \cltsCPEmbeddingDef, there is a CLTS $\fdsD= \langle \mathcal{V}_d, \theta_d,$ $\rho_d,$ $\mathcal{J}_d,$ $\mathcal{C}_d\rangle$, s.t. $\gsX \cup \gsY \subseteq \mathcal{V}_d$, $D$ is fairness-free, complete w.r.t. \gsX that satisfies $\fdsD \models \varphi$
	if and only if there is a CLTS $C$ legal w.r.t. ($E,\mathcal{C}$) s.t. $C \parallel E$ is deadlock-free and $C \parallel E \models \varphiCLTS$.
	\normalsize
\end{theorem}

%, with an additional condition for LTL based reactive systems ($\sigma \models \theta_d \wedge \square \rho_d$) ruling out the cases where $val(\varphi)$ is never satisfied, since $\varphi_{LTL}$ could be realized by achieving $\neg\theta \vee \diamond \neg \rho$. In such a case no controller can be built for $\mathcal{I}$ s.t. $\varphi$ is satisfied since it will violate the safety restrictions related to the behavior of the plant $E_G$.

The proof is split in two parts, first we prove that if an FDS \fdsD controller exists for $\fdsControlProblem$, a CLTS controller $C$ exists for $\cltsCPEmbedding$($\exists \fdsD \rightarrow \exists C$) by constructing an CLTS from controller \fdsD through a specific embedding $ctrl'(\fdsD)$ and then proving that each property holds. For the case where if a CLTS controller $C$ exists for the embedding \cltsCPEmbedding there must exist an FDS controller for \fdsControlProblem ($\exists C \rightarrow \exists \fdsD$) we construct an FDS controller from $C$ through a specific embedding, similar to the one used in the previous section, and then prove each property for it. The outline of the proof is as follows:


\begin{description}
	\item[($\exists \fdsD \rightarrow \exists C$)] Given $\fdsControlProblemDef$ and FDS $\fdsD$
		\begin{itemize}
			\item $ctrl'(\fdsD)$ is legal w.r.t. $E$ and $\mathcal{C}$
			\item $ctrl'(\fdsD)$ is a deadlock-free CLTS			
			\item $ctrl'(\fdsD) \models \varphiCLTS$
		\end{itemize}	
	\item[($\exists C \rightarrow \exists \fdsD$)] Given \cltsCPEmbeddingDef and $C$
\begin{itemize}
	\item $fds'(C)$ is a fairness-free FDS			
	\item $fds'(C)$ is complete w.r.t. $\mathcal{X}$
	\item $fds'(C) \models \varphi$
	%	\item $\exists \sigma \in D: \sigma \models \theta_e \wedge \square \rho_e$			
\end{itemize}	
\end{description}

First,  for the $\exists \fdsD \rightarrow \exists C$ step, $ctrl'$ constructs an embedding from an FDS into a CLTS. In the following definitions, let $\mathcal{M}= \mathcal{V}_d\setminus (\mathcal{X} \cup \mathcal{Y})$ be called the \emph{memory variables} of $\fdsD$ as before.
Again and following ~\cite{bloem2012synthesis} we will suppose that \fdsD is constructed from a strategy $f: M \times \Sigma$ $\times$ $\Sigma_{\gsX} \mapsto M \times \Sigma_{\gsY}$
that wins the game $G_{\varphi}=\langle \gsX, \gsY, true,$ $true, true,$ $true,$ $\varphi \rangle$. Considering that $\fdsD= \langle \mathcal{V}_d, \theta_d, \rho_d, \mathcal{J}_d, \mathcal{C}_d\rangle$ is a solution to $\varphi$, we can define the embedding from an FDS controller \fdsD into a CLTS controller ctrl'(\fdsD)  as:

\vspace{1em}
\begin{tabular}{ l c l }
	$S$ &$=$& $\{(s_{i+1},\hat{m}) : i \in [0\ldots2^{|\gsV|}-1] \in [0\ldots2^{|\mathcal{M}|}-1], \hat{m} \in [0\ldots2^{|\mathcal{M}|}-1] \}\cup\{s_0\}$\\
	$\Sigma$ &$=$&$\controlSet \cup \nonControlSet$\\	
	$\Delta$&$=$&$\{((s_i,\enumSet{\hat{s}_i|_{\mathcal{M}}}),\actionLabel,(s_j,\enumSet{\hat{s}_j|_{\mathcal{M}}})) | s_i = \enumStateSet{\hat{s}_i|_{\gsV}}$\\
	&&$\wedge  s_j = \enumStateSet{\hat{s}_j|_{\gsV}} \wedge (\hat{s}_i, \hat{s}_j) \models \rho_d$\\
	&&$\wedge (\forall v \in \gsV: (v\uparrow \in \actionLabel \iff v \in \hat{s}_j \wedge v \not\in \hat{s}_i)$\\
	&&$\wedge (\forall v \in \gsV: (v\downarrow \in \actionLabel \iff v \not\in \hat{s}_j \wedge v \in \hat{s}_i))\}$\\
	&&$\cup \{ (s_0,\actionLabel,(s_i,\enumSet{\hat{s}_i|_{\mathcal{M}}}))| s_i = \enumStateSet{\hat{s}_i|_{\gsV}} \wedge \hat{s}_i \models \theta_d$\\
	&&$\wedge (\forall v \in \gsV: (v\uparrow \in \actionLabel \iff v \in \hat{s}_i)\}$\\
	$\propositions$&$=$&$\{\dot{v} | v \in \gsV\}$\\
	$\valuations((s,\hat{m}))$&$=$&$\{\dot{v}| v \in \hat{s} \wedge s = \enumStateSet{\hat{s}}\}$\\
\end{tabular}
\vspace{1em}
\\
Where $S$ is the cross product between the range of values related to the set of valuations over \gsV and those induced by the memory of $\fdsD$ through $\#$, the sets of actions and atomic propositions remain unchanged from $E$, \valuations are evaluated over the first component of $S$ in order to get the set of propositions that are satisfied by the values of \gsV. The transition relation is defined by adding those CLTS states and labels whose FDS state and label variables are satisfied according to $\rho_d$ in $\fdsD$, setting also the second element of each state, i.e. its memory, according to the projection of the valuation over the memory domain $\mathcal{M}$. 

We can now show that $ctrl'(\fdsD)$ satisfies the properties that make it a solution to $\cltsCPEmbedding$. The first property we need to prove is legality w.r.t $E$ and $\mathcal{C}$, which means that for all $(s_e,s_c) \in E \parallel C$ it holds that:\\

\begin{center}
	\begin{tabular}{r l}
		
		$\forall L_{\nonControlSet}\text{ s.t. }(s_E, L, s'_E) \in \Delta_E \wedge L|_{\nonControlSet} = L_{\nonControlSet}:( \exists (s_C, L', s'_C) \in \Delta_C \wedge L'|_{\nonControlSet} = L_{\nonControlSet})$ &(legality openness)\\
		$\forall (s_C, L, s'_C) \in \Delta_C: ( \exists (s_E, L, s'_E) \in \Delta_E)$& (legality inclusion)\\
	\end{tabular}
\end{center}

Since $\fdsD$ is complete w.r.t. $\mathcal{X}$ and only accepts infinite runs there must exist a choice in $\rho$ to reach the next state, we can prove that all non controllable options from $E$ at each state are kept in $ctrl'(\fdsD)$, proving legality openness. Remember that $\Delta$ in $E$ had all the possible transitions between states related to valuations of \gsV, thus trivilly proving legality inclusion.
Now for the LTL formula $\varphiCLTS$, we prove that for every execution $\execution$ in $clts(\fdsD)$ there is a run $\sigma$ in $\fdsD$ that conforms to it. We use $\sigma(\execution) \in 2^{|\mathcal{V}|^{\omega}}$ as defined in the previous section.
Now, since for each execution $\sigma(\execution)$ in $clts(\fdsD)$ we have a run $\sigma$ in $\fdsD$, and since given that we evaluate only runs that follow $\theta$ and $\rho$, then $\fdsD$ satisfies $\varphi$ and we can show that $\sigma(\execution)$ satisfies $\varphiCLTS$ since the satisfaction of the atomic propositions is defined completely by the embedding of states $\varState{i} \in \sigma$ and $s_i \in \execution(\sigma)$.

%%
Now given \cltsCPEmbeddingDef and $C$ we build an FDS $\fdsD=\langle \mathcal V_d, \theta_d, \rho_d, \mathcal{J}_d, \mathcal{C}_d \rangle$ through $fds'(C)$, where $C=\langle S_C,$ $\Sigma_C,$ $\Delta_C,$ $s_0^C,$ $\propositions_C,$ $\valuations_C\rangle$ as:

\begin{tabular}{ l c l }
	$\gsV_d$ & $=$ & $\gsX \cup \gsY \cup \{\dot{s}| s \in S_C\}$\\	
	$\theta_d$ & $=$ & $\bigvee_{s,\exists \actionLabel:(s_0^C,\actionLabel,s)\in\Delta_C}(\bigwedge\{ v | \dot{v}\in \valuations_C(s)\}\wedge mux'(\dot{s}) )$\\
	$\rho_d$ & $=$ &$\bigwedge_{s,\exists \actionLabel,s':(s,\actionLabel,s')\in \Delta_C, s\neq s_0^C}\dot{s}\implies\bigvee_{s':\exists \actionLabel':(s,\actionLabel',s')\in \Delta_C}(\bigwedge\{v | \dot{v}\in \valuations_C(s') \}\wedge mux'(\dot{s}'))$\\	
	$\mathcal{J}_d$ & $=$ & $\emptyset$\\
	$\mathcal{C}_d$ & $=$ & $\emptyset$\\
\end{tabular}

We are using a slightly different version of the mutual exclusion formula for the state variables in $fds'$:

\begin{center}
	\begin{tabular}{r r r l}
		$mux':$&$\gsV_d \mapsto WFF(\gsV_d),$&$mux'(\dot{s}_i) = $&$(\dot{s}_i \wedge\bigwedge_{s_j \neq s_i, s_j \in S_C}\neg \dot{s}_j)$
	\end{tabular}
\end{center}

Where $\gsV_d$ is the extension of the original variables with the a set of variables $\dot{s}$ for each state in the CLTS controller. Both the initial and transition formulae are built in an extensive fashion, restricting the set of states in $\Sigma_{\gsV_d}$ or pair of states respectively that are to take part in the FDS controller. The initial formula $\theta_d$ is built as the disjunctive extension of conjunctions of original valuations (as kept in $\valuations_C$) plus the mutual exclusion over the state variables that keep up only that related to the current state in the CLTS controller. The transition formula $\rho_d$ is a conjunction of implications, where the precedent is the occurrence of a state variables and what it is implied is the set of variables that can hold in the next state, as a disjunctive conjunction that includes the original variables (as kept in $\valuations_C$) and the ones conforming to states in $C$.

%fairness free is trivial
%complete w.r.t. X comes from the fact that E kept all possible valuations over X and C must be legal, thus not restricting X choices
The fairness-free property holds by construction. The resulting FDS is complete w.r.t \gsX first because $C$ is legal w.r.t. $E$ and \controlSet, which means that it wont restrict environment options, and second by $E$ being non restrictive with respect to valuations over \gsV. This implies that any restriction imposed by $C$ has to be over \gsY in a completely open environment.
To show that $fds(C)$ satisfies $\varphi$  we prove that for every run $\sigma$ in $fds(C)$ there is an execution $\execution$ in $C$ that conforms to it using $exec(\sigma) \in S^{\omega}$ once again. Such execution is possible in $C$ because \fdsD fully emulates its behavior, for each transition $(s,\actionLabel, s')\in \Delta_C$ there is pair of states $(\dot{s}_i,\dot{s}_j) \in \rho_d$ s.t. $\dot{v} \in \valuations_C(s) \iff v \in \dot{s}_i|_{\gsV}$ and $\dot{v}' \in \valuations_C(s') \iff v' \in \dot{s}_j|_{\gsV}$ .
A trace $\pi$ in controller $C$ exists for each run $\sigma$ that satisfies $\theta_d \wedge \square \rho_d$, and since $C \models \varphiCLTS$ we can show that $\sigma \models \varphi$. This is proved over the algebraic construction of $\varphi$ as before.
