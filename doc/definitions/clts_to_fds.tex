
To show that the CLTS and FDS control problems are equivalent we  present a translation that preserves realizability, starting with the direction from the CLTS into the FDS domain first. We will start by introducing a set of embeddings that constitute the translation and then state and prove a realizability preserving theorem.


In order to embed the CLTS control problem \controlProblemDef into the FDS control problem \fdsEmbeddingDef, the first thing to note is that the LTL formula over atomic propositions $\varphi$ is translated as $val(\varphi)$ by replacing each occurrence of an atomic proposition $\proposition \in \propositions$ with the boolean variable $v_{\proposition}$. Remember that we denote $\Sigma \setminus \controlSet$ as $\nonControlSet$. 
%\begin{definition}
%	\label{def:val_ltl} \emph{(LTL over $\propositions$ to LTL over $\mathcal{V}$ translation)} 
%	Let $\varphi$ be a LTL formula over set of atomic propositions $\propositions$, and $\mathcal{V}$ the minimal set of boolean variables containing a variable $v_{\proposition}$ for each $\proposition \in \propositions$, then $val(\varphi)$ is an equivalent LTL formula over boolean variables and is defined as follows:\\
%	
%	\begin{tabular}{ l c l }
%		$val(\proposition)$ & $\triangleq$ & $v_{\proposition}$\\	
%		$val(\neg \varphi)$ & $\triangleq$ & $\neg val(\varphi)$\\
%		$val(\varphi \vee \psi$ & $\triangleq$ & $val(\varphi) \vee val(\psi)$\\
%		$val(\bigcirc \varphi)$ & $\triangleq$ & $\bigcirc val(\varphi)$\\
%		$val(\varphi \U \psi)$ & $\triangleq$ & $val(\varphi) \U val(\psi)$\\
%	\end{tabular}	
%\end{definition}

%\begin{definition}
%	\label{def:clts_to_gs_translation} \emph{(CLTS to Game Structure translation)} 
%	Let $A = \langle S_A, \Sigma_A, \Delta_A, s_{0_A} \rangle$ be a clts, and $\mathcal{C}\subseteq \Sigma_A$ its controllable alphabet, $gs(A,\mathcal{C})=B$ is a game structure such that $A \models \varphi$ $\iff$ $B \models val(\varphi)$.
%\end{definition}
%
%Let $A$ be the CLTS to be translated and $B =  \langle \gsX, \gsY, \gsTheE, \gsTheS, \gsRhoE, \gsRhoS, \varphi \rangle$ the game structure resulting after applying $gs(A)$. 
Let \cltsDef, the variables of the embedding will be comprised of two sets, \gsX and \gsY, declared as follows:
\[\gsX = \{\hat{x}_i : x_i \in \nonControlSet \}\]
\[\gsY_c = \{\hat{y}_i : y_i \in \controlSet \}, \gsY_p = \{v_{\proposition} : \proposition \in \propositions\}, \gsY_s = \{\varState{j} : s_j \in S \}\]
\[\gsY = \gsY_c \cup \gsY_p \cup \gsY_s \]
Here $\varState{j}$ being set to $true$ at state $\sigma_i$ of a run $\sigma$ in the translation relates it to the execution \execution over $E$ where $\execution_i$ is at state $j$, $\hat{x}_i$ is a Boolean variable representing the occurrence of a monitored action leading to the current state, i.e. if $\varState{q}$ and $\hat{x}_i$ are up at state $\sigma_i$ and $\varState{p}$ is up at $\sigma_{i-1}$, then it can be read as \emph{$x_i$ was part of the label of the transition taken when moving from $s_p$ to $s_q$ at $E$}, in a similar way, $\hat{y}_i$ is a Boolean variable representing the occurrence of a controllable action leading to the current state. Finally, $v_{\proposition}$ indicates that state $j$ satisfies proposition $\proposition$. In the context of this embedding, $\hat{x}_i$ and $\hat{y}_i$ will be called \emph{label variables}, $\varState{j}$ will be called a \emph{state variable} and $v_{\proposition}$ will be called a \emph{natural variable}. The information modeled at each state $v$ in the embedding \fdsEmbedding will be comprised of the \emph{state variable} $\varState{i}$ defining to which explicit state $s_i$ in the automaton $E$ $v$ is related, the \emph{natural variables} defining which propositions hold at $s_i$ and the \emph{label variables} ($\hat{x}_i^1\ldots \hat{x}_i^n \hat{y}_i^1\ldots \hat{y}_i^m$) defining the set of actions ($x_i^1\ldots x_i^n y_i^1\ldots y_i^m$) that constitute the label of the transition taken when moving from $s_{i-1}$ to $s_i$.

The possible behavior of $E$ is emulated by the pair of initial ($\theta$) and transition ($\rho$) formulae, the latter being expressed as a conjunction of smaller terms in order to clarify its construction. For $\theta$ and $\rho$ we write the restriction for environment first ($\theta_e$,$\theta_s$), and then the one for the system ($\rho_e$,$\rho_s$).  This is needed in order to allow the solution to be complete w.r.t \gsX, which is the way to account for the non controllability or openness of the problem.  The way in which this is achieved will become clear soon.  
As a brief side note, a somewhat similar translation is presented in ~\cite{bloem2012synthesis} where, while introducing the AMBA AHB case study, the authors replace the automaton that captures the assumption that an unspecified length locked burst, leaving \emph{HBUSREQ[i]} high locks the bus, with a set of LTL formulae. Auxiliary system variables are added to specification to represent the states of the automaton. These are also used as Büchi acceptance condition, which are easily included in the GR(1) formula by adding the conjunction of accepting states to the right side of the implication.
Two main differences between their translation and ours are, first,  that we restrict the occurrence of certain combination of label variables to enable only those that are conforming to an existing label in the original automaton and second, that we add additional system variables to add explicit valuations in our states following the definition of \valuations.
%From now on we will write $s(j)$ to represent the valuation of the system variables representing a given state $j$, i.e.: 
%\[s(j) = \bigwedge\limits^{\lfloor log_2(|S_A|)\rfloor - 1}_{i=0}  s_{bit}(i,j)\]
%\[
%s_{bit}(i,j) = \begin{cases}
%s_i & j / 2^i \\
%\neg s_i & otherwise
%\end{cases}
%\]

The translation is now presented in a bottom-up fashion, introducing first the translation of each element, to then compose the smaller formulae into a single one that captures both the behavior of $E$ and the property $\varphi$ that we want to evaluate. In order to embed satisfaction of proposition $\proposition$ at state $i$ we use $v$:
\begin{center}
	\begin{tabular}{r r r l}
$v:$ & $\propositions \times S \mapsto \gsY,$
$
v(\proposition,s_i) = $ & $\begin{cases}
v_{\proposition} & \proposition \in \valuations(s_i) \\
\neg v_{\proposition} & otherwise
\end{cases}
$
\end{tabular}
\end{center}
We will use the following three functions ($var$, $labels_e$ and $labels_s$) to build the conjunction of system or environment variables that will represent the occurrence of a set of actions as described by the labels of the transitions in $E$.
The enabling formula, which is a mutually exclusive condition over label variables for any given state $s_i$ that restricts the set of choices as the transition relation $\Delta$ does in $E$, is constructed through $labels_e$ and $labels_s$ as (note that here that \labelSet is a set of labels \actionLabel enabled for a particular state and \actionLabel is a label composed of several actions \action ):
%\[labels_e(\labelSet) = \bigvee_{\actionLabel \in \labelSet}(label_e(\actionLabel) \wedge \bigwedge_{\action' \in  \nonControlSet \setminus \actionLabel}\neg \varLabel{}') \]
%\[labels_s(\labelSet) = \bigvee_{\actionLabel \in \labelSet}(label_s(\actionLabel) \wedge \bigwedge_{\action' \in  \Sigma \setminus \actionLabel}\neg \varLabel{}') \]
\vspace{1em}
\footnotesize
%\begin{center}
	\begin{tabular}{l l r l}
		$var$&$:2^{|\Sigma|}\times \Sigma \mapsto WFF(\gsV),$ &
		$
		var(\actionLabel, \action)$&$= \begin{cases}
		\varLabel{} & \action \in \actionLabel \\
		\neg \varLabel{} & otherwise
		\end{cases}
		$\\
		$labels_e$&$:2^{2^{|\Sigma|}}\mapsto WFF(\gsX),$ &
		$labels_e(\labelSet)$&$= \bigvee_{\actionLabel \in \labelSet}\bigwedge_{\action \in \nonControlSet}var(\actionLabel, \action)$\\
		$labels_s$&$:2^{2^{|\Sigma|}}\mapsto  WFF(\gsV),$ &
		$labels_s(\labelSet)$&$= \bigvee_{\actionLabel \in \labelSet}\bigwedge_{\action \in \Sigma}var(\actionLabel, \action)$\\
	\end{tabular}
%\end{center}
\normalsize

The initial condition formula is defined as:

\vspace{1em}
\footnotesize
%\begin{center}
	\begin{tabular}{l l r l}
		$\gsTheE$&$:WFF(\gsX),$ &
		$\gsTheE$&$= \bigwedge_{x_i \in \nonControlSet}\neg \hat{x}_i$\\
		$\gsTheS$&$:WFF(\gsV),$ &
		$\gsTheS$&$= (\varState{0} \wedge\bigwedge_{s_j \neq s_0, s_q \in S}\neg \varState{j}) \wedge \bigwedge_{a_i \in \Sigma}\neg \hat{a}_i \wedge \bigwedge_{\proposition \in \propositions}v(\proposition,0)$\\
		$\theta$&$:WFF(\gsV),$ &
		$\theta$&$= \gsTheE \wedge \gsTheS$\\		
	\end{tabular}
%\end{center}
\normalsize
\vspace{1em}

Where $\gsTheE$ assigns all environment label variables to $false$ since no transition has been taken yet at the start of execution, $\gsTheS$ does the same for the controllable actions, also ensures that only the state variable related to initial state $s_0$ holds and sets the natural variables related to propositions enabled at the start of the execution to $true$.
The transition relation is constructed as a conjunction of safety formulae. These can be split into two types, \emph{enabling} and \emph{updating} formulae. The first type restricts the set of primed label variables that can be set to $true$ at state $s_i$, one mutually exclusive option for each label \actionLabel in the transition relation ($(s_i,\actionLabel,s_j) \in \Delta$), indicating the actions that ocurred when going from $s_i$ to $s_j$. The second type defines the state $s_j$ reached for each combination of current state and enabled label variables. Notice that we rely on $E$ being deterministic in order to build emulate the transition relation in this way.
We will use the following formula to set mutually exclusive values over state variables (exactly one state variable is set to $true$ at any given time):

\begin{center}
	\begin{tabular}{r r r l}
$mux:$&$\gsV \mapsto WFF(\gsV),$&$mux(\varState{i}) = $&$(\varState{i} \wedge\bigwedge_{s_j \neq s_i, s_j \in S}\neg \varState{j})$
	\end{tabular}
\end{center}

Transition formulae $\gsRhoE$ and $\gsRhoS$ are written conforming to $\Delta$ as a conjunction of the enabling and transition formulae:

\vspace{1em}
\begin{tabular}{ r c l }
	$\rho_{env.enabling}$&$:$&$WFF(\gsV \cup \gsX')$\\
	$\rho_{env.enabling}$ &$=$& $\bigwedge_{s_i \in S} \varState{i} \implies \bigvee_{(s_i,\actionLabel,s_j) \in \Delta}labels_e(\actionLabel)'$\\
	$\rho_{sys.enabling}$&$:$&$WFF(\gsV \cup \gsV')$\\
	$\rho_{sys.enabling}$ &$=$& $\bigwedge_{s_i \in S} \varState{i} \implies \bigvee_{(s_i,\actionLabel,s_j) \in \Delta}labels_s(\actionLabel)'$\\
	&&\\
	$\rho_{update.states}$&$:$&$WFF(\gsV \cup \gsV')$\\
$\rho_{update.states}$&$=$&$\bigwedge_{\exists s_j:(s_i,\actionLabel,s_j) \in \Delta} (\varState{i} \wedge label_s(\actionLabel) \implies mux(\varState{j})' ) $\\	
	$\rho_{update.propositions}$&$:$&$WFF(\gsV \cup \gsV')$\\
	$\rho_{update.propositions}$&$=$&$\bigwedge_{s_i \in S} (\varState{i} \implies \bigwedge_{\proposition \in \propositions}v(\proposition,i) )$\\
	&&\\
	$\gsRhoE$&$:$&$WFF(\gsV \cup \gsX')$\\
	$\gsRhoE$&$=$&$\rho_{env.enabling}$\\		
	$\gsRhoS$&$:$&$WFF(\gsV \cup \gsV')$\\	
	$\gsRhoS$&$=$&$\rho_{sys.enabling} \wedge  \rho_{update.states} \wedge  \rho_{update.propositions}$\\	
\end{tabular}
\vspace{1em}

The first formula, $\rho_{env.enabling}$ restricts the set of environmental label variables that can be enabled in mutually exclusive fashion at any given state $\sigma_j$ in the reactive system, when \varState{i} holds, capturing the fixed set of options of actions the environment can execute at any given point in the run $\sigma$ that conforms with the enabled transitions going out of $s_i$, note that the application of both $labels_e$ and $labels_s$ are primed to ensure that the restriction is placed on next-state variables, according to our interpretation of the problem. $\rho_{sys.enabling}$ does the same for system label variables that relate to controllable actions, $\rho_{update.states}$ defines the next state variables according to the satisfaction of the state and label variables at the moment that will conform to the state reached in $E$, remember again the deterministic assumption, after taking the transition whose label conforms to the variables chosen for the state of the run. $\rho_{update.propositions}$ sets the natural variables according to the selected state to ensure that these are set according to \valuations. Then $\gsRhoS$ is the system transition formula, defined as a conjunction of $\rho_{sys.enabling}$, $\rho_{update.states}$ and $\rho_{update.propositions}$.
%
%Now we have the following mapping:
%
%\[\xymatrix@C+1pc{
%	\langle E = \langle S, \Sigma, \Delta, s_0 \rangle, \mathcal{C} \rangle \ar@{<->}[d]^{clts}_{gs}
%	& \varphi_{CLTS}\ar@{<->}[d]^{fl}_{val}
%	& C\ar@{<->}[d]^{clts}_{gs}
%	& \exists C: E\parallel C \models  \varphi_{CLTS}\ar@{<=>}[d]\\
%	G = \langle \mathcal{X},\mathcal{Y},\theta_e,\theta_s,\rho_e,\rho_s, \emptyset \rangle
%	& \varphi_G
%	& D
%	&\exists D: D \models \varphi_G\\
%}\]

%The following will be used to later define an FDS:
%\[\theta = \theta_e \wedge \theta_s, \rho = \rho_e \wedge \rho_s \]

The LTL formula that contains the behavior of $E$ w.r.t. the separation of $\Sigma$ between $\controlSet$ and $\nonControlSet$ while being complete w.r.t. \gsX can be defined as:

\[ \varphiLTLDef \]

Now we can prove that realizability is preserved between CLTS control problem $\controlProblem$ and the embedding \fdsEmbedding.

\begin{theorem}(\emph{$ltl$ preserves realizability})\label{theorem:gs_preserves_realizability}\\
	Let \controlProblemDef be a CLTS control problem with \cltsDef and $\Sigma = \mathcal{C}\uplus \mathcal{U}$ then if \fdsEmbeddingDef, there is an FDS $\fdsD= \langle \mathcal{V}_d, \theta_d,$ $\rho_d,$ $\mathcal{J}_d,$ $\mathcal{C}_d\rangle$, s.t. $\gsX \cup \gsY \subseteq \mathcal{V}_d$, $D$ is fairness-free, complete w.r.t. \gsX that satisfies $\fdsD \models \varphiLTL$
	if and only if there is a CLTS $C$ legal w.r.t. ($E,\mathcal{C}$) s.t. $C \parallel E$ is deadlock-free and $C \parallel E \models \varphi$.
	\normalsize
\end{theorem}

%, with an additional condition for LTL based reactive systems ($\sigma \models \theta_d \wedge \square \rho_d$) ruling out the cases where $val(\varphi)$ is never satisfied, since $\varphi_{LTL}$ could be realized by achieving $\neg\theta \vee \diamond \neg \rho$. In such a case no controller can be built for $\mathcal{I}$ s.t. $\varphi$ is satisfied since it will violate the safety restrictions related to the behavior of the plant $E_G$.

The proof is split in two parts, first we prove that if a CLTS controller $C$ exists for $\controlProblem$, an FDS controler \fdsD exists for $\fdsEmbedding$($\exists C \rightarrow \exists \fdsD$) by constructing an FDS from controller $C$ through a specific embedding $fds(C)$ and then proving that each property holds. For the case where if an FDS \fdsD exists for the embedding \fdsEmbedding there must exist a CLTS controller for \controlProblem ($\exists \fdsD \rightarrow \exists C$) we construct a CLTS controller from \fdsD through a specific embedding $ctrl(\fdsD)$ and then prove each property for it. The outline of the proof is as follows:

\begin{description}
	\item[($\exists C \rightarrow \exists \fdsD$)] Given \controlProblemDef and $C$
		\begin{itemize}
			\item $fds(C)$ is a fairness-free FDS			
			\item $fds(C)$ is complete w.r.t. $\mathcal{X}$
			\item $fds(C) \models \varphiLTL$
		%	\item $\exists \sigma \in D: \sigma \models \theta_e \wedge \square \rho_e$			
		\end{itemize}
	\item[($\exists \fdsD \rightarrow \exists C$)] Given $\fdsEmbeddingDef$ and FDS $\fdsD$
		\begin{itemize}
			\item $ctrl(\fdsD)$ is legal w.r.t. $E$ and $\mathcal{C}$
			\item $ctrl(\fdsD)$ is a deadlock-free CLTS			
			\item $ctrl(\fdsD) \models \varphi$
		\end{itemize}	
\end{description}

First,  for the $\exists C \rightarrow \exists \fdsD$ step, $fds$ constructs an FDS $\fdsD=\langle \mathcal V, \theta_d, \rho_d, \mathcal{J}_d, \mathcal{C}_d \rangle$ from a control problem $\controlProblem$ and a controller $C$ using the variables and formulas defined for $\varphiLTL$, i.e. $\gsX, \gsY, \gsTheE, \gsTheS, \gsRhoE, \gsRhoS$, but with the consideration that instead of using $E$ as the base CLTS we will use $C$.

\begin{tabular}{ l c l }
	$\gsV$ & $=$ & $\gsX \cup \gsY$\\	
	$\theta_d$ & $=$ & $(\theta_e \implies \theta_s)$\\
	$\rho_d$ & $=$ & $(\rho_e \implies \rho_s)$\\	
	$\mathcal{J}_d$ & $=$ & $\emptyset$\\
	$\mathcal{C}_d$ & $=$ & $\emptyset$\\
\end{tabular}

We can easily see that $fds(C)$ is fairness-free and complete w.r.t. $\mathcal{X}$ by construction, the first property is obvious ($\mathcal{J}_d=\emptyset$,$\mathcal{C}_d=\emptyset$), the second holds because no restriction is placed upon environment variables. Note that the environmental propositions appear on the left side of the implication in both $\theta_d$ and $\rho_d$. Take for instance the first condition of the complete w.r.t. $\gsX$ property, for any assignment $s_{\gsX}$ exists $s \in \Sigma_{\gsV}$ s.t. $s|_{\gsX}=s_{\gsX}$ and $s \models \theta_d$. Now, to prove that $fds(C)$ satisfies $\varphiLTL$ we will show that the condition holds for all runs in $\fdsD$: 

\[\forall \sigma \in \fdsD: \sigma \models \varphiLTL \]

We split the proof in two again, first assuming that the run violates the restrictions that follow the behavior of $E$, i.e.: $\sigma \not\models \theta_e \wedge \square \rho_e$, i.e.: $\sigma \models \neg\theta_e \vee \Diamond \neg\rho_e$, in which case $\varphiLTL$ is trivially satisfied since all the precedents in the general conjunction terms are falsified, then if the behavior restrictions are held, i.e.: $\sigma \models \theta_e \wedge \square \rho_e$, we must prove that $\theta_s$, $\square\rho_s$ and $val(\varphi)$ are also satisfied. Since $\sigma$ is a play in $fds(C)$ then $\theta_d=(\theta_e \implies \theta_s)$ and $\rho_d=(\rho_e \implies \rho_s)$ hold, and since it also satisfies $\theta_e \wedge \square \rho_e$ we observe the following:
\[((\theta_e \implies \theta_s) \wedge \theta_e))\implies \theta_s\]
\[((\rho_e \implies \rho_s) \wedge \rho_e))\implies \rho_s\]

To show that $\sigma$ also satisfies $val(\varphi)$ we prove that for every run $\sigma$ in $fds(C)$ there is an execution $\execution$ in $C$ that conforms to it.
For a run $\sigma \in 2^{|\mathcal{V}|^{\omega}}$ we define $exec(\sigma) \in S^{\omega}$ as follows, suppose that $\sigma = \sigma_1 \sigma_2 \ldots \sigma_i \ldots$ and
$exec(\sigma)= \execution_1 \execution_2 \ldots \execution_i \ldots$, then $\execution_i = s_j \iff \sigma_i \models \varState{j}$, to prove that such execution is possible in $C$ we observe that since $\sigma$ satisfies $\rho_e \wedge \rho_s$, for each pair ($\sigma_i$,$\sigma_{i+1}$),  $\rho_e \wedge \rho_s$ forces
the run to set the value of its label variables according to exactly one of the enabled transitions label ($\rho_{env.enabling}$,$\rho_{sys.enabling}$) and to assign states according to transitions in $\Delta_C$ ($\rho_{update.states}$). Once the current state and label are fixed, then if $\sigma_i$ satisfies $s_j$ and $\sigma_{i+1}$ satisfies $s_k$ then a transition $(j, l, k) \in \Delta$ should exist. For the initial state in $\sigma$ we know that $\sigma_0$ satisfies $\varState{0}$ by construction, therefore showing that $exec(\sigma)$ is possible in $C$.
Since we have a trace $\pi$ in controller $C$ for each run $\sigma$ that satisfies $\theta_e \wedge \square \rho_e$, and since $C \models \varphi$ we can show that $\sigma \models val(\varphi)$. This is proved over the algebraic construction of $\varphi$, and we will only show the base case of atomic satisfaction, where
$exec(\sigma) \models \proposition$ implies $\sigma \models v_{\proposition}$. Again, for any given state $s_i \in exec(\sigma)$, satisfaction of the natural variable $v_{\proposition}$ will be preserved by  $\rho_{update.propositions}$ through $\bigwedge_{\proposition \in \propositions}v(\proposition,j)$.\\
\\

Now,  for the $\exists \fdsD \rightarrow \exists C$ direction, we propose an embedding $clts$ from an FDS into a CLTS that will preserve only the behavior that satisfies $\theta_e \wedge \square \rho_e$, since otherwise we will be keeping traces in $clts(\fdsD)$ that are not feasible in $clts(\fdsD)\parallel E$. We can show that even after removing the set of plays that satisfy $\neg \theta_e \vee \Diamond \rho_e$, a non empty controller exists, since otherwise it would contradict the existence of a FDS $\fdsD$ that is also complete w.r.t. $\gsX$ . In the following definitions, let $\mathcal{M}= \mathcal{V}_d\setminus (\mathcal{X} \cup \mathcal{Y})$ be called the \emph{memory variables} of $\fdsD$, these are the variables added by the controller to be enable to enact a memoryful strategy. We will use an arbitrary bijective function \enumSetDef that maps the valuation of memory variables with a single integer thus defining the second component in the set of states of the CLTS controller. 
Following ~\cite{bloem2012synthesis} we will suppose that \fdsD is constructed from a strategy $f: M \times \Sigma$ $\times$ $\Sigma_{\gsX} \mapsto M \times \Sigma_{\gsY}$
that wins the game $G_{\varphiLTL}=\langle \gsX, \gsY, true,$ $true, true,$ $true,$ $\varphiLTL \rangle$. We know from ~\cite{bloem2012synthesis} that \varphiLTL is realizable if and only if there is a winning strategy $f$ for $G_{\varphiLTL}$. In this case finite memory $M$ suffices and, in particular, for any given state in \fdsD, once the the values in \gsV and \gsX' are fixed, there is only one possible valuation for the variables in $\mathcal{M}$. 
Considering that $\fdsD= \langle \mathcal{V}_d, \theta_d, \rho_d, \mathcal{J}_d, \mathcal{C}_d\rangle$ is a solution to $\varphiLTL$ that was in turn built from \controlProblemDef, we can define the embedding from an FDS controller \fdsD into a CLTS controller \ctrlEmbeddingDef{\fdsD}  as:

\vspace{1em}
\begin{tabular}{ l c l }
	$S$ &$=$& $\{(s,\hat{m}) : s \in S_E, \hat{m} \in [0\ldots2^{|\mathcal{M}|}-1] \}$\\
	$\Sigma$ &$=$&$\Sigma_E$\\	
	$s_0$&$=$&$(s_0^{E}, \enumSet{s_0|_{\mathcal{M}}})$ where $s_0 \models \theta_d$ \\
	$\valuations$&$=$&$\pi_1 \circ \valuations_E$\\
	&&\\
	$\Delta$&$=$&$\{((s_i,\enumSet{s|_{\mathcal{M}}}),\actionLabel,(s_j,\enumSet{s'|_{\mathcal{M}'}}): (s, s') \models \rho_d \wedge \xspace s|_{\gsY_s} = mux(\varState{i})  \wedge s'|_{\gsY'_s} = mux(\varState{j})'$ \\
	&&$\wedge (\forall y_i \in \controlSet : y_i \in \actionLabel \iff \hat{y'}_i \in s'|_{\gsY'_c})  \wedge (\forall x_i \in \nonControlSet : x_i \in \actionLabel \iff \hat{x'}_i \in s'|_{\gsX'})  \}$\\
\end{tabular}
\vspace{1em}
\\
Where $S$ is the cross product between $S_E$ and the states induced by the memory valuations of $\fdsD$ through $\#$, the sets of actions and atomic propositions remain unchanged from $E$, $\valuations=\pi_1 \circ \valuations_E$ describes the application of $\valuations_E$ over the first component of $S$ in order to get the set of propositions that are satisfied by the projected state in $E$. The transition relation is defined by adding those CLTS states and labels whose FDS state and label variables are satisfied according to $\rho_d$ in $\fdsD$, setting also the second element of each state, i.e. its memory, according to the projection of the valuation over the memory domain $\mathcal{M}$.

We can now show that $clts(\fdsD)$ satisfies the properties that make it a solution to $\controlProblem$. The first property we need to prove is legality w.r.t $E$ and $\mathcal{C}$, which means that for all $(s_e,s_c) \in E \parallel C$ it holds that:\\

\begin{center}
\begin{tabular}{r l}

	$\forall L_{\nonControlSet}\text{ s.t. }(s_E, L, s'_E) \in \Delta_E \wedge L|_{\nonControlSet} = L_{\nonControlSet}:( \exists (s_C, L', s'_C) \in \Delta_C \wedge L'|_{\nonControlSet} = L_{\nonControlSet})$ &(legality openness)\\
	$\forall (s_C, L, s'_C) \in \Delta_C: ( \exists (s_E, L, s'_E) \in \Delta_E)$& (legality inclusion)\\
\end{tabular}
\end{center}

	Since $\fdsD$ is complete w.r.t. $\mathcal{X}$ and we are only restricting environmental variables by forcing $\theta_e \wedge \square\rho_e$, sets of environment label variables enabled by $\rho_e$ are held in $\fdsD$, which are then included as set of monitored actions in $\Delta$, and since \fdsD only accepts infinite runs there must exist a choice in $\rho_s$ to reach the next state, and since the next state variables are completely defined by the selection of label variables in $\rho_{update.states}$, we can prove that all non controllable options from $E$ at each state are kept in $ctrl(\fdsD)$, proving legality openness.
	Since $\fdsD$ satisfies $\varphiLTL$, which implies that it also satisfies $\rho_e \implies \rho_s$, and we are only considering plays where $\theta_e \wedge \square \rho_e$ holds, for any transition $((s_i, \enumSet{m_1}),\actionLabel,(s_j,\enumSet{m_j})$ in $clts(D)$ a transition $(s_i, \actionLabel, s_j)$ should exist in $\Delta_E$. This proves legality inclusion. Now is easy to show that such a controller needs to be deadlock free, because otherwise there would be at least one reachable state $s_b$ that would not have a \fdsD-successor, thus implying that there is a contradiction when at a point of the run $\sigma$ in $\fdsD$ the variable $\varState{s_b}$ holds, which will prevent $\fdsD$ from satisfying the $\rho$ part of $\varphiLTL$ ($\theta_e \implies \square((\boxdot \rho_e) \implies \rho_s)$).  
Now for the LTL formula $\varphi$ in \controlProblem, we prove that for every execution $\execution$ in $clts(\fdsD)$ there is a run $\sigma$ in $\fdsD$ that conforms to it.
For an execution $\execution \in S^{\omega}$ we define the run $\sigma(\execution) \in 2^{|\mathcal{V}|^{\omega}}$ as follows, suppose that $\execution= \execution_1 \execution_2 \ldots \execution_i \ldots$ and $\sigma(\execution) = \sigma_1 \sigma_2 \ldots \sigma_i \ldots$ 
 then $\execution_i = s_j \iff \sigma_i \models \varState{j}$, the proof that such run is possible in $D$ is similar to the one for $exec(\sigma)$.
Now, since for each execution $\sigma(\execution)$ in $clts(\fdsD)$ we have a run $\sigma$ in $\fdsD$, and since given that we evaluate only runs that follow $\theta$ and $\rho$, then $\fdsD$ satisfies $val(\varphi)$ and we can show that $\sigma(\execution)$ satisfies $\varphi$ since the satisfaction of the atomic propositions is defined completely by the embedding of states $\varState{i} \in \sigma$ and $s_i \in \execution(\sigma)$.