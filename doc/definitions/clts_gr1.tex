In the present work controller synthesis is applied to specifications where the environment behavior is defined by a set of CLTS instances and games structures. The properties to be satisfied by the specification fall into the GR(1) category and are expressed as LTL formulas.

Since the goal $\varphi$ is restricted to GR(1) formulas we will follow what has been done with LTS based specifications in \cite{DBLP:phd/ethos/DIppolito13}. A translation between CLTS control problems and GR(1) games, and one from strategies into CLTS controller are presented, yielding a framework for CLTS GR(1) control problem synthesis. The GR(1) games used here and in in \cite{DBLP:phd/ethos/DIppolito13} are the same.

\begin{definition}\label{def:gr1_clts_control_problem} \emph{(GR(1) CLTS control problem)} 
	Let \controlProblemDef be a CLTS control problem, $\controlProblem$ is a GR(1) CLTS control problem if $\varphi$ satisfies: 
	\[\varphi = (\bigwedge_{i=1}^k\square \Diamond \gamma_i \implies \bigwedge_{j=1}^l\square \Diamond \psi_j)\]
	In the previous definition $\gamma_1, \ldots , \gamma_k$, $\psi_1, \ldots , \psi_l$ are propositional LTL formulas over accepting symbols that represent a set of assumptions over the environment and a set of guarantees the system should satisfy.
\end{definition}

%\begin{definition}\label{def:clts_to_gr1_translation} \emph{(GR(1) CLTS control problem to GR(1) game translation)} 
%For $I = \langle E, \mathcal{C}, \mathcal{F}=\lbrace fl_1, \ldots, fl_{k+l} \rbrace, \varphi \rangle$ a GR(1) CLTS control problem $gr1(I)=G$ is a GR(1) game $G = \langle S_g,$ $\Gamma^-,$ $\Gamma^+,s_{g_0}\varphi \rangle$ where $S_g = S_e \times \mathbb{B}^{k+l}$ is the set of states and for a state $s_g=(s_e,\alpha_1,\ldots,\alpha_{k+l}) \in S_g$ a fluent
%$fl_i$ is said to be satisfied at $s_g$ if and only if $\alpha_i$ is true.
%$s_{g_0}=(s_0,Init_1,\ldots,Init_{k+l})$ is the initial state and for every $(s,l,s')\in \Delta_e$, $(s_g,(s'_e,\alpha'_1,\ldots,\alpha'_{k+l}))$ is added to $\Gamma^-$ if $l \in \mathcal{P}(\mathcal{U})$ or $\Gamma^+$ if $l \in \mathcal{P}(\mathcal{C})$, $\alpha'_i$ is set as follows:
%\[
%\alpha'_i = \begin{cases}
%\alpha_i & \text{if } l \notin I_{fl_i} \cup T_{fl_i} \\
%\top & \text{if } l \in I_{fl_i}\\
%\bot & \text{if } l \in T_{fl_i}
%\end{cases}
%\]
%\end{definition}
%
%If the game $G$ constructed when applying $gr1$ on $I$ is realizable, i.e. if there exists a winning strategy for the system that satisfies $\varphi$ over $G$, then such a strategy can be encoded as a pair of functions $\sigma
%$ and $u$. Since $G$ is memory dependent the strategy is split into the function $\sigma$ that picks a successor for any given controllable and reachable state  and $u$ that updates the memory in order to keep track of the guarantee that needs to be satisfied next. 

