In this section we present the background required to define a CLTS control problem.  In the context of this work we will use open transition systems, where the alphabet is split between controllable and monitored actions, i.e.: $\Sigma=\controlSet \; \uplus \;\nonControlSet$.

%, and in particular with instances that are bi-partite with respect to a \controlSet, making states either controllable or non controllable, and because of this every label is composed of actions that are all within $\mathcal{C}$ or all within $\mathcal{U}$. 

%\begin{definition}\label{def:bi-partite_clts} \emph{(Bi-partite w.r.t. $\controlSet$)} 
%A CLTS instance \cltsDef, is considered to be bi-partite with respect to a set of controllable actions $\controlSet \subseteq \Sigma$ if and only if each transition is either fully controllable or fully non controllable:
%\[ \actionLabel_{\controlSet}((s,\actionLabel,s'))= ( \forall \action \in \actionLabel:(\action \in \controlSet))\]
%\[ \actionLabel_{\nonControlSet}((s,\actionLabel,s'))= ( \forall \action \in \actionLabel:(\action \not\in \controlSet))\]
%\[ (\forall (s,\actionLabel,s') \in \Delta_e| \actionLabel_{\mathcal{C}}((s,\actionLabel,s')) \vee \actionLabel_{\controlSet}((s,l,s'))) \]
%And that the same criteria is satisfied by every state:
%\[ (\forall s \in S | (\forall (\actionLabel, s'): (s,\actionLabel,s') \in \Delta|\actionLabel_{\controlSet}((s,\actionLabel,s')))\vee (\forall (\actionLabel, s'): (s,\actionLabel,s') \in \Delta|\actionLabel_{\nonControlSet}((s,\actionLabel,s'))))\]
%\end{definition}

\begin{definition}
	\label{def:legal_clts} \emph{(Legality w.r.t. $E$ and $\controlSet$)} 
	Given two CLTS instances \cltsDefShareSigmaIdx{C} and \cltsDefShareSigmaIdx{E}, where $\Sigma$ is partitioned into actions controlled and monitored, i.e.: $\Sigma=\controlSet \; \uplus \;\nonControlSet$, we say that $C$ is a legal CLTS for $E$ w.r.t. \controlSet if the following holds for all $(s_E,s_C) \in E \parallel C$:
	\footnotesize
	\begin{align}\begin{aligned}
	\forall \actionLabel_{\nonControlSet}\text{ s.t. }  (s_E, \actionLabel, s'_E) \in \Delta_E \wedge \actionLabel|_{\nonControlSet} = \actionLabel_{\nonControlSet}:( \exists (s_C, \actionLabel', s'_C) \in \Delta_C \wedge\\ \actionLabel'|_{\nonControlSet} = \actionLabel_{\nonControlSet})
	\end{aligned}\end{align}	
	\begin{align}
	\forall (s_C, \actionLabel, s'_C) \in \Delta_C:( \exists (s_E, \actionLabel, s'_E) \in \Delta_E)
	\end{align}	
	\normalsize
	In the previous and following definitions $\actionLabel|_{\nonControlSet}$ stands for the restriction of the elements of $\actionLabel$, it can be read as the intersection $\actionLabel \cap \nonControlSet$.
	The first property will be referred to as \emph{legality openness} and the second as \emph{legality inclusion}.
\end{definition}

\begin{definition}
	\label{def:clts_control_problem} \emph{(CLTS Control Problem)} 
	Let \cltsDefShareSigmaIdx{E} be a deterministic CLTS then \controlProblemDef constitutes a CLTS control problem. If a solution to $\controlProblem$ exists, such a solution will be a CLTS $M$, deadlock-free, legal w.r.t. $E$ and $\controlSet$, such that, $E \parallel M \models \varphi$. If such a solution exists, we will say that \controlProblem is \emph{realizable}, otherwise we say it is \emph{unrealizable}.
\end{definition}
