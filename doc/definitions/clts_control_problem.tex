\textcolor{blue}{In this section} we present the background required to relate the model checking approach in game structures with an equivalent control problem in the CLTS domain.  The main idea behind this is that any game structure specified as a set of LTL formulas induces an equivalent automaton. A direct relation between both structures is presented, but the reader can think of it as equivalent to translating the game structure to Kripke first and then to CLTS.

Game structures variables will be mapped to CTLS labels as following:

\[
\cdot(v) = \begin{cases}
v\uparrow & \text{if } v \equiv \top \\
v\downarrow & \text{if } v \equiv \bot
\end{cases}
\]

\[\cdot(\overline{v}) = \lbrace \cdot (v) | v \in \overline{v} \rbrace \]
CTLS labels will be mapped to game structures variables as following:
\[
\times (a?) = \begin{cases}
a & \text{if } a? \equiv a\uparrow \\
\neg a & \text{if } a? \equiv a\downarrow
\end{cases}
\]

\[\times(\overline{a}) = \lbrace \times (a) | a \in \overline{a} \rbrace \]

\begin{definition}
	\label{def:gs_clts_equivalence} \emph{(Equivalence between Game Structure and CLTS)} 
	Let $G=\langle \mathcal{V}, \mathcal{X}, \theta_{e}, \theta_{s}, \rho_{e}, \rho_{s}\rangle$, be a game structure and $E=\langle S, \Sigma, \Delta, s_{0}\rangle$ a CLTS automaton, $G$ is equivalent to $E$, written: $E \approx G$ if the automaton follows the game structure:
	\[
	\forall x, y: \theta_e(x) \wedge \theta_s(x,y) \implies \exists s',s'' : (s_0, \cdot (x), s'), (s', \cdot (y), s'') \in \Delta 
	\] 
	\[
	\forall x, y : \exists v \rho_e(v,x) \wedge \rho_s(v,x,y) \implies \exists s',s'',s''' : (s', \cdot (x), s''), (s'', \cdot (y), s''') \in \Delta 
	\] 
And the game structure follows the automaton:	
\[
\forall  s',s'',l,l' : (s_0, l, s'), (s', l', s'') \in \Delta  \implies \exists x,y: \theta_e(\times(l)) \wedge \theta_s(\times(l),\times(l'))
\] 
\[
\forall  s',s'',s''',l,l' : (s', l, s''), (s', l', s''') \in \Delta  \implies \exists v,x,y: \rho_e(v,\times(l)) \wedge \rho_s(v,\times(l),\times(l'))
\] 	%2^{|\mathcal{V}|} 
\end{definition}

\begin{definition}
	\label{def:gs_conversion} \emph{(Game Structure induced CLTS)} 
	Let $G_i=\langle \mathcal{V}, \mathcal{X}, \theta_{e}, \theta_{s}, \rho_{e}, \rho_{s}\rangle$, be a game structure then $clts(G)$ is a CLTS instance that has the same observational behavior, i.e.: $clts(G) \approx G$. 
\end{definition}

\begin{lemma}
	\label{def:gs_clts_equiv_implies_satifaction} \emph{(Consistency implies formula satisfaction)} 
	Given $E$ a CLTS instance, $G$ a game structure such that $E \equiv G$ and $\mathcal{C} \in \Sigma$, if $\forall v \in \mathcal{V}: v\uparrow, v\downarrow \in \Sigma$, $\forall v\downarrow \in \Sigma : v \in \mathcal{V}$,
$\mathcal{Y} = \mathcal{V}\setminus \mathcal{X}$, $v \in \mathcal{Y}$ if and only if $v\uparrow, v\downarrow \in \mathcal{C}$, $\dot{v} = \lbrace v\uparrow, v\downarrow \rbrace \in \mathcal{F}$ and both $E$ and $clts(G)$ are consistent w.r.t. $\mathcal{F}$ it holds that for every LTL formula $\varphi$ over $\mathcal{V}, G \models \varphi \iff \lbrace E, \mathcal{C} \rbrace \models \cdot(\varphi)$.
\end{lemma}

\begin{proof}
\emph{Rough sketch.} An inductive proof over $\varphi$ ($\cdot(\varphi)$) shows that every trace $\pi (\cdot(\pi))$ in one structure satisfies the formula if and only if it is satisfied in the other one. 
\end{proof}

\begin{definition}
	\label{def:mixed_env} \emph{(Mixed Environment Definition)} 
	A mixed environment $D = \langle E, G \rangle$ is defined over both CTLS instances and game structures as follows,
	let $E_1,\ldots,E_n$ be a set of CLTS automata and $G_1,\ldots,G_m$ a set of game structures,
	then each component is defined as $G=\langle \mathcal{V}, \mathcal{X}, \cap_{i \in 1..m}\theta_{e_i}, \cap_{i \in 1..m}\theta_{s_i},$ $\cap_{i \in 1..m}\rho_{e_i},$ $\cap_{i \in 1..m}\rho_{s_i}\rangle$ and $E = \parallel_{*,i \in 1..n} E_i$, where $\parallel_*$ stands for either the asynchronous or synchronous composition semantic.
	The composed behavior of $D$ is defined by the automaton $E \parallel_* clts(G)$.
\end{definition}

\begin{definition}
	\label{def:legal_clts} \emph{(Legality w.r.t. $E$ and $\mathcal{C}$)} 
	Given CLTS $C = \langle S_C, \Sigma$, $\Delta_C$, $s_{C_0}\rangle$ and $E = \langle S_E,\Sigma,$ $\Delta_E,$ $s_{E_0}\rangle$, where $\Sigma$ is partitioned into actions controlled and monitored (non controllable) by $C$ ($\Sigma=\mathcal{C} \; \cup \;\mathcal{U}$), we say that $C$ is a legal CLTS for $E$ if for all $(s_E,s_C) \in E \parallel_* C$ it holds that
	$\Delta_{E}(s_E)\cap \mathcal{P}(\mathcal{C}) \supseteq \Delta_{C}(s_C)\cap \mathcal{P}(\mathcal{C})$ and also that  $\Delta_{E}(s_E)\cap \mathcal{P}(\mathcal{U}) \subseteq \Delta_{C}(s_C)\cap \mathcal{P}(\mathcal{U})$.
\end{definition}

\begin{definition}
	\label{def:mixed_control_problem} \emph{(Mixed Control Problem)} 
	Let $D= \langle E, G \rangle$ be a mixed environment, if $\mathcal{F}$ is a set of fluents over $\Sigma$ and $\varphi$ an LTL formula over $\mathcal{F}$, then $I = \langle E, G, \mathcal{F}, \varphi \rangle$ conforms a mixed control problem. If $I$ is consistent and a solution to $I$ exists, such a solution will be a CLTS $M$, legal w.r.t. $D$, such that, $D \parallel M \models \varphi$.
\end{definition}

\begin{definition}
	\label{def:consistent_mixed_control_problem} \emph{(Consistent Mixed Control Problem)} 
	Let $I = \langle E, G, \mathcal{F}, \varphi \rangle$ be a mixed control problem, $I$ is consistent if $\forall v \in \mathcal{V}: v\uparrow, v\downarrow \in \Sigma$, let
	$\mathcal{Y} = \mathcal{V}\setminus \mathcal{X}$, $v \in \mathcal{Y}$ implies $v\uparrow, v\downarrow \in \mathcal{C}$, $\dot{v} = \lbrace v\uparrow, v\downarrow \rbrace \in \mathcal{F}$ and both $E$ and $clts(G)$ are consistent w.r.t. $\mathcal{F}$.
\end{definition}