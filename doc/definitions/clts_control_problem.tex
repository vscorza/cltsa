\textcolor{blue}{In this section} we present the background to define the notion of control problem in the CLTS domain.  In the context of this work specifications come from the digital design domain, where the environmental signals are set first and the system signals are set afterwards, making states either controllable or non controllable, and because of this every transition's elements are all within $\mathcal{C}$ or all within $\mathcal{U}$. 

\begin{definition}\label{def:bi-partite_clts} \emph{(Bi-partite w.r.t. $\mathcal{C}$)} 
A CLTS instance $E=\langle S_e, \Sigma_e, \Delta_e, s_{e_0} \rangle$ is considered to be bi-partite with respect to a set of controllable actions $\mathcal{C} \subseteq \Sigma$ if and only if each transition is fully (non) controllable:
\[ l_{\mathcal{C}}((s,l,s')): ( \forall l_i \in l:(l_i \in \mathcal{C}))\]
\[ l_{\mathcal{U}}((s,l,s')): ( \forall l_i \in l:(l_i \not\in \mathcal{C}))\]
\[ \forall (s,l,s') \in \Delta_e: l_{\mathcal{C}}((s,l,s')) \vee l_{\mathcal{U}}((s,l,s')) \]
And that the same criteria is satisfied by every state:
\[ \forall s \in S_e : (\forall l, s' \text{ s.t. } (s,l,s') \in \Delta_e:l_{\mathcal{C}}((s,l,s')))\vee (\forall l, s' \text{ s.t. } (s,l,s') \in \Delta_e:l_{\mathcal{U}}((s,l,s')))\]
\end{definition}

\begin{definition}
	\label{def:legal_clts} \emph{(Legality w.r.t. $E$ and $\mathcal{C}$)} 
	Given CLTS $C = \langle S_c, \Sigma$, $\Delta_c$, $s_{c_0}\rangle$ and $E = \langle S_e,\Sigma,$ $\Delta_e,$ $s_{e_0}\rangle$, where $\Sigma$ is partitioned into actions controlled and monitored (non controllable) by $C$ ($\Sigma=\mathcal{C} \; \cup \;\mathcal{U}$), we say that $C$ is a legal CLTS for $E$ if for all $(s_e,s_c) \in E \parallel_* C$ it holds that:
	$\Delta_{E}(s_E)\cap \mathcal{P}(\mathcal{C}) \supseteq \Delta_{C}(s_C)\cap \mathcal{P}(\mathcal{C})$ and also that  $\Delta_{E}(s_E)\cap \mathcal{P}(\mathcal{U}) \subseteq \Delta_{C}(s_C)\cap \mathcal{P}(\mathcal{U})$.
\end{definition}

\begin{definition}
	\label{def:clts_control_problem} \emph{(CLTS Control Problem)} 
	Let $E = \langle S_e,\Sigma,$ $\Delta_e,$ $s_{e_0}\rangle $ be bi-partite w.r.t. $\mathcal{C} \subseteq \Sigma$, if $\mathcal{F}$ is a set of fluents over $\Sigma$ and $\varphi$ an LTL formula over $\mathcal{F}$, then $I = \langle E, \mathcal{C}, \mathcal{F}, \varphi \rangle$ constitutes a CLTS control problem. If a solution to $I$ exists, such a solution will be a CLTS $M$, legal w.r.t. $E$ and $\mathcal{C}$, such that, $E \parallel_* M \models \varphi$.
\end{definition}
