\textcolor{blue}{In this section} we present the background to define the notion of control problem in the CLTS domain.  When defining what is a legal restriction of an environment over a set of controllable actions $\mathcal{C}$, the asymmetry of choice will be tilted in favor of the adversary, thus if a concurrent transition holds both controllable and non controllable actions it will be considered non controllable as a whole.

\begin{definition}
	\label{def:legal_clts} \emph{(Legality w.r.t. $E$ and $\mathcal{C}$)} 
	Given CLTS $C = \langle S_C, \Sigma$, $\Delta_C$, $s_{C_0}\rangle$ and $E = \langle S_E,\Sigma,$ $\Delta_E,$ $s_{E_0}\rangle$, where $\Sigma$ is partitioned into actions controlled and monitored (non controllable) by $C$ ($\Sigma=\mathcal{C} \; \cup \;\mathcal{U}$), we say that $C$ is a legal CLTS for $E$ if for all $(s_E,s_C) \in E \parallel_* C$ it holds that
	$\Delta_{E}(s_E)\cap \mathcal{P}(\mathcal{C}) \supseteq \Delta_{C}(s_C)\cap \mathcal{P}(\mathcal{C})$ and also that  $\Delta_{E}(s_E)\cap \mathcal{P}(\mathcal{U}) \subseteq \Delta_{C}(s_C)\cap \mathcal{P}(\mathcal{U})$.
\end{definition}

\begin{definition}
	\label{def:mixed_control_problem} \emph{(CLTS Control Problem)} 
	Let $E$ be a a CLTS instance, if $\mathcal{F}$ is a set of fluents over $\Sigma$ and $\varphi$ an LTL formula over $\mathcal{F}$, then $I = \langle E, \mathcal{F}, \varphi \rangle$ constitutes a CLTS control problem. If a solution to $I$ exists, such a solution will be a CLTS $M$, legal w.r.t. $E$, such that, $E \parallel M \models \varphi$.
\end{definition}
