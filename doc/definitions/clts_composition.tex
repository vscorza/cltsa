
When having to compose concurrent systems we define three different semantics that can be applied pairwise. Synchronous semantic $A ||_s B$ (Figure~\ref{fig:synchronous_composition})captures the composition of two components that share a single synchronizing event (implicit), where all participants should be able to make progress concurrently. The main motivation being digital components sharing a single clock. 
Asynchronous semantic $A ||_a B$ (Figure~\ref{fig:asynchronous_composition}) captures the interleaving interpretation of concurrency as found in LTS systems. Concurrent semantic $A ||_c B$ (Figure~\ref{fig:concurrent_composition}) captures the behavioral superset of the latter two. It can be used when composing two processes that may not share an implicit synchronizing event, as in the synchronous semantic, but can be observed by a third component that over samples the other two, allowing for the possibility of both concurrent an independent occurrence when observed as a composed system. A synthetic example is shown in Figure~\ref{fig:concurrent_systems}, motivation for this semantic can be found in clock domain crossing examples and micro architecture buffers.

\begin{figure}[bt]
	\centering
	%\SmallPicture
	%\ShowFrame
		\begin{VCPicture}{(-3,-3)(3,2.5)}
			\SetStateLabelScale{1}
			\SetEdgeLabelScale{1}
			\State[1_a]{(-3,1.5)}{A}
			\State[1_b]{(-3,-1.5)}{B}			
			\State[1_c]{(2.5,0)}{C}						
			\Initial[n]{A}
			\Initial[n]{B}
			\Initial[w]{C}						
			%\ChgEdgeLineStyle{dashed} %\EdgeLineDouble
			%\ChgEdgeLineWidth{1.5}
			%\EdgeL{1}{2}{req}
			%\ArcR[.3]{6}{1}{reset}        
			\CLoopSW[.5]{A}{data_a}        
			%\CLoopSE[.5]{A}{idle_a}
			\CLoopSW[.5]{B}{data_b}        
			%\CLoopSE[.5]{B}{idle_b}				
			%\CLoopSW[.5]{C}{idle_c}        
			\CLoopSE[.5]{C}{data_b}					
			\CLoopNW[.5]{C}{data_a}					
			\CLoopNE[.5]{C}{<data_a,data_b,goal>}					
			%\RstEdgeLineWidth{1}
			%\RstEdgeLineStyle %\EdgeLineSimple
			%\EdgeL{2}{3}{grant}
			%\EdgeL[.75]{2}{5}{\overline{grant}}
			%\VArcR{arcangle=-20}{3}{6}{timeout}
			%\ArcR[.6]{3}{6}{timeout}
			%\ArcL{5}{1}{hready}
			%\VArcR{arcangle=-30}{3}{4}{hready}
		\end{VCPicture}

	\caption{A, B and C systems}
	\label{fig:concurrent_systems}
	%%\vspace*{-4mm}
	\MediumPicture
\end{figure}
\begin{figure}[bt]
	\centering
\minipage{0.32\textwidth}	
\centering
	%\SmallPicture
	%\ShowFrame
	\begin{VCPicture}{(-1.5,-1.5)(1.5,1.5)}
		\SetStateLabelScale{.8}
		\SetEdgeLabelScale{1}
		\State[1_{a\parallel_a b}]{(0,0)}{C}
		\Initial[n]{C}
		%\ChgEdgeLineStyle{dashed} %\EdgeLineDouble
		%\ChgEdgeLineWidth{1.5}
		%\EdgeL{1}{2}{req}
		%\ArcR[.3]{6}{1}{reset}        
		\CLoopSW[.5]{C}{data_a}        
		\CLoopSE[.5]{C}{data_b}					
		%\RstEdgeLineWidth{1}
		%\RstEdgeLineStyle %\EdgeLineSimple
		%\EdgeL{2}{3}{grant}
		%\EdgeL[.75]{2}{5}{\overline{grant}}
		%\VArcR{arcangle=-20}{3}{6}{timeout}
		%\ArcR[.6]{3}{6}{timeout}
		%\ArcL{5}{1}{hready}
		%\VArcR{arcangle=-30}{3}{4}{hready}
	\end{VCPicture}
	\caption{A $\parallel_a$ B}
	\label{fig:asynchronous_composition}
\endminipage\hfill
\minipage{0.32\textwidth}%
\centering
	%%\vspace*{-4mm}
	%\SmallPicture
	%\ShowFrame
	\begin{VCPicture}{(-1.5,-1.5)(1.5,1.5)}
		\SetStateLabelScale{.8}
		\SetEdgeLabelScale{1}
		\State[1_{a \parallel_s b}]{(0,0)}{C}
		\Initial[n]{C}
		%\ChgEdgeLineStyle{dashed} %\EdgeLineDouble
		%\ChgEdgeLineWidth{1.5}
		%\EdgeL{1}{2}{req}
		%\ArcR[.3]{6}{1}{reset}        
		\CLoopS[.5]{C}{<data_a, data_b>}        
		%\RstEdgeLineWidth{1}
		%\RstEdgeLineStyle %\EdgeLineSimple
		%\EdgeL{2}{3}{grant}
		%\EdgeL[.75]{2}{5}{\overline{grant}}
		%\VArcR{arcangle=-20}{3}{6}{timeout}
		%\ArcR[.6]{3}{6}{timeout}
		%\ArcL{5}{1}{hready}
		%\VArcR{arcangle=-30}{3}{4}{hready}
	\end{VCPicture}
	\caption{A $\parallel_s$ B}
	\label{fig:synchronous_composition}
\endminipage\hfill
\minipage{0.32\textwidth}%	
\centering
	%%\vspace*{-4mm}
	%\SmallPicture
	%\ShowFrame
	\begin{VCPicture}{(-1.5,-1.5)(1.5,1.5)}
		\SetStateLabelScale{.8}
		\SetEdgeLabelScale{1}
		\State[1_{a\parallel_c b}]{(0,0)}{C}
		\Initial[n]{C}
		%\ChgEdgeLineStyle{dashed} %\EdgeLineDouble
		%\ChgEdgeLineWidth{1.5}
		%\EdgeL{1}{2}{req}
		%\ArcR[.3]{6}{1}{reset}        
		\CLoopSW[.5]{C}{data_a}        
		\CLoopSE[.5]{C}{data_b}					
		\CLoopNE[.5]{C}{<data_a, data_b>}					
		%\RstEdgeLineWidth{1}
		%\RstEdgeLineStyle %\EdgeLineSimple
		%\EdgeL{2}{3}{grant}
		%\EdgeL[.75]{2}{5}{\overline{grant}}
		%\VArcR{arcangle=-20}{3}{6}{timeout}
		%\ArcR[.6]{3}{6}{timeout}
		%\ArcL{5}{1}{hready}
		%\VArcR{arcangle=-30}{3}{4}{hready}
	\end{VCPicture}
	\caption{A $\parallel_c$ B}
	\label{fig:concurrent_composition}
\endminipage\hfill	
	%%\vspace*{-4mm}
\end{figure}
%\begin{figure}[bt]
%\centering
%\SmallPicture
%%\ShowFrame
%\VCDraw{
%    \begin{VCPicture}{(-4,-1.5)(4,2.3)}
%        \SetEdgeLabelScale{1.4}
%        \State[1]{(-3,0)}{1}
%        \State[2]{(0,0)}{2}
%        \State[3]{(3,1)}{3}
%        \State[4]{(-0.5,3)}{4}
%        \State[6]{(0,1.5)}{6}        
%        \State[5]{(3,-1)}{5}
%		\Initial[w]{1}
%        \ChgEdgeLineStyle{dashed} %\EdgeLineDouble
%        %\ChgEdgeLineWidth{1.5}
%        \EdgeL{1}{2}{req}
%        \ArcR[.3]{6}{1}{reset}        
%        \VArcR{arcangle=-30}{4}{1}{process}        
%        %\RstEdgeLineWidth{1}
%        \RstEdgeLineStyle %\EdgeLineSimple
%        \EdgeL{2}{3}{grant}
%        \EdgeL[.75]{2}{5}{\overline{grant}}
%        %\VArcR{arcangle=-20}{3}{6}{timeout}
%        \ArcR[.6]{3}{6}{timeout}
%        \ArcL{5}{1}{hready}
%        \VArcR{arcangle=-30}{3}{4}{hready}
%    \end{VCPicture}
%}
%\caption{Bus Access example ($E$)}
%\label{fig:req_grant}
%%%\vspace*{-4mm}
%\MediumPicture
%\end{figure}
%\begin{figure}[bt]
%\centering
%\SmallPicture
%%\ShowFrame
%\VCDraw{
%    \begin{VCPicture}{(-4,-1.5)(4,2)}
%        \SetEdgeLabelScale{1.4}
%        \State[1]{(-3,0)}{1}
%        \State[2]{(0,-1)}{2}
%        \State[3]{(3,0)}{3}
%        \State[6]{(0,1)}{6}        
%		\Initial[w]{1}
%        \ChgEdgeLineStyle{dashed} %\EdgeLineDouble
%        %\ChgEdgeLineWidth{1.5}
%        \ArcR{1}{2}{req}
%        \ArcR{6}{1}{reset}        
%        %\RstEdgeLineWidth{1}
%        \RstEdgeLineStyle %\EdgeLineSimple
%        \ArcR{2}{3}{grant}
%        %\VArcR{arcangle=-20}{3}{6}{timeout}
%        \ArcR{3}{6}{timeout}
%    \end{VCPicture}
%}
%\caption{Minimized Bus Access ($E_1$)}
%\label{fig:req_grant_sub_1}
%%\vspace*{-4mm}
%\MediumPicture
%\end{figure}
%\begin{figure}[bt]
%\centering
%\SmallPicture
%%\ShowFrame
%\VCDraw{
%    \begin{VCPicture}{(-4,-1.5)(4,2)}
%        \SetEdgeLabelScale{1.4}
%        \State[1]{(-3,0)}{1}
%        \State[2]{(0,0)}{2}
%        \State[3]{(3,1)}{3}
%        \State[6]{(0,1.5)}{6}        
%        \State[5]{(3,-1)}{5}
%		\Initial[w]{1}
%        \ChgEdgeLineStyle{dashed} %\EdgeLineDouble
%        %\ChgEdgeLineWidth{1.5}
%        \EdgeL{1}{2}{req}
%        \ArcR[.3]{6}{1}{reset}        
%        %\RstEdgeLineWidth{1}
%        \RstEdgeLineStyle %\EdgeLineSimple
%        \EdgeL{2}{3}{grant}
%        \EdgeL[.75]{2}{5}{\overline{grant}}
%        %\VArcR{arcangle=-20}{3}{6}{timeout}
%        \ArcR[.7]{3}{6}{timeout}
%        \ArcL{5}{1}{hready}
%    \end{VCPicture}
%}
%\caption{Minimized Bus Access ($E_2$)}
%\label{fig:req_grant_sub_2}
%%\vspace*{-4mm}
%\MediumPicture
%\end{figure}

\begin{definition} 
	Let \automaton{M},\automaton{N}, with $\Delta_M : S_M \times \mathcal{P}(\Sigma_M) \times S_M$, be two CLTS instances, then CLTS concurrent parallel composition is defined as \ltsComposition{M}{N}{c} where $\Delta$ is the smallest relation s.t:
	\begin{center}
		\begin{equation}
		\AxiomC{$(s, l_M, s') \in \Sigma_M,(t, l_N, t') \in \Sigma_N  $}\RightLabel{$l_M \cap \Sigma_N = l_N \cap \Sigma_M$}
		\UnaryInfC{$((s,t),l_M \cup l_N,(s',t')) \in \Delta$}
		\DisplayProof
		\end{equation}	
		\begin{equation}
		\AxiomC{$(s, l_M, s') \in \Sigma_M $}\RightLabel{$l_M \cap \Sigma_N = \emptyset$}
		\UnaryInfC{$((s,t),l_M,(s',t))$}
		\DisplayProof
		\quad\quad
		\AxiomC{$(t, l_N, t') \in \Sigma_N $}\RightLabel{$l_N \cap \Sigma_M = \emptyset$}
		\UnaryInfC{$((s,t),l_N,(s,t'))$}
		\DisplayProof
		\end{equation}
	\end{center}
\end{definition}

\begin{definition} 
	Let \automaton{M},\automaton{N}, with $\Delta_M : S_M \times \mathcal{P}(\Sigma_M) \times S_M$, be two CLTS instances, then CLTS asynchronous parallel composition is defined as \ltsComposition{M}{N}{a} where $\Delta$ is the smallest relation s.t:
	\begin{center}
		\begin{equation}
		\AxiomC{$(s, l_M, s') \in \Sigma_M,(t, l_N, t') \in \Sigma_N  $}\RightLabel{$l_M \cap \Sigma_N = l_N \cap \Sigma_M \neq \emptyset$}
		\UnaryInfC{$((s,t),l_M \cup l_N,(s',t')) \in \Delta$}
		\DisplayProof
		\end{equation}	
		\begin{equation}
		\AxiomC{$(s, l_M, s') \in \Sigma_M $}\RightLabel{$l_M \cap \Sigma_N = \emptyset$}
		\UnaryInfC{$((s,t),l_M,(s',t))$}
		\DisplayProof
		\quad\quad
		\AxiomC{$(t, l_N, t') \in \Sigma_N $}\RightLabel{$l_N \cap \Sigma_M = \emptyset$}
		\UnaryInfC{$((s,t),l_N,(s,t'))$}
		\DisplayProof
		\end{equation}
	\end{center}
\end{definition}

\begin{definition} 
	Let \automaton{M},\automaton{N}, with $\Delta_M : S_M \times \mathcal{P}(\Sigma_M) \times S_M$, be two CLTS instances, then CLTS synchronous parallel composition is defined as \ltsComposition{M}{N}{s} where $\Delta$ is the smallest relation s.t:
	\begin{center}
		\begin{equation}
		\AxiomC{$(s, l_M, s') \in \Sigma_M,(t, l_N, t') \in \Sigma_N  $}\RightLabel{$l_M \cap \Sigma_N = l_N \cap \Sigma_M$}
		\UnaryInfC{$((s,t),l_M \cup l_N,(s',t')) \in \Delta$}
		\DisplayProof
		\end{equation}	
	\end{center}
\end{definition}

%\begin{definition} 
%	Let $M_1 \ldots M_N$ be CLTS instances s.t. $M_i= (S_{i}, \Sigma_{i}, \Delta_{i}, s_{0}^{i})$ with $\Delta_i : S_i \times l \subseteq \Sigma_i \times S_i$, the concurrent parallel composition \ltsComposition{}{M_i}{} where $\Delta_i$ is the smallest relation s.t:
%	\begin{center}
%		\begin{equation}
%		\AxiomC{$\forall i,j \in 1 \ldots N, i \neq j: ((s_i, l_i, t_i) \in \Delta_i \wedge \forall a \in l_i : a \in (\Sigma_i \cap \Sigma_j) \rightarrow (a \in l_i \wedge a \in l_j)) \vee (l_i = \emptyset \wedge t_i = s_i)$}
%		\UnaryInfC{$((s_1 \ldots s_N),\cup_{i=1}^{N}l_i,(t_1 \ldots t_N)) \in \Delta$}
%		\DisplayProof
%		\end{equation}		
%	\end{center}
%\end{definition}

%\begin{figure}[ht]
%	\begin{center}
%		\renewcommand{\ttdefault}{pcr}
\begin{lstlisting}[escapeinside={[*}{*]},basicstyle=\scriptsize\ttfamily,columns=flexible,frame=lines,mathescape=true,xleftmargin=3.0ex,keywordstyle=\textbf,morekeywords={if,while,do,else,fork,int,null, algorithm, is, input, output, return},numbers=left,numberstyle=\scriptsize]
algorithm compose_automata is
	input: $A_1 \ldots A_N$  the set of CLTS automata to be composed
	input: $type$ an element of $\{synch,asynch,concurrent\}$ the type of composition to be applied
	output: $A$ the CLTS automaton that results from applying $\parallel_{type}$ to $A$
	$s_0$ = $(s^1_{0},\ldots,s^N_{0})$
	$frontier$ = [$s_0$]
	$A$ = create_automaton($s_0$)
	$visited$ = $[]$
	while(|$frontier$| > 0)
		$\Delta_{union}$ = $\emptyset$
		$\Sigma_{union}$ = $\emptyset$		
		
		$s$ = $frontier$.pop()
		$visited$.push($s$)
		for $s_i \in s$ 
			if $\Delta_{union} = \emptyset$
				$\Delta_{union}$ = $\Delta_i(s)$
			else if $\Delta_i(s) \cap \Sigma_{union} = \emptyset$		
				if $type$ = $asynch$
					$\Delta_{union}$ = $\Delta_{union} \cup \Delta_i(s)$
				else
					$\Delta_{union}$ = $\Delta$_cross_product($\Delta_{union}$, $\Delta_i(s)$, $type$)
			else
				$\Delta_{union}$ = $\Delta$_partial_composition($\Delta_{union}$, $\Delta_i(s)$, $\Sigma_{union}$)
		$\Sigma_{union}$ = $\Sigma_{union} \cup \Sigma_i$
		$\Delta$_expand_frontier($frontier$,$\Delta_{union}$, $visited$)				
		add_transitions($A$, $\Delta_{partial}$)
	return $A$

algorithm $\Delta$_cross_product is
	input: $\Delta$ the composite set of transitions already computed for a given state
	input: $\Delta_{i}(s)$ the local set of transitions for a given state $s$ at automaton $A_i$
	input: $type$ an element of $\{synch,asynch,concurrent\}$ the type of composition to	
	output: $\Delta'$ the cross product of the computed set $\Delta$ with $\Delta_{i}(s)$
	$\Delta'$ = $\emptyset$
	for $d \in \Delta$
		for $d' \in \Delta_{i}(s)$
			$\Delta'$ = $\Delta' \cup$ {$s(d)$,($label(d) \cup label(d')$,($t(d)_1, \ldots, t(d')_i, \ldots, t(d)_N$))}
	if $type \neq synch$
		return $\Delta \cup \Delta' \cup \Delta_{i}(s)$
	else
		return $\Delta'$
	
algorithm $\Delta$_partial_composition is
	input: $\Delta$ the composite set of transitions already computed for a given state
	input: $\Delta_{i}(s)$ the local set of transitions for a given state $s$ at automaton $A_i$
	input: $\Sigma$ the composite set of labels already computed for a given state		
	output: $\Delta'$ the partial composition of the computed set $\Delta$ with $\Delta_{i}$ according to $type$
	$\Delta'$ = $\emptyset$
	for $d \in \Delta$
		for $d' \in \Delta_{i}(s)$
			if $\Sigma \cap label(d') = \Sigma_i \cap label(d)$
				$\Delta'$ = $\Delta' \cup$ {$s(d)$,($label(d) \cup label(d')$,($t(d)_1, \ldots, t(d')_i, \ldots, t(d)_N$))}
	return $\Delta'$
				

\end{lstlisting} 
%		\caption{Concurrent Composition Algorithm}
%		\label{fig:dfs-code}
%	\end{center}
%\end{figure}
%\begin{figure}[ht]
%	\begin{center}
%		\renewcommand{\ttdefault}{pcr}
\begin{lstlisting}[escapeinside={[*}{*]},basicstyle=\scriptsize\ttfamily,columns=flexible,frame=lines,mathescape=true,xleftmargin=3.0ex,keywordstyle=\textbf,morekeywords={if,while,do,else,fork,int,null, algorithm, is, input, output, return, for},numbers=left,numberstyle=\scriptsize]
algorithm compose_automata is
	input: $A_1 \ldots A_N$  the set of CLTS automata to be composed
	input: $type$ an element of $\{synch,asynch,concurrent\}$ the type of composition to be applied
	output: $A$ the CLTS automaton that results from applying $\parallel_{type}$ to $A_1 \ldots A_N$
	$s_0$ = $(s^1_{0},\ldots,s^N_{0})$
	$frontier$ = [$s_0$]
	$A$ = create_automaton($s_0$)
	$visited$ = $[]$
	while(|$frontier$| > 0)
		$\Delta_{I}$ = $\emptyset$
		$\Delta_{a}$ = $\emptyset$
		$\Sigma_{a}$ = $\emptyset$		
		$s$ = $frontier$.pop()
		if $s \in visited$
			continue
		$visited$.push($s$)
		$\Delta_{I}$ = $\Delta_0(s)$
		for $s_{i > 0} \in s$ 
			for $d:(s,l,t) \in \Delta_{I}$
				for $d':(s',l',t') \in \Delta_{i}(s)$
					if $\Sigma_{a} \cap l' = \Sigma_i \cap l \wedge \Sigma_i \cap l \neq \emptyset $
						$\Delta_{a}$ = $\Delta_{a} \cup$ {($s$,$l \cup l'$,($t_1, \ldots, t\prime_i, \ldots, t_N$))}
					if $\Sigma_{a} \cap l' = \Sigma_i \cap l \wedge \Sigma_i \cap l = \emptyset$
						if $type \neq asynch$
							$\Delta_{a}$ = $\Delta_{a} \cup$ {($s$,$l \cup l'$,($t_1, \ldots, t\prime_i, \ldots, t_N$))}
						if $type \neq synch$							
							$\Delta_{a}$ = $\Delta_{a} \cup$ {$(s$,$l$,($t_1, \ldots, t_N$))} 
								$\cup$ {$(s$,$l'$,($s_1, \ldots, t'_i, \ldots, s_N$))}
					if $l \cap l' = \emptyset \wedge type \neq synch$
						if $\Sigma_a \cap l' = \emptyset $
							$\Delta_{a}$ = $\Delta_{a} \cup$ {$(s$,$l'$,($s_1, \ldots, t'_i, \ldots, s_N$))} 
						if $\wedge \Sigma_i \cap l = \emptyset$
							$\Delta_{a}$ = $\Delta_{a} \cup$ {$(s$,$l$,($t_1, \ldots,  t_N$))} 							
			$\Delta_{I} = \Delta_{a}$						
			$\Delta_{a} = \emptyset$						
			$\Sigma_{a}$ = $\Sigma_{a} \cup \Sigma_i$
		for $d:(s,l,t) \in \Delta_{I}$
			if $t \notin visited \wedge t \notin frontier$ 
				 $frontier$.push($t$)
			add_transition($A$, $d$)			
	return $A$

\end{lstlisting} 
%		\caption{Concurrent Composition Algorithm (closer to code version)}
%		\label{fig:dfs-code2}
%	\end{center}
%\end{figure}