The following is an example of an embedding from a CLTS control problem into an FDS control problem, for which a strategy is built, converted into an FDS controller and then translated into a CLTS controller. 

The CLTS control problem tries to process two items before resetting the system. The initial state requires that the environment signals that the conditions are met through the action $r$ (ready) otherwise it will keep informing that the status is pending through action $p$.  The system can then process an item through action $P$ or issue a reset $R$. Proposition $e$ indicates that an enabled state has been reached after which the processing can start, propositions $a$ and $b$ denote that the first and second items, respectively, have been processed. 
The property to be satisfied is captured by $\varphi: \square \Diamond e$ $\implies$ $(\square \Diamond a$ $\wedge$ $\square \Diamond b)$. The CLTS control problem is noted as $\mathcal{I}=\langle E, \{P,R\}, \varphi \rangle$ with $E$ being the automaton depicted in figure ~\ref{fig:clts_to_fds_E}. The labels next to the state numbers indicate which propositions hold at any given point.

\begin{figure}[bt]
	\centering
	%\minipage{0.32\textwidth}%
	%\centering
	\VCDraw{
	\begin{VCPicture}{(-2.5,-1)(2.5,3.5)}
		\SetStateLabelScale{1}
		\SetEdgeLabelScale{1.2}
		\State[0:-]{(-2,2.3)}{0}
		\State[1:e]{(2,2.3)}{1}
		\State[2:a]{(2,-0.3)}{2}
		\State[3:b]{(-2,-0.3)}{3}
		\Initial[w]{0}
		\ArcL{0}{1}{<r>}        
		\ArcL{1}{0}{<R>}        
		\LoopN{0}{<p>}		
		\ArcL{1}{2}{<P>}        
		\ArcL{2}{0}{<R>}        				
		\ArcL{2}{3}{<P>}        
		\ArcL{3}{0}{<R>}        						
	\end{VCPicture}
	}
	\caption{Automaton $E$ from $\mathcal{I}=\langle E, \{P,R\}, \varphi \rangle$}
	\label{fig:clts_to_fds_E}
	%\endminipage\hfill
\end{figure}

The FDS equivalent formula to $E$ is built using the translation proposed before, with $\gsX$ being the set $\{\hat{p},\hat{r}\}$ and $\gsY$ being $\{s_0,s_1,\hat{P},\hat{R},\hat{e},\hat{a},\hat{b}\}$.

\footnotesize
\vspace{1em}
\begin{tabular}{ r c l }
	$\theta_{e}$ &$=$& $(\bar{\hat{p}} \wedge \bar{\hat{r}})$\\
	$\theta_{s}$ &$=$& $\bar{s_0} \wedge \bar{s_1} \wedge ((\bar{\hat{P}} \wedge \bar{\hat{R}})) \wedge \bar{\hat{e}} \wedge \bar{\hat{a}} \wedge \bar{\hat{b}}$\\	
	$\rho_{env.enabling}$ &$=$& $((\bar{s_0} \wedge \bar{s_1} ) \implies ((\hat{p}'\wedge \bar{\hat{r}}') \vee (\bar{\hat{p}}' \wedge \hat{r}'))) \wedge$\\
	&& $((s_0 \vee s_1 ) \implies (\bar{\hat{p}}' \wedge \bar{\hat{r}}'))$\\	
	$\rho_{sys.enabling}$ &$=$& $((\bar{s_0} \wedge \bar{s_1} ) \implies (\bar{\hat{P}}' \wedge \bar{\hat{R}}')) \wedge$\\
	&& $((s_0 \vee s_1 ) \implies ((\hat{P}'\wedge \bar{\hat{R}}') \vee (\bar{\hat{P}}' \wedge  \hat{R})))$\\		
	$\rho_{update.states}$ &$=$& $((\bar{s_0} \wedge \bar{s_1} \wedge \hat{p}'\wedge \bar{\hat{r}}' \wedge \bar{\hat{P}}' \wedge \bar{\hat{R}}') \implies (\bar{s_0}' \wedge \bar{s_1}')) \wedge$\\
	&&$((\bar{s_0} \wedge \bar{s_1} \wedge \bar{\hat{p}}' \wedge \hat{r}'\wedge \bar{\hat{P}}' \wedge \bar{\hat{R}}') \implies (s_0' \wedge \bar{s_1}')) \wedge$\\	
	&&$((s_0 \wedge \bar{s_1} \wedge \bar{\hat{p}}' \wedge \bar{\hat{r}}' \wedge \bar{\hat{P}}' \wedge \hat{R}')\implies (\bar{s_0}' \wedge \bar{s_1}')) \wedge$\\	
	&&$((s_0 \wedge \bar{s_1} \wedge \bar{\hat{p}}' \wedge \bar{\hat{r}}' \wedge \hat{P}'\wedge \bar{\hat{R}}') \implies (\bar{s_0}' \wedge s_1')) \wedge$\\	
	&&$((\bar{s_0} \wedge s_1 \wedge \bar{\hat{p}}' \wedge \bar{\hat{r}}' \wedge \bar{\hat{P}}' \wedge \hat{R}')\implies (\bar{s_0}' \wedge \bar{s_1}')) \wedge$\\		
	&&$((\bar{s_0} \wedge s_1 \wedge \bar{\hat{p}}' \wedge \bar{\hat{r}}' \wedge \hat{P}'\wedge \bar{\hat{R}}') \implies (s_0' \wedge s_1')) \wedge$\\		
	&&$((s_0 \wedge s_1 \wedge \bar{\hat{p}}' \wedge \bar{\hat{r}}' \wedge \bar{\hat{P}}' \wedge \hat{R}')\implies (\bar{s_0}' \wedge \bar{s_1}'))$\\	
	$\rho_{update.propositions}$ &$=$& $((\bar{s_0} \wedge \bar{s_1}) \implies (\bar{\hat{e}}' \wedge \bar{\hat{a}}' \wedge \bar{\hat{b}}')) \wedge$\\
	&& $((s_0 \wedge \bar{s_1}) \implies (\hat{e}' \wedge \bar{\hat{a}}' \wedge \bar{\hat{b}}')) \wedge$\\
	&& $((\bar{s_0} \wedge s_1) \implies (\bar{\hat{e}}' \wedge \hat{a}' \wedge \bar{\hat{b}}')) \wedge$\\
	&& $((s_0 \wedge s_1) \implies (\bar{\hat{e}}' \wedge \bar{\hat{a}}' \wedge \hat{b}'))$\\	
	$\gsRhoE$&$=$&$\rho_{env.enabling}$\\		
	$\gsRhoS$&$=$&$\rho_{sys.enabling} \wedge  \rho_{update.states} \wedge  \rho_{update.propositions}$\\	
\end{tabular}
\vspace{1em}
\normalsize

If we keep only the behavior that conforms to $\theta_e, \theta_s, \rho_e$ and $\rho_s$ this translation induces the Kripke automaton depicted in figure ~\ref{fig:clts_to_fds_K}.

\begin{figure}[bt]
	\centering
	%\minipage{0.32\textwidth}%
	%\centering
	\VCDraw{
		\begin{VCPicture}{(-6.5,-2.5)(6.5,3.5)}
			\SetStateLabelScale{1}
			\SetEdgeLabelScale{1.2}
			\StateVar[0:\overline{s_0 s_1 e a b p r P R}]{(-6,2.3)}{0}
			\StateVar[1:\overline{s_0 s_1 e a b}p \overline{r P R}]{(-2,2.3)}{1}
			\StateVar[2:\overline{s_0 s_1 e a b p r P} R]{(-2,-0.3)}{2}
			\StateVar[3:s_0\overline{s_1} e  \overline{a b p} r \overline{P R}]{(2,2.3)}{3}
			\StateVar[4:\overline{s_0}s_1 \overline{e} a  \overline{b p r } P \overline{R}]{(6,1.3)}{4}
			\StateVar[5:s_0 s_1 \overline{e a} b  \overline{ p r} P \overline{R}]{(2,-0.3)}{5}			
			\Initial[w]{0}
			\CLoopSW{1}{}			
			\EdgeL{0}{1}{}        
			\ArcL{0}{3}{}  			
			\EdgeL{2}{1}{}        		
			\EdgeL{1}{3}{}        		
			\EdgeR{3}{2}{}        		
			\EdgeR{2}{3}{}        					
			\ArcL{3}{4}{}        		
			\ArcR{4}{2}{}        		
			\ArcL{4}{5}{}        		
			\EdgeL{5}{2}{}        		
		%	\VArcR[.5]{arcangle=45,ncurv=.6}{6}{0}{}        						
		%	\VArcL[.5]{arcangle=-45,ncurv=.6}{6}{1}{}        									
		\end{VCPicture}
	}	
	\caption{Kripke structure induced from $\theta$ and $\rho$}
	\label{fig:clts_to_fds_K}
	%\endminipage\hfill
\end{figure}

We then use $\theta_e$, $\theta_s$, $\rho_e$ and $\rho_s$ to write \varphiLTL and solve \fdsEmbeddingDef. Following the construction of the FDS controller from a given strategy as in ~\cite{bloem2012synthesis}, we describe the transformation of $f$ into $\hat{f}$ as:
\[\hat{f} = \bigwedge_{m \in M}\bigwedge_{s \in W_s}\bigwedge_{s'_{\gsX} \in \Sigma_{\gsX}} ((m \wedge s \wedge s'_{\gsX})\implies f(m,s,s'_{\gsX})')\]
Where $M$ is the memory domain and $W_s$ is the set of winning states. We can see that this transformation maps the function:
\[f: M \times \Sigma \times \Sigma_{\gsX} \mapsto M \times \Sigma_{\gsY}\ \]
To:
\[\hat{f}: M \times \Sigma \times \Sigma_{\gsX'} \mapsto M' \times \Sigma_{\gsY'}\ \]
While restricting only those states within the winning region.
The FDS controller is built as $\mathcal{D} = \langle \hat{\gsV}, \hat{\theta}, \hat{\rho} \rangle$ where:

\vspace{1em}
\begin{tabular}{ l c l }
	$\hat{\gsV}$ & $=$ & $\gsX \cup \gsY \cup \mathcal{M}$\\	
	$\hat{\theta}$ & $=$ & $(\theta_e \implies (\theta_s \wedge m_0 \wedge W_s))$\\
	$\hat{\rho}$ & $=$ & $((W_s \wedge \rho_e) \implies \hat{f})$\\	
\end{tabular}
\vspace{1em}

We describe $\hat{f}$ extensively only for the part of its domain that satisfies $\theta_e$, $\rho_e$ and $W_s$ since any other values are assigned arbitrarily. The rightmost term indicates the transition from the automaton in figure ~\ref{fig:clts_to_fds_K} being kept for each row.

\vspace{1em}
\begin{tabular}{ r r r l l r r l l r}
	$\hat{f}($ & $\overline{m_0},$ & $\overline{s_0 s_1},$ &$\overline{r'}$ & $) =$ & $($ & $\overline{P'R'},$ & $\overline{m_o'}$ & $)$ & $(0 \rightarrow 0)$\\
	$\hat{f}($ & $\overline{m_0},$ & $\overline{s_0 s_1},$ &$r'$ & $) =$ & $($ & $P'\overline{R'},$ & $\overline{m_o'}$ & $)$ & $(0 \rightarrow 1)$\\	
	$\hat{f}($ & $\overline{m_0},$ & $s_0\overline{s_1},$ &$\top$ & $) =$ & $($ & $P'\overline{R'},$ & $m_o'$ & $)$ & $(1 \rightarrow 2)$\\	
	$\hat{f}($ & $m_0,$ & $\overline{s_0} s_1,$ &$\top$ & $) =$ & $($ & $P'\overline{R'},$ & $\overline{m_o'}$ & $)$ & $(2 \rightarrow 3)$\\	
	$\hat{f}($ & $\overline{m_0},$ & $s_0 s_1,$ &$\top$ & $) =$ & $($ & $\overline{P'}R',$ & $\overline{m_o'}$ & $)$ & $(3 \rightarrow 0)$\\	
\end{tabular}
\vspace{1em}



If we restrict the behavior of the model that was shown in figure ~\ref{fig:clts_to_fds_K} after applying $\hat{f}$ we get the Kripke automaton from figure  ~\ref{fig:clts_to_fds_F}.

\begin{figure}[bt]
	\centering
	%\minipage{0.32\textwidth}%
	%\centering
	\VCDraw{
		\begin{VCPicture}{(-6.5,-2.5)(6.5,3.5)}
			\SetStateLabelScale{1}
			\SetEdgeLabelScale{1.2}
			\StateVar[0:\overline{s_0 s_1 e a b p r P R}]{(-6,2.3)}{0}
			\StateVar[1:\overline{s_0 s_1 e a b}p \overline{r P R}]{(-2,2.3)}{1}
			\StateVar[2:\overline{s_0 s_1 e a b p r P} R]{(-2,-0.3)}{2}
			\StateVar[3:s_0\overline{s_1} e  \overline{a b p} r \overline{P R}]{(2,2.3)}{3}
			\StateVar[4:\overline{s_0}s_1 \overline{e} a  \overline{b p r } P \overline{R}]{(6,1.3)}{4}
			\StateVar[5:s_0 s_1 \overline{e a} b  \overline{ p r} P \overline{R}]{(2,-0.3)}{5}			
			\Initial[w]{0}
			\CLoopSW{1}{}			
			\EdgeL{0}{1}{}        
			\ArcL{0}{3}{}  			
			\EdgeL{2}{1}{}        		
			\EdgeL{1}{3}{}        		
			\EdgeR{2}{3}{}        					
			\ArcL{3}{4}{}        		
			\ArcL{4}{5}{}        		
			\EdgeL{5}{2}{}        		
			%	\VArcR[.5]{arcangle=45,ncurv=.6}{6}{0}{}        						
			%	\VArcL[.5]{arcangle=-45,ncurv=.6}{6}{1}{}        									
		\end{VCPicture}
	}	
	\caption{Kripke structure induced from $\theta$ and $\rho$ after strengthening through $\hat{f}$}
	\label{fig:clts_to_fds_F}
	%\endminipage\hfill
\end{figure}

We can get our CLTS controller from \fdsD by applying $clts(\fdsD)$ to get the CLTS depicted in figure ~\ref{fig:clts_to_fds_C} which is a solution to \controlProblem.

\begin{figure}[H]
	\centering
	%\minipage{0.32\textwidth}%
	%\centering
	\VCDraw{
		\begin{VCPicture}{(-2.5,-1)(2.5,3.5)}
			\SetStateLabelScale{1}
			\SetEdgeLabelScale{1.2}
			\State[0]{(-2,2.3)}{0}
			\State[1]{(2,2.3)}{1}
			\State[2]{(2,-0.3)}{2}
			\State[3]{(-2,-0.3)}{3}
			\Initial[w]{0}
			\ArcL{0}{1}{<r>}        
			\LoopN{0}{<p>}		
			\ArcL{1}{2}{<P>}        
			\ArcL{2}{3}{<P>}        
			\ArcL{3}{0}{<R>}        						
		\end{VCPicture}
	}
	\caption{Controller $C$ translated from $\fdsD$}
	\label{fig:clts_to_fds_C}
	%\endminipage\hfill
\end{figure}