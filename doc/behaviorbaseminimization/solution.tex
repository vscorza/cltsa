\section{Behavior Minimization Procedures}\label{sect:solution}

The strategy to find a minimal non realizable control problem is to incrementally remove transitions preserving an alternating Sub-LTS relation until a local minimum is reached.  A minimal non realizable control problem is reached when any further reduction of the current Sub-LTS renders the problem realizable.
The first procedure explores the semi-lattice defined by Alternating Sub-LTS relation starting with the original automaton and moving downwards by performing minimal reductions. 
  
\begin{figure}[bt]
\centering
\SmallPicture
%\ShowFrame
\VCDraw{
    \begin{VCPicture}{(-4,-1.5)(4,4)}
        \SetEdgeLabelScale{1.4}
        \State[1]{(-3,0)}{1}
        \State[2]{(0,1.5)}{2}        
        \State[3]{(3,3)}{3}
        \State[4]{(3,1.5)}{4}        
        \State[5]{(0,0)}{5}                
        \State[6]{(3,0)}{6}                        
        \State[7]{(0,-1.5)}{7}                                
        \State[8]{(3,-1.5)}{8}                                        
		\Initial[w]{1}
		\EdgeL{1}{2}{u_1}
		\EdgeL{1}{5}{u_2}						
		\LoopE{6}{g}
		\EdgeL{1}{7}{u_3}										
        \ChgEdgeLineStyle{dashed} 
		\EdgeL{2}{3}{c_1}		
		\EdgeL{2}{4}{c_2}				
		\EdgeL{5}{6}{c_1}								
		\EdgeL{7}{8}{c_2}												
    \end{VCPicture}
}
\caption{Several conflicts example ($E$).}
\label{fig:fig.several-conflicts}
\MediumPicture
\end{figure}

\begin{figure}[bt]
\centering
\SmallPicture
%\ShowFrame
\VCDraw{
    \begin{VCPicture}{(-4,0)(4,1.5)}
        \SetEdgeLabelScale{1.4}
        \State[1]{(-3,0)}{1}
        \State[2]{(0,0)}{2}        
        \State[3]{(3,1.5)}{3}
        \State[4]{(3,0)}{4}        
		\Initial[w]{1}
		\EdgeL{1}{2}{u_1}
        \ChgEdgeLineStyle{dashed} 
		\EdgeL{2}{3}{c_1}		
		\EdgeL{2}{4}{c_2}				
    \end{VCPicture}
}
\caption{Several conflicts example minimization($E_1$).}
\label{fig:fig.several-conflicts-min1}
\MediumPicture
\vspace{-1em}
\end{figure}

\begin{figure}[bt]
\centering
\SmallPicture
%\ShowFrame
\VCDraw{
    \begin{VCPicture}{(-4,0)(4,1)}
        \SetEdgeLabelScale{1.4}
        \State[1]{(-3,0)}{1}
        \State[7]{(0,0)}{7}        
        \State[8]{(3,0)}{8}        
		\Initial[w]{1}
		\EdgeL{1}{7}{u_3}
        \ChgEdgeLineStyle{dashed} 
		\EdgeL{7}{8}{c_2}				
    \end{VCPicture}
}
\caption{Several conflicts example minimization($E_2$).}
\label{fig:fig.several-conflicts-min2}
\MediumPicture
\vspace{-1em}
\end{figure}

Consider the example of Figure~\ref{fig:fig.several-conflicts} where the system has to satisfy the liveness goal of
achieving $g$ infinitely often (expressed
by the LTL formula $\square \Diamond g$)
with $\Sigma = \{u_1, u_2, u_3, c_1, c_2, g\}$ and 
$\mathcal{C} = \{c_1, c_2, g\}$, the original LTS $E$ will be
at the top of the semi lattice while $E_1$ (depicted in figure \ref{fig:fig.several-conflicts-min1}) and $E_2$ (depicted in 
figure \ref{fig:fig.several-conflicts-min2}) will be at 
the bottom.  In between there is, for example, the Alternating Sub-LTSs
obtained by removing just one non controllable transition.  
The automaton obtained by removing $\{u_1, u_3\}$ is not shown
since it is realizable. 
%
%\begin{definition}\label{def:minimal-non realizability legal-plant}\emph{(Minimal Counter-Strategy Preserving Sub-LTS)}
%Given a control problem $\mathcal{I'}=<E', \mathcal{C}, \varphi>$
%where $E' = (S_{E'}, \Sigma, \Delta_{E'}, s_0)$ we say that 
%$E'$ is a minimal Alternating Sub-LTS for $E$
%if $I'$ is not realizable then the following holds:
%\[
%\begin{matrix}
%\forall E'' \text{s.t. } E'' \text{ is a closest Alternating Sub-LTS for } E'\\
%\implies\\
%\mathcal{I}''=<E'', \mathcal{C}, \varphi> \text{ is realizable}\\
%\end{matrix}
%\]
%
%\end{definition}
%
%This definition (\ref{def:minimal-non realizability legal-plant}) encapsulates 
%what we expect from $min(\mathcal{I})$, i.e., if 
%$min(\mathcal{I})= \mathcal{I}'$, 
%$\mathcal{I}'=<E', \mathcal{C}, \varphi>$ then
%$E'$ should be a minimal Alternating Sub-LTS for $E$.

%The minization algorithm described below yields
%one minimal $E'$ Alternating Sub-LTS
%for $E$ and $\mathcal{C}$.
%First we remove all controllable transitions from mixed states since this
%is exactly what the realizability check does before computing the winning region
 
%then
%non controllable transitions are progressively removed from the automaton under
%minimization (this is what we call $\mathcal{T}_u$ or candidate set),
%the closest Alternating Sub-LTS is computed from the resulting partial structure 
%and a one step back track is fired every time a realizable Alternating Sub-LTS is found.
% A minimal Alternating Sub-LTS is found when no further reduction is available.

\begin{figure}[ht]
  \begin{center}
    \renewcommand{\ttdefault}{pcr}
\begin{lstlisting}[escapeinside={[*}{*]},basicstyle=\scriptsize\ttfamily,columns=flexible,frame=lines,mathescape=true,keywordstyle=\textbf,morekeywords={if,while,do,else,fork,int,null, algorithm, is, input, output, return},numbers=left,numberstyle=\scriptsize,numbers=none]
algorithm dfs_min is
	input: $E$ an LTS representing the original plant
	output: $E'$ an LTS that satisfies the problem statement
	$E$ = $E$.prune_mixed_states()[*\listRef{A}*]
	$E_{new}$ = $E'$ = $E$ 
	$\mathcal{T}_u$ = $E$.non_controllable_transitions() [*\listRef{B}*]
	to_remove = [] [*\listRef{C}*]
	while(|$\mathcal{T}_u$| > 0)
		$t$ = $\mathcal{T}_u$.pop()
		while($\neg E'$.contains($t$) or $d_{out}$($E'$,t.from) == 1)[*\listRef{D}*]
			if (|$\mathcal{T}_u$| == 0)
				return $E$
			$t$ = $\mathcal{T}_u$.pop() [*\listRef{E}*]		
		to_remove.add($t$)
		$E_{new}$ = $E$.remove(to_remove) [*\listRef{F}*]
		
		is_realizable = $E_{new}$.realizable() [*\listRef{G}*]
		
		if ($\neg$ is_realizable):
			$E'$ = $E_{new}$
		else
			to_remove.remove($t$) [*\listRef{H}*]
	return $E'$  [*\listRef{I}*]
\end{lstlisting} 
    \caption{Minimization Algorithm}
    \label{fig:dfs-code}
  \end{center}
\vspace{-1em}
\end{figure}







%\begin{proofsketch}
%By way of contradiction, if $\mathcal{I}_1$ is realizable but
%$\mathcal{I}_2$ is not, then the strategy for the environment
%on $E_2$ should not hold on $E_1$.  But since we know that
%$E_2 \sublts E_1$, this does not hold as a consequence of lemma ~\ref{theorem:theo.preserves-non-realizability}, hence proving the 
%contradiction.
%\end{proofsketch}

%Proof for Lemma \ref{def:induction-preserves}


\textbf{HAY QUE REVER ESTA PORQUE CAMBIO LA DEFINICI‰ON DE SUBLTS}
\begin{proof}
If the system has a winning strategy in $G(\mathcal{I}_1)$ but no
winning strategy in $G(\mathcal{I}_2)$, this would imply that
the environment is able to either take the play into a deadlock state
or a $\sigma$-trap that always falsifies the property $\varphi$. 
Suppose that the system is able to play according to the strategy up
to state $s_i$ in $G(\mathcal{I}_2)$, after this point, for every choice the
system makes the environment has a way to construct a play
$\pi$ that is winning for him, either finitely or infinitely.
The departure at state $s_i$ must have been introduced by 
restricting the system choice, removing a controllable transition
which is not allowed for Alternating Sub-Games, or giving the environment
more power, by adding a non controllable transition or removing all of them
at a non controllable state thus introducing a deadlock which is also
forbidden since by Alternating Sub-Game definition
inclusion is satisfied.  Is proven by contradiction
that realizability is preserved between Alternating Sub-LTSs.
\end{proof}



We can now describe the algorithm as
presented in figure \ref{fig:dfs-code}. The algorithm first removes all controllable transitions from
mixed states \listRef{A}, this step can be performed since it
can be interpreted as the successive removals of controllable 
transitions with each satisfying the Alternating Sub-LTS definition 
and preserving non realizability because if it was non realizable
for the mixed state it will remain non realizable when the controllable transitions are removed since int the game built from
the automaton at each mixed state the environment
wins the race condition, always picking the next move when allowed.
The next step initializes the set of 
candidate transitions
for removal at \listRef{B} and the effective list
to be removed at each iteration \texttt{to\_remove} at
\listRef{C}.  The latter keeps track of the transitions
removed along our search down the space of
Alternating Sub-LTSs for $E$.  The exploration
will continue until the candidate set is empty.
At each cycle starting at \listRef{D} the set is updated \listRef{E} by 
removing those transitions that were already removed by the reduction
at \listRef{F} or because they would induce a deadlock 
($d_{out}(E', t.from) == 1$ checks if there is only one
outgoing transition in the state from which \textbf{t} originates).
The
reduction, at \listRef{F}, looks for the closest
Alternating Sub-LTS obtained after removing 
transitions in \texttt{to\_remove} from $E$.
Then it checks for realizability \listRef{G}.  If the control problem is non
realizable the candidate set and the last 
Alternating Sub-LTS $E'$ are updated (since further minimization is 
possible), otherwise a one step backtrack
is triggered by removing the transition under test from
the effective set \texttt{to\_remove} \listRef{H}.
This is actually the backtracking point where the algorithm
shifts the search sideways in the semi-lattice instead of downwards.
Once the candidate set has been completely explored
the last non realizability preserving LTS $E'$ is
returned as a minimal Alternating Sub-LTS
for $E$ and $\mathcal{C}$ \listRef{I}.
Notice that
in the step \listRef{H} the transition $t$ was removed
from the candidate set but was retained in $E'$.  What happens if
the transition could have been removed further down the semi lattice?  Removing $t$ from $E'$ and reducing the Alternating Sub-LTS 
$E_{new}$ accordingly would render the control problem realizable.  Suppose that
the search continued preserving $t$ and removing other candidates
$t_1, \ldots, t_n$ from $E'$ getting the Alternating Sub-LTS
$E_{1,\ldots,n}$ that still contains $t$ while preserving non realizability.
If we were to remove $t$ from $E_{1,\ldots,n}$ after reducing it
to the closest Alternating Sub-LTS we would get a new LTS
$E'_{new}$ that can be Alternating Sub-LTS for $E_{new}$, and from lemma
\ref{def:induction-preserves} it will render the control problem
realizable.  We showed this way that if we are looking for just one
conflict we can immediately remove the transition that yielded realizability 
from the candidate set without jeopardizing minimality.
%%optimizations
\subsection{Strip Representatives}
We introduce now two different optimizations.  The first
takes advantage of the nature of induced plants.  
No new deadlock can be introduced when searching for induced plants
(since it will violate the non realizability legal property), 
this leads to an optimization where we can take a representative 
for every strip within the
automaton that contains a monitored transition.

\begin{definition}\label{def:strip}\emph{(Strip)}
Given $M = (S_M, L, \Delta_M, s_{M_0})$ an LTS and 
$\mathcal{C}$ a set of controllable actions, 
$\varpi \in \Delta_M*$ and $\varpi_i = (s_i, a_i, t_i)$ stand for
the transition at position $i$ within the sequence, we will
say that $\varpi$ is a strip if the following holds: 
\[
\begin{matrix}{r l}
\forall i \in 1 \ldots |\varphi| - 1:& \varpi_i = (s_i,a_i,t_i), \varphi_{i+1} = (s_{i+1} = t_i, a_{i+1}, t_{i+1})\\
& d_{out}(t_i) = 1\\
\exists i \in 1 \ldots |\varphi| - 1: & a_i \not\in \mathcal{C}\\
\end{matrix}
\]
\end{definition}

We want to identify these strips since taking any monitored transition
from the strip out the plant and then reducing will lead to same
induced plant.  It is clear to see that under the non realizability
legal reduction it is equivalent to remove any subset of monitored transitions
within the strip since the algorithm will keep removing them until
no new deadlock is found, and since the strip has outgoing degree $=1$
all the way through, if we were to remove an element of $\varpi$ we will
be forced to remove all of them.\\
This optimization gains relevance since the complexity of our minimization
is driven by the number of monitored transitions to check.
W.l.o.g we can pick any monitored transition for each strip and use them
as the set of elements to check for minimization.
We can build this set as explained in the algorithm from Figure
\ref{fig:strip-code}.
\begin{figure}[ht]
  \begin{center}
    \renewcommand{\ttdefault}{pcr}
\begin{lstlisting}[escapeinside={[*}{*]},basicstyle=\scriptsize\ttfamily,columns=flexible,frame=lines,mathescape=true,xleftmargin=3.0ex,keywordstyle=\textbf,morekeywords={if,while,do,else,fork,int,null},numbers=left,numberstyle=\scriptsize]
build_representatives($\mathcal{I} = <E, \mathcal{C}, \varphi>$):
	candidates = $E$.get_monitored_transitions() [*\listRef{A}*]
	representatives = []
	while candidates != $\emptyset$
		$t$ = candidates.pop() [*\listRef{B}*]
		head = $t$
		tail = $t$
		representatives.push($t$)
		while $d_{in}$(head.from_state) == 1 
			or $d_{out}$(tail.to_state) == 1:
			if $d_{in}$(head.from_state) == 1: [*\listRef{C}*]
				if head.action $\not\in \mathcal{C}$:
					candidates.remove(head) [*\listRef{D}*]
				head = head.predecessor
			if $d_{out}$(tail.to_state) == 1: [*\listRef{E}*]
				if tail.action $\not\in \mathcal{C}$:
					candidates.remove(tail)
				tail = tail.successor
	return representatives
\end{lstlisting} 
    \caption{Strip representatives computation}
    \label{fig:strip-code}
  \end{center}
\end{figure}
The set of monitored transitions is taken as the initial candidates
\listRef{A} and then checks what we could call the strip closure
for the remaining elements \listRef{B} by adding them to the representatives
array. Then it recursively picks the preceding transition whenever
the source state has an incoming degree of 1 \listRef{C}, it removes
monitored transitions found this way \listRef{D}.  The same procedure
is performed onwards for target states with an outgoing degree of 1
\listRef{E}. We can see that the set of representatives is constructed 
this way after a linear scan of a collection of size at most $|\Delta|$.

\subsection{Radial Approximation}
Under the hypothesis that for certain scenarios (see section
\ref{section:validation-ctv}) counterstragies
 can be built over a set of states close to $s_0$ it is tempting
 to perform what we will call a radial approximation 
 of $E$ when checking for a minimal induced plant.

\begin{figure}[ht]
  \begin{center}
    \renewcommand{\ttdefault}{pcr}
\begin{lstlisting}[escapeinside={[*}{*]},basicstyle=\scriptsize\ttfamily,columns=flexible,frame=lines,mathescape=true,xleftmargin=3.0ex,keywordstyle=\textbf,morekeywords={if,while,do,else,fork,int,null},numbers=left,numberstyle=\scriptsize]
smallest_non_realizable_approximation($E$):
	max_d = $E$.get_maximum_distance()
	low = 0
	hi = max_d
	$k$ = $\lfloor \frac{hi}{2}\rfloor$ [*\listRef{A}*]
	$E_{old}$ = $E$
	$E_{new}$ = $E$
	while $k$ != -1:
		if hi - low < 1:
			return $E_{old}$
		$E_{new}$ = $E$.clone()
		$E_{new}$.prune_at($k$) [*\listRef{B}*]
		$E_{new}$.get_legal_reduction($E$)[*\listRef{C}*]
		is_realizable = check($E_{new}$)[*\listRef{D}*]
		if is_realizable:
			low = $k$ [*\listRef{E}*]
		else:
			hi = $k$ [*\listRef{F}*]
			$E_{old}$ = $E_{new}$ [*\listRef{G}*]
		$k$ = low + $\lfloor \frac{hi - low}{2}\rfloor$ [*\listRef{H}*]
		if $k$ <= 1:
			return $E_{old}$
	return $E_{old}$
\end{lstlisting} 
    \caption{Binary approximation computation}
    \label{fig:kapprox-code}
  \end{center}
\end{figure}

For the approximation described in Figure \ref{fig:kapprox-code}
we use the distance of any state to $s_0$ (\texttt{max\_d}) to guide the binary search of (non minimal) induced plant.  We will prune $E$ removing states
at distances greater than $k$ from $s_0$ performing a non realizability legal
reduction and the checking for realizability.
This approximation can be regarded as a pre processing of the plant, since it
will (hopefully) perform a faster legal reduction by sacrificing minimality.
It computes $k$ by performing a binary search over the distance at which the
plant will be pruned and reduced.  The algorithm starts by assigning the
value $\lfloor \frac{hi}{2}\rfloor$ \listRef{A} for $k$ 
and then it keeps pruning \listRef{B}, reducing \listRef{C} and checking
for realizability \listRef{D} as long as the search can continue.  The search
range is updated by raising the lower bound if the check yielded a positive 
value \listRef{E} or lowering the upper bound if the check informed that the
problem was unrealizable \listRef{F}.  The last non realizable plant is updated 
in this case since the reduction ensures legality while preserving 
non realizability.  $K$ is consequently updated \listRef{H} and the search
carries on until the boundaries converge.  This procedure will perform
up to $log_2(|\Delta|)$ checks to yield an approximated induced plant, which
may not be minimal but can still be given to one of the other minimization
techniques to achieve a minimal induced plant.



%\begin{figure}[ht]
%  \begin{center}
%    \renewcommand{\ttdefault}{pcr}
\begin{lstlisting}[escapeinside={[*}{*]},basicstyle=\scriptsize\ttfamily,columns=flexible,frame=lines,mathescape=true,xleftmargin=5.0ex,keywordstyle=\textbf,morekeywords={if,while,do,else,fork,int,null},numbers=left,numberstyle=\scriptsize]

dd_min($E$):
	$\mathcal{T}_u$ = $E$.get_monitored_transitions()
	return dd_min'($\mathcal{T}_u$, $[ ]$, 2, $E$) [*\listRef{A}*]
dd_min'($\mathcal{T}_{u}$, $\mathcal{T}_{r}$, $n$, $E$):
	if $n$ <= $|\mathcal{T}_{u}|$:
		res = dd_subsets($\mathcal{T}_{u}$, $\mathcal{T}_{r}$, $n$, $E$, $\bot$)
		if res != $None$:
			return res
		if $n$ > 2:
			res = dd_subsets($\mathcal{T}_{u}$, $\mathcal{T}_{r}$, $n$, $E$, $\top$)
			if res != $None$:
				return res		
		if $n$ < $|\mathcal{T}_{u}|$:
			$n'$ = min($|\mathcal{T}_{u}|$, 2 * $n$)
			return dd_min'($\mathcal{T}_{u}$, $\mathcal{T}_{r}$, $n'$, $E$)
	$E_{new}$ = $E$.clone()	
	for $t$ in $\mathcal{T}_{u}$:
		$E_{new}$ = $E_{new}$.remove($t$)
	$E_{new}$.get_legal_reduction() [*\listRef{B}*]
	$\mathcal{T}'_u$ = $E_{new}$.get_monitored_transitions()	
	return $E_{new}$
dd_subsets($\mathcal{T}_{u}$, $\mathcal{T}_{r}$, $n$, $E$, check_complement):
	step = $\lfloor \frac{|\mathcal{T}_{u}|}{n}\rfloor$ [*\listRef{A}*]
	rem = $|\mathcal{T}_{u}| \% n$
	first = 0, last = -1
	for $i$ in $[0\ldots n-1]$:
		first = last + 1
		last = first + step - 1
		if rem > 0:
			last += 1
			rem -= 1
		if check_complement:
			$\mathcal{T}'_{u}$ = $\mathcal{T}_{u} \setminus \mathcal{T}_{u}[first\ldots last]$
			$\mathcal{T}'_{r}$ = $\mathcal{T}_{r} \cup (\mathcal{T}_{u} \setminus \mathcal{T}'_{u})$
			$n'$ = max(2, $n$ -1)
		else:
			$\mathcal{T}'_{u}$ = $\mathcal{T}_{u}[first\ldots last]$
			$\mathcal{T}'_{r}$ = $\mathcal{T}_{r} \cup (\mathcal{T}_{u} \setminus \mathcal{T}'_{u})$
			$n'$ = 2
		if test_subset($\mathcal{T}'_{r}$) == $\bot$:
			return dd_min'($\mathcal{T}'_{u}$, $\mathcal{T}'_{r}$, $n'$, $E$)						
	return $None$
\end{lstlisting} 
%    \caption{DD exploration code}
%    \label{fig:dd-code}
%  \end{center}
%\end{figure}



We introduced two modifications to the algorithm
to improve the time taken to compute the minimization.
The first optimization transforms the candidate set $\mathcal{T}_u$
into a list and orders it according to the distance of its elements 
with respect to the initial state.  The idea is to try and cut
as aggressively as possible first and see if a big cut still
preserves the non realizability cause.  The second optimization
looks for non controllable strips within the  input automaton
$E$, that is, sequences of non controllable actions without branches.
For each of these sequences only one action is
added to $\mathcal{T}_u$ since, if removed from $E'$
when looking for a minimal sub LTS all the others have to be removed
as well to preserve alternation.



The second procedure uses the Delta Debugging technique presented in ~\cite{DBLP:journals/tse/ZellerH02} to accelerate the search space exploration. Instead of reducing the automaton by applying an atomic reduction at a time Delta Debugging tries to achieve a minimal diagnosis by partitioning the problem $\Delta$ into $n$ uniform subsets ($\Delta = \bigcup_{i=1}^n \Delta_i $) and then it first checks its elements $\Delta_i$ for a non realizable smaller case, if none is found it proceeds by looking at the complement $\Delta \setminus \Delta_i$ of each element in the partition. If no smaller candidate was found it increases the granularity by updating $n$ as $max(2n,|\Delta|)$. This cycle is repeated until $n$ can not be further increased ($n \geq |\Delta|$). 
In our application of Delta Debugging the set $\Delta$ is the set of non controllable transitions in the original plant. Instead of descending regularly down the semi lattice, Delta Debugging tries to jump further down by removing a subset of transitions from the current candidate. We can briefly prove that the application of the technique is sound since Delta Debugging ensures 1-minimality, i.e. that removing any single element from the resulting minmization breaks non realizability which in conjunction with lemma ~\ref{def:induction-preserves} implies minimality.
