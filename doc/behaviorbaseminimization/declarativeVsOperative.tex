\subsection{On The Declarative vs Operational Approach}
To the best of our knowledge,
 previous work on the minimization of unrealizable
specifications has been focused on reducing the set of requirements expressed as LTL formulas.  
This is the approach taken in \cite{DBLP:conf/hvc/KonighoferHB10} and the
initial inspiration for our work.  Könighoefer et al. built their framework
from what was presented in \cite{DBLP:conf/vmcai/CimattiRST08} and in
\cite{DBLP:conf/vmcai/2008} where they took the \textit{delta debugging}
algorithm as a cost-effective minimization technique.\\
Even though the kind of properties being satisfied is the same, we differ
not only in the formalism under use but also in the nature of our minimization,
since we are not reducing the set of formulas but the explicit behavior expressed
as an automaton.\\
To clarify the difference we apply an automated translation
scheme to transform our operational control problem  into
a declarative specification.
The later is written as the conjunction of four \texttt{LTL} \cite{pnueli1977temporal}
formulas as presented in Table \ref{table:planar-robot-declarative-spec}. 
We follow the definitions from \cite{DBLP:conf/vmcai/PitermanPS06}. Our goal is to get a game structure that properly emulates the semantics
of the source control problem.  The idea is that both structures should
satisfy equivalent sets of LTL formulas.\\
The translation can be explained as follows:
we encode the automaton state space through a set of boolean system variables
that will express the binary representation at each particular state.  The 
target specification will have a boolean variable (or signal) for each source
event.  Monitored events from the source specification will be 
represented through environment variables ($\mathcal{X}$) and
controlled events through system variables ($\mathcal{Y}$).  
Suppose that we have the following control problem $\mathcal{I}=\langle E, \mathcal{C}, \varphi \rangle$,
with plant $E$ described as the LTKS $E = (S, \Sigma,\Delta, v:S \rightarrow 2^{\mathcal{F}} ,s_0)$, $\mathcal{F}=\{f_1, f_2 \}$ and where
$a,b,c \in \Sigma$, $c \in \mathcal{C}$, $v(s_i)=\{ f_1 \}, v(s_j)=\{f_2\}$ and
$(s_i, a, s_j)$,
$(s_j, a, s_k)$,
$(s_j,b,s_l)$,
$(s_j, c, s_m) \in \Delta$.
Each transition from the source automaton is translated into three 
different formulas ($enable_{sys}$, $enable_{env}$, $update$), two standing for enabling requirements:\\
\textbf{[$enable_{sys}$]}One for the environment ($env.enable_{i,a} \in \rho_e$) as an implication from current state 
and currently selected action into a disjunction of mutually exclusive conjunction 
of monitored event related signals declaring the set of next-step enabled
environment actions. Suppose that $s, act_c \in \mathcal{Y}$
and $act_a, act_b \in \mathcal{X}$.
\[
\square(s = i \wedge act_a \rightarrow \Circle((\neg act_a \wedge act_b)\vee(act_a \wedge \neg act_b) \vee (\neg act_a \wedge \neg act_b)))\]
\textbf{[$enable_{env}$]} One for the system ($sys.enable_{i,a} \in \rho_s$) as an implication from current state
and currently selected action into another implication 
from the negation of all monitored events to the
disjunction of mutually exclusive conjunction of controlled
event related signals declaring the set of next-step enabled
system actions.
\[
\square(s = i \wedge act_a \rightarrow \Circle((\neg act_a \wedge \neg act_b)\rightarrow (act_c)))\]
\textbf{[$update$]} The third transition requirement ($s.update_{i,a} \in \rho_s$) is the one 
effectively updating the automaton state representation.
\[
\square(s = i \wedge act_a \rightarrow \Circle(s = j))\]
Initial conditions ($\theta_s, \theta_e$) are set according to the initial state 
$s_0$ in the automaton and initially enabled actions.
The LTKS valuations, that predicate over the set $\mathcal{F}$ of 
fluents, is translated in with the proper system safety requirement.\\
Following this translation and starting with
the control problem presented for the planar robot
we get an equivalent game structure shown (abbreviated
leaving only formulas relevant to state $4$) in 
Table \ref{table:planar-robot-declarative-spec}.
\begin{center}
\begin{table}[h]
  \begin{tabular}{ L{2cm} c L{5cm} }
  \toprule
	Name  && Requirement \\
    Init. env.  &($\theta_e$) & $(sensor.disabled \veebar sensor.enabled)$ $ \wedge \neg timeout \wedge \neg data.rcvd$\\    
    Init. sys. &($\theta_s$) & $s = 0 \wedge \neg update.direction \wedge$ $\neg get.SLAM.location \wedge \neg get.GPS.location$\\
    
    \ldots  && \ldots \\
    
    $_{4 \rightarrow 3}$ $timeout$ &($\rho_s$)
    &$\square((s=4 \wedge timeout)\rightarrow \Circle(s=3))$ \\
    
    $_{4 \rightarrow 5}$ $data.rcvd$ &($\rho_s$)
    &$\square((s=4 \wedge data.rcvd)\rightarrow \Circle(s=5))$ \\
    
    \ldots  && \ldots \\
    
    $_{4\rightarrow 5}$ $data.rcvd$ enable$_{sys}$ &($\rho_s$)
    &$\square((s=4 \wedge data.rcvd)\rightarrow \Circle(update.direction \wedge \neg get.SLAM.location \wedge \neg get.GPS.location))$ \\    
    
    $_{4\rightarrow 5}$ $data.rcvd$ enable$_{env}$ &($\rho_e$)
    &$\square((s=4 \wedge data.rcvd)\rightarrow \Circle(\neg timeout \wedge \neg sensor.enabled \wedge \neg sensor.disabled \wedge \neg data.rcvd))$ \\        
    
    \ldots  && \ldots \\
    Goal   &($\varphi_s$) & $\square\Diamond(update.direction)$\\
    \bottomrule
  \end{tabular}
  \caption{Planar Robot Equivalent Game Structure}
  \label{table:planar-robot-declarative-spec}
 \end{table}
\end{center}
The Könighoefer et al. technique removes system requirements
only.  After applying their minimization the 
$_{4 \rightarrow 5} data.rcvd$ and $_{4 \rightarrow 5}$ $data.rcvd$ enable$_{sys}$ requirements are removed,
since relaxing the system this way does not affect 
the environment winning strategy.  The syntactic 
simplification actually augments the system transition
relation $\rho_{s}$ for the case where
$s=4 \wedge data.rcvd$ holds, implicitly adding a
$data.rcvd$ transition to every other state.  This
is feasible since the environment will always pick
$timeout$ over $data.rcvd$ when reaching state $4$. 
Aforementioned requirements 
($_{4 \rightarrow 5} data.rcvd$ and $_{4 \rightarrow 5}
data.rcvd$ enable$_{sys}$) are irrelevant to the 
non realizability cause, but removing them 
bloats the specification in terms of explicit behavior
since it adds more transitions.  We want to
improve the understanding of the problem by way
of simplification, this adds another point in favor
o a novel minimization technique, designed specifically
for behavior based (or operational) specifications.