\section{Introduction}\label{sec:introduction}
%specifications are hard to write properly and stay unrealizable for a long time


Requirements are naturally split between goals the system-to-be is expected to achieve and
assumptions that the system-to-be can rely on to fulfill its goals ~\cite{Jackson:1995,Letier:2002}. The question to be asked then is one of realizability: Is it possible to build a system that can monitor its environment and react through its capabilities in order to guarantee its goals as long as the environment fulfills the assumptions?

At early stages specifications are usually unrealizable ~\cite{Letier:2002}. There can be 
multiple causes for unrealizability including lack of monitorability and controllability 
and misconstrued goals or assumptions. Goals in their first formulations can  be 
stronger than what can be reasonably required and assumptions can be weak, failing to 
rule out exceptional circumstances that the system cannot deal with 
~\cite{vanLamsweerde:2000}. 

Unfortunately, the cause for unrealizability tends to be the result of a combination of issues and is not easy to detect or understand. Providing engineers feedback that allows them to understand the cause for unrealizability is highly desired if specifications are to be evolved into realizable ones that can then be implemented. 

The process of checking realizability is associated with producing a controller that can 
enact a strategy that correctly implements the specification. If such a controller exists, 
the specification is realizable and, in some contexts, the controller can be used as part of 
the system-to-be. If no such controller exists, the specification is unrealizable. 

Manually checking realizability is a laborious task. However, in some settings this 
process can be done automatically, ensuring as a by-product a controller  that correctly 
implements the specification.  Controller synthesis algorithms only produce a system 
strategy if the original specification is realizable. When the specification is unrealizable, 
the engineer is left with the onerous task of understanding why this is the case and, only 
after that, with the task of changing the specification.

Synthesis has been studied in three different fields: automata-based 
~\cite{allen1997formal, maggee1999concurrency, DBLP:journals/cacm/Hoare78, ramadge:1989, ClassicalMindBook}, automated planning ~\cite{fikes1971strips,penberthy1992ucpop} and reactive synthesis ~\cite{harel1985development,pnueli1985transition}.  Each one has different ways of defining the synthesis problem 
which impacts profoundly not only in the algorithms and data structures used to solve 
them but also the suitability for modeling different domains. 

Reactive synthesis assumes both a memory sharing communication model (where each party reads 
or writes a set of propositions that the other party can write or read) and 
\textit{synchronous execution} (where the environment and system-to-be advance 
taking turns to change all the propositions they control). This approach 
originates from its applications to hardware verification and synthesis~\cite{DBLP:conf/popl/PnueliR89} but has 
been applied to software too~\cite{manna1980deductive}. Temporal logics 
are commonly used to describe goals and assumptions. 

Event-based approaches adopt a communication model based on events (where only one 
instantaneous event can occur at any instant) and suppose \textit{asynchronous execution} (where parties can execute at 
different relative speeds) which is well suited for modeling message-based 
communication~\cite{DBLP:journals/cacm/Hoare78, Milner:1982} at the software architecture level, for instance. 
Communicating automata  and (asynchronous time or event-based) temporal logics are commonly used 
to model such synthesis problems~\cite{ramadge:1989}. 

Various techniques for providing feedback on unrealizability for reactive synthesis 
approaches have been proposed. In \cite{DBLP:conf/fmcad/KonighoferHB09, 
DBLP:journals/scp/Schuppan12,DBLP:conf/fmcad/AlurMT13,maoz2021unrealizable}
an expressive subset of Linear Temporal 
Logic (LTL) formulae, called \gr is considered, in the spirit of unsatisfiable cores~\cite{Torlak:2008} the result of these techniques is a minimal subset of formulae that is still 
unrealizable. The intention is to allow engineers to focus on a portion of the original 
specification to identify the causes for unrealizability. 
~\cite{DBLP:conf/fmcad/KonighoferHB09} also provides a counter-strategy over which 
either a counter-trace can be produced as a witness for unrealizability or a game can be 
interactively played with the user in order to aid in understanding the cause of the 
problem. In ~\cite{DBLP:conf/sigsoft/KuventMR17} a simplification of the 
counterstrategy in terms of attractors and cycles is provided, going further in the 
simplification of the cause by providing only relevant information in terms of valuation 
changes, this is also the starting point of other techniques, such as 
~\cite{maoz2019symbolic}, where ~\cite{DBLP:conf/sigsoft/KuventMR17} output is 
used to repair the initial GR(1) specification. In ~\cite{maoz2021unrealizable} the authors introduce a conceptual framework for incremental unrealizability core computation that lays the foundation for writing more effective minimization algorithms (e.g. \texttt{QuickCore}) while also providing a procedure that can compute all existing unrealizable cores (\texttt{Punch}), as in ~\cite{Torlak:2008} the output of these techniques is a minimal subset of formulae that preserve
unrealizability.

There are, however, no techniques for providing feedback on unrealizability for 
event-based control problems. \emph{In this paper, we propose an approach for  
providing  feedback on unrealizabilty of event-based control problems specified with a combination of 
automata and temporal logic}.
%
%- EXPLICAR PLANTA
%
%- Explicar minimización
%
%- Explicar ideas claves del cómo. Delta debugging, etc.

In the context of this work we refer to the automaton representation of all 
possible execution involving the environment and the system as the plant. Our 
technique reduces the plant to a minimal representation such that we do not restrict the 
controller capabilities but allow the 
environment to 
enact a strategy that both violates the desired property and can be applied in the 
original plant. We define a relation between automata that encapsulates the notion of 
only weakening the environment options while preserving those of the system. This 
relation is used to induce a semi-lattice, which serves as our search space, that in conjunction with the 
monotonicity of unrealizability is used to incrementally reduce the plant until a local 
minimum is reached. We present one algorithm that performs a linear exploration of the 
space and another that uses Delta-Debugging ~\cite{DBLP:journals/tse/ZellerH02} to 
 improve the time spent while diagnosing.


The paper is structured as follows. We first provide a motivating example as an informal 
overview of the proposed technique, then present some preliminary definitions, 
namely to define formally a control problem and the notion of realizability, and also define minimization and unrealizability preservation. We then present the minimization algorithm. We finalize with a discussion on 
related work and conclusions.
%
%
%TAL VEZ ALGO DE ESTO SE PUEDE REUTILIZAR  ARRIBA
%. In \textit{differs} from existing techniques in two ways. First and foremost, it works 
%with automata-based (explicit) models. To the best of our knowledge, no previous work 
%approached the problem of diagnosing unrealizability over explicit models.  When 
%working with symbolic definitions our tool creates an automaton representation of the 
%plant by translating an OBDD to a LTS. Automata-based specifications of the 
%environment (e.g., labeled transition systems~\cite{Keller:1976}, 
%statecharts~\cite{Harel:1987}, and process 
%algebra~\cite{Milner:1982,Hoare:1983})satisfy are common approaches to 
%environment description in software engineering literature. There are a variety of 
%software engineering tasks that have been approached using synthesis and 
%automata-based descriptions 
%(e.g,~\cite{Letier:2013:RMS,DIppolito:2013,Pistore:2004:PMW}. The explicit nature 
%of our output enables the use of the minimization when building a test suite or test 
%bench. The second difference is that the approach minimizes the initial plant, resulting 
%in an explicit model that exhibits a minimal representation of the original behavior,  
%Existing approaches either allow more behavior than originally specified when reducing 
%the number of formulae ~\cite{DBLP:conf/fmcad/KonighoferHB09} or require a 
%symbolic description of the plant in order to provide a automata-based representation of 
%the abstracted counter strategy ~\ref{DBLP:conf/sigsoft/KuventMR17}. 
%
%
%
%
%\textit{What does it mean to minimize behavior while preserving unrealizability?} A 
%specification determines a game in which one player, the environment, tries to  the 
%assumptions while preventing the other player, the system-to-be, from achieving its 
%goals. If the specification is unrealizable the environment has a playing strategy that 
%always beats its opponent. In other words, no matter what the system does, it cannot 
%achieve its goals. When talking about a minimization that preserves unrealizability we 
%refer to a process that produces a new model that has less behavior and for which the 
%winning strategy of the environment also works in the original specification. This allows 
%engineers to focus on the behavior of the original specification that causes 
%unrealizability. The importance of the technique lies not in how much of the general 
%plant is removed but on the benefits it presents during diagnosis, because a smaller plant 
%will in turn produce smaller, simpler traces for the user to explore, analyze and use as 
%guide towards the understanding of the underlying unrealizability cause.
%
%
%
%%The technique has also the potential to complement existing techniques that assume a specification only in LTL form and that work by performing a declarative, rather than a behavior-based, minimization.
%
%The paper is structured as follows. We first provide a motivating example as an informal overview of the proposed technique, we then present some preliminary definitions, namely to define formally a control problem and the notion of realizability, and then go on to define minimization and unrealizability preservation. In Section~\ref{sect:solution} we present the minimization algorithm and then discuss our way of validating the technique. We finally proceed with a discussion on related work and conclusions.