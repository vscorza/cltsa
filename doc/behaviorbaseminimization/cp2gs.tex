We propose a translation from control problems to game structures in order
to reason about our behavior based specifications in terms of existing 
work from the literature.  
Our goal is to get a game structure that properly emulates the meaning
of the source control problem.  The idea is that both structures should
satisfy equivalent sets of LTL formulas.  We show that given the proper
transformation from control problems to game structures this equivalence 
is quite straightforward.\\
The translation can be explained as follows:
we encode the automaton state space through a set of boolean system variables
that will express the binary representation of each particular state.  The 
target specification will have a boolean variable (or signal) for each source
event.  Monitored events from the source specification will be 
represented through environment variables ($\mathcal{X}$) and
controlled events through system variables ($\mathcal{Y}$).  
Suppose that we have the following control problem $\mathcal{L}=\langle E, H, L_c\rangle$,
with plant $E$ described as the LTKS $E = (S, L,\Delta, v:S \rightarrow 2^{\mathcal{F}} ,s_0)$, $\mathcal{F}=\{f_1, f_2 \}$ and where
$a,b,c \in L$, $c \in L_c$, $v(s_i)=\{ f_1 \}, v(s_j)=\{f_2\}$ and
$(s_i, a, s_j)$,
$(s_j, a, s_k)$,
$(s_j,b,s_l)$,
$(s_j, c, s_m) \in \Delta$.
Each transition from the source automaton is translated into three 
different formulas, two standing for enabling requirements:
One for the environment ($env.enable_{i,a} \in \rho_e$) as an implication from current state 
and currently selected action into a disjunction of mutually exclusive conjunction 
of monitored event related signals declaring the set of next-step enabled
environment actions. Suppose that $s, act_c \in \mathcal{Y}$
and $act_a, act_b \in \mathcal{X}$.
\[
\square(s = i \wedge act_a \rightarrow \Circle((\neg act_a \wedge act_b)\vee(act_a \wedge \neg act_b) \vee (\neg act_a \wedge \neg act_b)))\]
One for the system ($sys.enable_{i,a} \in \rho_s$) as an implication from current state
and currently selected action into another implication 
from the negation of all monitored events to the
disjunction of mutually exclusive conjunction of controlled
event related signals declaring the set of next-step enabled
system actions.
\[
\square(s = i \wedge act_a \rightarrow \Circle((\neg act_a \wedge \neg act_b)\rightarrow (act_c)))\]
The third transition requirement ($s.update_{i,a} \in \rho_s$) is the one 
effectively updating the automaton state representation.
\[
\square(s = i \wedge act_a \rightarrow \Circle(s = j))\]
Initial conditions ($\theta_s, \theta_e$) are set according to the initial state 
$s_0$ in the automaton and initially enabled actions.
The LTKS valuations, that predicate over the set $\mathcal{F}$ of 
fluents, is translated in with the proper system safety requirement.
We will present some basic definitions to simplify the translation:
\[
\square(s = i \rightarrow (f_1 \wedge \neg f_2))\]
\begin{definition}\label{def:state-u-events}\emph{(Monitored events per state)}
In the context of a control problem, let $s \in S$ be any given state 
\[u(s) = \{ u: \exists s' s.t. (s,u,s') \in \Delta \wedge u \in \Sigma \setminus \mathcal{C} \}\]
\end{definition}

\begin{definition}\label{def:state-c-events}\emph{(Controlled events per state)}
In the context of a control problem, let $s \in S$ be any given state 
\[c(s) = \{ c: \exists s' s.t. (s,u,s') \in \Delta \wedge u \in \mathcal{C} \}\]
\end{definition}

When not explicitly stated we will
assume that there is a control problem $\mathcal{I}= <E, \mathcal{C}, \varphi>$ with
$E = (S, \Sigma, \Delta, s_{M_0})$ and 
$\varphi = (\bigwedge_{i \in [1 \ldots n]} \phi_i \rightarrow \bigwedge_{j \in [1\ldots m]}\gamma_i)$.
The following equivalences will hold for every fluent
$f = <I_f, T_f>_{intially_f}$ and every $a \in \Sigma$:
\[
\begin{array}{r c l}
var(f) &=& \{ var(i) : i \in I_f \cup T_f \}\\
var(a) &=& 
\begin{cases}
x(a) \quad \text{if } a \in \Sigma \setminus \mathcal{C}\\
y(a) \quad \text{if } a \in \mathcal{C}\\
\end{cases}
\end{array}
\]

To represent the states of the automaton in the game structure
we will assume the existence of a mapping $index : S \rightarrow \{ 0 \ldots |S| -1 \}$.

\begin{definition}\label{def:state-encoding}\emph{(Automaton state encoding)}
Assume that $bit(i, v)$ gives the i-th boolean value of 
the binary representation of the integer $v$:
\[encode(s) = \bigwedge_{i=0}^{|S|-1} bit(i, index(s)) \]
\end{definition}



LTL formulas have to be adjusted by replacing control
problem fluent variables with their game structures counterparts.
Since propositions will predicate over different sets of variable
we proceed to define the \texttt{adequate} function.
Remember that (F)LTL formulas are constructed as follows:
\[\varphi ::= p | \neg \varphi | \varphi \vee \varphi | \Circle \varphi | \varphi \mathcal{U} \varphi\]
\begin{definition}\label{def:ltl-adequate}\emph{(FLTL to LTL formula translation)}
Let $\mathcal{F}$ be the set of fluents for the control problem 
$\mathcal{I}$ where each fluent can be described as the tuple
$fl = <I_{fl}, T_{fl}>_{initially_{fl}}$ and 
$y:FLTL \rightarrow \mathcal{Y}$ is a function mapping
control problem's fluents into game structure's system variables,
we introduce  $adequate:FLTL \rightarrow LTL$ a
translation from FLTL formulas into LTL formulas:
\[
\begin{array}{r c l}
adequate(\neg \varphi) &=& \neg adequate(\varphi)\\
adequate(\varphi \vee \psi) &=& adequate(\varphi) \vee adequate(\psi)\\
adequate(\Circle \varphi) &=& \Circle adeaquate(\varphi)\\
adequate(\varphi \mathcal{U} \psi) &=& adequate(\varphi) \mathcal{U} adequate(\psi)\\
adequate(fl) &=& y(fl)\\
\end{array}
\]
\end{definition}

We define the mutually exclusive conjunction of signals
as it conveniently simplifies other safety formulas.

\begin{definition}\label{def:mutex-signal}\emph{(Mutually exclusive conjunction of signals)}
Let $\sigma \subseteq \Sigma$:
\[mutex(\sigma) = \bigvee_{a \in \sigma} var(a) \wedge \bigwedge_{a' \in \sigma, a \neq a'} \neg var(a')\]
\end{definition}

Next we define the safety enabling conditions and requirements.

\begin{definition}\label{def:env-enabling-condition}\emph{(Environment enabling condition)}
Let $\sigma \subseteq \Sigma \setminus \mathcal{C}$ we define an environment enabling condition $enableCond_e(\sigma)$ for purely monitored states in $\mathcal{I}$ and 
$enableCond'_e(\sigma)$ for mixed states:
\[
\begin{array}{r c l}
enableCond_e(\sigma) &=& mutex(\sigma)\\
enableCond'_e(\sigma) &=& mutex(\sigma) \vee \bigwedge_{a \in \sigma} \neg var(a)\\
\end{array}
\]
\end{definition}


\begin{definition}\label{def:sys-enabling-condition}\emph{(System enabling condition)}
Let $\sigma \subseteq \mathcal{C}$ we define a system enabling condition $enableCond_s(\sigma)$ for controllable or mixed states in $\mathcal{I}$:
\[
enableCond_s(\sigma) = mutex(\sigma) \rightarrow \bigwedge_{u \in \Sigma \setminus \mathcal{C}}\neg var(u)
\]
\end{definition}

\begin{definition}\label{def:state-upadting-requirement}\emph{(State Updating Requirement)}
We define a state updating requirement $update_s(s,a,s')$ for states $s, s' \in S$ and event $a \in \Sigma$ in $\mathcal{I}$:
\begin{footnotesize}
\[
\begin{array}{r c l}
update_s(s, a, s') &=& 
\square(encode(s) \wedge var(a) \rightarrow \Circle(encode(s')))\\
\end{array}
\]
\end{footnotesize}
\end{definition}

\begin{definition}\label{def:env-enabling-requirement}\emph{(Environment enabling requirement)}
We define an environment enabling requirement $enable_e(s, a, s')$ for monitored states $s, s' \in S$ and event $a \in \Sigma \setminus \mathcal{C}$ in $\mathcal{I}$:
\begin{footnotesize}
\[
\begin{array}{r c l}
enable_e(s, a, s') &=& 
\begin{cases}
\square(encode(s) \wedge var(a) \rightarrow \Circle(enableCond_e(u(s'')))) 
\\
\quad \quad |c(s')| = 0\\
\square(encode(s) \wedge var(a) \rightarrow \Circle(enableCond'_e(u(s')))) 
\\
\quad \quad otherwise\\
\end{cases}
\\
\end{array}
\]
\end{footnotesize}
\end{definition}

\begin{definition}\label{def:system-enablig-requirement}\emph{(System Enabling Requirement)}
We define a system enabling requirement $enable_s(s, a, s')$ for states $s, s' \in S$ and event $a \in \Sigma$ in $\mathcal{I}$:
\begin{footnotesize}
\[
\begin{array}{r c l}
enable_s(s, a, s') &=& 
\square(encode(s) \wedge var(a) \rightarrow \Circle(enableCond_s(c(s'))))\\
\end{array}
\]
\end{footnotesize}
\end{definition}

To represent that a deadlock will falsify the formula
we need to introduce a new requirement for every deadlock state
in the source automaton.

\begin{definition}\label{def:deadlock-state}\emph{(Deadlock requirement)}
\[deadlock(s) = \square(encode(s) \rightarrow \Circle(false)) \]
\end{definition}

In order to keep the atomic propositions of the formulas consistent
we need set the value of the fluent-related signals in the
same way they do in the control problem.

\begin{definition}\label{def:fluent-set}\emph{(Set Fluent requirement)}\\
\[fluent_s(s) = \square(encode(s) \rightarrow \bigwedge_{f \in v(s)} y(f) \wedge \bigwedge_{f \in \mathcal{F} \setminus v(s)}\neg y(f)) \]

\end{definition}

\begin{definition}\label{def:cp2gs}\emph{(Control Problem to Games Structure Translation)}
For a control problem $\mathcal{I}= <E, \mathcal{C}, \varphi>$ with
$E = (S, \Sigma, \Delta, s_{M_0})$ and 
$\varphi = (\bigwedge_{i \in [1 \ldots n]} \phi_i \rightarrow \bigwedge_{j \in [1\ldots m]}\gamma_i)$, we define a control problem to game structure translation as:
\[
gs(<E,\mathcal{C},\varphi>) = \{\mathcal{V},\mathcal{X},\mathcal{Y},\theta_e, \theta_s,\rho_e,\rho_s, \varphi_e, \varphi_s\}\]
With the following auxiliary definitions:
\[
\begin{array}{r c l}
init_e(\mathcal{I},s) &=&
\begin{cases}
enabledCond_e(u(s)) \quad |c(s)| = 0\\
enabledCond'_e(u(s)) \quad otherwise\\
\end{cases}\\

init_s(\mathcal{I},s) &=& enabled_s(c(s)) \cup \{ encode(s_0) \}\\
init_{fl}(\mathcal{I}, f) &=& 
\begin{cases}
y(f)  \quad initially_f = true\\
\neg y(f)  \quad otherwise\\
\end{cases}\\
fluents_s(\mathcal{I}) &=& \bigwedge_{s \in S} fluents_s(s)\\
deadlocks(\mathcal{I}) &=& \bigwedge_{s \in S, |\Delta(s)| = 0} deadlock(s)\\
enable_e(\mathcal{I}) &=& \{ \bigwedge_{(s, a, s') \in \Delta \wedge a \in \Sigma \setminus \mathcal{C}} enable_e(\mathcal{I}, s, a, s') \}\\
enable_s(\mathcal{I}) &=& \{ \bigwedge_{(s, a, s') \in \Delta \wedge a \in \mathcal{C}} enable_s(\mathcal{I}, s, a, s') \}\\
update_s(\mathcal{I}) &=& \{ \bigwedge_{(s, a, s') \in \Delta} update_s(\mathcal{I}, s, a, s') \}\\
\end{array}
\]
We decompose the variables and the assertions in the games structure as follows:
\[
\begin{array}{r c l}
\mathcal{V} &=& \mathcal{X} \cup \mathcal{Y}\\
\mathcal{X} &=& \{ x(u) : u \in \Sigma \setminus \mathcal{C} \}\\
\mathcal{Y} &=& Y_c \cup Y_s \cup Y_f \\
Y_c &=& \{y(c) : c \in \mathcal{C}\} \\
Y_s &=& \{s_i : i \in [0\ldots \lceil log_2(|S|) \rceil ] \\
Y_f &=& \{ y(f) : f \in \mathcal{F} \}\\
\theta_e &=& \{ init_e(\mathcal{I}, s_0) \}\\
\theta_s &=& \{ init_s(\mathcal{I}, s_0) \cup \{ \bigwedge_{f \in \mathcal{F}} init_{fl}(\mathcal{I}, f)  \}\\
\rho_e &=& enable_e(\mathcal{I})\\
\rho_s &=& enable_s(\mathcal{I}) \cup update_s(\mathcal{I}) 
\cup fluents_s(\mathcal{I}) \\
&& \cup enable_{fl}(\mathcal{I}) \cup deadlocks(\mathcal{I})\\
\varphi_e &=& adequate(\bigwedge_{i \in [1 \ldots n]} \phi_i)\\
\varphi_s &=& adequate(\bigwedge_{i \in [1 \ldots m]} \gamma_i)\\
\end{array}
\]

\end{definition}
EXPLAIN NO CONTRADICTION IN FORMULAS\\
EXPLAIN $\varphi$ PREDICATE OVER $y(fl)$\\
EXPLAIN ONLY ONE EVENT PICKED AT EACH TRANS\\
We need to show that the translation preserves the semantic of
the model.  In order to do this we need to compare satisfaction of
LTL formulas in the control problem against the constructed
game structure.  Let $\mathcal{I} = <E, \mathcal{C}, \varphi>$
be our source control problem and $\mathcal{G} = gs(\mathcal{I})$ be our target
game structure. We will suppose, w.l.o.g that $E$ is a LTKS where the
valuation for each state holds the value of the fluents involved in
$\varphi$, if the structure of the initial LTS could not assign
a unique fluent value to each state we can always apply the parallel
composition of the original automaton with the induced automata for
the fluents, and then assign the a unique valuation to the resulting
states.\\
We will use this property to define a relation 
$\mathcal{Z} \subseteq S \times S'$ where $S'$ is the set of state
system variables in $\mathcal{G}$ used to encode the value of each
particular source state.  We define the relation as:
\[ \mathcal{Z} : {\forall s \in S : (s, encode(s))} \]
We proceed now to prove that $\mathcal{Z}$ is a bisimulation relation,
since for every $(s, s') \in \mathcal{Z}$ the following holds:
\begin{itemize}
	\item [(prop)] $\forall f \in \mathcal{F}: s \models f \iff s' \models y(f)$ 
	\item[(forth)] \[
		\begin{array}{l}
			\forall a,t s.t. (s,a,t) \in \Delta: \exists t' \in S', v',w' \in \mathcal{V}\setminus S' \text{s.t.}:\\
			(s' \wedge v',t' \wedge w') \in \rho
			\wedge (t, t') \in \mathcal{Z} 
		\end{array}	
		\]
	\item[(back)] \[
		\begin{array}{l}
			\forall v,w \in \mathcal{V} \setminus S' s.t. (s \wedge v,t \wedge w) \in \rho:\\
			\exists t' \in S, a \in \Sigma \text{s.t.}:
			(s',a,t') \in \Delta
			\wedge (t, t') \in \mathcal{Z} 
		\end{array}	
		\]		
\end{itemize}
\begin{proof}{(prop)}
Since $fluents_s(s) \in \rho_s$ we know that \[encode(s) \rightarrow \bigwedge_{f \in v(s)} y(f) \wedge \bigwedge_{f \in \mathcal{F} \setminus v(s)}\neg y(f) \]
But then is easy to see that:
\[s' \models y(f) \iff \bigwedge_{f' \in v(s)} y(f') \models y(f) \iff f \in v(s)  \]
Which is the definition of $s \models f$
\end{proof}
\begin{proof}{(forth)}
Since the safety requirements are defined for all pairs of states in $\Delta$
we can assume that $t' = encode(t)$ thus ensuring $(t,t') \in \mathcal{Z}$.
In order to find $v',w'$ satisfying $(s' \wedge v',t' \wedge w') \in \rho$ we
set $v'$ as $v_1 \wedge fluents_s(s')$ where
$v_1 = var(a) \wedge \bigwedge_{a'\in \Sigma, a' \neq a}\neg var(a)$, that holds since $enabling_{\sigma}(s) \in \rho$, $enabling_s{\sigma}(s) $ $\models v_1$. 
And we set $w'$ in the same fashion according to the safety restrictions for
$t'$ in $fluents_s(t')$ and pick any mutex conjunct satisfying $enable_\sigma(t')$.
\end{proof}
\begin{proof}{(back)}
Again we can assume that $t' = encode(t)$.
Since we are working with deterministic automaton we know that
for any given pair $v,w$ s.t. $(s' \wedge v', t' \wedge w') \in \rho$
we can pick a unique $a \in \Sigma$ s.t. $(s, a, t) \in \Delta$ since
$update_s(s,a,t) \in \rho_s$.
\end{proof}