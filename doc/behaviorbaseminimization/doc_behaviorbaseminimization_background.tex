\section{Preliminaries}
%In this section we present background on realisability, which is linked to the problem of controller synthesis~\cite{DBLP:conf/popl/PnueliR89,DBLP:journals/jcss/BloemJPPS12} which, in turn, requires a formal specification that, in this paper, is assumed to be, as in~\cite{DBLP:conf/sigsoft/DIppolitoBPU10} provided as a labelled transition system and a subset of linear temporal logic formula called Generalised Reactivity 1~\cite{DBLP:journals/jcss/BloemJPPS12}.

We model safety assumptions on the environment and goals for the system as composition of Labelled Transition Systems (LTS). 


\begin{definition}
\label{def:LTS} \emph{(Labelled Transition Systems)} 
A \emph{Labelled Transition System} (LTS) is $E =  (S, \Sigma, \Delta, s_0)$, where $S$ is a finite set of states, $\Sigma \subseteq Act$ is its {\em communicating alphabet}, $\Delta \subseteq (S \times \Sigma \times S)$ is a transition relation, and $s_0 \in S$ is the initial state.  We denote $\Delta(s)=\set{\act~|~(s,\act,s') \in \Delta}$. 
An LTS is deterministic if $(s,\act,s')$ and $(s,\act,s'')$ are in $\Delta$ implies $s'=s''$.
An execution of $E$ is a word $\varepsilon=s_0, \act_0, s_1, \ldots$ where $(s_i, \act_i, s_{i+1}) \in \Delta$. 
A word $\pi$ is a trace (induced by $\varepsilon$) of $E$ if its the result of removing every $s_i$ from an execution $\varepsilon$ of $E$. 
We denote the set of infinite traces of $E$ by $\traces(E)$. 
\end{definition}



\begin{definition}\label{def:parallelComposition}\emph{(Parallel Composition)}
Let $P=(S_P,\Sigma_P,\Delta_P,s_{0_P})$ y $Q=(S_Q,\Sigma_Q,\Delta_Q,s_{0_Q})$ be two LTSs, then 
the parallel composition $P\|Q$ is defined as $P \| Q = (S_P \times S_Q, \Sigma_P \cup \Sigma_Q, \Delta_{P\|Q}, (s_{0_P}, s_{0_Q}))$
where $\Delta_{P\|Q}$ is the smallest relation satisfying that for all $s, t \in S_P$, $s',t' \in S_Q$ and $\ell \in \Sigma_P \cup \Sigma_Q$:
\small
\begin{gather*}
 (s,\ell,t) \in \Delta_{P} \wedge a \not\in \Sigma_Q \Rightarrow ((s,s'),\ell,(t,s')) \in \Delta_{P\|Q}\\
(s',\ell,t') \in \Delta_{Q} \wedge \ell \not\in \Sigma_P \Rightarrow ((s,s'),\ell,(s,t')) \in \Delta_{P\|Q}\\
(s,\ell,t) \in \Delta_{P} \wedge (s',\ell,t') \in \Delta_{Q} \Rightarrow ((s,s'),\ell,(t,t')) \in \Delta_{P\|Q} 
\end{gather*}
\normalsize
\end{definition}

The distinction between what an LTS can control and what it can monitor is enforced through the notion of 
{\em legal environment} taken from~\cite{DIppolito:2013}, and inspired in that of Interface Automata~\cite{DBLP:conf/sigsoft/AlfaroH01}.
Intuitively, a controller does not block the actions that it does not control, and dually, the environment does not restrict controllable actions. 

\begin{definition} \label{def:IALTS} \emph{(Legal LTS)}
Given LTSs $C = (S_C, \Sigma$, $\Delta_C$, $s_{C_0})$ and $E = (S_E,\Sigma,\Delta_E,s_{E_0})$, where $\Sigma$ is partitioned into actions controlled and monitored (non controllable) by $C$ ($\Sigma=\mathcal{C} \; \cup \;\mathcal{U}$), we say that $C$ is a legal LTS for $E$ if for all $(s_E,s_C) \in E\|C$ it holds that
$\Delta_{E}(s_E)\cap \mathcal{C} \supseteq \Delta_{C}(s_C)\cap \mathcal{C}$ and also that  $\Delta_{E}(s_E)\cap \mathcal{U} \subseteq \Delta_{C}(s_C)\cap \mathcal{U}$.

\end{definition}


%We use $\Delta(s)$ to denote the set $\{\ell~|~(s,\ell,s') \in 
%\Delta \}$ and $\Delta(s,\ell)=\set{s'~|~(s,\ell,s')\in\ \Delta_E}$.
%We say that an LTKS is deterministic if, whenever $(s,\ell,s')$ and 
%$(s,\ell,s'')$ are in $\Delta$, $s'=s''$. 
%We use $\initialState{E}$ and $\states{E}$ to refer to $S_0$ and $S$ respectively.
%For deterministic LTKS, we also use $E\executes(a)$ to denote the LTKS $(S, A, P, \Delta, v, s)$ if $(s_0, a, s) \in \Delta$, and $E$ otherwise.
%\end{definition}
%\begin{definition}\emph{(Traces)}
%A trace of $E$ is $\pi\!=\!s_0,\ell_0,s_1,\ell_1,\cdots$, where $s_0$ is an initial state of $E$ and, for every $i\geq 
%0$, we have $(s_i,\ell_i,s_{i+1})\in \Delta$.
% We denote as $\pi|_{\Sigma}$ the 
%sequence that is the result of removing from $\pi$ all actions not in
%$\Sigma$. 
%We denote the set of infinite traces of $E$ by $\traces(E)$. 
%We will assume systems yield infinite traces. 
%\end{definition}

%\begin{definition}\emph{(Traces)}


%We describe liveness goals and assumptions using fluent linear temporal \gr formulae. Linear
%temporal logics (LTL) are widely used to describe behaviour requirements~\cite{CITA INCORRECTA: DBLP:conf/sigsoft/GiannakopoulouM03}.
%The motivation for
We use linear temporal logics on fluents (FLTL) since they provide a uniform framework
for specifying state-based temporal properties in event-based models~\cite{DBLP:conf/sigsoft/GiannakopoulouM03}. 
A \emph{fluent} \fluent is defined by a pair of sets and a Boolean value: $\emph{\fluent} = \langle I_{\emph{\fluent}}, T_{\emph{\fluent}}, \emph{Init}_{\emph{\fluent}} \rangle$, where $I_{\emph{\fluent}}\subseteq Act$ is the set of initiating actions, $T_{\emph{\fluent}}\subseteq Act$ is the set of terminating actions and $I_{\emph{\fluent}}\cap T_{\emph{\fluent}}=\emptyset$. 
A fluent may be initially \true or \false as indicated by \emph{Init}$_{\emph{\fluent}}$. 
Every action $\ell\in Act$ induces a fluent, namely $\fluentp{\ell}=\langle \ell, Act\setminus \set{\ell}, \false\rangle$. 
Finally, the alphabet of a fluent is the union of its terminating and initiating actions.

Let $\mathcal{F}$ be the set of all possible fluents over $Act$. 
An FLTL formula is defined inductively using the standard Boolean connectives and temporal operators $\bigcirc$~(next), $U$ (strong until) as follows: 
$\varphi ::= \fluent \mid \neg \varphi \mid \varphi \vee \psi \mid \X \varphi \mid \varphi U \psi,$
where $\fluent\in\mathcal{F}$. 
As usual we introduce $\wedge$, $\Diamond$ (eventually), and $\square$ (always) as syntactic sugar. 
Let $\Pi$ be the set of infinite traces over \emph{Act}.
The trace $\pi=\ell_0,\ell_1,\ldots$ satisfies a fluent $\emph{Fl}$ at position $i$, denoted $\pi,i \models \emph{Fl}$, if and only if at least one of the following conditions holds:
%\begin{list}{-} %{\leftmargin=3em}
\begin{itemize}
\item $\emph{Init}_{\emph{Fl}} \wedge (\forall j \in \mathbb{N} \cdot 0 \leq j \leq i \rightarrow \ell_j \notin T_{\fluent})$
\item $\exists j \in \mathbb{N} \cdot (j \leq i \wedge \ell_j \in I_{\fluent}) \wedge (\forall k \in \mathbb{N} \cdot j < k \leq i \ \rightarrow \ell_k \notin T_{\fluent})$
%\end{list}
\end{itemize}
We will assume that FLTL formulae are equipped with the corresponding
fluent definition triples.
Formula satisfaction for FLTL is defined as follows: 
given an infinite trace $\pi$, the satisfaction of a formula $\varphi$ at position $i$, denoted $\pi,i\models\varphi$, is defined as shown in Figure~\ref{fig:back:satisfactionop}.  
\begin{figure}[bt]
$$
\begin{array}{lcl}
\pi,i \models \fluent &\triangleq & \pi,i \models \fluent  \\
\pi,i \models \neg \varphi & \triangleq & \neg(\pi,i \models \varphi) \\
\pi,i \models \varphi \vee \psi &\triangleq& (\pi,i\models \varphi) \vee (\pi,i models \psi) \\
\pi,i \models \bigcirc \varphi &\triangleq & \pi,1 \models \varphi \\
\pi,i \models \varphi \bf U \psi &\triangleq&\exists j \geq i \cdot \pi,j \models \psi \wedge \forall \mbox{ }i \leq k < j \cdot \pi,k \models \varphi\\
\end{array}
$$
\caption{Semantics for the satisfaction operator}
\label{fig:back:satisfactionop}
\end{figure}


We say that $\varphi$ holds in $\pi$, denoted $\pi\models\varphi$, if $\pi,0\models\varphi$. 
A formula $\varphi \in \mbox{FLTL}$ holds in an LTS $E$ (denoted $E \models \varphi$) if it holds on every infinite trace produced by $E$.

In this paper we restrict attention to \gr~\cite{DBLP:journals/jcss/BloemJPPS12} formulas as there are effective synthesis algorithms for this sub-logic~\cite{DBLP:journals/jcss/BloemJPPS12}. \gr formulae are of the form $\varphi = \bigwedge_{i=1}^n \square\Diamond \assume_i \implies \bigwedge_{j=1}^m \square\Diamond \guarantee_j$, where $\{\assume_1 \ldots \assume_n\}$ and $\{\guarantee_1 \ldots \guarantee_m\}$ are propositional FLTL formulae.

%%%represent the set of
%%%assumptions that should
%%%always eventually be satisfied in order to 
%%%always eventually satisfy the set of goals
%%%$\{\guarantee_1 \ldots \guarantee_m\}$.
%%%\gr is an expressive subset that has gained increased interest recently due to the development of tractable synthesis algorithms for it~\cite{DBLP:journals/jcss/BloemJPPS12}.

%From the FLTL definition it follows that many results for LTL can be
%easily extended to FLTL.

We now provide a standard definition of parallel composition of LTSs to support compositional construction of environment models~\cite{DBLP:journals/cacm/Hoare78}.
% which is defined as an LTS that models the asynchronous execution of composed models, interleaving non-shared actions but forcing synchronisation on shared actions. 

%%%%\begin{definition}\label{def:legalEnvironment}\emph{(Legal Environment)}
%%%%Given $M = (S_M, \Sigma_M, \Delta_M, s_{M_0})$ and  $P = (S_P,\Sigma_P,\Delta_P,s_{P_0})$ LTSs, where $L_M=\Sigma_{M_c}\cup \Sigma_{M_u}$, $\Sigma_{M_c}\cap
%%%%\Sigma_{M_u}=\emptyset$, $\Sigma_P=\Sigma_{P_c}\cup \Sigma_{P_u}$ and $\Sigma_{P_c}\cap
%%%%\Sigma_{P_u}=\emptyset$. We say that $M$ is a legal environment for $P$
%%%%if the interface automaton $M'=\langle S_M, \set{s_{M_0}}, \Sigma_{M_u},
%%%%\Sigma_{M_c}, \emptyset, \Delta_M \rangle$ is a \emph{legal environment}
%%%%for the interface automaton $P'=\langle S_P, \set{s_{P_0}},$
%%%%$\Sigma_{P_u}, \Sigma_{P_c}, \emptyset, \Delta_P \rangle$.
%%%%\end{definition}

%\begin{definition}\label{def:legalEnvironment}\emph{(Legal Environment)}
%Given $M = (S_M, \Sigma_{M_c}\cup \Sigma_{M_u},$ $ \Delta_M, s_{M_0})$ and  $P = (S_P,\Sigma_{P_c}\cup \Sigma_{P_u},\Delta_P,s_{P_0})$ LTSs.
%%where $\Sigma_M=\Sigma_{M_c}\cup \Sigma_{M_u}$, $\Sigma_{M_c}\cap \Sigma_{M_u}=\emptyset$, $\Sigma_P=\Sigma_{P_c}\cup \Sigma_{P_u}$ and $\Sigma_{P_c}\cap \Sigma_{P_u}=\emptyset$. 
%We say that $M$ is a legal environment for $P$ if the composition of the interface automata $M'=(S_M, s_{M_0}, \Sigma_{M_u}, \Sigma_{M_c}, \emptyset, \Delta_M)$ and $P'=(S_P, s_{P_0}, \Sigma_{P_u},$ $ \Sigma_{P_c}, \emptyset, \Delta_P)$ has no illegal states~\cite{DBLP:conf/sigsoft/AlfaroH01}. 
%\end{definition}

%

The following definitions are based on those presented in~\cite{alfaro01}.

\begin{definition}\label{def:IA}\emph{(Interface Automata~\cite{alfaro01})}
An \emph{Interface Automata} (IA) is $P=\langle V, V^{init}, \mathcal{A}^I, \mathcal{A}^O, \mathcal{A}^H, \mathcal{T} \rangle$ where,
\begin{itemize}
  \item $V$ is a set of states
  \item $V^{init} \subseteq V$ is a set of initial states, with
      $|V^{init}| \leq 1$.
  \item $\mathcal{A}^I$, $\mathcal{A}^O$ and $\mathcal{A}^H$
      are mutually disjoint sets of input, output and internal
      actions. $\mathcal{A}=\mathcal{A}^I \cup \mathcal{A}^O
      \cup \mathcal{A}^H$ denotes the set of all actions.
  \item $\mathcal{T} \subseteq V \times \mathcal{A} \times V$
      is the set of transitions.
\end{itemize}
\end{definition}

If $a\in \mathcal{A}^I$ (resp. $a\in \mathcal{A}^O$, $a\in
\mathcal{A}^H$), then $(v, a, v')$ is called an input (resp.
output, internal) transition. $\mathcal{T}^I$ (resp.
$\mathcal{T}^O$, $\mathcal{T}^H$) denotes the set of input (resp.
output, internal) transitions. We say that an interface automaton
$P$ is \emph{closed} if it only has internal actions (i.e.
$\mathcal{A}^I = \mathcal{A}^O = \emptyset$); otherwise, we say
that $P$ is \emph{open}. An action $a$ is \emph{enabled} at a state
$v\in V$ if there exists a $v'\in V$ such that
$(v,a,v')\in\mathcal{T}$. The sets $\mathcal{A}^I(v)$,
$\mathcal{A}^O(v)$ and $\mathcal{A}^H(v)$ denote the set of input,
output and internal actions that are enabled from state $v$, and
$\mathcal{A}(v)=\mathcal{A}^I(v) \cup \mathcal{A}^O(v)\cup
\mathcal{A}^H(v)$ is the set of all enabled actions from state $v$.
As an example consider the drill modelled with the IA automata in
Figure~\ref{fig:drillIA}. Whenever the drill receives a product to
be processed ($put$) it signals an actuator to start processing the
product ($process$). Afterwards, the sensors signal if the product
has been successfully processed ($processOk$) or not
($processFail$). If the processing was successful, the drill
informs that the product is now ready ($ack$) and can be picked up
($get$); or report an error ($nack$) and waits for the product to
be discarded ($discard$) otherwise.


\begin{definition}\label{def:composable}\emph{(Composable)}
Two interface automata $P=\langle V_P, V_P^{init}, \mathcal{A}_P^I,
\mathcal{A}_P^O, $ $\mathcal{A}_P^H, \mathcal{T}_P \rangle$ and
$Q=\langle V_Q, V_Q^{init}, \mathcal{A}_Q^I, \mathcal{A}_Q^O,
\mathcal{A}_Q^H, \mathcal{T}_Q \rangle$ are \emph{composable} if
\begin{center}
\begin{displaymath}
    \begin{array}{cc}
     \mathcal{A}_P^H \cap \mathcal{A}_Q = \emptyset & \mathcal{A}_P^I \cap \mathcal{A}_Q^I = \emptyset \\
     \mathcal{A}_P^O \cap \mathcal{A}_Q^O = \emptyset & \mathcal{A}_Q^H \cap \mathcal{A}_P = \emptyset
    \end{array}
\end{displaymath}
\end{center}
Let $shared(P, Q)= \mathcal{A}_P \cap \mathcal{A}_Q$.
\end{definition}

The composition of IAs $P$ and $Q$ is defined in stages. First, the
product automaton $P\otimes Q$, which coincides with the
composition of I/O automata~\cite{Lynch:1987:HCP:41840.41852},
except that since $P$ and $Q$ are not necessarily input-enabled,
some transitions present in $P$ or $Q$ may not be present in the
product.

\begin{definition}\label{def:ia-product}\emph{(Product)}
Given $P=\langle V_P, V_P^{init}, \mathcal{A}_P^I, \mathcal{A}_P^O, \mathcal{A}_P^H, \mathcal{T}_P \rangle$ and $Q=\langle V_Q, V_Q^{init}, \mathcal{A}_Q^I, \mathcal{A}_Q^O, \mathcal{A}_Q^H, \mathcal{T}_Q \rangle$ composable interface automata, their product is the interface automaton $P\otimes Q=\langle V_{P\otimes Q}, V_{P\otimes Q}^{init}, \mathcal{A}_{P\otimes Q}^I, \mathcal{A}_{P\otimes Q}^O, \mathcal{A}_{P\otimes Q}^H, \mathcal{T}_{P\otimes Q} \rangle$, where:\\
\begin{eqnarray*}
  V_{P\otimes Q} &=& V_P \times V_Q \nonumber\\
  V_{P\otimes Q}^{init} &=& V_{P}^{init} \times V_{Q}^{init} \nonumber\\
  \mathcal{A}_{P\otimes Q}^I &=& (\mathcal{A}_{P}^I \cup \mathcal{A}_{Q}^I)\setminus shared(P,Q)\nonumber\\
  \mathcal{A}_{P\otimes Q}^O &=& (\mathcal{A}_{P}^O \cup \mathcal{A}_{Q}^O)\setminus shared(P,Q)\nonumber\\
  \mathcal{A}_{P\otimes Q}^H &=& (\mathcal{A}_{P}^H \cup \mathcal{A}_{Q}^H)\cup shared(P,Q)\nonumber
\end{eqnarray*}
The transition relation is defined as follows:
\begin{eqnarray*}
% \nonumber to remove numbering (before each equation)
  \mathcal{T}_{P\otimes Q} &=& \set{((u,v),a,(u,v')) | (v,a,v')\in\mathcal{T}_{P} \wedge a\notin shared(P,Q) \wedge u\in V_Q} \nonumber\\
      &\cup & \set{((u,v),a,(u',v)) | (u,a,u')\in\mathcal{T}_{Q} \wedge a\notin shared(P,Q) \wedge v\in V_P} \nonumber\\
      &\cup& \set{((u,v),a,(u',v')) | (v,a,v')\in\mathcal{T}_{P} \wedge (u,a,u')\in\mathcal{T}_{Q} \wedge a\in shared(P,Q)} \nonumber
\end{eqnarray*}
\end{definition}

Recall that interface automata do not require input-enabledness.
Therefore given the product $P\otimes Q$ of two interface automata
$P$ and $Q$, may have states in which one of them produces an
output action that is input action for the other but it is not
enabled in that state, hence it is not accepted. Such a state is
called an \emph{Illegal} state. $Illegal(P,Q)$ denotes the set of
illegal states in the product $P\otimes Q$. In
Figure~\ref{fig:armIA} we present a robotic arm that places
products in the drill ($put$), waits for the drill to finish
($ack$) and finally picks the resulting product up ($get$). The
robotic arm assumes that once products have been placed in the
drill, they will be successfully processed. Hence, it cannot handle
failures. The IA $\drill \otimes \arm$ is shown in
Figure~\ref{fig:IAproduct}. As in the case of LTS parallel
composition, each state in $\drill \otimes \arm$ consists of a
state of $\drill$ together with a state of $\arm$, and each
transition represents either a joint step on both $\drill$ and
$\arm$ processes or an individual step on one of them. Consider the
following scenario, the arm places a product in the drill ($put$),
the drill starts processing ($process$) and its sensors signal a
failure ($processFail$). This sequence lead us to state $5$ of
$\drill \otimes \arm$, which corresponds to state $5$ of $\drill$
and $1$ of $\arm$. In state $5$, the sensors are signalling the
drill that something went wrong while processing the current
product, but the $\arm$ is not prepared to handle such situation.
Such state is called $illegal$ in $\drill\otimes\arm$.

\begin{definition}\label{def:illegalStates}\emph{(Illegal States)}
Given two composable interface automata $P$ and $Q$, the set of
$Illegal(P,Q)\subseteq V_P \times V_Q$ of illegal states of
$P\otimes Q$ is defined as follows:

\begin{eqnarray*}
    Illegal(P, Q) &=& \set{(v,u)\in V_P\times V_Q | \exists a \in shared(P,Q) \cdot \\
        & & ((a\in\mathcal{A}_P^O(v) \wedge a\notin\mathcal{A}_Q^I(u)) \vee (a\in\mathcal{A}_Q^O(u) \wedge a\notin\mathcal{A}_P^I(v)))}
\end{eqnarray*}
\end{definition}

If the product $P\otimes Q$ is closed, $P$ and $Q$ are said to be
\emph{compatible} if no illegal state of $P\otimes Q$ is reachable.
If $P\otimes Q$ is open, the fact that a states is in
$Illegal(P,Q)$ is reachable does not necessarily indicate an
incompatibility, because by generating appropriate inputs, the
environment of $P\otimes Q$ may be able to ensure that no state in
$Illegal(P,Q)$ is visited. Such an environment is called a
\emph{legal environment}. The existence of legal environment
indicates that there is a way to use the interfaces $P$ and $Q$
together without giving rise to incompatibilities. A legal
environment for $R$ needs to satisfy the following side conditions.

\begin{definition}\label{def:IAlegalEnvironment}\emph{(Legal Environment)}
An environment for an interface automaton $R$ is an interface
automaton $E$ such that, (i) $E$ is composable with $R$, (ii) $E$
is nonempty, (iii) $\mathcal{A}_E^I=\mathcal{A}_R^O$, and (iv)
$Illegal(R,E)=\emptyset$.
\end{definition}

\begin{definition}\label{def:IAComposition}\emph{(IA Composition)}
The composition of two interface automata is obtained by
restricting the product of the two automata to the set of
compatible states, which are the states from which the environment
can prevent \emph{illegal states} being visited.
\end{definition}


\begin{definition} \label{def:comp-states}\emph{(Compatible States)}
Consider two composable interface automata $P$ and $Q$. A pair $(v,
u)\in V_P\times V_Q$ of states is compatible if there is an
environment $E$ for $P\otimes Q$ such that no state in
$Illegal(P\otimes Q)\times V_E$ is reachable in $(P\otimes Q)\times
V_E$ from the state $\set{(v,u)}\times V_E^{init}$.
\end{definition}

According to the definitions above, the IAs presented in
Figures~\ref{fig:drillIA} and~\ref{fig:armIA} are compatible.
Hence, there exists an environment that in which there are no
reachable \emph{illegal} states. For instance, an environment that
never fails ensures that the illegal state $5$ in
$\drill\otimes\arm$ is not reachable. In
Figure~\ref{fig:IAcomposition}, we show the composition
$\drill\|\arm$, obtained by restricting $\drill\otimes\arm$ to its
compatible states, which corresponds to the composition between
$\drill$, $\arm$ and an environment which never lets the drill to
fail processing products.

\begin{figure}[bt]
\centering
    \VCDraw{
%        \ShowFrame
        \begin{VCPicture}{(-10,-1)(10,5)}
        \State[0]{(-5,0)}{0}
        \Initial{0}
        \State[1]{(0,0)}{1}
        \State[2]{(5,0)}{2}
        \State[3]{(5,3.5)}{3}
        \State[4]{(-5,3.5)}{4}
        \EdgeL{0}{1}{put?}
        \EdgeL{1}{2}{process!}
        \EdgeL{2}{3}{processOk?}
        \EdgeR{3}{4}{ack!}
        \EdgeL{4}{0}{get?}
        \end{VCPicture}
    }
\caption{$\drill \| \arm$.}
\label{fig:IAcomposition}
\end{figure}

Based on the definitions above we define the notion of \emph{LTS
legal environment} which will be used in future definitions.

\begin{definition}\label{def:ltslegalEnvironment}
\emph{(LTS Legal Environment)} Given $M = (S_M, $ $ L_M, \Delta_M,
s_{M_0})$ and $P =$ $(S_P,L_P,\Delta_P,s_{P_0})$ LTSs, where $L_M$
$=L_{M_c}\sqcup L_{M_u}$ and $L_P=L_{P_c}\sqcup L_{P_u}$. We say
that $M$ is an \emph{LTS legal environment} for $P$ with controlled
actions $L_{M_c}$, if the interface automaton $M'=\langle S_M,
\set{s_{M_0}}, L_{M_u}, L_{M_c}, \emptyset, \Delta_M \rangle$ is a
\emph{legal environment} for the interface automaton $P'=\langle
S_P, \set{s_{P_0}},$ $L_{P_u}, L_{P_c}, \emptyset, \Delta_P
\rangle$.
\end{definition}

We now formally define the control problem for \gr formulas.


\begin{definition}\label{def:lts-control}\emph{(LTS \gr Control)}
Given a specification for a problem domain in the form of an environment LTS $E =  (S, \Sigma, \Delta, s_0)$, a set of controllable actions $\mathcal{C}$, and a \gr formula  $\varphi$, the solution for the LTS control problem $\mathcal{I}=\langle E, \mathcal{C}, \varphi\rangle$ is to find an LTS $M$ such that $M$ is a legal LTS for $E$ ,
$E\|M$ is deadlock free, and $M\|E \models \varphi$.
\end{definition}

\begin{definition}\label{def:state-characterization}\emph{((Non) Controllable and Mixed States)}
Given an LTS $E =  (S, \Sigma, \Delta, s_0)$, a set of controllable actions $\mathcal{C} \subseteq \Sigma$ and its complement $\mathcal{U} = \Sigma \setminus \mathcal{C}$,  a state $s \in S$ is called controllable if $\Delta(s) \cap \mathcal{C} = \Delta(s)$, non controllable if $\Delta(s) \cap \mathcal{U} \neq \emptyset$ and mixed otherwise.  We will refer to non mixed states pure states.
\end{definition}

\begin{definition}\label{def:twoplayer-game}\emph{(Two-player Game)}
A two player game $G$ is defined as the tuple $G=(S_g,\Gamma^{-},\Gamma^{+},s_{g0}, \varphi)$ where $S_{g}$ is a finite set of states, $\Gamma^{-}$,$\Gamma^{+}$ $\subseteq S\times S$ are transition relations defined for the uncontrollable and controllable transtiions respectively, $s_{g0}\in S_{g}$ is the initial condition and $\varphi \subseteq S_{g}^{\omega}$ is the winning condition.  We denote $\Gamma^{-}(s) =$ $\{s'|(s,s')$ $\in \Gamma^{-}\}$ and in a similar fashion for $\Gamma^{+}(s)$. A state $s$ is uncontrollable if $\Gamma^{-}(s)\neq \emptyset$ and controllable otherwise.  A play on $G$ is a sequence $p=s_0,s_1,\ldots$ an a play $p$ ending in $s_{g_{n}}$ is extended by the controller choosing a subset $\gamma \subseteq \Gamma^{+}(s_{g_{n}})$.  Then the environment chooses a state $s_{g_{n+1}}$ $\in \gamma \cup \Gamma^{-}(s_{g_{n}})$ and adds $s_{g_{n+1}}$ to $p$.
\end{definition}

\begin{definition}\label{def:strategy}\emph{((Counter)Strategy with memory)}
A strategy with memory $\Omega$ for the controller is a pair of functions $(\delta, u)$ where $\Omega$ is some memory domain, $\delta:\Omega\times S \rightarrow 2^{S}$ such that $\delta(\omega, s) \subseteq \Gamma^{+}(s)$ and $u:\Omega \times S \rightarrow \Omega$.
A counter strategy with memory $\nabla$ for the environment is a pair of functions $(\kappa, v)$ where $\nabla$ is some memory domain, $\kappa:\nabla\times S \rightarrow 2^{S}$ such that $\kappa(\nabla, s) \subseteq \Gamma^{-}(s)$ and $v:\nabla \times S \rightarrow \nabla$.
\end{definition}


\begin{definition}\label{def:consistency}\emph{(Consistency under (Non)Controllability and Winning (Counter)Strategy)}
A finite or infinite play $p= s_0,s_1,\ldots$ is consistent under controllability if for every $n$ we have $s_{n+1} \in \delta(\omega_n,s_n)$, where $\omega_{i+1}=u(\omega_i,s_{i+1})$ for all
$i \geq 0$. A strategy $(\delta, u)$ for controller from state $s$ is 
winning if every maximal play starting in $s$ and consistent under controllability with $(\delta, u)$ is infinite and remains within $\varphi$.  We say that the controller wins the game $G$ if it has a winning strategy from the initial state. For the non controllable case,
a finite or infinite play $p= s_0,s_1,\ldots$ is consistent under non-controllability if for every $n$ we have $s_{n+1} \in \kappa(\nabla_n,s_n)$, where $\nabla_{i+1}=v(\nabla_i,s_{i+1})$ for all
$i \geq 0$. A strategy $(\kappa, v)$ for environment from state $s$ is 
winning if every maximal play starting in $s$ and consistent under non-controllability with $(\kappa, v)$ is infinite and exits $\varphi$ at some point.  We say that the environment wins the game $G$ if it has a winning strategy from the initial state.
\end{definition}

\begin{definition}\label{def:generalized-reactivity}\emph{(Generalized Reactivity(1))}
Given an infinite sequence of states $p$, let $inf(p)$ denote the states that occur infinitely often in $p$, let $\phi_1,\ldots,\phi_n$ and $\gamma_1,\ldots,\gamma_m$ be subsets of $S$.  Let $gr((\phi_1,\ldots,\phi_n),(\gamma_1,\ldots,\gamma_m))$ denote the set of infinite sequences $p$ such that either for some $i$ we have $inf(p) \cap \phi_i = \emptyset$ or for all $j$ we have $inf(p)\cap \gamma_j \neq \emptyset$. A GR(1) game is a game where the winning condition $\varphi$ is $gr((\phi_1,\ldots,\phi_n),(\gamma_1,\ldots,\gamma_m))$.
\end{definition}


%\begin{definition}\label{def:counter-strategy}\emph{(Counter-strategy with memory)}
%A counter strategy with memory $\nabla$ for the environment is a pair of functions $(\kappa, v)$ where $\nabla$ is some memory domain with designate start value $\nabla_{0}$, $\kappa:\nabla\times S \rightarrow 2^{S}$ such that $\kappa(\nabla, s) \subseteq \Gamma^{-}(s)$ and $v:\nabla \times S \rightarrow \nabla$.
%\end{definition}
%
%\begin{definition}\label{def:counter-consistency}\emph{(Consistency under Non-Controllability and Winning Strategy for the Environment)}
%A finite or infinite play $p= s_0,s_1,\ldots$ is consistent under non-controllability if for every $n$ we have $s_{n+1} \in \kappa(\nabla_n,s_n)$, where $\nabla_{i+1}=v(\nabla_i,s_{i+1})$ for all
%$i \geq 0$. A strategy $(\kappa, v)$ for environment from state $s$ is 
%winning if every maximal play starting in $s$ and consistent under non-controllability with $(\kappa, v)$ is infinite and remains exits $\varphi$ at least at some point.  We say that the environment wins the game $G$ if it has a winning strategy from the initial state.
%\end{definition}

\begin{definition}\label{def:lts-2-game}\emph{(($G_{\mathcal{I}}$) Two-Player Game for $\mathcal{I}$ )}
We convert the GR(1) LTS control problem $\mathcal{I}=\langle E, \mathcal{C}, \varphi \rangle$ to a two-player GR(1) game $G (\mathcal{I})=(S_{g},\Gamma{-},\Gamma{+},s_{g0},\varphi)$ as follows: every state in $S_{g}$ encodes a state in $E$ and a valuation of all fluents
appearing in $\varphi$. We build $S_g$ from $E$ in such a way that states in the game encode a state in $E$ and truth values for all fluents in $\varphi$, let $S_g = S_e \times \Pi_{i=0}^{k}\{true,false\}$.  Consider state $s_g=(s_e,\alpha_1,\ldots,\alpha_k)$, given fluent $fl_i$ we say that $s_g$ satisfies $fl_i$ if and only if $\alpha_i$ is $true$. We generalize satisfaction to boolean combination of fluents in the natural way.  We build the transition relations $\Gamma^{-}$,$\Gamma^{+}$ through the following rules.  Consider state $s_g=(s_e,\alpha_1,\ldots,\alpha_k)$, for every transition $(s_e,l,s'_e) \in \Delta$ we include $(s_g,(s'_e,\alpha'_1,\ldots,\alpha'_k))$ in $\Gamma^{\beta}$ where
$\beta$ is $+$ if $l \in \mathcal{C}$ and in $\Gamma^{-}$ otherwise.
$\alpha'_i$ is $\alpha_i$ if $l \neg\in I_{fl_i} \cup T_{fl_i}$, $\alpha'_i$ is $true$ if $l \in I_{fl_i}$ and $false$ if $l \in T_{fl_i}$.  
If $s_e$ is a mixed state (i.e., $\Delta(s_e) \cap \mathcal{C} \neq \emptyset$
$\wedge$
$\Delta(s_e) \cap \mathcal{U} \neq \emptyset$) we only include
$(s_g, (s'_e,\alpha'_1,\ldots,\alpha'_k))$ in $\Gamma^{-}$
if $l \in \mathcal{U}$ since we will consider mixed states as 
non-controllable (antagonistic) in the game.
$s_{g0}$ is $(s_0, Init_{fl_1},\ldots, Init_{fl_k})$.
Let $\varphi_g \subseteq S_g^{\omega}$ be the set of sequences that satisfy $gr((\phi_1,\ldots,\phi_n),(\gamma_1,\ldots,\gamma_m))$.
\end{definition}

%Although the synthesis problem for general FLTL goals is 2EXPTIME
%complete~\cite{DBLP:conf/popl/PnueliR89}. Nevertheless, restrictions on the form of
%the goal and assumptions specification have been studied and found
%to be solvable in polynomial time. The formulation above, which is restricted to \gr, has a polinomial solution~\cite{DBLP:conf/vmcai/PitermanPS06}. An adaptation of \gr in the context
%of LTS has been presented in~\cite{DBLP:conf/sigsoft/DIppolitoBPU10} 


%
%DDDDDDDDDDDDDDDDDDDDDDDDd NADA DE ACA PARA ABAJO. CREO
%
%Starting with \cite{church1962logic} the synthesis problem is presented
%as a correct by construction, automated way of producing the expected system.
%A linear temporal logic specification solution was proposed in 
%\cite{DBLP:conf/popl/PnueliR89} and a computational efficient 
%approach developed in \cite{DBLP:journals/jcss/BloemJPPS12}.
%This  set of techniques expand the domain of non realizability diagnosis
%beyond model checking for manually written specification .
%It is only reasonable to shed some light on the causes of non realizability
%if our initial claim was that engineers will be relieved from the
%eternal cycle of suffering a specification through manual trial and error.
%Otherwise we would be exchanging the problem of hopefully
%progressive model checking with the one of hopefully progressive
%realizability checking.
%% (A)
%In the context of the present work we put the focus on non realizability.
%
%In such scenarios, when there is no way
%to build a system that satisfies $\varphi$, it is insightful
%to turn the table and look at
%the realizability game from the other side of
%the board.  Instead of reasoning about the satisfaction
%of the formula, we look at the way the environment
%falsifies it by picking a winning move for
%every choice available to the system at every step. 
%In this work we focus on simplifying the understanding of
%this kind of problem.\\
%If we think in game theoretical terms we could be tempted
%to assume that we are providing an explicit representation
%of the counter strategy, when in reality we are providing
%a subset of the explicit behavior of the plant that allows
%a minimal counter strategy to be enacted.
%%
\begin{definition}\label{def:fulfillable}\emph{((un)fulfillable, (in)sufficient)}
Let $\mathcal{A}$ be a set of available assuptions, let $\mathcal{G}$ be a set of available
guarantees, and let $A \subseteq \mathcal{A}$ and $G \subseteq \mathcal{G}$.\\
If a specification $<A,G>$ is realizable, we say that $G$ is fulfillable w.r.t. $A$,
and, conversely, $A$ is sufficient w.r.t. G. Otherwise, $G$ is unfulfillable w.r.t. 
$A$, and $A$ is insufficient w.r.t. G, respectively.\\
$G$ is minimally unfulfillable w.r.t.
$A$ iff $<A,G>$ is unrealizable and removal of any element of $G$ leads to realizability:
$\forall g \in \mathcal{G}. <A, G \setminus {g}>$ is realizable.\\
$G$ is maximally fulfillable w.r.t. $A$ in $\mathcal{G}$ iff $<A,G>$ is realizable and addition
of any element of $G$ leads to unrealizability: 
$\forall g \in \mathcal{G} \setminus G. <A, G \cup {g}>$ is unrealizable.\\
$A$ is minimally sufficient w.r.t. $G$ iff $<A,G>$ is realizable and removal of any element
of $A$ leads to unrealizability: $\forall a \in A. <A\setminus {a},G>$ is unrealizable.\\
$A$ is maximally insufficient w.r.t. $G$ in $\mathcal{A}$ iff $<A,G>$ is unrealizable and addition
of any element of $\mathcal{A}\setminus A$ leads to realizability: 
$\forall a \in \mathcal{A}\setminus A.<A \cup {a},G>$ is realizable.
\end{definition}
\begin{definition}\label{def:helpful}\emph{((un)helpful)}
Let $<A,G>$ be a specification. \\
An assumption $a \in A$ is unhelpful if:\\
$\forall G' \subseteq G. (<A,G'> \text{is realizable} \iff <A\setminus {a},G'> \text{is realizable})$.\\
A guarantee $g \in G$ is unhelpful if:\\ $\forall A' \subseteq A. (<A',G> \text{is realizable} \iff <A', G \setminus {g}> \text{is realizable})$.\\
An assumption or a guarantee is helpful iff is not unhelpful.
\end{definition}
\begin{definition}\label{def:ddmin}\emph{Minimizing Delta Debugging algorithm}
Let $C$ be a set, and $test:2^C\rightarrow \{\times, \checkmark, ?\}$ such that
$test(\emptyset)=\checkmark$ and $test(C)=\times$ holds.  The minimizing Delta
Debugging algorithm $ddmin(c)$ is:
\begin{footnotesize}
\[
\begin{array}{r c l}
ddmin(c_\checkmark) & = & ddmin_2(c_\times , 2) \textbf{ where}\\
ddmin_2(c'_\times, n) & = & 
  \begin{cases}
  ddmin_2(\Delta_i, 2) \text{ ("reduce to subset")}\\
  \quad \text{if } \exists i \in \{1,\ldots,n\}.test(\Delta_i) = \times \\
  ddmin_2(\nabla_i, max(n-1, 2)) \text{ ("reduce to complement")}\\
  \quad \text{else if } \exists i \in \{1,\ldots,n\}.test(\nabla_i) = \times \\  
  ddmin_2(c'_\times, min(|c'_\times|, 2n)) \text{ ("increase granularity")}\\
  \quad \text{else if } n < |c'_\times|\\    
  c'_\times \text{otherwise ("done")}\\    
  \end{cases}
\end{array}  
\]\\
\end{footnotesize}
Where $\nabla_i = c'_\times - \Delta_i$, $c'_\times = \Delta_1 \cup \ldots \cup \Delta_n$, all
$\Delta_i$ are pairwise disjoint, and $\forall \Delta_i. |\Delta_i| \approx |c'_\times|/n$ holds.
The recursion invariant (and thus precondition) for $ddmin_2$ is $test(c'_\times) = \times$
and $n \leq |c'_\times|$.
\end{definition}
%
%
%
%DDDDDDDDDDDDDDD
%
%\begin{definition}
%\label{def:game} \emph{(Two-player Game)} A \emph{Two-player}
%\emph{Ga\-me} (Game) is
%$G=(S_g,\Gamma^{-},\Gamma^{+},s_{g_0},\varphi)$, where $S_g$ is a
%finite set of states, $\Gamma^{-},\Gamma^{+}\subseteq S_g\times S_g$
%are transition relations, $s_{g_0}\in S_g$ is the initial state, and
%$\varphi\subseteq S_g^\omega$ is a winning condition. We denote
%$\Gamma^{-}(s_g) = \{s_g' ~|~ (s_g,s_g') \in \Gamma^{-}\}$ and
%similarly for $\Gamma^{+}$. A state $s_g$ is \emph{uncontrollable}
%if $\Gamma^{-}(s_g)\neq \emptyset$ and \emph{controllable}
%otherwise. A \emph{play} on $G$ is a sequence
%$p=s_{g_0},s_{g_1},\ldots$. A play $p$ ending in $s_{g_n}$ is
%extended by the controller choosing a subset $\gamma \subseteq
%\Gamma^{+}(s_{g_n})$. Then, the environment chooses a state
%$s_{g_{n+1}} \in \gamma \cup \Gamma^{-}(s_{g_{n+1}})$ and adds
%$s_{g_{n+1}}$ to $p$.
%
%A \emph{strategy with memory $\Omega$} for the controller is a pair
%of functions $(\sigma, u)$ where, $\sigma:\Omega\times S_g
%\rightarrow 2^{S_g}$ such that $\sigma(\varpi,s_g) \subseteq
%\Gamma^{+}(s_g)$ and $u:\Omega\times S_g \rightarrow \Omega$ such
%that $\Omega$ is some memory domain with a designated start value
%$\varpi_0$. Intuitively, $\sigma$ tells controller which states to
%enable as possible successors and $u$ tells controller how to update
%her memory. If $\Omega$ is finite, we say that the strategy uses
%finite memory. A finite or infinite play $p=s_{g_0},s_{g_1},\ldots$
%is \emph{consistent} with $(\sigma, u)$ if for every $n$ we have
%$s_{g_{n+1}} \in \sigma(\varpi_n,s_{g_n}) \cup \Gamma^{-}(s_{g_n})$,
%where $\varpi_{i+1}=u(\varpi_i,s_{g_{i+1}})$ for all $i\geq 0$. A
%strategy $(\sigma, u)$ for controller is \emph{winning} if every
%maximal play consistent with $(\sigma, u)$ is infinite and in
%$\varphi$. We say that controller \emph{wins} the game $G$ if it has
%a winning strategy.
%\end{definition}
%
%\begin{definition}\label{def:strategy}\emph{(Strategy with memory)}
%Given an LTS control problem $\mathcal{I}=\langle E, \mathcal{C}, \varphi\rangle$, where $E = (S,\Sigma_{c}\uplus \Sigma_{u},\Delta,s_{0})$
%A \emph{strategy $\strategy$ with memory $\Omega$}  for controller is a pair of functions $(\tran, \upd)$ where, $\tran:\Omega\times S \rightarrow 2^{\Sigma}$ and $\upd:\Omega\times S \rightarrow \Omega$ such that $\Omega$ is some memory domain with a designated start value $\varpi_0$, and for all $s\in S$ it holds that $\tran(\varpi,s) \subseteq (\Sigma_c\cap\Delta(s))$ and $\tran(\varpi,s) = (\Sigma_u\cap\Delta(s))$.
%Intuitively, $\tran$ tells controller which actions to enable as possible successors from a state $s$ and $\upd$ tells controller how to update her memory. 
%If $\Omega$ is finite, we say that the strategy uses finite memory. 
%An execution $\varepsilon=s_0, \act_0, s_1, \ldots$ is \emph{consistent} with $\strategy$ if for every $n$ we have $\act\in\tran(\varpi_n,s_{n})$, where $\varpi_{i+1}=\upd(\varpi_i,\act)$ for all $i\geq 0$. 
%We say that the trace $\pi$ induced by $\varepsilon$  is also \emph{consistent} with \strategy.
%A strategy \strategy for controller is \emph{winning} if every maximal trace in $E$ consistent with \strategy is infinite and satisfies $\varphi$. 
%Finite traces are losing for the controller.
%%We say that controller \emph{wins} the game $G$ if it has a winning strategy.
%A strategy for environment is defined in a similar fashion being the only difference that if \strategy is a strategy for environment then for all $s\in S$ it holds that $\tran(\varpi,s) \subseteq (\Sigma_u\cap\Delta(s))$ and $\tran(\varpi,s) = (\Sigma_c\cap\Delta(s))$.
%\end{definition}
%
%\begin{definition}
%Given an LTS control problem $\mathcal{I}=\langle E, \mathcal{C}, \varphi\rangle$ and a strategy 
%\end{definition}
%\begin{property}
%Let $\mathcal{I}=\langle E, \mathcal{C}, \varphi\rangle$ be an LTS control problem and $M$ an LTS solution to $\mathcal{I}$. 
%\end{property}
