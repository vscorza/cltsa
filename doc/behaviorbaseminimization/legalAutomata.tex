

The following definitions are based on those presented in~\cite{alfaro01}.

\begin{definition}\label{def:IA}\emph{(Interface Automata~\cite{alfaro01})}
An \emph{Interface Automata} (IA) is $P=\langle V, V^{init}, \mathcal{A}^I, \mathcal{A}^O, \mathcal{A}^H, \mathcal{T} \rangle$ where,
\begin{itemize}
  \item $V$ is a set of states
  \item $V^{init} \subseteq V$ is a set of initial states, with
      $|V^{init}| \leq 1$.
  \item $\mathcal{A}^I$, $\mathcal{A}^O$ and $\mathcal{A}^H$
      are mutually disjoint sets of input, output and internal
      actions. $\mathcal{A}=\mathcal{A}^I \cup \mathcal{A}^O
      \cup \mathcal{A}^H$ denotes the set of all actions.
  \item $\mathcal{T} \subseteq V \times \mathcal{A} \times V$
      is the set of transitions.
\end{itemize}
\end{definition}

If $a\in \mathcal{A}^I$ (resp. $a\in \mathcal{A}^O$, $a\in
\mathcal{A}^H$), then $(v, a, v')$ is called an input (resp.
output, internal) transition. $\mathcal{T}^I$ (resp.
$\mathcal{T}^O$, $\mathcal{T}^H$) denotes the set of input (resp.
output, internal) transitions. We say that an interface automaton
$P$ is \emph{closed} if it only has internal actions (i.e.
$\mathcal{A}^I = \mathcal{A}^O = \emptyset$); otherwise, we say
that $P$ is \emph{open}. An action $a$ is \emph{enabled} at a state
$v\in V$ if there exists a $v'\in V$ such that
$(v,a,v')\in\mathcal{T}$. The sets $\mathcal{A}^I(v)$,
$\mathcal{A}^O(v)$ and $\mathcal{A}^H(v)$ denote the set of input,
output and internal actions that are enabled from state $v$, and
$\mathcal{A}(v)=\mathcal{A}^I(v) \cup \mathcal{A}^O(v)\cup
\mathcal{A}^H(v)$ is the set of all enabled actions from state $v$.
As an example consider the drill modelled with the IA automata in
Figure~\ref{fig:drillIA}. Whenever the drill receives a product to
be processed ($put$) it signals an actuator to start processing the
product ($process$). Afterwards, the sensors signal if the product
has been successfully processed ($processOk$) or not
($processFail$). If the processing was successful, the drill
informs that the product is now ready ($ack$) and can be picked up
($get$); or report an error ($nack$) and waits for the product to
be discarded ($discard$) otherwise.


\begin{definition}\label{def:composable}\emph{(Composable)}
Two interface automata $P=\langle V_P, V_P^{init}, \mathcal{A}_P^I,
\mathcal{A}_P^O, $ $\mathcal{A}_P^H, \mathcal{T}_P \rangle$ and
$Q=\langle V_Q, V_Q^{init}, \mathcal{A}_Q^I, \mathcal{A}_Q^O,
\mathcal{A}_Q^H, \mathcal{T}_Q \rangle$ are \emph{composable} if
\begin{center}
\begin{displaymath}
    \begin{array}{cc}
     \mathcal{A}_P^H \cap \mathcal{A}_Q = \emptyset & \mathcal{A}_P^I \cap \mathcal{A}_Q^I = \emptyset \\
     \mathcal{A}_P^O \cap \mathcal{A}_Q^O = \emptyset & \mathcal{A}_Q^H \cap \mathcal{A}_P = \emptyset
    \end{array}
\end{displaymath}
\end{center}
Let $shared(P, Q)= \mathcal{A}_P \cap \mathcal{A}_Q$.
\end{definition}

The composition of IAs $P$ and $Q$ is defined in stages. First, the
product automaton $P\otimes Q$, which coincides with the
composition of I/O automata~\cite{Lynch:1987:HCP:41840.41852},
except that since $P$ and $Q$ are not necessarily input-enabled,
some transitions present in $P$ or $Q$ may not be present in the
product.

\begin{definition}\label{def:ia-product}\emph{(Product)}
Given $P=\langle V_P, V_P^{init}, \mathcal{A}_P^I, \mathcal{A}_P^O, \mathcal{A}_P^H, \mathcal{T}_P \rangle$ and $Q=\langle V_Q, V_Q^{init}, \mathcal{A}_Q^I, \mathcal{A}_Q^O, \mathcal{A}_Q^H, \mathcal{T}_Q \rangle$ composable interface automata, their product is the interface automaton $P\otimes Q=\langle V_{P\otimes Q}, V_{P\otimes Q}^{init}, \mathcal{A}_{P\otimes Q}^I, \mathcal{A}_{P\otimes Q}^O, \mathcal{A}_{P\otimes Q}^H, \mathcal{T}_{P\otimes Q} \rangle$, where:\\
\begin{eqnarray*}
  V_{P\otimes Q} &=& V_P \times V_Q \nonumber\\
  V_{P\otimes Q}^{init} &=& V_{P}^{init} \times V_{Q}^{init} \nonumber\\
  \mathcal{A}_{P\otimes Q}^I &=& (\mathcal{A}_{P}^I \cup \mathcal{A}_{Q}^I)\setminus shared(P,Q)\nonumber\\
  \mathcal{A}_{P\otimes Q}^O &=& (\mathcal{A}_{P}^O \cup \mathcal{A}_{Q}^O)\setminus shared(P,Q)\nonumber\\
  \mathcal{A}_{P\otimes Q}^H &=& (\mathcal{A}_{P}^H \cup \mathcal{A}_{Q}^H)\cup shared(P,Q)\nonumber
\end{eqnarray*}
The transition relation is defined as follows:
\begin{eqnarray*}
% \nonumber to remove numbering (before each equation)
  \mathcal{T}_{P\otimes Q} &=& \set{((u,v),a,(u,v')) | (v,a,v')\in\mathcal{T}_{P} \wedge a\notin shared(P,Q) \wedge u\in V_Q} \nonumber\\
      &\cup & \set{((u,v),a,(u',v)) | (u,a,u')\in\mathcal{T}_{Q} \wedge a\notin shared(P,Q) \wedge v\in V_P} \nonumber\\
      &\cup& \set{((u,v),a,(u',v')) | (v,a,v')\in\mathcal{T}_{P} \wedge (u,a,u')\in\mathcal{T}_{Q} \wedge a\in shared(P,Q)} \nonumber
\end{eqnarray*}
\end{definition}

Recall that interface automata do not require input-enabledness.
Therefore given the product $P\otimes Q$ of two interface automata
$P$ and $Q$, may have states in which one of them produces an
output action that is input action for the other but it is not
enabled in that state, hence it is not accepted. Such a state is
called an \emph{Illegal} state. $Illegal(P,Q)$ denotes the set of
illegal states in the product $P\otimes Q$. In
Figure~\ref{fig:armIA} we present a robotic arm that places
products in the drill ($put$), waits for the drill to finish
($ack$) and finally picks the resulting product up ($get$). The
robotic arm assumes that once products have been placed in the
drill, they will be successfully processed. Hence, it cannot handle
failures. The IA $\drill \otimes \arm$ is shown in
Figure~\ref{fig:IAproduct}. As in the case of LTS parallel
composition, each state in $\drill \otimes \arm$ consists of a
state of $\drill$ together with a state of $\arm$, and each
transition represents either a joint step on both $\drill$ and
$\arm$ processes or an individual step on one of them. Consider the
following scenario, the arm places a product in the drill ($put$),
the drill starts processing ($process$) and its sensors signal a
failure ($processFail$). This sequence lead us to state $5$ of
$\drill \otimes \arm$, which corresponds to state $5$ of $\drill$
and $1$ of $\arm$. In state $5$, the sensors are signalling the
drill that something went wrong while processing the current
product, but the $\arm$ is not prepared to handle such situation.
Such state is called $illegal$ in $\drill\otimes\arm$.

\begin{definition}\label{def:illegalStates}\emph{(Illegal States)}
Given two composable interface automata $P$ and $Q$, the set of
$Illegal(P,Q)\subseteq V_P \times V_Q$ of illegal states of
$P\otimes Q$ is defined as follows:

\begin{eqnarray*}
    Illegal(P, Q) &=& \set{(v,u)\in V_P\times V_Q | \exists a \in shared(P,Q) \cdot \\
        & & ((a\in\mathcal{A}_P^O(v) \wedge a\notin\mathcal{A}_Q^I(u)) \vee (a\in\mathcal{A}_Q^O(u) \wedge a\notin\mathcal{A}_P^I(v)))}
\end{eqnarray*}
\end{definition}

If the product $P\otimes Q$ is closed, $P$ and $Q$ are said to be
\emph{compatible} if no illegal state of $P\otimes Q$ is reachable.
If $P\otimes Q$ is open, the fact that a states is in
$Illegal(P,Q)$ is reachable does not necessarily indicate an
incompatibility, because by generating appropriate inputs, the
environment of $P\otimes Q$ may be able to ensure that no state in
$Illegal(P,Q)$ is visited. Such an environment is called a
\emph{legal environment}. The existence of legal environment
indicates that there is a way to use the interfaces $P$ and $Q$
together without giving rise to incompatibilities. A legal
environment for $R$ needs to satisfy the following side conditions.

\begin{definition}\label{def:IAlegalEnvironment}\emph{(Legal Environment)}
An environment for an interface automaton $R$ is an interface
automaton $E$ such that, (i) $E$ is composable with $R$, (ii) $E$
is nonempty, (iii) $\mathcal{A}_E^I=\mathcal{A}_R^O$, and (iv)
$Illegal(R,E)=\emptyset$.
\end{definition}

\begin{definition}\label{def:IAComposition}\emph{(IA Composition)}
The composition of two interface automata is obtained by
restricting the product of the two automata to the set of
compatible states, which are the states from which the environment
can prevent \emph{illegal states} being visited.
\end{definition}


\begin{definition} \label{def:comp-states}\emph{(Compatible States)}
Consider two composable interface automata $P$ and $Q$. A pair $(v,
u)\in V_P\times V_Q$ of states is compatible if there is an
environment $E$ for $P\otimes Q$ such that no state in
$Illegal(P\otimes Q)\times V_E$ is reachable in $(P\otimes Q)\times
V_E$ from the state $\set{(v,u)}\times V_E^{init}$.
\end{definition}

According to the definitions above, the IAs presented in
Figures~\ref{fig:drillIA} and~\ref{fig:armIA} are compatible.
Hence, there exists an environment that in which there are no
reachable \emph{illegal} states. For instance, an environment that
never fails ensures that the illegal state $5$ in
$\drill\otimes\arm$ is not reachable. In
Figure~\ref{fig:IAcomposition}, we show the composition
$\drill\|\arm$, obtained by restricting $\drill\otimes\arm$ to its
compatible states, which corresponds to the composition between
$\drill$, $\arm$ and an environment which never lets the drill to
fail processing products.

\begin{figure}[bt]
\centering
    \VCDraw{
%        \ShowFrame
        \begin{VCPicture}{(-10,-1)(10,5)}
        \State[0]{(-5,0)}{0}
        \Initial{0}
        \State[1]{(0,0)}{1}
        \State[2]{(5,0)}{2}
        \State[3]{(5,3.5)}{3}
        \State[4]{(-5,3.5)}{4}
        \EdgeL{0}{1}{put?}
        \EdgeL{1}{2}{process!}
        \EdgeL{2}{3}{processOk?}
        \EdgeR{3}{4}{ack!}
        \EdgeL{4}{0}{get?}
        \end{VCPicture}
    }
\caption{$\drill \| \arm$.}
\label{fig:IAcomposition}
\end{figure}

Based on the definitions above we define the notion of \emph{LTS
legal environment} which will be used in future definitions.

\begin{definition}\label{def:ltslegalEnvironment}
\emph{(LTS Legal Environment)} Given $M = (S_M, $ $ L_M, \Delta_M,
s_{M_0})$ and $P =$ $(S_P,L_P,\Delta_P,s_{P_0})$ LTSs, where $L_M$
$=L_{M_c}\sqcup L_{M_u}$ and $L_P=L_{P_c}\sqcup L_{P_u}$. We say
that $M$ is an \emph{LTS legal environment} for $P$ with controlled
actions $L_{M_c}$, if the interface automaton $M'=\langle S_M,
\set{s_{M_0}}, L_{M_u}, L_{M_c}, \emptyset, \Delta_M \rangle$ is a
\emph{legal environment} for the interface automaton $P'=\langle
S_P, \set{s_{P_0}},$ $L_{P_u}, L_{P_c}, \emptyset, \Delta_P
\rangle$.
\end{definition}