%optimizations
\subsection{Strip Representatives}
We introduce now two different optimizations.  The first
takes advantage of the nature of induced plants.  
No new deadlock can be introduced when searching for induced plants
(since it will violate the non realizability legal property), 
this leads to an optimization where we can take a representative 
for every strip within the
automaton that contains a monitored transition.

\begin{definition}\label{def:strip}\emph{(Strip)}
Given $M = (S_M, L, \Delta_M, s_{M_0})$ an LTS and 
$\mathcal{C}$ a set of controllable actions, 
$\varpi \in \Delta_M*$ and $\varpi_i = (s_i, a_i, t_i)$ stand for
the transition at position $i$ within the sequence, we will
say that $\varpi$ is a strip if the following holds: 
\[
\begin{matrix}{r l}
\forall i \in 1 \ldots |\varphi| - 1:& \varpi_i = (s_i,a_i,t_i), \varphi_{i+1} = (s_{i+1} = t_i, a_{i+1}, t_{i+1})\\
& d_{out}(t_i) = 1\\
\exists i \in 1 \ldots |\varphi| - 1: & a_i \not\in \mathcal{C}\\
\end{matrix}
\]
\end{definition}

We want to identify these strips since taking any monitored transition
from the strip out the plant and then reducing will lead to same
induced plant.  It is clear to see that under the non realizability
legal reduction it is equivalent to remove any subset of monitored transitions
within the strip since the algorithm will keep removing them until
no new deadlock is found, and since the strip has outgoing degree $=1$
all the way through, if we were to remove an element of $\varpi$ we will
be forced to remove all of them.\\
This optimization gains relevance since the complexity of our minimization
is driven by the number of monitored transitions to check.
W.l.o.g we can pick any monitored transition for each strip and use them
as the set of elements to check for minimization.
We can build this set as explained in the algorithm from Figure
\ref{fig:strip-code}.
\begin{figure}[ht]
  \begin{center}
    \renewcommand{\ttdefault}{pcr}
\begin{lstlisting}[escapeinside={[*}{*]},basicstyle=\scriptsize\ttfamily,columns=flexible,frame=lines,mathescape=true,xleftmargin=3.0ex,keywordstyle=\textbf,morekeywords={if,while,do,else,fork,int,null},numbers=left,numberstyle=\scriptsize]
build_representatives($\mathcal{I} = <E, \mathcal{C}, \varphi>$):
	candidates = $E$.get_monitored_transitions() [*\listRef{A}*]
	representatives = []
	while candidates != $\emptyset$
		$t$ = candidates.pop() [*\listRef{B}*]
		head = $t$
		tail = $t$
		representatives.push($t$)
		while $d_{in}$(head.from_state) == 1 
			or $d_{out}$(tail.to_state) == 1:
			if $d_{in}$(head.from_state) == 1: [*\listRef{C}*]
				if head.action $\not\in \mathcal{C}$:
					candidates.remove(head) [*\listRef{D}*]
				head = head.predecessor
			if $d_{out}$(tail.to_state) == 1: [*\listRef{E}*]
				if tail.action $\not\in \mathcal{C}$:
					candidates.remove(tail)
				tail = tail.successor
	return representatives
\end{lstlisting} 
    \caption{Strip representatives computation}
    \label{fig:strip-code}
  \end{center}
\end{figure}
The set of monitored transitions is taken as the initial candidates
\listRef{A} and then checks what we could call the strip closure
for the remaining elements \listRef{B} by adding them to the representatives
array. Then it recursively picks the preceding transition whenever
the source state has an incoming degree of 1 \listRef{C}, it removes
monitored transitions found this way \listRef{D}.  The same procedure
is performed onwards for target states with an outgoing degree of 1
\listRef{E}. We can see that the set of representatives is constructed 
this way after a linear scan of a collection of size at most $|\Delta|$.

\subsection{Radial Approximation}
Under the hypothesis that for certain scenarios (see section
\ref{section:validation-ctv}) counterstragies
 can be built over a set of states close to $s_0$ it is tempting
 to perform what we will call a radial approximation 
 of $E$ when checking for a minimal induced plant.

\begin{figure}[ht]
  \begin{center}
    \renewcommand{\ttdefault}{pcr}
\begin{lstlisting}[escapeinside={[*}{*]},basicstyle=\scriptsize\ttfamily,columns=flexible,frame=lines,mathescape=true,xleftmargin=3.0ex,keywordstyle=\textbf,morekeywords={if,while,do,else,fork,int,null},numbers=left,numberstyle=\scriptsize]
smallest_non_realizable_approximation($E$):
	max_d = $E$.get_maximum_distance()
	low = 0
	hi = max_d
	$k$ = $\lfloor \frac{hi}{2}\rfloor$ [*\listRef{A}*]
	$E_{old}$ = $E$
	$E_{new}$ = $E$
	while $k$ != -1:
		if hi - low < 1:
			return $E_{old}$
		$E_{new}$ = $E$.clone()
		$E_{new}$.prune_at($k$) [*\listRef{B}*]
		$E_{new}$.get_legal_reduction($E$)[*\listRef{C}*]
		is_realizable = check($E_{new}$)[*\listRef{D}*]
		if is_realizable:
			low = $k$ [*\listRef{E}*]
		else:
			hi = $k$ [*\listRef{F}*]
			$E_{old}$ = $E_{new}$ [*\listRef{G}*]
		$k$ = low + $\lfloor \frac{hi - low}{2}\rfloor$ [*\listRef{H}*]
		if $k$ <= 1:
			return $E_{old}$
	return $E_{old}$
\end{lstlisting} 
    \caption{Binary approximation computation}
    \label{fig:kapprox-code}
  \end{center}
\end{figure}

For the approximation described in Figure \ref{fig:kapprox-code}
we use the distance of any state to $s_0$ (\texttt{max\_d}) to guide the binary search of (non minimal) induced plant.  We will prune $E$ removing states
at distances greater than $k$ from $s_0$ performing a non realizability legal
reduction and the checking for realizability.
This approximation can be regarded as a pre processing of the plant, since it
will (hopefully) perform a faster legal reduction by sacrificing minimality.
It computes $k$ by performing a binary search over the distance at which the
plant will be pruned and reduced.  The algorithm starts by assigning the
value $\lfloor \frac{hi}{2}\rfloor$ \listRef{A} for $k$ 
and then it keeps pruning \listRef{B}, reducing \listRef{C} and checking
for realizability \listRef{D} as long as the search can continue.  The search
range is updated by raising the lower bound if the check yielded a positive 
value \listRef{E} or lowering the upper bound if the check informed that the
problem was unrealizable \listRef{F}.  The last non realizable plant is updated 
in this case since the reduction ensures legality while preserving 
non realizability.  $K$ is consequently updated \listRef{H} and the search
carries on until the boundaries converge.  This procedure will perform
up to $log_2(|\Delta|)$ checks to yield an approximated induced plant, which
may not be minimal but can still be given to one of the other minimization
techniques to achieve a minimal induced plant.



%\begin{figure}[ht]
%  \begin{center}
%    \renewcommand{\ttdefault}{pcr}
\begin{lstlisting}[escapeinside={[*}{*]},basicstyle=\scriptsize\ttfamily,columns=flexible,frame=lines,mathescape=true,xleftmargin=5.0ex,keywordstyle=\textbf,morekeywords={if,while,do,else,fork,int,null},numbers=left,numberstyle=\scriptsize]

dd_min($E$):
	$\mathcal{T}_u$ = $E$.get_monitored_transitions()
	return dd_min'($\mathcal{T}_u$, $[ ]$, 2, $E$) [*\listRef{A}*]
dd_min'($\mathcal{T}_{u}$, $\mathcal{T}_{r}$, $n$, $E$):
	if $n$ <= $|\mathcal{T}_{u}|$:
		res = dd_subsets($\mathcal{T}_{u}$, $\mathcal{T}_{r}$, $n$, $E$, $\bot$)
		if res != $None$:
			return res
		if $n$ > 2:
			res = dd_subsets($\mathcal{T}_{u}$, $\mathcal{T}_{r}$, $n$, $E$, $\top$)
			if res != $None$:
				return res		
		if $n$ < $|\mathcal{T}_{u}|$:
			$n'$ = min($|\mathcal{T}_{u}|$, 2 * $n$)
			return dd_min'($\mathcal{T}_{u}$, $\mathcal{T}_{r}$, $n'$, $E$)
	$E_{new}$ = $E$.clone()	
	for $t$ in $\mathcal{T}_{u}$:
		$E_{new}$ = $E_{new}$.remove($t$)
	$E_{new}$.get_legal_reduction() [*\listRef{B}*]
	$\mathcal{T}'_u$ = $E_{new}$.get_monitored_transitions()	
	return $E_{new}$
dd_subsets($\mathcal{T}_{u}$, $\mathcal{T}_{r}$, $n$, $E$, check_complement):
	step = $\lfloor \frac{|\mathcal{T}_{u}|}{n}\rfloor$ [*\listRef{A}*]
	rem = $|\mathcal{T}_{u}| \% n$
	first = 0, last = -1
	for $i$ in $[0\ldots n-1]$:
		first = last + 1
		last = first + step - 1
		if rem > 0:
			last += 1
			rem -= 1
		if check_complement:
			$\mathcal{T}'_{u}$ = $\mathcal{T}_{u} \setminus \mathcal{T}_{u}[first\ldots last]$
			$\mathcal{T}'_{r}$ = $\mathcal{T}_{r} \cup (\mathcal{T}_{u} \setminus \mathcal{T}'_{u})$
			$n'$ = max(2, $n$ -1)
		else:
			$\mathcal{T}'_{u}$ = $\mathcal{T}_{u}[first\ldots last]$
			$\mathcal{T}'_{r}$ = $\mathcal{T}_{r} \cup (\mathcal{T}_{u} \setminus \mathcal{T}'_{u})$
			$n'$ = 2
		if test_subset($\mathcal{T}'_{r}$) == $\bot$:
			return dd_min'($\mathcal{T}'_{u}$, $\mathcal{T}'_{r}$, $n'$, $E$)						
	return $None$
\end{lstlisting} 
%    \caption{DD exploration code}
%    \label{fig:dd-code}
%  \end{center}
%\end{figure}
