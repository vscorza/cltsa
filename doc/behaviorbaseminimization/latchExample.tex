To exemplify the difference, and the complementation between the model-based
and the behavior-based minimization we introduce a synthetic case where an engineer
wants to specify a zero triggered latch mechanism. The initial specification
involves only two signals: $x$ representing a single boolean input and $y$ the only
boolean output.  The idea is that the output should be kept constantly up 
once a zero boolean value has been fed to the system.
Before this $y$ must remain as a zero.  
The specification is written as the conjunction of four \texttt{LTL} \cite{pnueli1977temporal}
formulas as presented in table \ref{table:d-ff-spec}. 
Following the definitions from \cite{DBLP:conf/vmcai/PitermanPS06} $\theta_s$ stands for
an initial system formula, $\rho_s$ for a safety system formula and 
$\varphi_s$ for a system liveness formula.
The \textit{Initial condition} formula 
fixes the initial value for the output signal,
\textit{Wait for negative trigger} keeps the value of $y$ down as long as the input
is up and the output value has not yet been raised, \textit{Latch} keeps the 
output up after the initial raise and the \texttt{GR(1)} \textit{Goal} is to 
always eventually produce an output ($\square\Diamond(y)$).  After running our automated
translation we produced the \texttt{LTS} representation of the \textit{Game Structure}
definition.  We get the behavior depicted in Figure \ref{fig:fig.neg-trigger-latch-behavior},
where again solid lines indicate monitored events and dashed lines indicate controlled events.
The $\uparrow$ suffix after a signal name means the the value is being raised ($x\uparrow$),
$\downarrow$ consequently means that value is being lowered ($x\downarrow$).  A $tick$ event
is added to show when the values could be kept constant at the current time step.
\begin{center}
\begin{table}[h]
  \begin{tabular}{ l  c  l }
	Name & Type & Requirement \\
    \hline
    Initial condition & $\theta_s$ & $y = 0$\\
    Wait for negative trigger & $\rho_s$ &$\square((x \wedge \neg y) \rightarrow \Circle(\neg y) )$ \\
    Latch & $\rho_s$ &$\square (y \rightarrow \Circle(y))$ \\   
    Goal &  $\varphi_s$ & $\square\Diamond(y)$\\
  \end{tabular}
  \caption{Zero triggered latch specification}
  \label{table:d-ff-spec}
 \end{table}
\end{center}
We wrote this specification in \texttt{RATSY}, run the minimization presented in
\cite{DBLP:conf/hvc/KonighoferHB10} (where the \textit{Latch} requirement was
considered irrelevant and removed) and then translated the minimized specification
to produce the \texttt{LTS} depicted in Figure 
\ref{fig:fig.neg-trigger-latch-behavior-formula-minimization}.  Understanding the specification
in terms of the conjunction of formulas we can see that the $Latch$ requirement is irrelevant
to the unrealizability cause since even if we allow the plant to lower $y$ after its initial
raise, no output would be produced if the input is kept up.  Removing a formula is, in fact,
relaxing the behavior, this is expressed in the double lined states and transitions 
present in Figure \ref{fig:fig.neg-trigger-latch-behavior-formula-minimization}.  
We can see that the transitions added express the new behavior where the plant is allowed
to lower $y$ after being raised.  Let us note that removing a formula means relaxing, 
thus augmenting, the behavior of the plant.
\begin{figure}[bt]
\centering
\SmallPicture
%\ShowFrame
\VCDraw{
    \begin{VCPicture}{(-4,-5)(4,4)}
        \SetEdgeLabelScale{1.4}
        \State[1]{(0,3)}{1}
        \State[2]{(-3,2)}{2}        
        \State[3]{(-3,0)}{3}
        \State[4]{(3,-2)}{4}        
        \State[5]{(-3,-2)}{5}                
        \State[6]{(-3,-4)}{6}                        
        \State[7]{(3,2)}{7}                                
        \State[8]{(3,0)}{8}                                        
		\Initial[n]{1}
		\EdgeL{1}{2}{$x$\downarrow}
		\EdgeR{1}{7}{$x$\uparrow}		
		\EdgeR{8}{3}{$x$\downarrow}								
		\LoopW{3}{$tick$}
		\LoopE{8}{$tick$}		
		\LoopW{6}{$tick$}		
		\EdgeR{3}{4}{$x$\uparrow}
		\ArcR{5}{6}{$x$\uparrow}
		\ArcR{6}{5}{$x$\downarrow}		
        \ChgEdgeLineStyle{dashed} 
		\EdgeR{4}{6}{$y$\uparrow}		
		\EdgeL{2}{3}{$y$\downarrow}				
		\EdgeL{7}{8}{$y$\downarrow}						
		\EdgeR{3}{5}{$y$\uparrow}
		\EdgeR{4}{8}{$tick$}		
    \end{VCPicture}
}
\vspace*{-2mm}
\caption{Zero triggered latch behavior.}
\label{fig:fig.neg-trigger-latch-behavior}
\vspace*{-4mm}
\MediumPicture
\end{figure}

\begin{figure}[bt]
\centering
\SmallPicture
%\ShowFrame
\VCDraw{
    \begin{VCPicture}{(-4,-6)(4,5)}
        \SetEdgeLabelScale{1.4}
        \State[1]{(0,3)}{1}
        \State[2]{(-3,2)}{2}        
        \State[3]{(-3,0)}{3}
        \State[4]{(1,-1)}{4}        
        \State[5]{(-3,-2)}{5}                
        \State[6]{(-3,-4)}{6}                        
        \State[7]{(3,2)}{7}                                
        \State[8]{(3,0)}{8}                                        
        \StateLineDouble
		\State[9]{(1,-2.5)}{9}                         
        \State[10]{(-1,-2.5)}{10}                             
		\Initial[n]{1}
		\EdgeL{1}{2}{$x$\downarrow}
		\EdgeR{1}{7}{$x$\uparrow}		
		\ArcR{8}{3}{$x$\downarrow}								
		\LoopW{3}{$tick$}
		\LoopE{8}{$tick$}		
		\EdgeL[.8]{3}{4}{$x$\uparrow}
		\ArcR{5}{6}{$x$\uparrow}
		\ArcR{6}{5}{$x$\downarrow}	
		\EdgeLineDouble		
		\LoopS{9}{$tick$}	
		\ArcR[.3]{6}{9}{$tick$}
		\EdgeR{9}{10}{$x$\downarrow}
		\EdgeLineSimple
        \ChgEdgeLineStyle{dashed} 
		\EdgeL{2}{3}{$y$\downarrow}				
		\EdgeL{7}{8}{$y$\downarrow}						
		\ArcR{3}{5}{$y$\uparrow}
		\EdgeL{4}{8}{$tick$}		
		\EdgeLineDouble
		\EdgeR{10}{5}{$tick$}		
		\ArcR[.2]{10}{3}{$y$\downarrow}						
		\ArcR{4}{9}{$y$\uparrow}		
		\ArcR{9}{8}{$y$\downarrow}		
		\ArcR{5}{3}{$y$\downarrow}
		\VArcR{arcangle=-60,ncurv=1}{6}{8}{$y$\downarrow}				
    \end{VCPicture}
}
\vspace*{-2mm}
\caption{Zero trig. latch formula based minimization.}
\label{fig:fig.neg-trigger-latch-behavior-formula-minimization}
\vspace*{-4mm}
\MediumPicture
\end{figure}
This comparison may seem unfair, and to some extent it is, if one
forgets that we are looking at the same problem expressed in two
quite different ways.  One of our claims is that our approach is in
fact complimentary to that of the model-based minimization.
Our minimization produces the plant depicted in Figure 
\ref{fig:fig.neg-trigger-latch-behavior-minimization}, where it
shows that the environment can easily win by playing to keep
$x$ up.  
It is important to note that when minimizing the plant we must
keep it unrealizable while allowing the environment to have
a winning strategy that works in both the minimized and the original
plant.
\begin{figure}[bt]
\centering
\SmallPicture
%\ShowFrame
\VCDraw{
    \begin{VCPicture}{(-5,-2)(5,2)}
        \SetEdgeLabelScale{1.4}
        \State[1]{(-3,0)}{1}
        \State[2]{(0,0)}{2}        
        \State[3]{(3,0)}{3}
		\Initial[w]{1}
		\EdgeL{1}{2}{$x$\uparrow}
		\LoopE{3}{$tick$}		
        \ChgEdgeLineStyle{dashed} 
		\EdgeL{2}{3}{$y$\downarrow}				
    \end{VCPicture}
}
\vspace*{-2mm}
\caption{Zero trig. latch behavior based minimization.}
\label{fig:fig.neg-trigger-latch-behavior-minimization}
\vspace*{-4mm}
\MediumPicture
\end{figure}
In the model-based and the behavior-based approaches the
idea is that we are aiding the engineer in understanding the
cause on unrealizability.  If we think in terms of specification repair,
in this particular case if she were to add
a new assumption (\textit{input will eventually fall}), either as
$\square\Diamond(\neg x)$ or $\Diamond(\neg x)$ the specification
will become realizable.
