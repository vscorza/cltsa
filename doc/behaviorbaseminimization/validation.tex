\newpage 
\section{Validation}
The extent to which the technique serves it purpose is defined by its soundness, efficiency and, more importantly, information, understood as the ability to guide the engineer while writing a non trivial formal specification. The presented diagnosis should help by pointing out the causes of non realizability (expressed in a behavioral implicit language) that may appear while progressively gathering knowledge and building a complete and precise model.

\textbf{<EFFICIENCY CLAIM>}

The measure of information our technique may provide should ultimately be related to the impact it can have on the overall specification process, in particular how clear it is for the engineer  to interpret the diagnosis in terms of amount of data presented and its relevance in relation to the non realizability cause. The unavailability of skilled professionals and the amount of time and effort required to prepare a considerable set of external subjects to the point where they would be able to write non trivial formal specifications renders the perspective of a controlled experiment with external engineers writing specifications from scratch unfeasible. On the other hand, it is also imposible to propose a direct comparison with similar techniques, since, for instance, in ~\cite{DBLP:conf/hvc/KonighoferHB10} the minimization is applied on the set of environmental safety formulas, which in fact increments the volume of the plant when presented as an automaton. 

\textbf{<REF. TO JTVS (MAOZ)>}

Under the light of these facts we propose a small set of metrics to quantitatively evaluate our technique. The first metric is volume, defined as the relation of the minimized plant against the original instance.

\[ v_{\mathcal{U}} = \dfrac{|E'|}{|E|} \]

The number of transitions was chosen as the representation of size for each automaton. It also desirable to observe the relation of this metric with a similar notion of volume, but one defined between the size of the controller (for the realizable version of the problem) and the original plant.

\[ v_{\mathcal{C}} = \dfrac{|C|}{|E|} \]

A measure of controlability is defined, observed and related to each volume. Controlability is defined as the number of controllable transitions in the plant over the total number of transitions. The motivation behind it is that a plant with a higher controlability measure could be related to a more powerful system, against which the technique can not prune much behavior.

\[ c = \dfrac{|T_{\mathcal{C}}|}{|E|} \]

\newpage
"Te paso el lineamiento de lo que voy a escribir para el viernes, hay dos limitaciones fuertes para no presentar otro tipo de experimento para probar la adecuación de la técnica,

no vamos a realizar un estudio controlado con sujetos que escriban y diagnostiquen especificaciones
no hay una noción clara para comparar las técnicas existentes (minimización de fórmulas ltl) que de hecho agregan comportamiento, con nuestra técnica


Es por esto que vamos a utilizar un proxy para aproximar la calidad del recorte o minimización, la idea es utilizar el volumen (cantidad de transiciones) en comparación con la planta original. Suponiendo que E es la planta original, E' la planta minimizada y C el controlador del caso realizable vamos a presentar y analizar las relaciones |C|/|E| y |E'|/|E| donde |S| reporta la cantidad de transiciones de E.

Habrá que definir una magnitud que exprese que tan controlabe es un sistema y evaluar como se relaciona con la medida anterior. Por ejemplo |U|/|S| donde |U| expresa la cantidad de transiciones no controlables de E y |S| la cantidad de transiciones controlables."


The validation reported in this section is guided by two main concerns.  This first is the degree to which the initial control problem is reduced and the computational cost involved. The second is if the minimization offers insight into non realizability causes.
In order to avoid bias, we prioritized using third-party developed unrealizable specifications that include explanations for the causes of unrealizability.
Unfortunately, third party controller synthesis problems that are found in the literature are typically realizable. The only sources we found were from ~\cite{DBLP:conf/hvc/KonighoferHB10,DBLP:conf/kbse/DegiovanniRACA16,DBLP:conf/fmcad/AlurMT13}. More specifically, we used Generalized Buffer case study from~\cite{DBLP:conf/hvc/KonighoferHB10}, the Lift Controller case study from ~\cite{DBLP:conf/fmcad/AlurMT13} and the Mine Pump case study from~\cite{DBLP:conf/kbse/DegiovanniRACA16}.
The case studies taken from~\cite{DBLP:conf/hvc/KonighoferHB10,DBLP:conf/kbse/DegiovanniRACA16,DBLP:conf/fmcad/AlurMT13} all have  the specification of the control problem  provided as a set of LTL formulas rather than automata. Hence, they required a translation of the safety formulae of these specifications into LTS. To complete our validation, we also included a series of case studies that were provided in their original formulation as a composition of automata: the t-strong fairness case study from \cite{DBLP:conf/icse/DIppolitoBPU11}, a game of tic-tac-toe. For these we were forced to manually seed unrealizability causes. Finally, we were provided an unrealizable specification for the CTvPlatform Controller~\cite{DBLP:journals/computer/OmmeringLKM00} which we analyzed. 
In the following we report quantitatively on all case studies to show the degree to which the initial control problem is reduced and the computational cost involved. We also report qualitatively on the case studies taken from ~\cite{DBLP:conf/hvc/KonighoferHB10,DBLP:conf/kbse/DegiovanniRACA16,DBLP:conf/fmcad/AlurMT13} as they provide an unbiased basis against to which it is possible to evaluate the insight into non realizability causes that our technique provides. 

%The case study described here comes
%from automata based specifications and shows 
%the existence of relatively simple
%problems when highly modular specifications are involved.\\

All case studies were run using an extension of the MTSA tool \cite{DBLP:conf/kbse/DIppolitoFCU08}. MTSA natively supports specification of LTS \cite{DBLP:journals/cacm/Keller76} and properties using
 a textual, process algebraic notation and linear temporal logic.  The tool supports synthesis of controllers for GR(1) control problems where the safety part of the problem can be specified as an composition of LTS, and the liveness goals and assumptions are provided using temporal logic.   
 %Prior to this work the only feedback given to the user  when specifying an unrealizable problem was a legend  reading that \textit{there was no controller for the given
 %specification}.  
  The extension builds a minimized unrealizable version of the environment that can then be inspected through various methods including animations and visualization of the state space.
  
%\subsection{Quantitative Evaluation}


  
  
The quantitative results of our experiments are shown in table 
\ref{table:quantitative-results} and were run on an 
\texttt{Intel\textsuperscript{\textregistered} Core\textsuperscript{\texttrademark}
 i7-4790} CPU with 8 processors of 3.60GHz frequency
16 Gb of RAM memory and \texttt{ubuntu 14.04} as the operating system.
Total time taken by the algorithm is measured in milliseconds and
labeled as \emph{Time}, $|S_e|$ and $|S_{e\kappa}|$ stand for
the size of the original and minimized LTSs in terms of states,
\emph{State Reduction} is the reduction percentage of these.  
$|\Delta_e|$ and $|\Delta_{e\kappa}|$ stand for
the size of the original and minimized LTSs in terms of edges,
\emph{Transition Reduction} is the reduction percentage of the later two.  
%$Time/S_e$ is the relative measure of time taken per state.
$steps$ reports how many iterations the algorithm needed before finding
the first Minimal Alternating Sub-LTS. \emph{Edges Explored} reports the percentage of
edges the algorithm needed to explore before finding the first
conflict.
 
\begin{table*}
\centering
  \begin{tabular}{ | l | c | c | c | c | c | c | c | c | c | }
    \hline
 & Time & &  &State&  & &Transition&&Edge\\
Case & (ms) & $|S_e|$ & $|S_{e\kappa}|$ &Reduction& $|\Delta_e|$ & $|\Delta_{e\kappa}|$ &Reduction&steps&Explored\\
    \hline
t strong fairness&10.65&7&4&57.14\%&9&4&44.44\%&3&33.33\%\\
%t strong fairness b &15&7&5&71.43\%&9&6&66.67\%&5&55.56\%\\
mine pump &8.7&14&5&35.71\%&20&5&25.00\%&6&30.00\%\\
lift controller&121&95&43&45.26\%&130&50&38.46\%&35&26.92\%\\
Tv controller&11031&218&8&3.67\%&1416&21&1.48\%&43&3.04\%\\
%tictactoe3 &66054&4658&176&3.78\%&18561&354&1.91\%&414&2.23\%\\
%tictactoe4 &70896&4696&236&5.03\%&18647&472&2.53\%&529&2.84\%\\
%tictactoe5 &69738&4696&236&5.03\%&18647&472&2.53\%&529&2.84\%\\
tictactoe&75279&4696&236&5.03\%&18647&472&2.53\%&529&2.84\%\\
%tictactoe2&65449&4590&184&4.01\%&19241&420&2.18\%&418&2.17\%\\
gen buf (miss. assumpt.)&6688142&10964&2235&20.38\%&14894&2963&19.89\%&1093&7.34\%\\
gen buf (added goal)&7882862&10964&850&7.75\%&14894&1139&7.64\%&646&4.33\%\\
gen buf (added restr.)&73&39&26&66.66\%&47&28&59.57\%&10&21.27\%\\
\hline    
  \end{tabular}
  \vspace*{2mm}
  \caption{Quantitative results for minimized plants}
  \label{table:quantitative-results}
 \end{table*}

%DD complexity, DFS reduction

%\subsection{Optimization}
%\subsection{Test Cases}
%\subsection{Experimental Framework}
%\subsection{Results}

From the results it is possible to argue that the minimization technique can be applied to control problems of similar size to those reported in the literature. 
%In addition, it is possible to observe, as expected, that the  that spatial and temporal limitation behave linearly with respect to the realizability query. 
In addition, it can be seen that the technique can achieve a significant reduction in terms of states and transitions.


\subsection{CTvPlatform Controller}\label{section:validation-ctv}
In \cite{DBLP:journals/computer/OmmeringLKM00} the authors
introduce a modular specification for a TV System using
their Koala model. At the time of this writing  
another member of our group was working on a 
specification inspired in this particular system
and introducing some abstractions, such as the user 
module, from \cite{DBLP:conf/sigsoft/UchitelCKM04}.
As in the original scenario the specification was decoupled
in several modules to achieve cohesion.  
Relevant to this case study is the
TV set that can display random noise while
tuning, set the output to the specified channel afterwards
or eventually reset the channel selection and
go blank. The other modules describe the
behavior of the user (selecting a particular channel), 
the tuning sub-system (linking user selection with
frequency filtering) and the screen driver (that outputs
the corresponding image, be it a channel, random noise
or simply nothing when going blank). 
The goal was to synthesize a 
controller for a TV that must satisfy the following
liveness requirement: $\square \Diamond (select.i \implies display.i)$. 
In this example we found that 
compositionality trades cohesion for behavioral abstraction.
This means that even when the behavior of each component is fairly simple their concurrent interaction is difficult to
grasp.  Applying our minimization technique to several ongoing
specifications within our group we found that even though
the automaton exposed a complex potential behavior (in terms
of its state space) unrealizability could be explained without going too far away from the starting point.
The first CTvPlatform TV Controller diagnosis, as depicted in figure
\ref{fig:tv-controller-minimization} showed that the tuner
introduced a non controllable self loop where the environment
could flood the execution with incoming tuning signals ($signal.i$) and that
even if a controllable action was allowed to escape the
looping state the screen was allowed to produce 
the $blank$ action indefinitely. \\
\begin{figure}[bt]
\centering
\SmallPicture
%\ShowFrame
\VCDraw{
    \begin{VCPicture}{(-4,-4)(4,4)}
        \SetEdgeLabelScale{1.4}
        \State[1]{(-4,0)}{1}
        \State[2]{(0,0)}{2}
        \State[3]{(4,0)}{3}              
        \State[4]{(2,-3)}{4}
		\Initial[w]{1}
		\LoopN{1}{signal.1}
		\LoopN{2}{signal.1}		
		\ArcR{2}{3}{blank}				
		\LoopN{3}{blank}								
		\ArcR{2}{4}{unblank}
		\ArcR[.75]{4}{3}{blank}		
		\ArcR[.6]{3}{4}{unblank}				
        \ChgEdgeLineStyle{dashed} 
		\EdgeL{1}{2}{tick}		
		\ArcR{3}{2}{tick}						        
    \end{VCPicture}
}
\vspace*{-2mm}
\caption{CTvPlatform TV Controller minimization.}
\label{fig:tv-controller-minimization}
\vspace*{-4mm}
\MediumPicture
\end{figure}

\subsection{Mine Pump Controller (MPC)}
%introduce translation for GS specs
The Mine Pump Controller (MPC)
 originally presented in
\cite{:/content/journals/10.1049/ip-e.1983.0001} describes a controller that should
drain water from a mine whenever its level surpassed
a certain threshold, with the restriction that it should
stop the pump if the sensors indicated that a high level
of methane was present due to the risk of ignition.\\

We used the variant specification of the problem provided  in \cite{DBLP:conf/kbse/DegiovanniRACA16}
where they highlight that two safety goals ($\square(methane.hi \implies \Circle(\neg running))$
and $\square(water.hi \implies \Circle(running))$) are logically consistent but there is a \textit{boundary condition} 
that if true, makes the conjunction of the two safety goals impossible. The GR(1) liveness requirement states that the
 water level to be kept within reasonable limits infinitely often ($\square \Diamond \neg water.Hi$).
 %, as
 %already stated in \cite{:/content/journals/10.1049/ip-e.1983.0001}
 %when expecting that \emph{“the pump
 %shall be on when the water level is above the high threshold”}.


We applied our minimization on the translated specification from \cite{DBLP:conf/kbse/DegiovanniRACA16}
to get the LTS depicted in Figure
 \ref{fig:mine-pump-minimization}.  We can see here that
 the minimization properly slices the LTS to expose what
 is characterized
 as the boundary condition $(\Diamond hw \wedge m)$ and in our
 case is preventing the system from achieving its goal.
 \\

 
 This slice should lead the engineer towards adding
 the missing assumption $\square \Diamond (\neg methane.hi \wedge \neg water.hi)$.
 If the original problem and the implementation proposed in 
the appendix from \cite{:/content/journals/10.1049/ip-e.1983.0001}
are expected to be deployed, this assumption should be added for
the problem to be realizable.\\


Now we explain the behavior depicted in Figure \ref{fig:mine-pump-minimization}.
 After the initial condition is met (between states 1 and 
 5) the environment can then raise the water level (transition
 from state 6 to 7) forcing the system to start the pump
 (transition from state 7 to 8) but then raise the
 methane level (transition from 8 to 9) giving to option
 to the system but to turn off the pump (transition 9
 to 10).  After this the environment can keep the
 signals constant denying the system the opportunity
 to satisfy its goal.\\
 
%This example shows the relevance of behavior base minimization
%for unrealizable specifications initially described as
%a set of LTL formulae over boolean variables.  This is
%an explicit representation of the unrealizability cause
%that gives immediate feedback about the interaction
%of the system and the environment that leads to the
%violation of the liveness guarantee.
\begin{figure}[bt]
\centering
\SmallPicture
%\ShowFrame
\VCDraw{
    \begin{VCPicture}{(-6,-3)(6,3)}
        \SetEdgeLabelScale{1.4}
        \State[1]{(-6,2)}{1}
        \State[2]{(-2,2)}{2}
        \State[3]{(2,2)}{3}
        \State[4]{(6,2)}{4}
        \State[5]{(6,0)}{5}        
        \State[6]{(2,0)}{6}        
        \State[7]{(-2,0)}{7}        
        \State[8]{(-6,-2)}{8}        
        \State[9]{(-2,-2)}{9}        
        \State[10]{(2,-2)}{10}        
		\Initial[w]{1}
        \EdgeL{1}{2}{methane.hi \downarrow}		
        \EdgeL{2}{3}{water.lo \downarrow}		        
        \EdgeL{3}{4}{water.hi \downarrow}		                
        \EdgeR{5}{6}{water.lo \uparrow}
        \EdgeR{6}{7}{water.hi \uparrow}        
        \EdgeL{8}{9}{methane.hi \uparrow}                        
        \LoopE{10}{tick}                                
        \ChgEdgeLineStyle{dashed} %\EdgeLineDouble
        \EdgeL{9}{10}{running \downarrow}                        
        \EdgeR{4}{5}{running \downarrow}		                
        \ArcR{7}{8}{running \uparrow}                        
    \end{VCPicture}
}
\vspace*{-2mm}
\caption{Mine Pump Controller (MPC) minimization.}
\label{fig:mine-pump-minimization}
\vspace*{-4mm}
\MediumPicture
\end{figure} 
\subsection{Modified Lift Controller}
Another example that has been evaluated is that
of the Modified Lift Controller from 
\cite{DBLP:conf/fmcad/AlurMT13} first presented in
its realizable version in 
\cite{DBLP:journals/jcss/BloemJPPS12}.
In the original version a lift controller should serve
several floors.  It has a button sensor for each floor 
controlled by the environment.  There is an additional 
restriction that allows the lift to move up only if there
is a pending request in one of the floors 
($\square \Diamond (b_j \implies f_j)$), and another that
forces the system to visit the first floor infinitely
often if no button was pressed.\\
In the modified version from ~\cite{DBLP:conf/fmcad/AlurMT13} a new set of system liveness 
requirements is added, where the system should 
visit each floor infinitely often ($\square \Diamond f_j$).


The problem here is that this requirement collides with
the restriction that the lift will not move up if there
is no pending request.\\
We run the minimization over a simplified two floors
specifications and produced the LTS depicted in Figure
\ref{fig:lift-controller-minimization}.
It shows that after initial variable setting on behalf of the
environment (states 1 to 3) the system is free to choose 
opposite values for the initial floor (either start at
floor 1, states 3 to 5, or start at the second floor,
states 3 to 8), if it chooses to start at the first floor
the environment leaves the $b1, b2$ signals down, falsifying
the $\square \Diamond f_j$ requirement, since it cannot 
go up to visit the second floor unless $b2$ is up.  On the
other hand if the system chooses the second floor as the
initial configuration the environment keeps waiting until
the lift descends to win by letting the signals down.
The system is forced to descend eventually to satisfy
the restriction that it should visit the first floor when
$b1$ and $b2$ are down.  In this example we have to do
some more interpretation after the LTS has been minimized,
but we think this is a reasonable expectation.  The minimization removed 
84 states out of the 95 in the original specification, so
even if the engineer has to correlate liveness assumptions
with the minimized LTS 
the technique proves itself helpful by hiding
irrelevant behavior.\\
Going back to \cite{DBLP:conf/fmcad/AlurMT13} we see that our
slice is consistent with the diagnosis presented by the authors
where they say that \emph{(\ldots)A counter-strategy for the
environment is to always keep all $b_i$'s low(\ldots)}.
Indeed, this counter strategy is captured by the minimized 
LTS depicted in Figures~\ref{fig:lift-controller-minimization}.
% strategy for the environment can be expressed
%n this case as $f_{1}^{\neg \varphi}(s_1)= b1 \downarrow$,
% $f_{1}^{\neg \varphi}(s_2)= b2 \downarrow$.
\begin{figure}[bt]
\centering
\SmallPicture
%\ShowFrame
\VCDraw{
    \begin{VCPicture}{(-6,-3.5)(6,3.5)}
        \SetEdgeLabelScale{1.4}
        \State[1]{(-6,2)}{1}
        \State[2]{(-2,2)}{2}
        \State[3]{(2,2)}{3}
        \State[6]{(6,2)}{6}
        \State[4]{(-2,0)}{4}        
        \State[5]{(-6,-2)}{5}        
        \State[8]{(2,-2)}{8}        
        \State[7]{(6,-2)}{7}        
		\Initial[w]{1}
        \EdgeL{1}{2}{b1 \downarrow}		
        \EdgeL{2}{3}{b2 \downarrow}		        
		\LoopW{5}{tick}                
		\LoopE{7}{tick}                
        \ChgEdgeLineStyle{dashed} %\EdgeLineDouble		
        \EdgeL{3}{6}{f1 \downarrow}		                
        \EdgeR{3}{4}{f1 \uparrow}		                       
        \EdgeR{4}{5}{f2 \downarrow}                               	
		\EdgeL{6}{7}{f2 \uparrow}                		
		\EdgeR{7}{8}{f1 \uparrow}         		
		\EdgeR{8}{5}{f2 \downarrow}         				
    \end{VCPicture}
}
\vspace*{-2mm}
\caption{Modified Lift Controller minimization.}
\label{fig:lift-controller-minimization}
\vspace*{-4mm}
\MediumPicture
\end{figure}
%system allows noncontrollable cycle to avoid deadlock
%\subsection{t-strong-fairness}
\begin{figure}[bt]
\centering
\SmallPicture
%\ShowFrame
\VCDraw{
    \begin{VCPicture}{(-4,-1.5)(4,2.3)}
        \SetEdgeLabelScale{1.4}
        \State[1]{(-3,0)}{1}
        \State[2]{(0,0)}{2}
        \State[3]{(3,1)}{3}
        \State[4]{(-0.5,2)}{4}
        \State[5]{(3,-1)}{5}
		\Initial[w]{1}
        \ChgEdgeLineStyle{dashed} %\EdgeLineDouble
        %\ChgEdgeLineWidth{1.5}
        \EdgeL{1}{2}{try}
        %\RstEdgeLineWidth{1}
        \RstEdgeLineStyle %\EdgeLineSimple
        \EdgeL{2}{3}{succ}
        \EdgeL{2}{5}{fail}
        \VArcR{arcangle=-20}{3}{1}{\ell}
        \ArcL{5}{1}{A}
        \ArcR{3}{4}{A}
        \ArcR{4}{1}{G}
    \end{VCPicture}
}
\vspace*{-2mm}
\caption{t-strong fairness example.}
\label{fig:strongfairness}
\vspace*{-4mm}
\MediumPicture
\end{figure}
In \cite{DBLP:conf/icse/DIppolitoBPU11} a concise example was 
presented to show that the notion of t-strong-fairness was not
enough to achieve synthesis of controller over fallible domains.
The plant is depicted in figure \ref{fig:strongfairness}, where the only
controllable action is $try$.  The system should achieve
to produce event $G$ infinitely often provided that the environment
agrees to produce $A$ infinitely often.  The idea in 
\cite{DBLP:conf/icse/DIppolitoBPU11} is to treat a
failure as something not systematic, different from a truly antagonistic
event.  The example shows how this
could not be captured with weaker notions of fairness, we take 
the plant in two different incarnations.  In the first we keep
the liveness properties intact, i.e., the system should accomplish 
$\square \Diamond G$ if the environment agrees to comply with
$\square \Diamond A$.  After applying our technique we get the
minimized plant shown in figure \ref{fig:strongfairness-min1}.
\begin{figure}[bt]
\centering
\SmallPicture
%\ShowFrame
\VCDraw{
    \begin{VCPicture}{(-4,-1.5)(4,2.3)}
        \SetEdgeLabelScale{1.4}
        \State[1]{(-3,0)}{1}
        \State[2]{(0,0)}{2}
        \State[5]{(3,-1)}{5}
		\Initial[w]{1}
        \ChgEdgeLineStyle{dashed} %\EdgeLineDouble
        %\ChgEdgeLineWidth{1.5}
        \EdgeL{1}{2}{try}
        %\RstEdgeLineWidth{1}
        \RstEdgeLineStyle %\EdgeLineSimple
        \EdgeL{2}{5}{fail}
        \ArcL{5}{1}{A}
    \end{VCPicture}
}
\vspace*{-2mm}
\caption{First minimization.}
\label{fig:strongfairness-min1}
\vspace*{-4mm}
\MediumPicture
\end{figure}
The minimization clarifies the cause of non realizability
since it is clear that the environment can keep the play
within the language described as $(try, fail, A)^*$ .\\
Now, we can add a new environmental liveness assumption,
forcing the environment to produce infinitely many $succ$
events.  After applying our technique we get the result in
figure \ref{fig:strongfairness-min2}.  This minimization is 
better explained in conjunction with the strategy shown in 
figure \ref{fig:strongfairness-min-counter-strat}. We can see
that state 4 has been removed and state 2 preserves both
the $succ$ and $fail$ options since the environment will take
each one alternately to comply with the $\square \Diamond A$
assumption while avoiding state 4, since this will allow 
infinitely many $G$s.
\begin{figure}[bt]
\centering
\SmallPicture
%\ShowFrame
\VCDraw{
    \begin{VCPicture}{(-4,-1.5)(4,2.3)}
        \SetEdgeLabelScale{1.4}
        \State[1]{(-3,0)}{1}
        \State[2]{(0,0)}{2}
        \State[3]{(3,1)}{3}
        \State[5]{(3,-1)}{5}
		\Initial[w]{1}
        \ChgEdgeLineStyle{dashed} %\EdgeLineDouble
        %\ChgEdgeLineWidth{1.5}
        \EdgeL{1}{2}{try}
        %\RstEdgeLineWidth{1}
        \RstEdgeLineStyle %\EdgeLineSimple
        \EdgeL{2}{3}{succ}
        \EdgeL{2}{5}{fail}
        \VArcR{arcangle=-20}{3}{1}{\ell}
        \ArcL{5}{1}{A}
    \end{VCPicture}
}
\vspace*{-2mm}
\caption{Second minimization.}
\label{fig:strongfairness-min2}
\vspace*{-4mm}
\MediumPicture
\end{figure}
\begin{figure}[bt]
\centering
\SmallPicture
%\ShowFrame
\VCDraw{
    \begin{VCPicture}{(-4,-1.5)(4,2.3)}
        \SetEdgeLabelScale{1.4}
        \State[1]{(-3,0)}{1}
        \State[2]{(-1,-1)}{2}
        \State[3]{(1,-1)}{3}
        \State[4]{(3,0)}{4}        
        \State[5]{(1,1)}{5}
        \State[6]{(-1,1)}{6}        
		\Initial[w]{1}
        \ChgEdgeLineStyle{dashed} 
        \EdgeL{1}{2}{try}
        \RstEdgeLineStyle
        \EdgeL{2}{3}{fail}
        \EdgeL{3}{4}{A}
        \ChgEdgeLineStyle{dashed} 
        \EdgeR{4}{5}{try}
        \RstEdgeLineStyle
        \EdgeR{5}{6}{succ}        
        \EdgeR{6}{1}{\ell}                
    \end{VCPicture}
}
\vspace*{-2mm}
\caption{Environment counter strategy.}
\label{fig:strongfairness-min-counter-strat}
\vspace*{-4mm}
\MediumPicture
\end{figure}
This example is very synthetic but depicts a scenario
where behavior based diagnosis proves itself a useful
tool not only for repairing a specification but to reason 
incrementally about requirements' limitations.
%the two cases, the second introduces the problem of
%memory unfolding

\subsection{Generalized Buffer (GenBuf)}
Non realizable specifications of the Generalized Buffer Case Study~\cite{DBLP:journals/entcs/BloemGJPPW07} were studied in~\cite{DBLP:conf/fmcad/KonighoferHB09}.  
%Original LTL specifications were automatically translated to produce the expected LTS input.  
The original specification was introduced by 
IBM with their Rulebase verification tool~\cite{beer1996rulebase}.
It describes a family of buffers that transmit data from $n$ senders
$Sender_0,Sender_1$$,\ldots$ $,Sender_n$ to two receivers $Receiver_0$ and
$Receiver_1$.  Data is received from senders in a random order and should be
delivered to the receivers in round-robin order.  
It contains a handshake protocol to communicate with each sender/receiver device
by exchanging $StoB\_REQ(i) \leftrightarrow BtoS\_ACK(i)$ messages on the senders side
and $BtoR\_REQ(j) \leftrightarrow RtoB\_ACK(j)$ on the receivers.  
A controller should arbitrate between the senders and the receivers to
accomplish proper communication through a four slot FIFO queue.
Several restrictions and assumptions are declared in order
to ensure consistent behavior (e.g. \emph{G2: No request should be immediately
acknowledged since it is valid at least one step 
after the assertion of request}).\\

\begin{figure}[bt]
\centering
\SmallPicture
%\ShowFrame
\VCDraw{
    \begin{VCPicture}{(-6,-3.5)(6,3.5)}
        \SetEdgeLabelScale{1.4}
        \State[1]{(-7.5,0)}{1}
        \State[80]{(3,2)}{80}
        \State[79]{(6,2)}{79}
        \State[81]{(1.5,0)}{81}        
        \State[82]{(3,-2)}{82}        
        \State[83]{(6,-2)}{83}        
        \State[78]{(7.5,0)}{78}                
		\Initial[w]{1}
        \ArcR{81}{82}{tick\_e}		
        \ArcR{78}{79}{tick\_e}		                                        
        \ChgEdgeLineStyle{dashed} %\EdgeLineDouble				
        \ArcR{82}{83}{SLC0\uparrow}		                
        \ArcR{83}{78}{tick\_s}		                        
        \ArcR{79}{78}{tick\_s}		                                
        \ArcR{79}{80}{SLC0\downarrow}		                                      
        \ArcR{82}{81}{tick\_s}		        
        \ArcR[.35]{80}{81}{tick\_s}        
        \ZZEdgeL{1}{81}{\ldots StoB\_REQ0\uparrow \ldots BtoR\_REQ0\uparrow \ldots}	        
    \end{VCPicture}
}
\vspace*{-2mm}
\caption{GenBuf (missing assumption) Terminal Set.}
\label{fig:genbuf-terminal-set}
\vspace*{-4mm}
\MediumPicture
\end{figure}

We studied the three non-realizable specifications of GenBuf discussed in~\cite{DBLP:conf/fmcad/KonighoferHB09}. 
The goal is captured by the LTL formula $\square\Diamond($ $StoB\_REQ0  \leftrightarrow$ $BtoS\_ACK0))$.
 In the first the authors consider removing the assumption 
that expresses that always eventually if a request to read data
has been sent from the controller to $Receiver_0$ ($BtoR\_REQ0$) then
an acknowledge will be issued ($RtoB\_ACK0$).  
The minimized environment decreased 80\%
(from 10964 to 2235 states) and, amongst other things, excluded all actions related with $Receiver_1$,
since they are irrelevant to the environment's strategy.
In the minimized LTS we identified one terminal set
depicted in Figure~\ref{fig:genbuf-terminal-set} in which one guarantee is being 
neglected infinitely often: 
  In the trace leading to the terminal set
the buffer sends \\$BtoR\_REQ0$ message to $Receiver_0$ but
never gets the expected acknowledge ($RtoB\_ACK0$), thus
not being able to produce an acknowledge for the sender.  This
diagnosis should point the engineer in the direction to
enforce the specification by restricting the environment to always
eventually produce an acknowledge message for each requirement
sent to the receivers.\\
The second variation studied in~\cite{DBLP:conf/fmcad/KonighoferHB09}  incorporated the goal 
$\square \Diamond$ $(\neg EMPTY$ $\wedge$ $\neg DEQ)$.
The minimized environment decreased 92\%
(from 10964 to 850 states)
with only one strongly connected component at which point the buffer remains empty and the environment keeps exchanging the value of the
signal $SLC$ and never requests the bus ($StoB\_REQ(i)$), thus preventing the controller to achieve the new goal. This tells us that the problem is the lack of an assumption that states that the environment must always eventually achieve $ENQ$ thus lowering the $EMTPY$ signal. 
\\
The final variation analyzed in ~\cite{DBLP:conf/fmcad/KonighoferHB09} restricts specification by
adding a new safety requirement $\square \neg ENQ$. This already reduces the size of the 
problem significantly (39 states) leaving little room for minimization (resulting in 26 states). Indeed, in both the original and minimized environment LTS it is straightforward to visualize the deadlock state which is reached after the first
$BtoS\_ACK(i)$ message. This acknowledgement is required to be followed by raising the $ENQ$ signal which would violate the new requirement. 

%\begin{figure}[bt]
%\centering
%\SmallPicture
%%\ShowFrame
%\VCDraw{
%    \begin{VCPicture}{(-4,-2)(4,2)}
%        \SetEdgeLabelScale{1.4}
%        \LargeState\FinalState[\neg s]{(-4,0)}{1}
%        \LargeState\State[s]{(4,0)}{2}
%		\Initial[w]{1}
%		\LoopN[.5]{1}{EMPTY \vee DEQ}
%		\LoopN[.5]{2}{\neg DEQ}		
%		\ArcL[.5]{1}{2}{\neg EMPTY \vee \neg DEQ}				
%		\ArcL[.5]{2}{1}{DEQ}						
%    \end{VCPicture}
%}
%\vspace*{-2mm}
%\caption{Monitor for Guarantee 12.}
%\label{fig:genbuf_g12}
%\vspace*{-4mm}
%\MediumPicture
%\end{figure}

%\subsection{Tic Tac Toe}

%introduce optimizations

%brief explanation of examples

%explanation of representative minimizations

%table: time, sizes

%table explanation

