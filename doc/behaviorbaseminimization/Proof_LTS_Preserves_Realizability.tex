Proof for Lemma \ref{def:induction-preserves}


\textbf{HAY QUE REVER ESTA PORQUE CAMBIO LA DEFINICI‰ON DE SUBLTS}
\begin{proof}
If the system has a winning strategy in $G(\mathcal{I}_1)$ but no
winning strategy in $G(\mathcal{I}_2)$, this would imply that
the environment is able to either take the play into a deadlock state
or a $\sigma$-trap that always falsifies the property $\varphi$. 
Suppose that the system is able to play according to the strategy up
to state $s_i$ in $G(\mathcal{I}_2)$, after this point, for every choice the
system makes the environment has a way to construct a play
$\pi$ that is winning for him, either finitely or infinitely.
The departure at state $s_i$ must have been introduced by 
restricting the system choice, removing a controllable transition
which is not allowed for Alternating Sub-Games, or giving the environment
more power, by adding a non controllable transition or removing all of them
at a non controllable state thus introducing a deadlock which is also
forbidden since by Alternating Sub-Game definition
inclusion is satisfied.  Is proven by contradiction
that realizability is preserved between Alternating Sub-LTSs.
\end{proof}

