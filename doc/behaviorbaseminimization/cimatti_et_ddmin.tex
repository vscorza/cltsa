
\begin{definition}\label{def:fulfillable}\emph{((un)fulfillable, (in)sufficient)}
Let $\mathcal{A}$ be a set of available assuptions, let $\mathcal{G}$ be a set of available
guarantees, and let $A \subseteq \mathcal{A}$ and $G \subseteq \mathcal{G}$.\\
If a specification $<A,G>$ is realizable, we say that $G$ is fulfillable w.r.t. $A$,
and, conversely, $A$ is sufficient w.r.t. G. Otherwise, $G$ is unfulfillable w.r.t. 
$A$, and $A$ is insufficient w.r.t. G, respectively.\\
$G$ is minimally unfulfillable w.r.t.
$A$ iff $<A,G>$ is unrealizable and removal of any element of $G$ leads to realizability:
$\forall g \in \mathcal{G}. <A, G \setminus {g}>$ is realizable.\\
$G$ is maximally fulfillable w.r.t. $A$ in $\mathcal{G}$ iff $<A,G>$ is realizable and addition
of any element of $G$ leads to unrealizability: 
$\forall g \in \mathcal{G} \setminus G. <A, G \cup {g}>$ is unrealizable.\\
$A$ is minimally sufficient w.r.t. $G$ iff $<A,G>$ is realizable and removal of any element
of $A$ leads to unrealizability: $\forall a \in A. <A\setminus {a},G>$ is unrealizable.\\
$A$ is maximally insufficient w.r.t. $G$ in $\mathcal{A}$ iff $<A,G>$ is unrealizable and addition
of any element of $\mathcal{A}\setminus A$ leads to realizability: 
$\forall a \in \mathcal{A}\setminus A.<A \cup {a},G>$ is realizable.
\end{definition}
\begin{definition}\label{def:helpful}\emph{((un)helpful)}
Let $<A,G>$ be a specification. \\
An assumption $a \in A$ is unhelpful if:\\
$\forall G' \subseteq G. (<A,G'> \text{is realizable} \iff <A\setminus {a},G'> \text{is realizable})$.\\
A guarantee $g \in G$ is unhelpful if:\\ $\forall A' \subseteq A. (<A',G> \text{is realizable} \iff <A', G \setminus {g}> \text{is realizable})$.\\
An assumption or a guarantee is helpful iff is not unhelpful.
\end{definition}
\begin{definition}\label{def:ddmin}\emph{Minimizing Delta Debugging algorithm}
Let $C$ be a set, and $test:2^C\rightarrow \{\times, \checkmark, ?\}$ such that
$test(\emptyset)=\checkmark$ and $test(C)=\times$ holds.  The minimizing Delta
Debugging algorithm $ddmin(c)$ is:
\begin{footnotesize}
\[
\begin{array}{r c l}
ddmin(c_\checkmark) & = & ddmin_2(c_\times , 2) \textbf{ where}\\
ddmin_2(c'_\times, n) & = & 
  \begin{cases}
  ddmin_2(\Delta_i, 2) \text{ ("reduce to subset")}\\
  \quad \text{if } \exists i \in \{1,\ldots,n\}.test(\Delta_i) = \times \\
  ddmin_2(\nabla_i, max(n-1, 2)) \text{ ("reduce to complement")}\\
  \quad \text{else if } \exists i \in \{1,\ldots,n\}.test(\nabla_i) = \times \\  
  ddmin_2(c'_\times, min(|c'_\times|, 2n)) \text{ ("increase granularity")}\\
  \quad \text{else if } n < |c'_\times|\\    
  c'_\times \text{otherwise ("done")}\\    
  \end{cases}
\end{array}  
\]\\
\end{footnotesize}
Where $\nabla_i = c'_\times - \Delta_i$, $c'_\times = \Delta_1 \cup \ldots \cup \Delta_n$, all
$\Delta_i$ are pairwise disjoint, and $\forall \Delta_i. |\Delta_i| \approx |c'_\times|/n$ holds.
The recursion invariant (and thus precondition) for $ddmin_2$ is $test(c'_\times) = \times$
and $n \leq |c'_\times|$.
\end{definition}