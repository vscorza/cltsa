\clearpage
\section{Motivation}\label{sec:motivation}

We introduce a simple example to illustrate what we consider to be the complementary nature 
of both the presented approach and that of ~\cite{DBLP:conf/hvc/KonighoferHB10}. Suppose that the engineer is working on a specification consisting of two environmental boolean variables $x_1$ and $x_2$ and one system variable $y$. $x_2$ (initially false) is required to swap its valuation at each step of the execution, while $x_1$ remains free. The value of $y$ is derived from the conjunction of the two environmental variables.
It is assumed that $x_1$ will be set to true infinitely often while $y$ is required to also be satisfied infinitely often. The set of LTL formulae corresponding to this specification is as follow: 
\begin{center}
	\begin{tabular}{ r c l }
	$\theta_e$& $=$ &$\neg x_1 \wedge \neg x_2$\\
	$\theta_s$& $=$ &$\neg y$\\
	$\rho_e$& $=$ &$\square(x_2 \leftrightarrow \bigcirc(\neg x_2))$\\
	$\rho_s$& $=$ &$\square((x_1 \wedge x_2) \leftrightarrow y)$\\
	$\varphi_e$& $=$ &$\square \Diamond x_1$\\
	$\varphi_s$& $=$ &$\square \Diamond y$\\
\end{tabular}
\end{center}

The automaton induced from this set of formulae is depicted in Fig. ~\ref{fig:konig_original_plant}. The specification is not realizable because the environment can set $x_1$ to true only when $x_2$ is down, thus honoring the environmental assumption. The minimization presented in ~\cite{DBLP:conf/hvc/KonighoferHB10} removes $\rho_e$ since it is unnecessary for the environment to swap $x_2$ regularly in order to win the (non) realizability game. This approach can be characterized as a syntactic minimization, whose purpose is to help the engineer when viewing the specification (in particular the safety part describing the behavior of the plant) as a set of LTL formulae. This approach, nonetheless, yields the automaton depicted in Fig. ~\ref{fig:konig_original_plant_k} which is, in fact, more complex behavior-wise. Our approach is designed to simplify the semantic view of the specification, producing the automaton depicted in Fig. ~\ref{fig:konig_original_plant_c}. 

The syntactic minimization should help the engineer to focus on a smaller set of formulae while diagnosing the non realizability cause (in this case the too weak assumption $\varphi_e$) by getting rid of the irrelevant element $\rho_e$. Our approach simplifes the specification semantically, reducing the relevant behavior and showing that it is possible for the environment to avoid state 2 by never allowing $x_1$ and $x_2$ to happen simultaneously. If we were allowed to add the assumption that both environmental variables occur infinitely often ($\rho'_s = \square \Diamond x_1 \wedge x_2$) the specification would be realizable and the plant as is would be a proper controller, since it allows the environment to move freely as long as it visits state 2 infinitely often. If the engineer has this behavior in mind as her design intent, it should be easy for her to see that in our minimization the desired state is unreachable, hopefully guiding her to the strengthening of the assumption.  

\begin{figure}[bt]
\centering
\SmallPicture
%\ShowFrame
\VCDraw{
	\begin{VCPicture}{(-4,-2.5)(4,2.5)}
		\SetEdgeLabelScale{1.4}
		\StateVar[0:\overline{x_1 x_2 y}]{(-3,0)}{1}
		\StateVar[2:x_1 x_2 y]{(0,-2)}{2}
		\StateVar[1:\overline{x_1} x_2 \overline{y}]{(3,0)}{3}
		\StateVar[3:x_1 \overline{x_2 y}]{(0,2)}{6}        
		\Initial[w]{1}
		\ArcR{1}{2}{}
		\ArcR{2}{1}{}
		\ArcR{6}{1}{}        
		\ArcR{1}{6}{}
		\ArcR{2}{3}{}
		\ArcR{3}{2}{}
		\ArcR{3}{6}{}
		\ArcR{6}{3}{}
	\end{VCPicture}
}
\caption{Original Plant}
\label{fig:konig_original_plant}
\MediumPicture
\end{figure}
\begin{figure}[bt]
	\centering
	\SmallPicture
	%\ShowFrame
	\VCDraw{
		\begin{VCPicture}{(-4,-3)(4,3)}
			\SetEdgeLabelScale{1.4}
			\StateVar[0:\overline{x_1 x_2 y}]{(-3,0)}{1}
			\StateVar[2:x_1 x_2 y]{(0,-2)}{2}
			\StateVar[1:\overline{x_1} x_2 \overline{y}]{(3,0)}{3}
			\StateVar[3:x_1 \overline{x_2 y}]{(0,2)}{6}        
			\Initial[n]{1}
			\LoopW{1}{}
			\LoopN{6}{}
			\LoopE{3}{}
			\LoopS{2}{}
			\ArcR{1}{2}{}
			\EdgeL{2}{1}{}
			\ArcR{1}{3}{}
			\ArcR{3}{1}{}
			\ArcR{6}{1}{}        
			\EdgeL{1}{6}{}
			\ArcR{2}{3}{}
			\EdgeL{3}{2}{}
			\ArcR{2}{6}{}
			\ArcR{6}{2}{}
			\ArcR{3}{6}{}
			\EdgeL{6}{3}{}
		\end{VCPicture}
	}
	\caption{Minimization according to Könighoefer}
	\label{fig:konig_original_plant_k}
	\MediumPicture
\end{figure}
\begin{figure}[bt]
	\centering
	\SmallPicture
	%\ShowFrame
	\VCDraw{
		\begin{VCPicture}{(-4,-3)(4,3)}
			\SetEdgeLabelScale{1.4}
			\StateVar[0:\overline{x_1 x_2 y}]{(-3,0)}{1}
			\StateVar[2:x_1 x_2 y]{(0,-2)}{2}
			\StateVar[1:\overline{x_1} x_2 \overline{y}]{(3,0)}{3}
			\StateVar[3:x_1 \overline{x_2 y}]{(0,2)}{6}        
			\Initial[w]{1}       
			\EdgeL{1}{6}{}
			\ArcR{3}{6}{}
			\ArcR{6}{3}{}
		\end{VCPicture}
	}
	\caption{Behavior minimization}
	\label{fig:konig_original_plant_c}
	\MediumPicture
\end{figure}
\clearpage
We motivate and informally introduce relevant concepts by discussing a small example inspired by~\cite{DBLP:conf/vmcai/PitermanPS06}. Suppose an engineer is writing the specification for an industrial
controller that needs to coordinate communication between various devices consuming data from different sensors via a shared bus according to the scheduling policy defined by an arbiter. 
A device acting as a master will raise a bus access requirement ($req$) 
%on the related bus input line 
to communicate with a specific sensor acting as a slave. 
The arbiter can then either grant access to the device ($grant$) or give the bus
to another master ($\overline{grant}$). For the latter case, the device denied permission will have to wait for the slave currently using the bus to raise a
ready signal ($hready$) indicating that it has finished operation.  Only then
will the device issue another requirement.  If access is granted then the device
will wait either for a slave confirmation ($hready$) or a $timeout$ event.
After the slave has finished operation the device will start to process the
received information ($process$). Should the slave issue a timeout message, a reset will be triggered ($reset$) to try and restore normal operation.  
The initial environment specification $E$, including safety assumptions and safety goals, is presented in Figure~\ref{fig:req_grant}, where
solid lines indicate non controllable actions and dotted lines indicate
controlled actions.  Non controllable actions are those that the environment
can trigger while controlled actions are triggered by the system. 
\begin{figure}[bt]
\centering
\SmallPicture
%\ShowFrame
\VCDraw{
    \begin{VCPicture}{(-4,-1.5)(4,2.3)}
        \SetEdgeLabelScale{1.4}
        \State[1]{(-3,0)}{1}
        \State[2]{(0,0)}{2}
        \State[3]{(3,1)}{3}
        \State[4]{(-0.5,3)}{4}
        \State[6]{(0,1.5)}{6}        
        \State[5]{(3,-1)}{5}
		\Initial[w]{1}
        \ChgEdgeLineStyle{dashed} %\EdgeLineDouble
        %\ChgEdgeLineWidth{1.5}
        \EdgeL{1}{2}{req}
        \ArcR[.3]{6}{1}{reset}        
        \VArcR{arcangle=-30}{4}{1}{process}        
        %\RstEdgeLineWidth{1}
        \RstEdgeLineStyle %\EdgeLineSimple
        \EdgeL{2}{3}{grant}
        \EdgeL[.75]{2}{5}{\overline{grant}}
        %\VArcR{arcangle=-20}{3}{6}{timeout}
        \ArcR[.6]{3}{6}{timeout}
        \ArcL{5}{1}{hready}
        \VArcR{arcangle=-30}{3}{4}{hready}
    \end{VCPicture}
}
\caption{Bus Access example ($E$)}
\label{fig:req_grant}
%%\vspace*{-4mm}
\MediumPicture
\end{figure}
\begin{figure}[bt]
\centering
\SmallPicture
%\ShowFrame
\VCDraw{
    \begin{VCPicture}{(-4,-1.5)(4,2)}
        \SetEdgeLabelScale{1.4}
        \State[1]{(-3,0)}{1}
        \State[2]{(0,-1)}{2}
        \State[3]{(3,0)}{3}
        \State[6]{(0,1)}{6}        
		\Initial[w]{1}
        \ChgEdgeLineStyle{dashed} %\EdgeLineDouble
        %\ChgEdgeLineWidth{1.5}
        \ArcR{1}{2}{req}
        \ArcR{6}{1}{reset}        
        %\RstEdgeLineWidth{1}
        \RstEdgeLineStyle %\EdgeLineSimple
        \ArcR{2}{3}{grant}
        %\VArcR{arcangle=-20}{3}{6}{timeout}
        \ArcR{3}{6}{timeout}
    \end{VCPicture}
}
\caption{Minimized Bus Access ($E_1$)}
\label{fig:req_grant_sub_1}
%\vspace*{-4mm}
\MediumPicture
\end{figure}
\begin{figure}[bt]
\centering
\SmallPicture
%\ShowFrame
\VCDraw{
    \begin{VCPicture}{(-4,-1.5)(4,2)}
        \SetEdgeLabelScale{1.4}
        \State[1]{(-3,0)}{1}
        \State[2]{(0,0)}{2}
        \State[3]{(3,1)}{3}
        \State[6]{(0,1.5)}{6}        
        \State[5]{(3,-1)}{5}
		\Initial[w]{1}
        \ChgEdgeLineStyle{dashed} %\EdgeLineDouble
        %\ChgEdgeLineWidth{1.5}
        \EdgeL{1}{2}{req}
        \ArcR[.3]{6}{1}{reset}        
        %\RstEdgeLineWidth{1}
        \RstEdgeLineStyle %\EdgeLineSimple
        \EdgeL{2}{3}{grant}
        \EdgeL[.75]{2}{5}{\overline{grant}}
        %\VArcR{arcangle=-20}{3}{6}{timeout}
        \ArcR[.7]{3}{6}{timeout}
        \ArcL{5}{1}{hready}
    \end{VCPicture}
}
\caption{Minimized Bus Access ($E_2$)}
\label{fig:req_grant_sub_2}
%\vspace*{-4mm}
\MediumPicture
\end{figure}


The goal of continually processing data from the board is captured
by the LTL liveness formula $\square\Diamond process$.  The assumption
that access to the bus will always eventually be granted to the device
is captured by the liveness formula $\square\Diamond grant$.
 Note that without further assumptions the environment is able to
 systematically produce $timeout$ actions even after granting the device access
 to the bus.  These two choices, picking $grant$ at state 2 and then
 $timeout$ at state 3 conform a winning strategy for the environment. In other
 words, no matter what the system does, the environment, by making these two
 choices always prevents the system goals from being achieved.
 
The  specification $E_{1}$ in Figure~\ref{fig:req_grant_sub_1} is a subgraph of $E$, describing an environment with strictly less behavior than that of $E$. Furthermore, the winning environment strategy for specification $E_1$ is also a winning environment strategy in $E$. Indeed,  $E_1$ cannot be further reduced while satisfying this strategy preserving property. Thus, $E_1$ can be thought of as a slice of the original environment specification that has removed behavior that is irrelevant with respect to a cause for unrealizability.

Assume that after analyzing this first unrealizability cause another assumption is added, forcing the environment to produce
$hready$ infinitely often ($\square\Diamond hready$) hoping to reach state 4,
thus realizing the goal.  It turns out that even after adding this
new assumption the specification remains unrealizable:  The environment
can avoid state 4 if it keeps track of the most recently satisfied assumption.
It can first grant access to the device and then produce a $timeout$. 
This will be the first cycle of its strategy, then it will deny access
 ($\overline{grant}$) and start over again.  This alternating strategy will allow the environment
 to fulfill its assumptions (producing $grant$ in the first cycle
 and $hready$ in the second) while avoiding to produce the expected
 outcome.  The LTS in Figure
 ~\ref{fig:req_grant_sub_2} represents a minimal slice of the original specification $E$ that captures the winning strategy of the environment. 
 
 

 %  Note that we are
 %trying to progressively fix the specification with the information
 %provided by our behavior minimization feedback . $E_2$ is of complimentary value
 %with respect to the counter strategy since it presents a more compact 
 %representation of the conflict (avoiding memory induced unrolling)
 %that is also canonical (since it does not need to fix an order for the
 %satisfaction of the assumptions on behalf of the environment).
 
 \begin{comment}
  
Our technique is in fact quiet different, in terms of behavior, to what is
achieved by minimizing the set of LTL formulae in a declarative specification.
To exemplify the difference, and the complementation between the declarative
and the behavior minimization we introduce a synthetic case where an engineer
wants to specify a zero triggered latch mechanism. The initial specification
involves only two signals: $x$ representing a single boolean input and $y$ the only
boolean output.  The idea is that the output should be kept constantly up 
once a zero boolean value has been fed to the system.
Before this $y$ must remain low.  
The specification is written as the conjunction of four LTL 
formulae as presented in table \ref{table:d-ff-spec}. 
Following the definitions from \cite{DBLP:conf/vmcai/PitermanPS06} $\theta_s$ stands for
an initial system formula, $\rho_s$ for a safety system formula and 
$\varphi_s$ for a system liveness formula.
The \textit{Initial condition} formula 
fixes the initial value for the output signal,
\textit{Wait for negative trigger} keeps the value of $y$ down as long as the input
is up and the output value has not yet been raised, \textit{Latch} keeps the 
output up after the initial raise and the \texttt{GR(1)} \textit{Goal} is to 
always eventually produce an output ($\square\Diamond(y)$).  After running our automated
translation we produced the LTS representation of the \textit{Game Structure}
definition.  We get the behavior depicted in Figure \ref{fig:fig.neg-trigger-latch-behavior}.
The $\uparrow$ suffix after a signal name means the the value is being raised ($x\uparrow$),
$\downarrow$ consequently means that value is being lowered ($x\downarrow$).  A $tick$ event
is added to show when the values can be kept constant at the current time step.
\begin{center}
 %\vspace*{-4mm}
\begin{table}[h]
  \begin{tabular}{ l  c  l }
	Name & Type & Requirement \\
    \hline
    Initial condition & $\theta_s$ & $y = 0$\\
    Wait for negative trigger & $\rho_s$ &$\square((x \wedge \neg y) \rightarrow \Circle(\neg y) )$ \\
    Latch & $\rho_s$ &$\square (y \rightarrow \Circle(y))$ \\   
    Goal &  $\varphi_s$ & $\square\Diamond(y)$\\
  \end{tabular}
  \caption{Zero triggered latch specification}
  \label{table:d-ff-spec}
 \end{table}
 %\vspace*{-12mm}
\end{center}
We wrote this specification in \texttt{RATSY}, run the minimization presented in
\cite{DBLP:conf/hvc/KonighoferHB10} (where the \textit{Latch} requirement was
considered irrelevant and removed) and then translated the minimized specification
to produce the LTS depicted in Figure 
\ref{fig:fig.neg-trigger-latch-behavior-formula-minimization}.  Understanding the specification
in terms of the conjunction of formulas we can see that the $Latch$ requirement is irrelevant
to the unrealizability cause since even if we allow the system to lower $y$ after its initial
raise, no output would be produced if the input is kept up.  Removing a formula is, in fact,
relaxing the behavior, this is expressed in the double lined states and transitions 
present in Figure \ref{fig:fig.neg-trigger-latch-behavior-formula-minimization}.  
We can see that the transitions added express the new behavior where the system is allowed
to lower $y$ after being raised.  Let us note that removing a formula means relaxing, 
thus augmenting, the behavior of the system.
\begin{figure}[bt]
\centering
\SmallPicture
%\ShowFrame
\VCDraw{
    \begin{VCPicture}{(-4,-5)(4,4)}
        \SetEdgeLabelScale{1.4}
        \State[1]{(0,3)}{1}
        \State[2]{(-3,2)}{2}        
        \State[3]{(-3,0)}{3}
        \State[4]{(3,-2)}{4}        
        \State[5]{(-3,-2)}{5}                
        \State[6]{(-3,-4)}{6}                        
        \State[7]{(3,2)}{7}                                
        \State[8]{(3,0)}{8}                                        
		\Initial[n]{1}
		\EdgeL{1}{2}{$x$\downarrow}
		\EdgeR{1}{7}{$x$\uparrow}		
		\EdgeR{8}{3}{$x$\downarrow}								
		\LoopW{3}{$tick$}
		\LoopE{8}{$tick$}		
		\LoopW{6}{$tick$}		
		\EdgeR{3}{4}{$x$\uparrow}
		\ArcR{5}{6}{$x$\uparrow}
		\ArcR{6}{5}{$x$\downarrow}		
        \ChgEdgeLineStyle{dashed} 
		\EdgeR{4}{6}{$y$\uparrow}		
		\EdgeL{2}{3}{$y$\downarrow}				
		\EdgeL{7}{8}{$y$\downarrow}						
		\EdgeR{3}{5}{$y$\uparrow}
		\EdgeR{4}{8}{$tick$}		
    \end{VCPicture}
}
%\vspace*{-2mm}
\caption{Zero triggered latch behavior.}
\label{fig:fig.neg-trigger-latch-behavior}
%\vspace*{-8mm}
\MediumPicture
\end{figure}

\begin{figure}[bt]
\centering
\SmallPicture
%\ShowFrame
\VCDraw{
    \begin{VCPicture}{(-4,-6)(4,5)}
        \SetEdgeLabelScale{1.4}
        \State[1]{(0,3)}{1}
        \State[2]{(-3,2)}{2}        
        \State[3]{(-3,0)}{3}
        \State[4]{(1,-1)}{4}        
        \State[5]{(-3,-2)}{5}                
        \State[6]{(-3,-4)}{6}                        
        \State[7]{(3,2)}{7}                                
        \State[8]{(3,0)}{8}                                        
        \StateLineDouble
		\State[9]{(1,-2.5)}{9}                         
        \State[10]{(-1,-2.5)}{10}                             
		\Initial[n]{1}
		\EdgeL{1}{2}{$x$\downarrow}
		\EdgeR{1}{7}{$x$\uparrow}		
		\ArcR{8}{3}{$x$\downarrow}								
		\LoopW{3}{$tick$}
		\LoopE{8}{$tick$}		
		\EdgeL[.8]{3}{4}{$x$\uparrow}
		\ArcR{5}{6}{$x$\uparrow}
		\ArcR{6}{5}{$x$\downarrow}	
		\EdgeLineDouble		
		\LoopS{9}{$tick$}	
		\ArcR[.3]{6}{9}{$tick$}
		\EdgeR{9}{10}{$x$\downarrow}
		\EdgeLineSimple
        \ChgEdgeLineStyle{dashed} 
		\EdgeL{2}{3}{$y$\downarrow}				
		\EdgeL{7}{8}{$y$\downarrow}						
		\ArcR{3}{5}{$y$\uparrow}
		\EdgeL{4}{8}{$tick$}		
		\EdgeLineDouble
		\EdgeR{10}{5}{$tick$}		
		\ArcR[.2]{10}{3}{$y$\downarrow}						
		\ArcR{4}{9}{$y$\uparrow}		
		\ArcR{9}{8}{$y$\downarrow}		
		\ArcR{5}{3}{$y$\downarrow}
		\VArcR{arcangle=-60,ncurv=1}{6}{8}{$y$\downarrow}				
    \end{VCPicture}
}
%\vspace*{-8mm}
\caption{Zero trig. latch formula based minimization.}
\label{fig:fig.neg-trigger-latch-behavior-formula-minimization}
%\vspace*{-4mm}
\MediumPicture
\end{figure}
We are looking at the same problem expressed from two 
different directions, thus making impossible a direct comparison of
the techniques.  One of our claims is that our approach is in
fact complimentary to that of the declarative minimization.
Our minimization produces the LTS depicted in Figure 
\ref{fig:fig.neg-trigger-latch-behavior-minimization}, where it
shows that the environment can easily win by playing to keep
$x$ up.  
It is important to note that when minimizing the control problem we must
keep it unrealizable while allowing the environment to have
a winning strategy that works in both the minimized and the original
control problem.
\begin{figure}[bt]
\centering
\SmallPicture
%\ShowFrame
\VCDraw{
    \begin{VCPicture}{(-5,-2)(5,2)}
        \SetEdgeLabelScale{1.4}
        \State[1]{(-3,0)}{1}
        \State[2]{(0,0)}{2}        
        \State[3]{(3,0)}{3}
		\Initial[w]{1}
		\EdgeL{1}{2}{$x$\uparrow}
		\LoopE{3}{$tick$}		
        \ChgEdgeLineStyle{dashed} 
		\EdgeL{2}{3}{$y$\downarrow}				
    \end{VCPicture}
}
%\vspace*{-8mm}
\caption{Zero trig. latch behavior based minimization.}
\label{fig:fig.neg-trigger-latch-behavior-minimization}
%\vspace*{-4mm}
\MediumPicture
\end{figure}
In both the declarative and the behavior-based approaches the
idea is that we are aiding the engineer in understanding the
cause on unrealizability.  If we think in terms of specification repair,
in this particular case if she were to add
a new assumption (\textit{input will eventually fall}), either as
$\square\Diamond(\neg x)$ or $\Diamond(\neg x)$ the specification
will become realizable.
In the next sections we will show how to automatically build a minimal sub-LTS that preserves a winning strategy for the environment (which is a cause of unrealizability).
%\subsection{On The Declarative vs Operational Approach}
To the best of our knowledge,
 previous work on the minimization of unrealizable
specifications has been focused on reducing the set of requirements expressed as LTL formulas.  
This is the approach taken in \cite{DBLP:conf/hvc/KonighoferHB10} and the
initial inspiration for our work.  Könighoefer et al. built their framework
from what was presented in \cite{DBLP:conf/vmcai/CimattiRST08} and in
\cite{DBLP:conf/vmcai/2008} where they took the \textit{delta debugging}
algorithm as a cost-effective minimization technique.\\
Even though the kind of properties being satisfied is the same, we differ
not only in the formalism under use but also in the nature of our minimization,
since we are not reducing the set of formulas but the explicit behavior expressed
as an automaton.\\
To clarify the difference we apply an automated translation
scheme to transform our operational control problem  into
a declarative specification.
The later is written as the conjunction of four \texttt{LTL} \cite{pnueli1977temporal}
formulas as presented in Table \ref{table:planar-robot-declarative-spec}. 
We follow the definitions from \cite{DBLP:conf/vmcai/PitermanPS06}. Our goal is to get a game structure that properly emulates the semantics
of the source control problem.  The idea is that both structures should
satisfy equivalent sets of LTL formulas.\\
The translation can be explained as follows:
we encode the automaton state space through a set of boolean system variables
that will express the binary representation at each particular state.  The 
target specification will have a boolean variable (or signal) for each source
event.  Monitored events from the source specification will be 
represented through environment variables ($\mathcal{X}$) and
controlled events through system variables ($\mathcal{Y}$).  
Suppose that we have the following control problem $\mathcal{I}=\langle E, \mathcal{C}, \varphi \rangle$,
with plant $E$ described as the LTKS $E = (S, \Sigma,\Delta, v:S \rightarrow 2^{\mathcal{F}} ,s_0)$, $\mathcal{F}=\{f_1, f_2 \}$ and where
$a,b,c \in \Sigma$, $c \in \mathcal{C}$, $v(s_i)=\{ f_1 \}, v(s_j)=\{f_2\}$ and
$(s_i, a, s_j)$,
$(s_j, a, s_k)$,
$(s_j,b,s_l)$,
$(s_j, c, s_m) \in \Delta$.
Each transition from the source automaton is translated into three 
different formulas ($enable_{sys}$, $enable_{env}$, $update$), two standing for enabling requirements:\\
\textbf{[$enable_{sys}$]}One for the environment ($env.enable_{i,a} \in \rho_e$) as an implication from current state 
and currently selected action into a disjunction of mutually exclusive conjunction 
of monitored event related signals declaring the set of next-step enabled
environment actions. Suppose that $s, act_c \in \mathcal{Y}$
and $act_a, act_b \in \mathcal{X}$.
\[
\square(s = i \wedge act_a \rightarrow \Circle((\neg act_a \wedge act_b)\vee(act_a \wedge \neg act_b) \vee (\neg act_a \wedge \neg act_b)))\]
\textbf{[$enable_{env}$]} One for the system ($sys.enable_{i,a} \in \rho_s$) as an implication from current state
and currently selected action into another implication 
from the negation of all monitored events to the
disjunction of mutually exclusive conjunction of controlled
event related signals declaring the set of next-step enabled
system actions.
\[
\square(s = i \wedge act_a \rightarrow \Circle((\neg act_a \wedge \neg act_b)\rightarrow (act_c)))\]
\textbf{[$update$]} The third transition requirement ($s.update_{i,a} \in \rho_s$) is the one 
effectively updating the automaton state representation.
\[
\square(s = i \wedge act_a \rightarrow \Circle(s = j))\]
Initial conditions ($\theta_s, \theta_e$) are set according to the initial state 
$s_0$ in the automaton and initially enabled actions.
The LTKS valuations, that predicate over the set $\mathcal{F}$ of 
fluents, is translated in with the proper system safety requirement.\\
Following this translation and starting with
the control problem presented for the planar robot
we get an equivalent game structure shown (abbreviated
leaving only formulas relevant to state $4$) in 
Table \ref{table:planar-robot-declarative-spec}.
\begin{center}
\begin{table}[h]
  \begin{tabular}{ L{2cm} c L{5cm} }
  \toprule
	Name  && Requirement \\
    Init. env.  &($\theta_e$) & $(sensor.disabled \veebar sensor.enabled)$ $ \wedge \neg timeout \wedge \neg data.rcvd$\\    
    Init. sys. &($\theta_s$) & $s = 0 \wedge \neg update.direction \wedge$ $\neg get.SLAM.location \wedge \neg get.GPS.location$\\
    
    \ldots  && \ldots \\
    
    $_{4 \rightarrow 3}$ $timeout$ &($\rho_s$)
    &$\square((s=4 \wedge timeout)\rightarrow \Circle(s=3))$ \\
    
    $_{4 \rightarrow 5}$ $data.rcvd$ &($\rho_s$)
    &$\square((s=4 \wedge data.rcvd)\rightarrow \Circle(s=5))$ \\
    
    \ldots  && \ldots \\
    
    $_{4\rightarrow 5}$ $data.rcvd$ enable$_{sys}$ &($\rho_s$)
    &$\square((s=4 \wedge data.rcvd)\rightarrow \Circle(update.direction \wedge \neg get.SLAM.location \wedge \neg get.GPS.location))$ \\    
    
    $_{4\rightarrow 5}$ $data.rcvd$ enable$_{env}$ &($\rho_e$)
    &$\square((s=4 \wedge data.rcvd)\rightarrow \Circle(\neg timeout \wedge \neg sensor.enabled \wedge \neg sensor.disabled \wedge \neg data.rcvd))$ \\        
    
    \ldots  && \ldots \\
    Goal   &($\varphi_s$) & $\square\Diamond(update.direction)$\\
    \bottomrule
  \end{tabular}
  \caption{Planar Robot Equivalent Game Structure}
  \label{table:planar-robot-declarative-spec}
 \end{table}
\end{center}
The Könighoefer et al. technique removes system requirements
only.  After applying their minimization the 
$_{4 \rightarrow 5} data.rcvd$ and $_{4 \rightarrow 5}$ $data.rcvd$ enable$_{sys}$ requirements are removed,
since relaxing the system this way does not affect 
the environment winning strategy.  The syntactic 
simplification actually augments the system transition
relation $\rho_{s}$ for the case where
$s=4 \wedge data.rcvd$ holds, implicitly adding a
$data.rcvd$ transition to every other state.  This
is feasible since the environment will always pick
$timeout$ over $data.rcvd$ when reaching state $4$. 
Aforementioned requirements 
($_{4 \rightarrow 5} data.rcvd$ and $_{4 \rightarrow 5}
data.rcvd$ enable$_{sys}$) are irrelevant to the 
non realizability cause, but removing them 
bloats the specification in terms of explicit behavior
since it adds more transitions.  We want to
improve the understanding of the problem by way
of simplification, this adds another point in favor
o a novel minimization technique, designed specifically
for behavior based (or operational) specifications.
%To exemplify the difference, and the complementation between the model-based
and the behavior-based minimization we introduce a synthetic case where an engineer
wants to specify a zero triggered latch mechanism. The initial specification
involves only two signals: $x$ representing a single boolean input and $y$ the only
boolean output.  The idea is that the output should be kept constantly up 
once a zero boolean value has been fed to the system.
Before this $y$ must remain as a zero.  
The specification is written as the conjunction of four \texttt{LTL} \cite{pnueli1977temporal}
formulas as presented in table \ref{table:d-ff-spec}. 
Following the definitions from \cite{DBLP:conf/vmcai/PitermanPS06} $\theta_s$ stands for
an initial system formula, $\rho_s$ for a safety system formula and 
$\varphi_s$ for a system liveness formula.
The \textit{Initial condition} formula 
fixes the initial value for the output signal,
\textit{Wait for negative trigger} keeps the value of $y$ down as long as the input
is up and the output value has not yet been raised, \textit{Latch} keeps the 
output up after the initial raise and the \texttt{GR(1)} \textit{Goal} is to 
always eventually produce an output ($\square\Diamond(y)$).  After running our automated
translation we produced the \texttt{LTS} representation of the \textit{Game Structure}
definition.  We get the behavior depicted in Figure \ref{fig:fig.neg-trigger-latch-behavior},
where again solid lines indicate monitored events and dashed lines indicate controlled events.
The $\uparrow$ suffix after a signal name means the the value is being raised ($x\uparrow$),
$\downarrow$ consequently means that value is being lowered ($x\downarrow$).  A $tick$ event
is added to show when the values could be kept constant at the current time step.
\begin{center}
\begin{table}[h]
  \begin{tabular}{ l  c  l }
	Name & Type & Requirement \\
    \hline
    Initial condition & $\theta_s$ & $y = 0$\\
    Wait for negative trigger & $\rho_s$ &$\square((x \wedge \neg y) \rightarrow \Circle(\neg y) )$ \\
    Latch & $\rho_s$ &$\square (y \rightarrow \Circle(y))$ \\   
    Goal &  $\varphi_s$ & $\square\Diamond(y)$\\
  \end{tabular}
  \caption{Zero triggered latch specification}
  \label{table:d-ff-spec}
 \end{table}
\end{center}
We wrote this specification in \texttt{RATSY}, run the minimization presented in
\cite{DBLP:conf/hvc/KonighoferHB10} (where the \textit{Latch} requirement was
considered irrelevant and removed) and then translated the minimized specification
to produce the \texttt{LTS} depicted in Figure 
\ref{fig:fig.neg-trigger-latch-behavior-formula-minimization}.  Understanding the specification
in terms of the conjunction of formulas we can see that the $Latch$ requirement is irrelevant
to the unrealizability cause since even if we allow the plant to lower $y$ after its initial
raise, no output would be produced if the input is kept up.  Removing a formula is, in fact,
relaxing the behavior, this is expressed in the double lined states and transitions 
present in Figure \ref{fig:fig.neg-trigger-latch-behavior-formula-minimization}.  
We can see that the transitions added express the new behavior where the plant is allowed
to lower $y$ after being raised.  Let us note that removing a formula means relaxing, 
thus augmenting, the behavior of the plant.
\begin{figure}[bt]
\centering
\SmallPicture
%\ShowFrame
\VCDraw{
    \begin{VCPicture}{(-4,-5)(4,4)}
        \SetEdgeLabelScale{1.4}
        \State[1]{(0,3)}{1}
        \State[2]{(-3,2)}{2}        
        \State[3]{(-3,0)}{3}
        \State[4]{(3,-2)}{4}        
        \State[5]{(-3,-2)}{5}                
        \State[6]{(-3,-4)}{6}                        
        \State[7]{(3,2)}{7}                                
        \State[8]{(3,0)}{8}                                        
		\Initial[n]{1}
		\EdgeL{1}{2}{$x$\downarrow}
		\EdgeR{1}{7}{$x$\uparrow}		
		\EdgeR{8}{3}{$x$\downarrow}								
		\LoopW{3}{$tick$}
		\LoopE{8}{$tick$}		
		\LoopW{6}{$tick$}		
		\EdgeR{3}{4}{$x$\uparrow}
		\ArcR{5}{6}{$x$\uparrow}
		\ArcR{6}{5}{$x$\downarrow}		
        \ChgEdgeLineStyle{dashed} 
		\EdgeR{4}{6}{$y$\uparrow}		
		\EdgeL{2}{3}{$y$\downarrow}				
		\EdgeL{7}{8}{$y$\downarrow}						
		\EdgeR{3}{5}{$y$\uparrow}
		\EdgeR{4}{8}{$tick$}		
    \end{VCPicture}
}
\vspace*{-2mm}
\caption{Zero triggered latch behavior.}
\label{fig:fig.neg-trigger-latch-behavior}
\vspace*{-4mm}
\MediumPicture
\end{figure}

\begin{figure}[bt]
\centering
\SmallPicture
%\ShowFrame
\VCDraw{
    \begin{VCPicture}{(-4,-6)(4,5)}
        \SetEdgeLabelScale{1.4}
        \State[1]{(0,3)}{1}
        \State[2]{(-3,2)}{2}        
        \State[3]{(-3,0)}{3}
        \State[4]{(1,-1)}{4}        
        \State[5]{(-3,-2)}{5}                
        \State[6]{(-3,-4)}{6}                        
        \State[7]{(3,2)}{7}                                
        \State[8]{(3,0)}{8}                                        
        \StateLineDouble
		\State[9]{(1,-2.5)}{9}                         
        \State[10]{(-1,-2.5)}{10}                             
		\Initial[n]{1}
		\EdgeL{1}{2}{$x$\downarrow}
		\EdgeR{1}{7}{$x$\uparrow}		
		\ArcR{8}{3}{$x$\downarrow}								
		\LoopW{3}{$tick$}
		\LoopE{8}{$tick$}		
		\EdgeL[.8]{3}{4}{$x$\uparrow}
		\ArcR{5}{6}{$x$\uparrow}
		\ArcR{6}{5}{$x$\downarrow}	
		\EdgeLineDouble		
		\LoopS{9}{$tick$}	
		\ArcR[.3]{6}{9}{$tick$}
		\EdgeR{9}{10}{$x$\downarrow}
		\EdgeLineSimple
        \ChgEdgeLineStyle{dashed} 
		\EdgeL{2}{3}{$y$\downarrow}				
		\EdgeL{7}{8}{$y$\downarrow}						
		\ArcR{3}{5}{$y$\uparrow}
		\EdgeL{4}{8}{$tick$}		
		\EdgeLineDouble
		\EdgeR{10}{5}{$tick$}		
		\ArcR[.2]{10}{3}{$y$\downarrow}						
		\ArcR{4}{9}{$y$\uparrow}		
		\ArcR{9}{8}{$y$\downarrow}		
		\ArcR{5}{3}{$y$\downarrow}
		\VArcR{arcangle=-60,ncurv=1}{6}{8}{$y$\downarrow}				
    \end{VCPicture}
}
\vspace*{-2mm}
\caption{Zero trig. latch formula based minimization.}
\label{fig:fig.neg-trigger-latch-behavior-formula-minimization}
\vspace*{-4mm}
\MediumPicture
\end{figure}
This comparison may seem unfair, and to some extent it is, if one
forgets that we are looking at the same problem expressed in two
quite different ways.  One of our claims is that our approach is in
fact complimentary to that of the model-based minimization.
Our minimization produces the plant depicted in Figure 
\ref{fig:fig.neg-trigger-latch-behavior-minimization}, where it
shows that the environment can easily win by playing to keep
$x$ up.  
It is important to note that when minimizing the plant we must
keep it unrealizable while allowing the environment to have
a winning strategy that works in both the minimized and the original
plant.
\begin{figure}[bt]
\centering
\SmallPicture
%\ShowFrame
\VCDraw{
    \begin{VCPicture}{(-5,-2)(5,2)}
        \SetEdgeLabelScale{1.4}
        \State[1]{(-3,0)}{1}
        \State[2]{(0,0)}{2}        
        \State[3]{(3,0)}{3}
		\Initial[w]{1}
		\EdgeL{1}{2}{$x$\uparrow}
		\LoopE{3}{$tick$}		
        \ChgEdgeLineStyle{dashed} 
		\EdgeL{2}{3}{$y$\downarrow}				
    \end{VCPicture}
}
\vspace*{-2mm}
\caption{Zero trig. latch behavior based minimization.}
\label{fig:fig.neg-trigger-latch-behavior-minimization}
\vspace*{-4mm}
\MediumPicture
\end{figure}
In the model-based and the behavior-based approaches the
idea is that we are aiding the engineer in understanding the
cause on unrealizability.  If we think in terms of specification repair,
in this particular case if she were to add
a new assumption (\textit{input will eventually fall}), either as
$\square\Diamond(\neg x)$ or $\Diamond(\neg x)$ the specification
will become realizable.



\end{comment}
 