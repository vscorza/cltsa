\section{Motivation}\label{sec:motivation}
We motivate and informally introduce relevant concepts by discussing a small example. 
Suppose that an engineer is working on a specification consisting of two environmental 
binary variables $a$ and $b$ and one system variable $y$. Variable $b$ 
(initially false) is required to swap its value at each step of the execution, while $a$ 
remains free. The value of $y$ is set at the start and kept fixed throughout the 
execution. Setting and releasing a variable $v$ is indicated by the pair of events $v.on$ and $v.off$.
It is assumed that $a$ will hold true infinitely often while $a \wedge b \wedge y$ is also required to be satisfied infinitely often. This is the property to be satisfied expressed as a GR(1) formula: 
\begin{center}
	\begin{tabular}{ r c l }
		$\varphi$& $=$ &$\square \Diamond a\implies\square \Diamond (a \wedge b \wedge y)$\\
	\end{tabular}
\end{center}
The automaton induced from this set of formulae is depicted in Fig. ~\ref{fig:konig_original_plant_2_clts}. At this point, the specification is not realizable since the environment can set $a$ to true only when $b$ is down, thus honoring the environmental assumption while avoiding the system to reach the winning valuation $a \wedge b \wedge y$.  
Our approach simplifies the specification semantically, reducing the relevant behavior and showing that it is possible for the environment to avoid state 2 by never allowing $a$ and $b$ to happen simultaneously, while at the same time not restricting the system below its original capabilities.
 Notice that if we were allowed to add the assumption that both environmental variables occur simultaneously infinitely often ($\square \Diamond (a \wedge b)$) the specification would be realizable and choosing to raise $y$ at the beginning would be sufficient in order to guarantee our goal, as shown in Fig. ~\ref{fig:konig_strategy_2}. After fixing $y$ to true the environment can move freely as long as it visits state 2 infinitely often. If the engineer has this behavior in mind as their design intent, it should be easy to see that in the semantic minimization the desired state is unreachable, hopefully guiding them to the strengthening of the assumption.  In Fig. ~\ref{fig:konig_composition_2} we show the synchronous composition of the aforementioned strategy with our minimization. Note that after letting the system raise $y$ state the execution visits states 5,6,7,3,4,1 to keep alternating the values of $a$ and $b$ in mutual exclusion.

%\begin{figure}[bt]
%	\centering
%	\SmallPicture
%	%\ShowFrame
%	\VCDraw{
%		\begin{VCPicture}{(-4,-2.5)(4,9.5)}
%			\SetEdgeLabelScale{1.4}
%			\StateVar[0:\overline{x_1 x_2} y]{(-3,6)}{0}
%			\StateVar[1:x_1 x_2 y]{(0,8)}{1}
%			\StateVar[2:x_1 \overline{x_2} y]{(3,6)}{2}
%			\StateVar[3:\overline{x_1} x_2 y]{(0,4)}{3} 
%			\StateVar[4:\overline{x_1 x_2 y}]{(-3,0)}{4}			
%			\StateVar[5:x_1 x_2 \overline{y}]{(0,2)}{5}
%			\StateVar[6:x_1 \overline{x_2 y}]{(3,0)}{6}
%			\StateVar[7:\overline{x_1} x_2 \overline{y}]{(0,-2)}{7}   			  
%			\Initial[w]{0}
%			\Initial[w]{4}	
%			\ArcR{0}{1}{}
%			\ArcR{1}{0}{}
%			\ArcR{1}{2}{}						
%			\ArcR{2}{1}{}
%			\ArcR{2}{3}{}        
%			\ArcR{3}{2}{}
%			\ArcR{3}{0}{}
%			\ArcR{0}{3}{}
%			\ArcR{4}{5}{}
%			\ArcR{5}{4}{}
%			\ArcR{5}{6}{}						
%			\ArcR{6}{5}{}
%			\ArcR{6}{7}{}        
%			\ArcR{7}{6}{}
%			\ArcR{7}{4}{}
%			\ArcR{4}{7}{}			
%		\end{VCPicture}
%	}
%	\caption{Original Plant}
%\label{fig:konig_original_plant_2}
%\MediumPicture
%\end{figure}

\begin{figure}[bt]
\centering
\SmallPicture
%\ShowFrame
\VCDraw{
\begin{VCPicture}{(-5,-2)(4,7.5)}
	\SetEdgeLabelScale{1.4}
%	\State[0]{(-5,2.5)}{init}
	\State[0]{(-5,1.75)}{0}		
	\State[1]{(-3.5,6.5)}{1}
	\State[2]{(-0.5,7)}{2}	
	\State[3]{(2.5,6.5)}{3}
	\State[4]{(-0.5,6)}{4}
	\State[5]{(-3.5,4)}{5} 
	\State[6]{(-0.5,4.5)}{6} 
	\State[7]{(2.5,4)}{7}			
	\State[8]{(-0.5,3.5)}{8}	
	\State[9]{(-3.5,1)}{9}
	\State[10]{(-0.5,1.5)}{10}	
	\State[11]{(2.5,1)}{11}
	\State[12]{(-0.5,0.5)}{12}
	\State[13]{(-3.5,-1.5)}{13} 
	\State[14]{(-0.5,-1)}{14} 
	\State[15]{(2.5,-1.5)}{15}			
	\State[16]{(-0.5,-2)}{16}		
				
	\Initial[w]{0}
	%\EdgeL{init}{init_e}{\overline{x_1 x_2}}
	\ArcL{0}{5}{y}
	\ArcL{5}{1}{b}	
	\ArcL{1}{5}{\overline{b}}
	\ArcL{1}{2}{a}		
	\ArcL{2}{3}{\overline{b}}
	\ArcL{3}{4}{\overline{a}}
	\ArcL{4}{1}{b}		
	\ArcL{3}{7}{b}	
	\ArcL{7}{3}{\overline{b}}		
	\ArcL{7}{8}{\overline{a}}
	\ArcL{8}{5}{\overline{b}}
	\ArcL{5}{6}{a}
	\ArcL{6}{7}{b}	
	\ArcR{0}{13}{\overline{y}}
	\ArcL{13}{9}{b}	
	\ArcL{9}{13}{\overline{b}}
	\ArcL{9}{10}{a}		
	\ArcL{10}{11}{\overline{b}}
	\ArcL{11}{12}{\overline{a}}
	\ArcL{12}{9}{b}		
	\ArcL{11}{15}{b}	
	\ArcL{15}{11}{\overline{b}}		
	\ArcL{15}{16}{\overline{a}}
	\ArcL{16}{13}{\overline{b}}
	\ArcL{13}{14}{a}
	\ArcL{14}{15}{b}		
\end{VCPicture}
}
	\caption{Original Plant}
	\label{fig:konig_original_plant_2_clts}
	\MediumPicture
	\vspace{-1em}
\end{figure}

\begin{figure}[bt]
	\centering
	\SmallPicture
	%\ShowFrame
	\VCDraw{
	\begin{VCPicture}{(-5,.5)(4,6.5)}
		\SetEdgeLabelScale{1.4}
	\State[0]{(-5,2.5)}{0}		
\State[1]{(-3.5,5.5)}{1}
\State[3]{(2.5,5.5)}{3}
\State[4]{(-0.5,6)}{4}
\State[5]{(-3.5,4)}{5} 
\State[6]{(-0.5,3.5)}{6} 
\State[7]{(2.5,4)}{7}			
\State[9]{(-2,1)}{9}
\State[10]{(0.25,1.5)}{10}	
\State[11]{(2.5,1)}{11}
\State[12]{(0.25,0.5)}{12}
\State[13]{(-3.5,1)}{13} 
\Initial[w]{0}
%\EdgeL{init}{init_e}{\overline{x_1 x_2}}
\ArcL{0}{5}{y}
\ArcR{1}{5}{\overline{b}}
\ArcR{5}{6}{a}
\ArcR{6}{7}{b}
\ArcR{7}{3}{\overline{b}}		
\ArcR{3}{4}{\overline{a}}
\ArcR{4}{1}{b}		
\ArcR{0}{13}{\overline{y}}
\ArcR{13}{9}{b}	
\ArcL{9}{10}{a}		
\ArcL{10}{11}{\overline{b}}
\ArcL{11}{12}{\overline{a}}
\ArcL{12}{9}{b}		
	\end{VCPicture}
}
	\caption{Behavior minimization}
	\label{fig:konig_original_plant_c_2}
	\MediumPicture
\end{figure}
\begin{figure}[bt]
	\centering
	\SmallPicture
	%\ShowFrame
	\VCDraw{
		\begin{VCPicture}{(-5,2.5)(4,7.5)}
	\State[0]{(-5,4)}{0}		
\State[1]{(-3.5,6.5)}{1}
\State[2]{(-0.5,7)}{2}	
\State[3]{(2.5,6.5)}{3}
\State[4]{(-0.5,6)}{4}
\State[5]{(-3.5,4)}{5} 
\State[6]{(-0.5,4.5)}{6} 
\State[7]{(2.5,4)}{7}			
\State[8]{(-0.5,3.5)}{8}	
\Initial[w]{0}
%\EdgeL{init}{init_e}{\overline{x_1 x_2}}
\EdgeL{0}{5}{y}
\ArcL{5}{1}{b}	
\ArcL{1}{5}{\overline{b}}
\ArcL{1}{2}{a}		
\ArcL{2}{3}{\overline{b}}
\ArcL{3}{4}{\overline{a}}
\ArcL{4}{1}{b}		
\ArcL{3}{7}{b}	
\ArcL{7}{3}{\overline{b}}		
\ArcL{7}{8}{\overline{a}}
\ArcL{8}{5}{\overline{b}}
\ArcL{5}{6}{a}
\ArcL{6}{7}{b}	
		\end{VCPicture}
	}
	\caption{Realizable version strategy}
	\label{fig:konig_strategy_2}
	\MediumPicture
\end{figure}
\begin{figure}[bt]
	\centering
	\SmallPicture
	%\ShowFrame
	\VCDraw{
	\begin{VCPicture}{(-5,2.5)(4,6)}
		\State[0]{(-5,4)}{0}		
		\State[1]{(-3.5,5.5)}{1}
		\State[3]{(2.5,5.5)}{3}
		\State[4]{(-0.5,6)}{4}
		\State[5]{(-3.5,4)}{5} 
		\State[6]{(-0.5,3.5)}{6} 
		\State[7]{(2.5,4)}{7}			
		\Initial[w]{0}
		%\EdgeL{init}{init_e}{\overline{x_1 x_2}}
		\EdgeL{0}{5}{y}
		\ArcR{1}{5}{\overline{b}}
		\ArcR{5}{6}{a}
		\ArcR{6}{7}{b}
		\ArcR{7}{3}{\overline{b}}		
		\ArcR{3}{4}{\overline{a}}
		\ArcR{4}{1}{b}		
	\end{VCPicture}
}
	\caption{Composition of the strategy with the minimization}
	\label{fig:konig_composition_2}
	\MediumPicture
	\vspace{-1em}
\end{figure}
