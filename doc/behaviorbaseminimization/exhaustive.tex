For the exploration criterion we evaluated a version of Delta Debugging
that was adapted to our technique, a DFS search algorithm
and then an exhaustive first search approach that looks for all the
minimal induced plants.\\
We introduce the exhaustive exploration algorithm first,
we will use a mapping between the superset
of monitored transitions $\mathcal{T}_u$ in $E$ to the natural numbers.  
It allows to order the search by scanning the range 
$r_u = [0\ldots 2^{|\mathcal{T}_u|}-1]$ incrementally.  The elements
of $r_u$ encode members of the superset $\mathcal{P}(\mathcal{T}_u)$
as follows, let $\mathcal{T}'_u \subseteq \mathcal{T}_u$:
\[
get\_binary\_key(\mathcal{T}'_u, \mathcal{T}_u) = 
\sum\limits_{i \in [0\ldots 2^{|\mathcal{T}_u|}-1], \mathcal{T}'_u[i] \in \mathcal{T}_u}2^i
\]
The dual function $get\_transitions:\mathbb{N} \rightarrow \mathcal{P}(\mathcal{T}_u)$ will retrieve the pre image of $get\_binary\_key$.
The overview is linked to the pseudocode
in Figure \ref{fig:exhaustive-code} representing the runtime behavior of the
algorithm by means of the \listRef{A} \ldots \listRef{G} markers to aid the reader.
The algorithm starts by getting the set of monitored transitions as 
candidates to be removed from the plant \listRef{A} and then computing the 
binary representation for the whole set.  The set \texttt{candidates} of
induced plants is initialized containing the original automaton $E$ \listRef{B}.
The algorithm then proceeds to explore the search space by decrementing the 
\texttt{key} \listRef{C} and computing its pre image \listRef{D}.  
The automaton under observation is the updated according to the current
$\mathcal{T}'_u$ set \listRef{E}, reduced to its closest induced representation
and checked for realizability \listRef{F}.  If the automaton happens to be 
unrealizable it is added to the collected candidates list.  We then proceed
to get the minimal induced plants by removing any candidate that has an 
unrealizable induced plant \listRef{G}.  The remaining members of the list
are returned as the set of conflicts found for $E$.
\begin{figure}[ht]
  \begin{center}
    \renewcommand{\ttdefault}{pcr}
\begin{lstlisting}[escapeinside={[*}{*]},basicstyle=\scriptsize\ttfamily,columns=flexible,frame=lines,mathescape=true,xleftmargin=5.0ex,keywordstyle=\textbf,morekeywords={if,while,do,else,fork,int,null},numbers=left,numberstyle=\scriptsize]
exh_min($E$):
	$\mathcal{T}_u$ = $E$.get_monitored_transitions() [*\listRef{A}*]
	$key$ = get_binary_key($\mathcal{T}_u$)
	candidates = [$E$] [*\listRef{B}*]
	
	while $key$ != 0:
		$key$.decrement() [*\listRef{C}*]
		$E_{new}$ = $E$.clone()
		$\mathcal{T}'_u$ = get_transitions($\mathcal{T}_u$, $key$) [*\listRef{D}*]
		$\mathcal{T}_r$ = $\mathcal{T}_u \setminus \mathcal{T}'_u$ 
		for $t$ in $\mathcal{T}_r$:
			$E_{new}$ = $E_{new}$.remove($t$) [*\listRef{E}*]
		$E_{new}$.get_legal_reduction($E$)
		is_realizable = check($E_{new}$) [*\listRef{F}*]
		if is_realizable = $\bot$:		
			candidates.push($E_{new}$)
	
	for $E_i$ in candidates:
		for $E_j$ in candidates:
			if $i$ != $j$:
				if is_induced_from($E_j$,$E_i$): [*\listRef{G}*]
					candidates.remove($E_i$)
	return candidates
\end{lstlisting} 
    \caption{Exhaustive exploration code}
    \label{fig:exhaustive-code}
  \end{center}
\end{figure}