\section{Behavior Minimization Feedback}\label{sec:problemStatement}

In this section we formalize the problem of minimizing behavior while preserving unrealizability. In the next section we define a minimization procedure.

Assume an unrealizable specification of the form $\mathcal{I} = <E, \mathcal{C}, \varphi>$. Our aim is to automatically produce an unrealizable specification $\mathcal{I'} = <E', \mathcal{C}, \varphi>$ where $E'$ has less behavior than $E$ and the witnesses for the unrealizablity of $\mathcal{I'}$ (i.e., the counter strategies) are also witnesses for the unrealizability of $\mathcal{I}$.  Furthermore, we aim for $E'$ to be minimal in the sense that it cannot be further reduced while having its associated counter strategies preserved in  $\mathcal{I}$.
Thus, an engineer focusing on why  $\mathcal{I'}$ is unrealizable can gain insight on why $\mathcal{I}$ is unrealizable. 


\begin{definition}\label{}\emph{(Unrealizability Preservation)}
Given an unrealizable control problem $\mathcal{I} = <E, \mathcal{C}, \varphi>$ we say
that $\mathcal{I'} = <E', \mathcal{C}, \varphi>$ preserves unrealizability
 if for all  winning strategy \counterS for
the environment in $G(\mathcal{I}_P)$, $f_1^{\neg \varphi}$ is winning for the environment in
$G(\mathcal{I}_M)$.
\end{definition}


Now define the Alternating Sub-LTS relation to capture a notion of minimization that preserves unrealizability. Such a relation preserves counter strategies and forms a semi-lattice that can be used to incrementally minimize a unrealizable specification. 

%\textbf{
%CREO QUE HAY QUE BORRAR ESTE PARRAFO
%}In ~\cite{DBLP:phd/ethos/DIppolito13} the author shows that
%a control problem is realizable if and only if there is a winning strategy
%for the system in the game constructed following the conversion
%proposed in definition \ref{def:lts-2-game}.  The same work shows that GR(1) games 
%are determined, meaning that a control problem is unrealizable if
%and only if there exists a winning strategy for the environment.
%We will refer to the latter as the \emph{counter strategy}.
%
%\textbf{Y ESTE TAMBIEN}
%Formally, we aim to build a minimal unrealizable specification $\mathcal{I'} = <E', \mathcal{C}, \varphi>$ such that any counter strategy in $G(\mathcal{I'})$ is also a counter strategy in $G(\mathcal{I})$. 

%If $\mathcal{I} = <E, \mathcal{C}, \varphi>$ is a non realizable control problem then there is no LTS $M$ that is a legal and deadlock-free environment with respect to $E$ such that $E || X \models \varphi$. 


To define Alternating Sub-LTSs we first define the Sub-LTS relation.  A Sub-LTS is simply achieved by removing states and transitions from a LTS while keeping its initial state.
%~\ref{XXX}

\begin{definition}\label{def:lts-inclusion}\emph{(Sub-LTS)}
Given $M = (S_M, \Sigma_M, \Delta_M, s_{M_0})$ and
 $P =$ $(S_P,\Sigma_P,\Delta_P,s_{P_0})$ LTSs, 
we say that $P$ is a Sub-LTS of $M$ (noted $P \subseteq M$) if $S_P \subseteq S_M$,
$s_{M_0} = s_{P_0}$, $\Sigma_P \subseteq \Sigma_M$ and $\Delta_P \subseteq \Delta_M$.
\end{definition}


%We define an unrealizability preserving control problem as follows:

%\subsection{Behavior Minimization}
%  We expect our minimization to be sound, which means that
%for every counter strategy winning for the
%environment in $E_{\kappa}$ it must also be winning for the environment
%in $E$. To formalize the notion of soundness of
%our minimization procedure $min(\mathcal{I})$ we
%expect the resulting control problem $\mathcal{I}_{\kappa}$
%to preserve non realizability, to allow every winning (counter) strategy
%for the environment in $E_{\kappa}$ to hold in $E$ and
%to be minimal in terms of retained behavior.

To preserve unrealizability, the Alternating Sub-LTS definition refines that of Sub-LTS by restricting some of the original LTS transitions. 
These restrictions consider controllability. 

%In order to define when a minimized plant $E_{\kappa}$ preserves
%a counter strategy w.r.t. to an original plant $E$ and a controllable set
%$\mathcal{C}$ we introduce the following definitions:


\begin{definition}\label{def:nonreal-legalEnvironment}\emph{(Alternating Sub-LTS)}
Given $M = (S_M, \Sigma, \Delta_M, s_{M_0})$ and
 $P =$ $(S_P,\Sigma,\Delta_P,s_{P_0})$ LTSs, 
where $\Sigma =\mathcal{C}\cup \mathcal{U}$ and $\mathcal{C}\cap
\mathcal{U}=\emptyset$. We say that $P$ is a Alternating
 Sub-LTS of $M$, noted as $P \sublts_{\mathcal{C},\mathcal{U}} M$, if  $P \subseteq M$  and $\forall s \in S_{P}$ the following holds:
 \begin{itemize}
\item If $s$ is a pure controllable state then $\Delta_{P}(s) \cap \mathcal{C} = \Delta_{M}(s) \cap \mathcal{C} $
\item If $s$ is not a pure controllable state then  
%$\Delta_{M}(s)\neq\emptyset \rightarrow \Delta_{P}(s)\neq\emptyset$
%\item 
%$(\Delta_{M}(s) \cap \mathcal{C} \neq \emptyset$ $ \wedge $ $\Delta_{M}(s) \cap \mathcal{U} \neq \emptyset)$ $\rightarrow$
$\Delta_{P}(s) \cap \mathcal{U} \neq \emptyset$.
 \end{itemize}
 
We shall omit $\mathcal{C}$ and $\mathcal{U}$ in $\sublts_{\mathcal{C},\mathcal{U}}$ when the context is clear. 
 \end{definition}

The definition above introduces  restrictions on transitions of Alternating Sub-LTSs. 
First, in states where all outgoing transitions are controllable, all transitions must be preserved. 
If this were not the case, then a counter strategy in $P$ with $P \sublts M$ may not work in $M$ because the controller in $M$ has more controllable actions that the counter strategy in $P$ does not account for. 
The second restriction states that in all states where at least one uncontrollable action exists, at least one uncontrollable action must be preserved. This restriction ensures that in pure uncontrollable states, not new deadlock in $P$ is introduced. Otherwise an environment strategy could win by forcing that deadlock in $P$, even when the same strategy would not win in $M$ at that state. In mixed states, removing all uncontrolled transitions would allow a counter strategy in $P$ to force the controller to move (as not doing so would result in a deadlock). However a strategy in $P$ exploiting this would not work in $M$ because reaching the same state, now a mixed state, the controller may defeat the same environment strategy by not picking any controlled action since the environment is forced to progress in this case. 

The justification behind the definition of Alternating Sub-LTSs is that for two unrealizable control problems $\mathcal{I} = <E, \mathcal{C}, \varphi>$ and $\mathcal{I'} = <E', \mathcal{C}, \varphi>$ with $E' \sublts E$ any counter strategy for the game $G(\mathcal{I'})$ is also a counter strategy for $G(\mathcal{I})$. We first introduce a corresponding notion of Alternating Sub-Game and then use it to show that the Alternating Sub-LTS relation preserves counter strategies.  

\begin{definition}\label{def:sub-game}\emph{(Sub-Game)}
Given $G = (S_g, \Gamma^{-},$ $\Gamma^{+},$ $s_{g_{0}}$$,$ $\varphi)$ and
$G' = (S'_g, \Gamma^{-\prime}, \Gamma^{+\prime},s'_{g_{0}}, \psi)$ two-player games, we say that $G'$ is a sub-game of $G$, noted
$G' \subgame G$, if $S'_g \subseteq S_g$, $s'_{g_{0}}=s_{g_{0}}$, $\psi = \varphi$, $\Gamma^{-\prime}\subseteq \Gamma^{-}$ and
$\Gamma^{+\prime}\subseteq \Gamma^{+}$.
\end{definition}

\begin{definition}\label{def:alternating-sub-game}\emph{(Alternating Sub-Game)}
Given $G = (S_g, \Gamma^{-}, \Gamma^{+},$ $s_{g_{0}}, \varphi)$ and
$G' = (S'_g, \Gamma^{-\prime}, \Gamma^{+\prime},s'_{g_{0}}, \psi)$ both two-player games such that $G' \subgame G$, we say that $G'$ is an Alternating Sub-Game of $G$, noted
$G'\altsubgame G$, if the following holds: $\forall s'_g \in S'_g:$ 
$\Gamma^{+}(s'_g)=\Gamma^{+\prime}(s'_g)$ and $\Gamma^{-}(s'_g)\neq \emptyset$ $\rightarrow$ $\Gamma^{-\prime}(s'_g) \neq \emptyset$. 
\end{definition}

\begin{lemma}\emph{(Alternating Sub-LTS implies Alternating Sub-Game)}\label{theorem:alternating-sub-game}
Let $\mathcal{I}_1 = \langle E_1, \mathcal{C}, \varphi \rangle$ and
$\mathcal{I}_2 = \langle E_2, \mathcal{C}, \varphi \rangle$ be two
control problems, then if $E_2 \sublts E_1$ implies $G(\mathcal{I}_2)$
$\altsubgame$ $G(\mathcal{I}_1)$.
\end{lemma}

%Proof for Lemma~\ref{theorem:alternating-sub-game}


\textbf{HAY QUE REVER ESTA PORQUE CAMBIO LA DEFINICI‰ON DE SUBLTS}
\begin{proof}
$\mathcal{I}_1$ and $\mathcal{I}_2$ share the same winning condition
so they also share fluents definition.  Since
$E_2 \sublts E_1$ implies $E_2 \subseteq E_1$ we know that
$S_2 \subseteq S_1$ and then for every state in the game constructed from
the control problem holds that $S_2 \times \Pi_{i=0}^{k}\{true,false\}$
$\subseteq$ $S_1 \times \Pi_{i=0}^{k}\{true,false\}$ implying
that $S_{g_2}$ $\subseteq$ $S_{g_1}$.  
We can see that following the definition of the
game $G(\mathcal{I}_1)$ constructed from $\mathcal{I}_1$ if
$\Delta_2 \subseteq \Delta_1$ and 
$s_{g_1} = (s_e,\alpha_1,\ldots,\alpha_k)$ $\in S_{g_1}$
for all $(s_e, l, s'_e) \in \Delta_1$ the transition
$(s_{g_1},(s'_e,\alpha'_1,\ldots,\alpha'_k))$ will be added
to $\Gamma^{+}_1$ if $l \in \mathcal{C}$ or to $\Gamma^{-}_1$
if $l \in \mathcal{U}$.  Since $S_{g_2} \subseteq$ $S_{g_1}$ and 
$\Delta_2 \subseteq$ $\Delta_1$ it holds that
$\Gamma^{+}_2 \cup \Gamma^{-}_2$ $\subseteq$
$\Gamma^{+}_1 \cup \Gamma^{-}_1$. This far we have proven
$G(\mathcal{I}_2) \subseteq G(\mathcal{I}_1)$.  Now, if 
$E_2 \sublts E_1$ we know, starting with pure states, that 
for all $s \in S_2, \Delta_1(s)\cap \mathcal{U} = \emptyset$, $l \in \mathcal{C}$ if
$(s,l,s')\in \Delta_1$ then $(s,l,s') \in \Delta_2$ and
also for every state $s_{g_1}$ related to $s$ in the game: $s_{g_1}$ $\in$ $s \times \Pi_{i=0}^{k}\{true,false\}$ and
for all $l \in \mathcal{C}$ such that 
$(s, l, s') \in \Delta_1$, $(s,l,s') \in \Delta_2$ and for
$s'_{g_1}= (s', \alpha'_1. \ldots, \alpha'_k)$, 
$(s_{g_1}, s'_{g_1})$$\in \Gamma^{+}_1$ and $(s_{g_1}, s'_{g_1})$$\in \Gamma^{+}_2$
hold,
thus implying $\Gamma^{+}_1(s_{g_1})$$=$$\Gamma^{+}_2(s_{g_1})$. 
If $s$ is a mixed state in $E_1$, following the construction of 
$G(\mathcal{I}_1)$ we know that
$\Gamma^{+}(s, \alpha_1^{\prime} \ldots \alpha_k^{\prime}) = \emptyset$,
from the definition of Sub-LTS $s$ will be a mixed state in
$E_2$ and $\Gamma^{+ \prime}$
$(\alpha_1^{\prime} \ldots \alpha_k^{\prime}) = \emptyset$ implies that
$\Gamma^{+}(s_g)=\Gamma ^{+ \prime}(s_g)$.
If $s$ is not a mixed
state and non controllable for all $s \in S_2$ if there exists an
$l \in \mathcal{U}$ such that
$(s,l,s')\in \Delta_1$ then from $E_2 \sublts E_1$ it must exists an 
$l' \in \mathcal{U}$ such $(s,l',s'') \in \Delta_2$.  
With $s_{g_1}$ $\in$ $s \times \Pi_{i=0}^{k}\{true,false\}$, $l$, 
and $s'_{g_1}= (s', \alpha'_1. \ldots, \alpha'_k)$
we have $(s_{g_1}, s'_{g_1})$$\in \Gamma^{-}_1$ and for $l'$ in $\Delta_2$,
let $s''_{g_1}= (s'', \alpha''_1. \ldots, \alpha''_k)$, we have
 $(s_{g_1}, s''_{g_1})$$\in \Gamma^{-}_2$ implying $\Gamma^{-}_1(s_{g_1})$ $ \neq \emptyset$ $\rightarrow$ $\Gamma^{-}_2(s_{g_1}) \neq \emptyset$. 
 If $s$ is a mixed state there exists $l' \in \mathcal{U}$ s.t. 
 $(s,l', s'') \in \Delta_1$ and then it holds that 
 $(s_{g_1}, s'_{g_1}) \in \Gamma_1^{-}$.
 Following Sub-LTS definition there must exist $l'' \in \mathcal{U}$
 s.t. $(s, l'',s'') \in \Delta_2$ implying the existence of 
 $(s_{g_1},s_{g''_1}) \in \Gamma_{2}^{-}$.
\end{proof}

The following lemma states that removing transitions in the environment description of an unrealizable control problem, while preserving an Alternating Sub-LTS relation, yields a new unrealizable control problem in which all its counter strategies are also counter strategies of the original control problem. From this follows that inspecting the behavior of the smaller environment description provides insights on the causes for unrealizability of the original one.  

\begin{lemma}\emph{(Alternating Sub-LTS preserves counter strategy)}\label{theorem:theo.preserves-non-realizability}
Let $P$ and $M$ be two LTSs s.t. $P \sublts M$, 
$\mathcal{I}_P = \lbrace P, \mathcal{C}, \varphi \rbrace$,
$\mathcal{I}_M = \lbrace M, \mathcal{C}, \varphi \rbrace$
two control problems, if
 \counterS is a winning strategy for
the environment in $G(\mathcal{I}_P)$ 
then $f_1^{\neg \varphi}$ is winning for the environment in
$G(\mathcal{I}_M)$.
\end{lemma}

We provide now an informal sketch of the proof, if the environment has an infinite
win in $\mathcal{I}_P$ not feasible in $\mathcal{I}_M$ and since $\mathcal{I}_P \subseteq \mathcal{I}_M$
it follows that the system was able to take the play
out of the path of the counter strategy.  This could be due to the fact that the system wins $\mathcal{I}_P$ or the environment has a strategy for  $\mathcal{I}_P$ not feasible in $\mathcal{I}_M$. If the system wins a controllable state $s_a$ must be reached where the environment strategy was denied, but following alternation all controllable options are
preserved at $s_a$ from $M$.  If this is the case the strategy should
hold since it wins in $P$ against every system choice for
the preserved states. If the environment wins with a strategy not feasible in $\mathcal{I}_M$ there should exist $s_b$ where either the environment reaches a deadlock not present in $\mathcal{I}_P$ or all non controllable options are disabled forcing the system to take an action not feasible in $\mathcal{I}_M$, but this two cases are again prevented by the Alternating Sub-LTS relation.  

Lemma ~\ref{def:induction-preserves} shows that we can move from $E$ to its closest Alternating Sub-LTS until we reach a Sub-LTS $E'$ ($E' \sublts E$) that preserves unrealizability but also has the property that any closest Alternating Sub-LTS $E''$ ($E'' \sublts E'$)  makes the problem realizable.
\begin{lemma}\label{def:induction-preserves}(\emph{Alternating Sub-LTSs preserve realizability})
	Let $E$ be the LTS for the non realizable control problem
	$\mathcal{I}=<E,\mathcal{C}, \varphi>$, and $E_1$, $E_2$ two LTSs s.t. 
	$E_2 \sublts E_1 \sublts E$ and
	$\mathcal{I}_1=<E_1, \mathcal{C}, \varphi>$, $\mathcal{I}_2=<E_2, \mathcal{C}, \varphi>$, if $\mathcal{I}_1$ is realizable then $\mathcal{I}_2$ is also realizable.
\end{lemma}

The general idea of why this property holds is that once a Sub-LTS is reached that allows the system to win $\mathcal{I}_1$ any reduction that follows alternation will only restrict environmental choices, allowing the system to keep its strategy.


We formally define the problem we solve as follows.

\begin{definition}\label{def:ProblemStatement}\emph{(Problem Statement)}
Given a unrealizable control problem $\mathcal{I} = <E, \mathcal{C}, \varphi>$, find an unrealizable control problem $\mathcal{I'} = <E', \mathcal{C}, \varphi>$ such that $E' \sublts E$, $\mathcal{I'}$ is unrealizable $\mathcal{I}$, and that there is no $\mathcal{I''} = <E'', \mathcal{C}, \varphi>$ such that $E' \sublts E''$ and $\mathcal{I''}$ is unrealizable. We refer to $\mathcal{I'}$ as a minimal unrealizable control problem.
\end{definition}

Suppose that a procedure reaches a minimal Sub-LTS by looking for candidates that satisfy the Alternating Sub-LTS property starting with $E$ and stopping once all its closest neighbors (w.r.t. to the Alternating Sub-LTS property) render the problem realizable (\emph{stopping criteria}). Such a procedure is 
complete w.r.t to the problem statement because
if a solution for the problem exists it can progressively,
reduce it to a minimal $E'$, lemma ~\ref{def:induction-preserves} proves that the \emph{stopping criteria}
implies minimality. If the problem has multiple solutions the procedure
will yield one of those.
Such a procedure is also sound w.r.t to the problem statement because
at every step it holds that $E' \sublts E$, thus the alternating condition
is met and lemma ~\ref{theorem:theo.preserves-non-realizability} shows that
$E'$ preserves unrealizability.
%Proof for Theorem\ref{theorem:theo.preserves-non-realizability}

\textbf{HAY QUE REVER ESTA PORQUE CAMBIO LA DEFINICI‰ON DE SUBLTS}
\begin{proof}
By way of contradiction suppose that 
$P$ is Alternating Sub-LTS of $M$ and $\mathcal{C}$ and that there
exists a counter strategy in $G(\mathcal{I}_P)$ that loses
$G(\mathcal{I}_M)$.
 Let $\pi$ be a play 
$s_0 s_1 \ldots$ consistent under non controllability with
the counter strategy that wins in $G(\mathcal{I}_P)$ but loses in $G(\mathcal{I}_M)$. 
$\pi$ leading to a finite win in $G(\mathcal{I}_P)$ at $s_{\bot}$
(not a finite state in $M$) will be prevented by the
alternation requirement where 
$\forall s_P \in S_P: |\Delta_M(s_p)| > 0 \rightarrow |\Delta_P(s_p)| > 0$ 
(implying $\forall s_P\prime=(s_P,\alpha_1,\ldots,\alpha_k) \in S_{G_P}: |\Gamma^{\beta}_M(s_p\prime)| > 0 \rightarrow |\Gamma^{\beta}_P(s_p\prime)| > 0$ ).  
If the environment has an infinite
win in $G(\mathcal{I}_P)$ not feasible in $G(\mathcal{I}_M)$ and since $G(\mathcal{I}_P) \subseteq G(\mathcal{I}_M)$
it follows that the system was able to take the play
out of the domain of the counter strategy.  In order
to accomplish this a state $s_a$ must be reached in
$\pi = s_0  \ldots s_a  \ldots$ where the environment
can no longer play according to the counter strategy.
The offending move has to be performed by the system, since
the environment would otherwise play according to his
winning strategy.
If $s_a \in S_P$ and $s_a$ is controllable we know that
under alternation $\Gamma^{+}_P(s_a) = \Gamma^{+}_M(s_a)$, i.e., all controllable options are
preserved at $s_a$.  If this is the case the strategy should
hold since it wins in $P$ against every system choice for
the preserved states.  
If $s_a \in S_{G_M} \setminus S_{G_P}$ there should exist $s_b$ 
the last state before $s_a$ ($\pi = s_0 \ldots s_b \ldots s_a \ldots$)
where the play stepped out of $S_{G_P}$.  If $s_b$ is non controllable
it would keep playing according to the counter strategy, 
so we can assume that $s_b$ is controllable, but if this is
the case, again, since all the controllable options are
preserved in $G(\mathcal{I}_P)$ and the winning strategy holds, by definition,
against every system choice, the environment should be able
to keep playing accordingly, not stepping out of $G(\mathcal{I}_P)$.  
\end{proof}
