\subsection{t-strong-fairness}
\begin{figure}[bt]
\centering
\SmallPicture
%\ShowFrame
\VCDraw{
    \begin{VCPicture}{(-4,-1.5)(4,2.3)}
        \SetEdgeLabelScale{1.4}
        \State[1]{(-3,0)}{1}
        \State[2]{(0,0)}{2}
        \State[3]{(3,1)}{3}
        \State[4]{(-0.5,2)}{4}
        \State[5]{(3,-1)}{5}
		\Initial[w]{1}
        \ChgEdgeLineStyle{dashed} %\EdgeLineDouble
        %\ChgEdgeLineWidth{1.5}
        \EdgeL{1}{2}{try}
        %\RstEdgeLineWidth{1}
        \RstEdgeLineStyle %\EdgeLineSimple
        \EdgeL{2}{3}{succ}
        \EdgeL{2}{5}{fail}
        \VArcR{arcangle=-20}{3}{1}{\ell}
        \ArcL{5}{1}{A}
        \ArcR{3}{4}{A}
        \ArcR{4}{1}{G}
    \end{VCPicture}
}
\vspace*{-2mm}
\caption{t-strong fairness example.}
\label{fig:strongfairness}
\vspace*{-4mm}
\MediumPicture
\end{figure}
In \cite{DBLP:conf/icse/DIppolitoBPU11} a concise example was 
presented to show that the notion of t-strong-fairness was not
enough to achieve synthesis of controller over fallible domains.
The plant is depicted in figure \ref{fig:strongfairness}, where the only
controllable action is $try$.  The system should achieve
to produce event $G$ infinitely often provided that the environment
agrees to produce $A$ infinitely often.  The idea in 
\cite{DBLP:conf/icse/DIppolitoBPU11} is to treat a
failure as something not systematic, different from a truly antagonistic
event.  The example shows how this
could not be captured with weaker notions of fairness, we take 
the plant in two different incarnations.  In the first we keep
the liveness properties intact, i.e., the system should accomplish 
$\square \Diamond G$ if the environment agrees to comply with
$\square \Diamond A$.  After applying our technique we get the
minimized plant shown in figure \ref{fig:strongfairness-min1}.
\begin{figure}[bt]
\centering
\SmallPicture
%\ShowFrame
\VCDraw{
    \begin{VCPicture}{(-4,-1.5)(4,2.3)}
        \SetEdgeLabelScale{1.4}
        \State[1]{(-3,0)}{1}
        \State[2]{(0,0)}{2}
        \State[5]{(3,-1)}{5}
		\Initial[w]{1}
        \ChgEdgeLineStyle{dashed} %\EdgeLineDouble
        %\ChgEdgeLineWidth{1.5}
        \EdgeL{1}{2}{try}
        %\RstEdgeLineWidth{1}
        \RstEdgeLineStyle %\EdgeLineSimple
        \EdgeL{2}{5}{fail}
        \ArcL{5}{1}{A}
    \end{VCPicture}
}
\vspace*{-2mm}
\caption{First minimization.}
\label{fig:strongfairness-min1}
\vspace*{-4mm}
\MediumPicture
\end{figure}
The minimization clarifies the cause of non realizability
since it is clear that the environment can keep the play
within the language described as $(try, fail, A)^*$ .\\
Now, we can add a new environmental liveness assumption,
forcing the environment to produce infinitely many $succ$
events.  After applying our technique we get the result in
figure \ref{fig:strongfairness-min2}.  This minimization is 
better explained in conjunction with the strategy shown in 
figure \ref{fig:strongfairness-min-counter-strat}. We can see
that state 4 has been removed and state 2 preserves both
the $succ$ and $fail$ options since the environment will take
each one alternately to comply with the $\square \Diamond A$
assumption while avoiding state 4, since this will allow 
infinitely many $G$s.
\begin{figure}[bt]
\centering
\SmallPicture
%\ShowFrame
\VCDraw{
    \begin{VCPicture}{(-4,-1.5)(4,2.3)}
        \SetEdgeLabelScale{1.4}
        \State[1]{(-3,0)}{1}
        \State[2]{(0,0)}{2}
        \State[3]{(3,1)}{3}
        \State[5]{(3,-1)}{5}
		\Initial[w]{1}
        \ChgEdgeLineStyle{dashed} %\EdgeLineDouble
        %\ChgEdgeLineWidth{1.5}
        \EdgeL{1}{2}{try}
        %\RstEdgeLineWidth{1}
        \RstEdgeLineStyle %\EdgeLineSimple
        \EdgeL{2}{3}{succ}
        \EdgeL{2}{5}{fail}
        \VArcR{arcangle=-20}{3}{1}{\ell}
        \ArcL{5}{1}{A}
    \end{VCPicture}
}
\vspace*{-2mm}
\caption{Second minimization.}
\label{fig:strongfairness-min2}
\vspace*{-4mm}
\MediumPicture
\end{figure}
\begin{figure}[bt]
\centering
\SmallPicture
%\ShowFrame
\VCDraw{
    \begin{VCPicture}{(-4,-1.5)(4,2.3)}
        \SetEdgeLabelScale{1.4}
        \State[1]{(-3,0)}{1}
        \State[2]{(-1,-1)}{2}
        \State[3]{(1,-1)}{3}
        \State[4]{(3,0)}{4}        
        \State[5]{(1,1)}{5}
        \State[6]{(-1,1)}{6}        
		\Initial[w]{1}
        \ChgEdgeLineStyle{dashed} 
        \EdgeL{1}{2}{try}
        \RstEdgeLineStyle
        \EdgeL{2}{3}{fail}
        \EdgeL{3}{4}{A}
        \ChgEdgeLineStyle{dashed} 
        \EdgeR{4}{5}{try}
        \RstEdgeLineStyle
        \EdgeR{5}{6}{succ}        
        \EdgeR{6}{1}{\ell}                
    \end{VCPicture}
}
\vspace*{-2mm}
\caption{Environment counter strategy.}
\label{fig:strongfairness-min-counter-strat}
\vspace*{-4mm}
\MediumPicture
\end{figure}
This example is very synthetic but depicts a scenario
where behavior based diagnosis proves itself a useful
tool not only for repairing a specification but to reason 
incrementally about requirements' limitations.
%the two cases, the second introduces the problem of
%memory unfolding