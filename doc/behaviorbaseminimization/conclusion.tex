\section{Conclusions and Future Work}\label{sec:conclusion}
In this paper we have presented a technique that minimizes behavior while
preserving unrealizability in order to provide feedback to an engineer.
The problem was introduced in the context of operational specifications where
the liveness guarantees and assumptions are provided as LTL formulas and safety behavior
is expressed as a composition of LTS automata.  %The characterization of
%the search space as a semi lattice of sub LTSs derived from the original
%plant allowed us to introduce an eager algorithm that finds a minimal
%representative of non realizability preserving behavior.  
We believe that our work complements pre-existing approaches based on LTL minimization. 
%and it should serve
%as a kick start to the development of other diagnosis techniques
%for this kind of specifications.  
%It should also promote the publication of non realizable operational specifications
%from industrial domains.  
Future work can specialize
our approach by defining specific exploration heuristics of the search space for particular fragments of LTL such as GR(1) or reactiveness.  
%One should
%be able to use this to improve the search for %a minimal non realizability
%preserving representative. 
%In addition, we believe there may be opportunities in improving feedback by providing a decomposed feedback in terms of communicating automata.
% Beyond this, we think that irrelevant automata 
%can also be marked out of the
%diagnosis to improve understanding. In order to expand applicability 
%the ground formalism can be slightly adapted to cope better with
%signal-based specifications.\\
In addition, we believe that our technique can help and provide complimentary value to the
specification-repair approaches, and could be used alongside assumption mining
tools. 

